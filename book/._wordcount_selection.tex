
\section*{\normalsize Est-ce possible d'assurer une vie décente à chacun dans un monde décarboné~?}\label{q:decent}
\addcontentsline{toc}{section}{\nameref{q:decent}}

Oui. Le problème n'est pas technique, mais politique. Il existe de nombreux scénarios montrant comment opérer une mue de notre société pour atteindre la neutralité climatique dans le monde entier. Le GIEC publie des scénarios compatibles avec un réchauffement limité à 1,5\textdegree{}C, d'autres à 2\textdegree{}C, etc. Les scénarios les plus ambitieux sur le plan du climat et de la réduction de la pauvreté requièrent une baisse importante de la consommation dans les pays à hauts revenus, qui passe à la fois par des gains d'efficacité et par la sobriété. \cite{oneill_good_2018,hickel_is_2019} montrent qu'une vie décente pourrait être assurée à 7 milliards d'humains tout en respectant les limites planétaires, à condition d'opérer une décroissance dans les pays à hauts revenus. \cite{millward-hopkins_providing_2020} calculent qu'une vie décente pourrait être assurée à tous les humains en 2050 tout en ramenant la consommation d'énergie à son niveau des années 1960, malgré une population trois fois plus importante (cela nécessiterait de réduire de 60~\% la consommation d'énergie par humain). Même s'il est clair que la sobriété faciliterait grandement l'atteinte des objectifs écologiques, certains scénarios détaillent comment limiter le réchauffement à 1,5\textdegree{}C avec une <<~croissance verte~>>. Ainsi, l'\cite{agence_internationale_de_lenergie_net_2023} présente un scénario où la planète atteindrait la neutralité climatique en 2050 tout en doublant le PIB mondial d'ici-là. 

Des modèles détaillés déclinent la mue nécessaire pour décarboner chaque secteur de chaque pays. Dans ces scénarios, l'essentiel de la décarbonation s'appuie sur des technologies déjà déployées à grande échelle (énergies renouvelables, batteries, isolation des bâtiments, pompes à chaleur) ou en cours de déploiement (hydrogène vert, acier et ciment bas carbone, capture du carbone). En d'autres termes, des technologies peuvent être déployées pour se passer d'énergies fossiles et pour éliminer de l'atmosphère le CO$_\text{2}$ dû aux émissions résiduelles. Sans ces technologies, nous n'aurions aucun moyen de mettre fin au changement climatique, puisque la réduction des émissions ne suffit pas --- il faut les ramener à zéro. 

Pour autant, il est peu probable que les technologies soient déployées assez rapidement pour limiter le réchauffement à 1,5\textdegree{}C. Les politiques et actions 
actuelles nous orientent vers un réchauffement de 2,6\textdegree{}C à 2,9\textdegree{}C en 2100\footnote{Cf. \href{https://climateactiontracker.org/global/temperatures/}{climateactiontracker.org/global/temperatures}.} et une température qui continuerait de croître à un rythme alarmant après 2100. Dans ce contexte, plus les efforts de sobriété seront importants, plus le réchauffement sera ralenti. 

Ces observations devraient mettre d'accord les tenants de la décroissance avec ceux de la croissance verte. D'une part, il faut stimuler la croissance de la \textit{productivité}, notamment pour améliorer notre efficacité énergétique et déployer des technologies plus écologiques. D'autre part, il faut favoriser la décroissance de la \textit{surconsommation}, pour réduire les dégâts que le dépassement des limites planétaires inflige aux plus vulnérables. 

\section*{\normalsize Qui paie dans le système proposé~: les entreprises ou les consommateurs~?}\label{q:incidence}
\addcontentsline{toc}{section}{\nameref{q:incidence}}

En économie, il faut distinguer l'incidence \textit{légale} de l'incidence \textit{économique}. Légalement, ce sont les entreprises en amont de la chaîne de production qui seraient assujetties et seraient obligées d'acheter des permis d'émissions. Mais cette incidence légale ne permet pas de savoir qui va payer. En principe, les entreprises assujetties peuvent réagir de trois façons différentes~: en réduisant leurs profits, en réduisant leurs salaires, ou en augmentant leurs prix. Le coût de la mesure serait alors respectivement reporté sur les actionnaires, les travailleurs des secteurs polluants, ou les consommateurs. \cite{ganapati_energy_2020} estiment qu'environ 70~\% des hausses de prix énergétiques auxquels font face le secteur manufacturier sont reportés sur les consommateurs à court ou moyen terme, le reste étant absorbé par les actionnaires. À long terme, on peut s'attendre à ce que le taux de profit se stabilise, et que les consommateurs paient l'intégralité du coût. C'est le mécanisme qui est décrit au Chapitre \ref{ch:coeur}~: le prix du carbone est payé par les consommateurs, en proportion de leur empreinte carbone. C'est d'ailleurs le but de la tarification carbone~: sans hausse du prix des biens carbonés relativement aux options bas carbone, personne ne serait incité à réduire ses émissions.

\section*{\normalsize Quid des autres gaz à effet de serre~? Des autres limites planétaires~? De la biodiversité~?}\label{q:scope}
\addcontentsline{toc}{section}{\nameref{q:scope}}

Hélas, le Plan proposé ne traite pas ces problèmes. Il faut donc consulter d'autres travaux pour constituer un programme qui répondrait à tous les défis écologiques\footnote{Par exemple \citet{strassburg_reducing_2009,karsenty_geopolitique_2021} dans le cas des forêts.}. %
Notons simplement que la logique du Plan pourrait être répliquée pour résoudre d'autres problèmes (pas tous). Par exemple, on pourrait imaginer un système de quotas équivalent sur les ressources épuisables telles que les métaux ou les stocks de poisson. Les permis d'extraction %
ou de pêche seraient vendus aux enchères, et les recettes distribuées à part égale entre tous les humains.

\section*{\normalsize Les émissions ne vont-elles pas augmenter si on double les revenus des plus pauvres~?}\label{q:emissions}
\addcontentsline{toc}{section}{\nameref{q:emissions}}

Si on ne faisait que redistribuer les revenus, les émissions augmenteraient, car les plus modestes consacrent une plus grande part de leurs revenus à la consommation de biens carbonés que les plus aisés. Et encore, la hausse serait assez limitée. \cite{sager_income_2019} estime qu'une égalisation complète des revenus des États-uniens (à revenu moyen constant) augmenteraient leurs émissions de gaz à effet de serre de 2~\%. De même, \cite{oswald_global_2021} trouvent qu'avec une égalisation quasi-complète des revenus des humains (ramenant le revenu maximal à deux fois le revenu minimal), la consommation d'énergie augmenterait de 7~\%. 
Mais, par construction, le Plan mondial pour le climat plafonnerait les émissions selon une trajectoire baissière. Ainsi, les émissions ne pourraient que baisser. La hausse spectaculaire de la consommation (donc des émissions) des plus pauvres serait plus que compensée par la baisse de la consommation des plus riches et par la décarbonation (c'est-à-dire la baisse des émissions liées à un niveau de consommation donné).

\section*{\normalsize Ce système ne profite-t-il pas aux plus riches, en leur permettant d'acheter un droit à polluer~?}\label{q:riches}
\addcontentsline{toc}{section}{\nameref{q:riches}}

Un argument qui revient souvent contre la tarification carbone est qu'elle donnerait un passe-droit aux plus riches pour polluer, et ferait reposer le coût de la décarbonation sur les classes moyennes, qui n'ont pas les moyens de faire face aux hausses de prix. 

La réponse courte à cette objection, c'est qu'il faut bien spécifier à quoi on se compare. Si on compare le Plan mondial pour le climat à une proposition bien plus radicale telle qu'une sortie du capitalisme où les revenus seraient plafonnés à 3~000~\euro{} par mois, effectivement, les plus riches s'en sortent bien. En revanche, si on le compare au statu quo, les plus riches sont perdants, et ce d'autant plus que sont mises en œuvre les mesures complémentaires décrites au Chapitre \ref{ch:premier_pas}. 

Pour la réponse longue, commençons par rappeler les effets distributifs du Plan~: les individus avec une empreinte carbone élevée perdraient financièrement, tandis que les individus avec une empreinte carbone autour de la moyenne mondiale pourraient effectuer la mue écologique sans perdre en pouvoir d'achat. Ces individus <<~moyens~>> seraient incités à changer leurs équipements et leurs habitudes par le prix du carbone, et ceux qui basculeraient vers les options décarbonées plus vite que les autres deviendraient gagnants financièrement (puisque leur empreinte carbone passerait sous la moyenne mondiale). En revanche, les classes moyennes dans les pays à hauts revenus ont une empreinte carbone supérieure à la moyenne mondiale~: elles seraient donc perdantes financièrement si elles ne s'adaptent pas, et l'adaptation présente de toute façon elle-même un coût (monétaire ou de confort). Pour autant, les plus aisés perdraient davantage, puisqu'ils ont une empreinte carbone encore plus élevée. 

Certes, on peut légitimement trouver que les plus aisés ont davantage de marges de manœuvre pour s'adapter, et considérer qu'ils devraient être mis encore plus à contribution que ce qu'implique le Plan. C'est d'ailleurs pour cette raison que je préconise des mesures complémentaires de redistribution nationale (cf. Chapitre \ref{ch:premier_pas}) pour faire porter l'intégralité du coût de décarbonation sur les plus aisés, et préserver le niveau de vie des classes moyennes dans les pays à hauts revenus. Pour autant, si la majeure partie de la population soutient des mesures de redistribution nationale, c'est parce qu'elle considère la répartition des richesses trop inégale, qu'il y ait une politique climatique ou pas. En d'autres termes, on peut séparer les deux propositions~: d'une part, une politique climatique qui n'aggrave pas les inégalités (en fait, le Plan mondial les réduit)~; d'autre part, une politique de redistribution nationale qui met fin à un niveau indécent d'inégalité. Si on cherchait à constituer un programme politique complet, il faudrait y inclure ces deux propositions, ainsi que de nombreuses autres. Mais ce n'est pas l'objet de ce livre que de constituer un programme. Nous nous focalisons ici sur une proposition, et la concevons d'une façon qu'elle puisse être soutenue par des individus et des gouvernements de tous bords. 

Pour résumer, je considère que le Plan mondial pour le climat est préférable au statu quo, et qu'il doit être complété par des mesures de redistribution supplémentaires. Dans les questions suivantes, j'expliquerai en quoi il me semble préférable à des mesures climatiques alternatives. 


\section*{\normalsize Ce Plan ne permettrait-il pas au capitalisme de perdurer, alors qu'il faudrait le renverser~?}\label{q:capitalisme}
\addcontentsline{toc}{section}{\nameref{q:capitalisme}}

Pour ne pas rentrer dans un débat sans fin entre réformisme et révolution, je m'en tiendrai à trois arguments. Premièrement, on peut soutenir le Plan mondial pour le climat car il va dans la bonne direction, tout en préférant et en élaborant un plan plus ambitieux. Deuxièmement, l'ampleur de la redistribution opérée par les propositions de ce livre me semble être proche de celle qu'on peut espérer la plus radicale %
dans la décennie qui vient, étant donné le rapport de force actuel --- dominé par des groupes sociaux aisés qui tiennent à leur niveau de confort. %
Troisièmement, même dans une société post-capitaliste, il faudrait plafonner les émissions de façon contraignante, et le Plan proposé me paraît la meilleure option pour ce faire (comme expliqué ci-dessous). 

\section*{\normalsize Ne faut-il pas simplement %
interdire les activités vouées à disparaître et subventionner celles appelées à se développer~?}\label{q:interdiction}
\addcontentsline{toc}{section}{\nameref{q:interdiction}}

Il y a certainement des normes à instaurer, des activités ou des produits à interdire et d'autres à subventionner. Par exemple, on pourrait interdire la vente de véhicules thermiques et de chaudières à gaz ou au fioul, ainsi que la construction d'aciéries, de cimenteries et de centrales électriques qui dépassent un certain niveau d'émissions. Des subventions pourraient rendre le coût des équipements décarbonés supportable pour les ménages et les nouvelles usines bas carbone compétitives face aux usines polluantes existantes. Pour que la décarbonation s'effectue de la sorte dans les pays du Sud, il faudrait sans doute que les pays du Nord financent leurs subventions, mais c'est également envisageable. Rien n'empêche de mettre en place de telles mesures en complément du Plan mondial --- et c'est d'ailleurs le sens des propositions du Chapitre \ref{ch:premier_pas}. Si ces interdictions et ces subventions réduisaient les émissions sous le plafond qu'on s'est fixé, le prix du carbone serait de zéro, car il ne serait pas nécessaire d'inciter à davantage de réductions d'émissions. Le Plan mondial pour le climat serait alors indolore. Il n'en serait pour autant pas moins utile. En effet, il n'est pas certain que les interdictions et les subventions engendrent des réductions d'émissions suffisantes~: le plafond instauré par le Plan offre en la matière une garantie précieuse.

En outre, les normes, interdictions et subventions ont différents défauts, qui les rendent inadaptées à certaines situations. Premièrement, elles ne profitent pas toujours aux plus modestes~: par exemple, des subventions à la rénovation thermique peuvent bénéficier disproportionnément aux propriétaires (dont les logements vont gagner en valeur suite à la rénovation), tandis que l'interdiction des véhicules polluants dans les centres-villes affecte davantage les ménages modestes. Deuxièmement, les normes et les subventions créent souvent un effet rebond, qui en amoindrit l'efficacité en suscitant une hausse de la consommation. Par exemple, en subventionnant les voitures électriques, on encourage l'usage de la voiture plutôt que du vélo ou des transports en commun. 
Économiquement, un bonus/malus sur l'achat d'un véhicule électrique/polluant est équivalent à une norme sur les émissions (de la moyenne) des nouveaux véhicules, et ces mesures entraînent toutes deux un usage plus important de la voiture par rapport à une tarification du carbone\footnote{Cf. \cite{fullerton_suggested_2003}.}. Et si on subventionne tous les moyens de transport pour ne pas favoriser la voiture, on encourage sans raison la mobilité, et avec elle l'étalement urbain et les temps de transport. %
Troisièmement, les régulations portant sur un secteur ou une technologie spécifique sont moins efficientes que la tarification indiscriminée des émissions, car elles favorisent de façon discrétionnaire des secteurs ou des technologies spécifiques. Cet effet n'est même pas lié au fait que les régulations spécifiques peuvent être davantage sujettes au lobbying, à la corruption ou aux erreurs de l'administration. 
Pour comprendre, imaginons qu'un pays favorise l'électricité d'origine renouvelable par des subventions ou des cibles contraignantes, plutôt que de tarifer les émissions de CO$_\text{2}$ du secteur électrique. C'est à gros traits ce qu'ont fait les États-Unis. Le charbon n'est ainsi pas pénalisé par rapport au gaz (pourtant moins polluant), et son usage dans la génération d'électricité est alors excessif. %

Ces défauts sont à mettre en regard des défauts de la tarification carbone, et notamment les disparités de situations qu'elle crée, cf. la Section \ref{sec:mue_nationale}. En l'absence de solution miracle, l'optimum consiste à mettre en œuvre une panoplie de mesures complémentaires, dans laquelle aussi bien la tarification que les normes, interdictions et subventions ont leur place. Le Plan mondial a sa place dans cette panoplie car il présente deux avantages majeurs~: d'une part, il permet de définir et de garantir une ambition climatique internationale, sous la forme du budget carbone~; d'autre part, ce Plan est plus facile à négocier que des accords alternatifs. En effet, le principal élément du Plan à négocier est le budget carbone. On peut imaginer des accords alternatifs, par exemple des transferts venant des pays Nord en échange d'actions pour le climat de la part des pays du Sud\footnote{Des partenariats de ce genre sont en train d'être noués par certains pays du Sud. L'Afrique du Sud, l'Indonésie, le Sénégal et le Vietnam ont chacun signé un \textit{Partenariat pour une transition énergétique juste} (ou \textit{JETP}) avec un groupe de pays du Nord. Par exemple, l'Indonésie s'est engagée à accélérer sa sortie du charbon et la décarbonation de son système électrique en échange d'un financement de 20 milliards de dollar, principalement sous la forme de prêts concessionnels \citep{ha-duong_just_2023}. Conditionnellement au respect de cet engagement, un groupe de pays du Nord s'est engagé à fournir ce financement. Par rapport au Plan mondial pour le climat, les JETP souffrent de plusieurs défauts~: ils financent des pays à revenus intermédiaires, ne s'attaquent pas à la pauvreté et impliquent des transferts relativement faibles de la part des pays du Nord. En réalité, les JETP permettent à des banques de pays du Nord de financer des projets rentables dans les pays du Sud. En outre, la couverture des JETP est loin d'être systémique puisqu'ils ne concernent quelques pays du Sud, et il est peu probable qu'ils puissent s'étendre au-delà du secteur électrique. En résumé, les JETP ne permettent pas de garantir le respect du budget carbone ni de mettre fin à l'extrême pauvreté.} 
(tels qu'un plan crédible de décarbonation d'un secteur électrique en croissance). Si d'aventure les pays réussissent à négocier un accord qui détermine une trajectoire de décarbonation aussi ambitieuse que celle du Plan avec des transferts Nord--Sud équivalents (autour de 1~\% du PIB mondial), alors cet accord pourrait tout à fait remplacer le Plan. En attendant, il me semble plus facile de s'accorder sur le Plan que sur un ensemble de mesures différenciées par région et secteur.


\section*{\normalsize Pourquoi un marché carbone plutôt qu'une taxe~?}\label{q:taxe}
\addcontentsline{toc}{section}{\nameref{q:taxe}}

L'avantage d'un marché du carbone est qu'il instaure un plafond aux émissions, qui garantit qu'on suive la trajectoire d'émissions souhaitée. Si on fixe la trajectoire de la taxe à l'avance, il y a de fortes chances qu'on se trompe sur le niveau requis pour atteindre l'objectif climatique, et qu'on fixe la taxe à niveau soit trop faible, soit trop élevé. Si le niveau de la taxe est automatiquement réévalué chaque année de sorte à maintenir les émissions sur la trajectoire souhaitée, on retombe sur un système assez proche du marché du carbone. La différence entre les deux est de l'ordre du détail~: le marché s'ajuste immédiatement suite aux situations nouvelles (tels qu'une guerre ou l'entrée d'un nouveau pays dans le système), au prix d'un écosystème d'acteurs financiers qui analysent, opèrent et se rémunèrent sur ce marché. Les deux systèmes étant très proches, une taxe réévaluée automatiquement conviendrait tout autant. Je préfère toutefois présenter le Plan sous la forme d'un marché, pour qu'on ne le confonde pas avec les taxes carbones habituelles, dont la trajectoire est fixée à l'avance. 

Enfin, si le niveau de la taxe était réévalué chaque année par une décision politique (plutôt qu'automatique), cela offrirait à des groupes d'intérêts des occasions incessantes pour remettre en cause le niveau d'ambition climatique et s'écarter de l'objectif climatique. Pour mieux résister à de telles pressions, il me paraît préférable de fixer à l'avance la trajectoire d'émissions. Cette préférence consiste à prioriser l'objectif climatique, quitte à accepter des baisses de niveaux de vie importantes si la décarbonation se révèle plus coûteuse que prévue. 
Au contraire, d'aucuns préfèreraient sans doute limiter les efforts à court terme, quitte à infliger plus de dégâts aux générations futures si les efforts consentis se révèlent moins efficaces que prévus. Si cette dernière inclination était majoritaire, l'objectif climatique ne pourrait plus être garanti. Il faudrait alors se résoudre à avoir un prix plafond sur le marché carbone, ou une taxe carbone dont le niveau serait choisi politiquement\footnote{On pourrait alors fixer le niveau de la taxe à la médiane des niveaux préférés, comme proposé par \cite{weitzman_world_2017}.}. 

\section*{\normalsize Pourquoi pas une taxe carbone progressive~?}\label{q:taxe_progressive}
\addcontentsline{toc}{section}{\nameref{q:taxe_progressive}}

Certains auteurs comme Thomas \cite{piketty_capital_2019} préconisent une taxe carbone progressive, avec les premières tonnes d'émissions individuelles peu ou pas taxées, les suivantes taxées davantage, et ainsi de suite jusqu'à un niveau d'émissions maximal. À mon sens, c'est une fausse bonne idée. 

Tout d'abord, nous sommes loin d'avoir les moyens administratifs de calculer précisément l'empreinte carbone d'un individu~: pour ce faire, il faudrait un traité international obligeant les entreprises à déclarer leurs transactions. Certes, un tel traité serait bienvenu, et même sans lui, on pourrait approximer les empreintes carbone. Mais, en approximant les empreintes carbone, on réduirait les incitations à la décarbonation. Par exemple, en attribuant le même contenu carbone à chaque smartphone, un fabricant de smartphone n'aurait pas intérêt à faire d'efforts puisqu'il ne serait pas distingué des autres. Tant administrativement qu'économiquement, il est plus efficace de faire payer le prix au producteur en amont de la chaîne de valeur plutôt qu'au consommateur en aval. 

Mais le véritable écueil est ailleurs~: les effets d'une telle mesure ne seraient pas nécessairement désirables. En effet, prenons deux individus ayant un revenu de 2~000~\euro{} par mois~: l'un vit dans un pavillon mal isolé, chauffé au fioul, et roule 50 km par jour pour se rendre à son lieu de travail~; l'autre vit dans un immeuble bien isolé, chauffé par géothermie, et se déplace à vélo. Le premier a une empreinte carbone cinq fois plus élevée que le second. Avec une tarification du carbone classique, le premier perd déjà en pouvoir d'achat par rapport au second. Avec une taxation qui progresse avec l'empreinte carbone, la disparité se creuse encore plus. Au contraire, il faudrait probablement chercher à réduire les disparités d'effets des politiques climatiques entre ces deux types de personnes (c'est le sens de mesures complémentaires proposées au Chapitre \ref{ch:premier_pas}). À la limite, une telle proposition serait intéressante si l'empreinte carbone à partir de laquelle le taux de taxation augmente était suffisamment élevée pour épargner les classes moyennes (disons, à 30 tonnes de CO$_\text{2}$ par an) ou si elle ne s'appliquait qu'à l'aviation. %

Mais la taxation carbone progressive pourrait prendre une forme plus pertinente~: le taux de taxe pourrait progresser avec le \textit{revenu} de l'individu, plutôt qu'avec son \textit{empreinte carbone}. Ainsi, on éviterait d'exacerber les disparités d'effets pour un même niveau de revenu, tout en pénalisant disproportionnément les plus riches. Une telle solution permettrait d'imposer un niveau d'effort de décarbonation comparable à tous les niveaux de revenus. 

Ces deux formes de taxation progressive se heurteraient à un problème commun~: le cas des entreprises. À quel taux devrait-on taxer les émissions d'une société~? Si on les taxe à un taux faible, les plus aisés auront intérêt à faire passer leurs dépenses personnelles comme frais professionnels. Si on les taxe à un taux élevé, on favoriserait le travail informel ainsi que des fraudes dans l'autre sens, où les salariés feraient passer des dépenses professionnelles comme frais personnels (en échange de primes). Il faudrait donc probablement taxer les émissions des entreprises à un taux intermédiaire. Pour éviter les fraudes, on pourrait choisir un faible écart entre le taux minimal et le taux maximal, mais cela réduirait la portée de la mesure. Une autre piste serait d'attribuer certaines dépenses des entreprises (déplacements, frais de bouche, hébergement) aux individus qui en bénéficient (généralement des salariés), et de taxer ces individus, du moins à partir d'un certain seuil de dépenses. 

Au final, si l'objectif de ce genre de propositions est de réduire les inégalités, il me semble plus simple de compléter le Plan par une redistribution des richesses, comme proposée au Chapitre \ref{ch:premier_pas}. Cela dit, la dernière piste évoquée est à creuser. Elle n'est d'ailleurs pas incompatible avec le Plan mondial pour le climat~: celui-ci pourrait être complété par une taxation supplémentaire de l'empreinte carbone des plus aisés. 

\section*{\normalsize Pourquoi pas un rationnement de l'empreinte carbone individuelle~?}\label{q:rationnement}
\addcontentsline{toc}{section}{\nameref{q:rationnement}}

Quelques %
personnes\footnote{\cite{wood_rationing_2023}.} proposent un système de rationnement des émissions individuelles, où les échanges de permis d'émissions seraient interdits. En d'autres termes, contrairement à notre Plan qui revient à un système de quotas carbones \textit{échangeables}, le rationnement interdirait aux individus d'acheter des permis s'ils en manquent ou à en vendre s'ils en ont en excès. %
Le rationnement serait problématique à plusieurs titres. 

Déjà, si les permis d'émissions ne sont pas échangeables, cela signifie soit que des centaines de millions de personnes (notamment dans les pays du Nord) devraient diviser leurs émissions par deux ou trois du jour au lendemain, les mettant dans l'impossibilité de poursuivre leurs activités quotidiennes~; soit qu'on allouerait (dans un premier temps) davantage de permis d'émissions aux personnes qui polluent davantage, ce qui romprait avec le principe d'égalité cher aux défenseurs des quotas non échangeables. A contrario, si les permis sont échangeables, les pollueurs auraient une certaine latitude concernant leurs émissions et du temps pour adapter progressivement leurs activités et changer leur équipement. Quant aux personnes avec une faible empreinte carbone, elles pourraient revendre leurs permis d'émissions inutilisés et ainsi gagner du pouvoir d'achat. Ainsi, tant les pollueurs que les frugaux bénéficieraient de la flexibilité permise par le marché. D'ailleurs, le bénéfice potentiel serait tellement important que l'émergence d'un marché noir serait difficile à empêcher dans le cas d'un système de quotas non échangeables. 

\section*{\normalsize Est-il moral de laisser les riches acheter des droits à polluer~?}\label{q:moral}
\addcontentsline{toc}{section}{\nameref{q:moral}}

Vous vous dites peut-être qu'un marché du carbone serait immoral ou injuste, car il permettrait aux plus riches de continuer à polluer. Pourtant, un tel système opérerait une redistribution des pollueurs vers les frugaux~: ceux-là devant payer pour acheter des permis d'émissions à ceux-ci. En outre, mettre en place un marché du carbone n'empêche pas d'interdire par ailleurs les consommations jugées superflues, telles que les yachts, les jets privés, voire les SUV. Enfin, si on considère injuste que les plus riches soient capables de préserver un mode de vie dispendieux dans un tel système, n'est-ce pas parce qu'on considère l'extrême richesse comme injuste~? Si c'est le cas, autant s'attaquer à la fortune directement, plutôt que de passer par des moyens détournés. En effet, plafonner les émissions d'un Rupert Murdoch ne l'aurait pas empêché d'utiliser son empire médiatique pour minimiser, voire nier le changement climatique. D'ailleurs, plafonner les émissions des milliardaires ne les empêcherait même pas d'utiliser un jet privé~: ils remplaceraient simplement le kérosène par des agrocarburants ou de l'hydrogène. Si l'objectif du rationnement est d'empêcher les activités fastueuses, %
il vaut mieux passer par une interdiction\footnote{Si on souhaite limiter (plutôt qu'interdire) une activité particulière telle que les vols en avion, on pourrait rationner cette activité. Cependant, même dans cette optique où on ne se satisfait pas du prix du carbone, il semble préférable de concevoir un système d'autorisation (voire de taxation) différenciée pour distinguer les vols justifiés (motif professionnel ou familial) des vols superflus (tourisme). 
} de ces activités ou un plafonnement de la richesse.  %

\section*{\normalsize Le revenu de base est-il la meilleure façon de distribuer des ressources aux plus pauvres~?}\label{q:rdb}
\addcontentsline{toc}{section}{\nameref{q:rdb}}

Le revenu de base présente plusieurs avantages. Tout d'abord, s'il est correctement mis en œuvre, il permet d'atteindre tous les humains, sans laisser personne sur le bord de la route. A contrario, si les fonds étaient alloués à des institutions plutôt qu'aux personnes, certains groupes sociaux (tels que les urbains, les hommes, ou les groupes au pouvoir) pourraient être favorisés au détriment d'autres. Par ailleurs, le revenu de base permet de répondre aux besoins de chaque individu et d'émanciper tout un chacun. Enfin, les études scientifiques montrent que les transferts d'argent inconditionnels dans les pays à bas revenus procurent une amélioration significative de la nutrition, du bien-être, de la santé, ainsi qu'une hausse de l'activité (beaucoup profitant du revenu de base pour investir dans leur micro-entreprise) et un enrichissement durable\footnote{\cite{haushofer_short-term_2016,egger_general_2022,standing_little_2014}.}. Par rapport à des transferts ciblés (vers certains pays ou vers les plus démunis) ou conditionnés (à la présence des enfants à l'école par exemple), le revenu de base présente l'immense avantage de la simplicité et de l'universalité, ce qui rend sa distribution moins sujette aux erreurs, aux abus et aux critiques. Le principal inconvénient du revenu de base est que sa distribution nécessite une infrastructure qui reste à construire dans certains pays (cf. Section \ref{sec:implementation}).  %

Les alternatives ont elles aussi des avantages. Distribuer les ressources au gouvernement permet de développer les services publics et la protection sociale. Allouer l'argent aux autorités locales (communes, chefs de village, associations de citoyens, groupements d'intérêts économiques) permet de financer des travaux collectifs (assainissement, irrigation, etc.). En outre, le développement des pays du Sud requiert également de financer des projets à large échelle (barrages, réseau ferroviaire, etc.) pilotés par l'État ou des agences de développement. 

Si ces institutions méritent d'être financées, leur financement peut s'appuyer sur 
le revenu de base plutôt que de s'y substituer. En effet, en accroissant les revenus de la population et en développant une infrastructure de moyen de paiement, le revenu de base permettra aux autorités locales et nationales de prélever davantage d'impôts. Aussi, plutôt que de prélever la même somme à chaque humain (ce qui serait le cas si tout ou partie du revenu de base était directement versé aux autorités), ces impôts pourront être progressifs, c'est-à-dire concentrés sur les plus riches. 

Pour ces raisons, le revenu de base est l'option qui paraît la plus souhaitable. Cela dit, comme expliqué dans le calendrier du Plan (Annexe \ref{ch:details}), la forme que prendra le versement doit être choisi par la population locale. Si le revenu de base se révèle inadapté à certains contextes ou si l'infrastructure pour sa distribution n'est pas encore prête à certains endroits, une alternative pourrait lui être préférée, en respectant le principe d'allouer les recettes proportionnellement à la population locale. 

\section*{\normalsize Peut-on éviter la fraude~?}\label{q:fraude}
\addcontentsline{toc}{section}{\nameref{q:fraude}}

Un système bien conçu permettrait de largement éviter la fraude. La Section \ref{sec:implementation} montre qu'on peut contrôler que les émissions sont correctement reportées grâce à des observations par satellite, et garantir que personne ne touche plusieurs fois le revenu de base grâce à une identification biométrique. En outre, un pays qui n'appliquerait pas correctement la tarification carbone ou ne verserait pas correctement le revenu de base sur son territoire ferait face à des sanctions allant jusqu'à l'exclusion de l'union climatique. 

Un autre type de fraude a eu lieu lors du lancement du marché carbone européen en 2009~: la fraude à la TVA sur les quotas carbone. Cette fraude a été rendue possible par une faille de conception du régime d'assujettissement des quotas à la TVA, corrigée en 2010\footnote{\cite{cour_des_comptes_fraude_2012}.}. Une telle fraude ne s'est pas reproduite sur les autres marchés carbone, puisque les autorités sont désormais vigilantes à leur conception.


\section*{\normalsize La population ne va-t-elle pas s'opposer au Plan lorsqu'elle réalisera l'ampleur des efforts nécessaires~?}\label{q:soutien}
\addcontentsline{toc}{section}{\nameref{q:soutien}}

Même s'il est impossible de prévoir l'avenir, il paraît improbable que la population s'oppose au Plan, et encore moins probable %
qu'elle s'oppose davantage au Plan qu'à une politique climatique nationale d'ambition équivalente, et ce pour plusieurs raisons. 

D'une part, les enquêtes représentatives reflètent bien %
l'opinion publique. 
Par exemple, lors du mouvement des Gilets jaunes, 
seuls 13~\% des Français déclaraient soutenir la taxe carbone. Notons que quelques mois avant et quelques mois après, le soutien était à un niveau bien plus élevé. Ainsi, une nouvelle enquête indiquait 38~\% de soutien pour la même mesure deux ans après le début du mouvement\footnote{\cite{douenne_les_2020}.}. Si ces enquêtes révèlent que l'opinion sur une politique climatique peut changer substantiellement avec le contexte, la variation de soutien d'une année sur l'autre est restée contenue à moins de 25 points. De même, l'effet d'une campagne médiatique négative sur le Plan a été estimé à 11 points de soutien en moins 
aux États-Unis\footnote{\cite{fabre_international_2023}.}. Même dans l'hypothèse où le soutien au Plan baisserait de 25 points, il resterait majoritaire en Europe, puisqu'il est actuellement à 76~\% (cf. Chapitre \ref{ch:soutien}). Certes, avec un changement d'opinion, le soutien pourrait devenir minoritaire aux États-Unis, mais nous partons déjà du principe que les États-Unis ne participeront probablement pas au Plan. 

Par ailleurs, les enquêtes convergent sur différents aspects cohérents avec un fort soutien au Plan. D'abord, les enquêtes montrent toutes que la plupart des humains se préoccupent du changement climatique et soutiennent l'action climatique. Par exemple, des enquêtes représentatives dans 125 pays montre que 89~\% de la population mondiale trouve que leur gouvernement national devrait en faire plus pour combattre le changement climatique, et 69~\% déclarent être prêts à contribuer 1~\% de leur revenu pour ce faire\footnote{\cite{andre_globally_2024}.}. %
Ensuite, des enquêtes ont montré que deux tiers des États-uniens et huit Français sur dix sont <<~prêts à adopter un mode de vie écologique (c'est-à-dire à manger peu de viande rouge et à faire en sorte de ne presque pas utiliser d'essence, de diesel ou de
kérosène)~>>, <<~dans l'hypothèse où tous les États du monde se mettaient d'accord pour lutter fermement contre le changement climatique, notamment en effectuant une
transition vers les énergies renouvelables, en mettant à contribution les plus riches, et en imaginant que [le pays] étende très largement l'offre de transports non polluants~>>\footnote{En France, il y a 65~\% de \textit{Oui} vs. 17~\% de \textit{Non} \citep{douenne_french_2020}~; aux États-Unis, c'est 51~\% vs. 26~\% (résultats non publiés de \citealp{dechezlepretre_fighting_2022}), le reste ne se prononce pas.}. %
Ainsi, une large majorité des humains se déclare prête à la mue écologique même quand on explicite le coût monétaire ou le changement de mode de vie qu'elle requiert. 

Si ces déclarations semblent déconnectées des choix quotidiens, c'est principalement parce que les efforts individuels sont conditionnés à une mue systémique. Outre les aspirations de justice et les besoins matériels, cette conditionnalité est aussi liée à la tendance des individus à se conformer à la norme de leur groupe social. La recherche en psychologie sociale a montré que les individus ont des attitudes ambivalentes, liées à des aspirations en conflit les unes avec les autres et que, selon le contexte, une facette de leur identité est activée plutôt qu'une autre\footnote{\cite{fielding_social_2016}.}. %
Si, actuellement, le contexte social favorise l'activation d'une identité consumériste et individualiste, l'équilibre pourrait rapidement changer, puisque de nombreuses personnes sont prêtes à activer un autre mode de leur personnalité, plus frugal et altruiste.

Enfin, les études en la matière ont mis en évidence trois perceptions clés pour que la population soutienne une politique climatique~: le fait que la mesure soit (perçue comme) effective pour lutter contre le changement climatique, dans l'intérêt des plus modestes, et dans son intérêt personnel\footnote{\cite{dechezlepretre_fighting_2022}.}. L'importance de l'effectivité permet de comprendre pourquoi les politiques climatiques à l'échelle mondiale sont préférées à des mesures de décarbonation tout aussi rapides mais cantonnées à l'échelle nationale~: seule une action mondiale peut mettre fin au changement climatique. Le Plan mondial pour le climat vérifie les deux premiers critères (effectivité et justice sociale) et pour peu qu'il soit complété par une redistribution nationale (telle que proposée au Chapitre \ref{ch:premier_pas}), il préservera également l'intérêt des classes moyennes occidentales. En faisant peser de la sorte l'essentiel du coût de la mue écologique sur les plus aisés, cette solution a toutes les chances d'être soutenue par la majorité de la population, même dans les pays à hauts revenus. Dans la mesure où les classes moyennes occidentales ne seraient pas plus affectées dans la solution proposée que dans un programme de décarbonation national, elles n'auraient pas de raison de protester contre le Plan mondial. Au vu de ces éléments, il me paraît beaucoup moins probable de voir surgir un mouvement social contre le Plan mondial que contre une taxe carbone qui nuit aux classes moyennes (telle que celle à l'origine des Gilets jaunes) ou contre l'interdiction de la production de voitures thermiques (à venir en 2035 dans l'UE). Certes, il y a néanmoins des chances que les plus riches s'opposent au Plan, car ils en seront les grands perdants. Tout l'enjeu sera alors de réussir à ce que la majorité de la population l'emporte sur l'élite, ce qui est censé se produire en démocratie.


\section*{\normalsize En quelles devises s'effectueront les échanges de permis et la distribution du revenu de base~?}\label{q:devise}
\addcontentsline{toc}{section}{\nameref{q:devise}}

Lors des enchères, n'importe quelle monnaie nationale sera acceptée pour acheter des permis d'émissions\footnote{Le taux de conversion utilisé sera le taux du marché lors de la date butoir pour la transmission des options d'achat.}. Le revenu de base sera distribué dans la monnaie nationale. L'organisme chargé de la vente aux enchères et du versement du revenu de base contractera des \textit{swaps de change} avec les banques centrales de différents pays pour acquérir les devises locales nécessaires. En pratique, les banques centrales des pays à bas revenus accumuleront des réserves de devises fortement utilisées (dollar, euro, renminbi), ce qui améliorera la stabilité financière de ces pays et leur permettra de financer des importations.

\section*{\normalsize Quelles seront les conséquences macroéconomiques du Plan (croissance, inflation, chômage)~?}\label{q:macro}
\addcontentsline{toc}{section}{\nameref{q:macro}}

Comme toute politique de décarbonation, le Plan renforcera l'activité de certains secteurs (énergies renouvelables, construction, extraction minière) et réduire l'activité d'autres secteurs (énergies fossiles, aviation). Cela se traduira par des créations d'emplois qui surpasseront les destructions d'emplois dans la plupart des pays, ainsi que par des relocalisations de lieux de travail et de logements. Ainsi, le chômage devrait diminuer au total, même s'il augmenterait dans certaines zones. L'effet sur la croissance mondiale devrait être positif, bien que la consommation finale des ménages croîtrait moins vite qu'en absence de mue écologique. En effet, une plus forte part de l'activité serait dédiée aux investissements, dont les bénéfices (en économie d'énergie notamment) ne se matérialisent qu'après coup. 

Même si les pays exportateurs d'hydrocarbures perdront d'importantes ressources avec la décarbonation\footnote{Les pays les plus affectés seront les pays du Golfe. En particulier, l'Irak, l'Iran et l'Algérie combinent une économie fortement dépendante aux énergies fossiles et un PIB insuffisant pour surmonter la perte de recettes tirées de la vente d'hydrocarbures \citep{muttitt_equity_2020}. 
Grâce à la clause d'\textit{opt out} leur permettant de conserver les recettes de la tarification carbone liées aux fossiles qu'ils mettent sur le marché mondial, %
ces pays récupéreraient une part des recettes qu'ils auraient totalement perdues en cas de décarbonation unilatérale du reste du monde.
}, il ne faut pas surestimer les pertes d'emploi qu'elle engendrerait à l'échelle mondiale. Celles-ci sont estimées à 28 millions (contre 52 millions d'emplois créés) par une étude\footnote{Cf. \cite{jacobson_100_2017}. D'après cette étude, 22 pays feraient face à une perte nette d'emplois. Les pays les plus touchés seraient Brunei (9~\% des emplois détruits en net), la Libye (7~\%), le Qatar (6~\%), la Norvège (5~\%), le Koweït (5~\%), l'Arabie Saoudite (3~\%) et l'Irak (2~\%).}, et à 9 millions (contre 14 millions de créations) par une autre
\footnote{\cite{pai_meeting_2021}. Plus que la quantité d'emplois, c'est la qualité et la localisation des emplois créés qui peut poser problème \citep{haywood_welfare_2021}.}. En d'autres termes, la décarbonation détruirait au plus 1~\% des emplois mondiaux et en créerait deux fois plus. Au niveau de l'emploi, l'ampleur de la transition est donc limitée, comparée à l'automatisation, qui menace 10 à 50~\% des emplois\footnote{\cite{frey_future_2017,lassebie_what_2022,hatzius_global_2023}.}, ou à la mécanisation de l'agriculture, qui a par exemple fait passer la part d'agriculteurs dans l'emploi des Français de 36~\% à 10~\% entre 1946 et 1976\footnote{Cf. \href{https://ourworldindata.org/grapher/urbanization-last-500-years?country=~FRA}{ourworldindata.org} et \href{https://ourworldindata.org/grapher/share-of-the-labor-force-employed-in-agriculture?tab=chart&time=1800..latest&country=FRA}{\cite{herrendorf_chapter_2014}}.} et fait croître l'urbanisation au même rythme. En revanche, il ne faut pas sous-estimer les changements de mode de vie requis par la mue écologique, et notamment le déclin de la voiture individuelle et de la consommation de bœuf.

Comme toute redistribution mondiale des richesses, le Plan va faire croître la consommation et l'activité dans les pays du Sud, et décroître la consommation des plus riches. La consommation de biens et services de base (alimentation, infrastructures, soins, éducation) va croître, au détriment de secteurs tels que le luxe et le tourisme. Le temps que l'économie s'ajuste, cette redistribution entraînera une hausse des prix des denrées de base, et potentiellement une baisse des prix dans certains secteurs (art, hôtellerie%
). 
Notons que l'inflation initiale dans les pays à bas revenus sera nécessairement plus faible que la hausse du pouvoir d'achat due au revenu de base (l'inflation étant précisément causée par la hausse du pouvoir d'achat). 

Les exemples historiques d'afflux massifs d'aide au développement en Afrique ont montré que ces transferts ont des effets positifs et conformes aux attentes (sur l'inflation, le commerce, le taux de change, les taux d'intérêt et la croissance)\footnote{\cite{strand_revenue_2009,berg_macroeconomics_2007}.}. Autre exemple, le Guyana a pu absorber un doublement de son PIB entre 2021 et 2022 (suite à l'ouverture d'un puits de pétrole) tout en maîtrisant l'inflation\footnote{\citet{fmi_guyana_2023}.}. 

L'effet d'une telle redistribution sur le PIB d'un pays tel que la France est ambigu. D'une part, la croissance accrue des pays du Sud fera croître les exportations de biens et services des pays du Nord (machines-outils, services d'ingénierie, médicaments). 
D'autre part, l'importance de secteurs en déclin (luxe et tourisme représentent à eux deux 6~\% du PIB français%
) réduirait la production nationale. L'effet total sur le PIB n'excèdera sans doute pas un ou deux pourcents dans un sens ou dans l'autre, et restera dans tous les cas dérisoire au regard des bénéfices du Plan (fin du changement climatique et de l'extrême pauvreté). 

Même si les mécanismes macroéconomiques sont bien compris et que leur quantification est soumise aux incertitudes inhérentes à la modélisation, une telle modélisation est nécessaire pour anticiper au mieux les effets du Plan. Je compte m'y atteler prochainement avec d'autres universitaires.





\section*{\normalsize Comment se positionne le Plan par rapport à d'autres revendications telles que des transferts pour les pertes et dommages ou le traité de non-prolifération des fossiles~?}\label{q:climate_movt}
\addcontentsline{toc}{section}{\nameref{q:climate_movt}}

Les pays en développement revendiquent au moins 100 milliards de dollars annuels pour compenser les pertes et dommages climatiques causés par les émissions des pays développés\footnote{Cf. \cite{tc_proposal_2023}.}. Cette compensation découle de la responsabilité historique des États, tandis que le Plan mondial pour le climat concerne les émissions futures. En outre, ces financements n'auraient pas vocation à plafonner les émissions de CO$_\text{2}$ et n'auraient aucun effet sur la décarbonation. %
Ainsi, le Plan mondial pour le climat est une revendication parallèle, qui ne doit pas se substituer à celle liées aux pertes et dommages.

Le traité de non-prolifération des combustibles fossiles est une initiative internationale pour une décarbonation juste, portée conjointement par le mouvement pour le climat, des scientifiques, et des pays insulaires du Pacifique\footnote{Cf. \href{https://fossilfueltreaty.org/fra}{fossilfueltreaty.org/fra}}. Contrairement à ce que son nom pourrait laisser penser, aucun traité n'a été ébauché dans le cadre de cette initiative, et ses revendications restent d'ordre général. De ce fait, le Plan mondial pour le climat est tout à fait compatible avec cette initiative, et peut même être compris comme une instanciation possible d'un tel traité.

\section*{\normalsize Comment ce système s'articulerait avec les outils déjà en place, comme le marché carbone européen~?}\label{q:ets}
\addcontentsline{toc}{section}{\nameref{q:ets}} %

Pour éviter de ralentir la décarbonation de certains pays déjà ambitieux, le Plan devrait s'ajouter aux instruments en place. 

Prenons l'exemple de l'ETS, c'est-à-dire le marché du carbone européen. Deux cas sont possibles. Le cas le moins probable, c'est que le prix européen du carbone (juste avant l'entrée en vigueur du système mondial) soit inférieur au prix mondial. L'application du prix mondial suffirait alors à ramener les émissions européennes sous le plafond défini par l'ETS sans avoir besoin du mécanisme de prix européen~: le prix du carbone dans l'ETS serait de zéro. Dans le cas le plus probable, le prix mondial serait inférieur au prix nécessaire sur l'ETS pour que les émissions européennes respectent le plafond européen. Dans ce cas-là, le prix du carbone total (mondial + européen) payé dans l'UE ne varierait pas lors de l'introduction du prix mondial, mais le prix mondial absorberait une partie du prix européen. En d'autres termes, l'UE perdrait des recettes au profit du reste du monde. Il faudrait donc prévoir des recettes additionnelles si l'on souhaite perpétuer les programmes financés par les recettes de l'ETS, tels que des subventions à la rénovation thermique. Encore une fois, la taxation des plus fortunés pourrait faire l'affaire (cf. Tableau \ref{tab:redistr_policies}).











\appendix

\part*{Annexe technique et méthodologique}\label{annexe}
\addcontentsline{toc}{part}{\nameref{annexe}} 

\chapter{Les détails du Plan\label{ch:details}} 

Certains points restent à préciser pour que le Plan soit complet : son calendrier, son champ d'application, son cadre, sa gouvernance, l'organisation du marché et ses mécanismes de participation. 

\paragraph{Calendrier} 
Le Plan peut être inscrit à l'ordre du jour des COP et du G20, en vue d'une mise en œuvre progressive entre 2030 et 2035. Au cours de la phase de négociation et de préparation (avant 2030), il est essentiel de demander aux citoyens du monde entier s'ils souhaitent bénéficier d'un revenu de base et d'étudier leurs préoccupations potentielles. En effet, chaque communauté devrait avoir le droit de se retirer du revenu de base (ou de le recevoir sous une forme différente, par exemple sous la forme d'un transfert à l'ensemble de la communauté plutôt qu'à des individus), afin d'éviter de perturber les structures sociales. En outre, le revenu de base devrait commencer avec des montants très faibles pour s'assurer que sa mise en œuvre se déroule sans heurts. En effet, la redistribution opérée par le revenu de base entraînerait une augmentation de la demande (et du prix) des produits de base. Malgré l'inflation, le revenu de base augmenterait le pouvoir d'achat des personnes à bas revenus, mais il est important de ne laisser personne de côté et de s'assurer que tous ceux qui veulent le revenu de base le reçoivent. Si le système d'échange de quotas d'émission est prêt avant le revenu de base, il pourrait être mis en œuvre en allouant (dans un premier temps) les recettes aux États.

\paragraph{Périmètre} 
A priori, le Plan réglementerait exclusivement les émissions de CO$_\text{2}$\footnote{Cela dit, sur le modèle du marché carbone européen, le Plan pourrait aussi couvrir des gaz à effet de serre mineurs, tels que le N$_\text{2}$O ou les PFC.}. Bien que des politiques similaires puissent être conçues pour réglementer d'autres substances, comme le méthane, il est plus approprié de traiter le CO$_\text{2}$ séparément afin de mieux gérer ses spécificités. Idéalement, le Plan devrait couvrir toutes les émissions de CO$_\text{2}$, bien qu'il puisse être plus pratique de le limiter dans un premier temps au CO$_\text{2}$ provenant des combustibles fossiles et de la production de ciment dans les grandes unités industrielles (c'est-à-dire un champ d'application semblable aux deux marchés carbone européens combinés). Le Plan devrait également couvrir les émissions de CO$_\text{2}$ provenant du transport maritime international et de l'aviation. 

\paragraph{Cadre} %
Le traité international instituant le Plan devrait préciser certains éléments non modifiables, notamment son périmètre, l'utilisation de ses recettes, ses règles de gouvernance et le budget carbone. Le principal élément à négocier serait le budget carbone, et celui-ci devrait être défini en conformité avec l'accord de Paris. Une référence à l'accord de Paris permettrait de définir un budget carbone <<~Contenant l'élévation de la température moyenne de la planète nettement en dessous de 2 °C par rapport aux niveaux préindustriels et en poursuivant l'action menée pour limiter l'élévation de la température à 1,5 °C~>>. Une interprétation possible de cet objectif est de viser une température à long terme de +1,5\textdegree{}C mais d'autoriser un dépassement temporaire dans la limite de +2\textdegree{}C. 
En considérant que viser +1,5\textdegree{}C signifie adopter un budget carbone donnant une chance sur deux d'atteindre cette cible, cette interprétation définirait un budget carbone de 500 GtCO$_\text{2}$ à partir de 2020\footnote{Cf. \cite{ipcc_climate_2021}.}, 
qui déterminerait le budget carbone du Plan en soustrayant les émissions survenant entre 2020 et le lancement du Plan. Ce budget couvrirait à la fois le budget d'émissions positives de la première phase du Plan et les budgets d'émissions positives et négatives de la seconde phase (cf. Section \ref{sec:pcp_quota}). 
La trajectoire complète d'émissions %
pourrait alors être choisie conformément à l'objectif de dépassement maximal et à la compréhension actualisée du système climatique. 
En particulier, le budget d'émissions de la première phase permettant de maintenir le réchauffement climatique sous les 2\textdegree{}C pourrait être choisi suivant une probabilité de 67~\% (ce qui implique un budget de 1~150 GtCO$_\text{2}$ à partir de 2020, soit 1~000 Gt à partir de 2024) ou une probabilité de 83~\% (900 GtCO$_\text{2}$ à partir de 2020). 

Si certains pays ne participent pas au Plan, le budget carbone du Plan serait ajusté à la baisse sur la base d'un droit d'émission égal pour chaque adulte humain (en cohérence avec le principe de laisser les mêmes droits d'émission aux pays non participants).

\paragraph{Gouvernance} 
L'organe de gouvernance du Plan définirait le quota d'émissions annuel (conformément au cadre du Plan), l'organisation du marché (\textit{market design}) et les éventuelles sanctions à l'encontre des pays non participants. %
Le choix des sanctions serait la décision la plus politique de l'organe de gouvernance. Ces sanctions pourraient se retourner contre les membres de l'union en cas d'éventuelles représailles de la part des pays sanctionnés, et ces représailles affecteraient probablement les pays en proportion de leur poids économique et géopolitique. 
En outre, le système d'échange de quotas d'émissions imposerait un coût aux pays participants en proportion de leurs émissions. Il semble donc légitime d'accorder à chaque pays un droit de vote proportionnel à ses émissions de CO$_\text{2}$, en ce qui concerne les sanctions et les décisions techniques relatives au marché carbone (les décisions importantes étant déjà réglées dans le traité). 
Pour les décisions relatives au revenu de base, chaque pays aurait un droit de vote proportionnel à sa population adulte. 

Lorsque l'organe de gouvernance devrait choisir entre plusieurs options, il devrait utiliser le vote par approbation\footnote{Dans ce système, chaque votant approuve ou désapprouve chaque option. Le vote par approbation sélectionne l'option la plus largement approuvée.}, et lorsque ces options sont numériques, utiliser la valeur médiane préférée. Enfin, chaque pays devrait être autorisé à faire siéger plusieurs représentants plutôt qu'un seul, à choisir le mode de désignation de ses représentants (éventuellement par le biais d'élections) ainsi que la manière dont les droits de vote du pays sont répartis entre ces représentants. 


\paragraph{Organisation du marché} 
La période requise pour restituer les permis d'émission devrait être d'une année civile, et le quota devrait être ajusté chaque année. Les compensations carbone ne devraient pas être autorisées en remplacement des permis d'émission. L'emprunt et la mise en réserve de permis d'émission devraient être limités dans le temps et en quantité afin d'éviter la spéculation. %

\paragraph{Mécanismes de participation}

Le mécanisme de participation de base, qui empêcherait également les fuites de carbone\footnote{On appelle <<~fuite de carbone~>> le déplacement d'émissions d'un pays où la législation devient contraignante vers un pays où elle l'est moins.}, est une tarification carbone aux frontières~: les pays non participants seraient soumis à un droit de douane sur les biens qu'ils exportent vers les pays participants en proportion du contenu carbone de ces biens (ou en fonction d'une valeur de référence correspondant au pire cas possible si ces émissions ne peuvent pas être mesurées). Le prix du carbone appliqué à ces exportations serait au moins égal au prix du marché. L'organe de gouvernance pourrait décider d'appliquer un prix plus élevé, pour deux raisons. Premièrement, si des pays non participants (dont l'empreinte carbone est supérieure à la moyenne) rejoignaient le Plan, le budget carbone du Plan augmenterait moins que l'ensemble des émissions réglementées, de sorte que le prix du carbone sur le marché augmenterait. Le prix du carbone mondial devrait être égal à ce niveau plus élevé pour respecter le budget carbone. Par conséquent, la tarification carbone aux frontières pourrait être fixée à (la valeur estimée) de ce \textit{prix contrefactuel}, afin d'internaliser le prix que les entités participantes devraient payer pour ces biens importés si le Plan était réellement mondial et si le budget carbone était respecté. Deuxièmement, l'organe de gouvernance pourrait décider d'appliquer des sanctions sous la forme d'un tarif plus élevé que le prix contrefactuel. 

Dans certains pays fédéraux comme les États-Unis, certains États peuvent être disposés à adhérer au Plan alors que le niveau fédéral ne le souhaite pas. Le Plan inclurait des dispositions pour aider ces États à y adhérer. En particulier, les entités infranationales participantes seraient autorisées à ne pas prélever la tarification carbone aux frontières, et elles pourraient choisir librement l'utilisation des recettes qui leur sont allouées, sans être tenues au revenu de base. 
Ainsi, un État comme la Californie pourrait utiliser les recettes du Plan pour subventionner des entreprises manufacturières, perpétuant ainsi l'usage des recettes de son propre marché carbone, usage qui empêche les fuites de carbone tout en respectant l'union douanière nationale.

Bien que les pays à hauts revenus aient la capacité et le devoir d'aider les pays à bas revenus à se décarboner et à réduire la pauvreté, cette responsabilité ne semble pas s'appliquer aux \textit{pays intermédiaires} dont le revenu par habitant est inférieur à la moyenne mondiale. Pourtant, certains pays intermédiaires comme la Chine ont une empreinte carbone supérieure à la moyenne. Pour encourager ces pays à participer au Plan, on pourrait leur permettre de ne pas participer à la mutualisation des recettes et au revenu de base, sous certaines conditions. Pour bénéficier d'une dérogation complète et conserver les recettes des ventes aux enchères perçues sur son territoire, un pays devrait avoir un PIB par habitant inférieur à 1,5 fois la moyenne mondiale\footnote{Actuellement, la moyenne mondiale est de 21~000~\textit{\$} par an en parité de pouvoir d'achat, tandis que la Chine est à 21~500~\textit{\$} et la Russie à 34~600~\textit{\$}.}. 
Les pays plus riches que ce seuil pourraient bénéficier d'une dérogation partielle, dans la mesure où leur PIB par habitant reste inférieur au double de la moyenne mondiale. Par exemple, un pays dont le PIB est supérieur de 70~\% à la moyenne devrait mutualiser 40~\% des recettes provenant de ses émissions territoriales\footnote{En effet, 1,7 fois la moyenne mondiale correspond à 40~\% de l'écart entre 1,5 et 2 fois la moyenne mondiale.}, mais pourrait conserver 60~\% de ces recettes, auquel cas il ne recevrait que 40~\% du revenu de base. Les dérogations (ou \textit{opt out}) à la mutualisation des recettes réduiraient le revenu de base de 54 à 44 euros par mois en 2030 (dans les pays qui n'en bénéficient pas). 

Ce mécanisme de participation pourrait être critiqué comme un avantage trop important conféré aux grands exportateurs, c'est-à-dire aux pays qui activent la dérogation et dont les émissions territoriales sont nettement supérieures à leur empreinte carbone. En effet, ces pays conserveraient des recettes correspondant à leurs exportations de contenu carbone (nettes), ce qui rendrait le revenu de base inférieur à l'augmentation moyenne des dépenses individuelles pour les pays qui ne bénéficient pas de la dérogation. Il convient toutefois de noter que la tarification carbone aux frontières adoptée par l'UE confère exactement le même avantage aux pays exportateurs étrangers avec un prix interne du carbone égal au prix sur le marché européen~: les importations en provenance de tels pays seront exemptées de la tarification carbone aux frontières, et ces pays bénéficieront des recettes du prix du carbone payées en fin de compte par les consommateurs européens. Malgré tout, l'avantage de la dérogation peut être limité, par exemple en établissant une limite sur les recettes qui peuvent être retenues, par exemple à 50~\% au-dessus de la moyenne des recettes mondiales par adulte. %

Un autre problème potentiel de la dérogation est que les pays qui en bénéficient soient trop peu incités à réduire leurs émissions, ce qui ferait démesurément reposer l'effort de décarbonation sur le reste du monde. Pour vérifier que ces pays jouent le jeu, la dérogation pourrait être conditionnée au respect d'une trajectoire de réduction de l'intensité carbone conforme à la moyenne mondiale\footnote{Une autre possibilité serait de spécifier la dérogation différemment. Le revenu de base reçu par les pays bénéficiant de la dérogation serait accru d'un facteur égal à l'empreinte carbone du pays en 2025 rapporté à l'empreinte carbone moyenne de l'union à cette date. Ainsi, les incitations seraient préservées et le pays serait bénéficiaire du Plan si et seulement si son intensité carbone décroît plus vite que la moyenne de l'union.}. 
En outre, la dérogation pourrait être accordée en échange de certaines conditions, telles que la participation à un impôt mondial sur la fortune dont une partie des recettes serait mise en commun pour financer les pays à bas revenus.

Réciproquement, certains pays à hauts revenus pourraient à l'avenir avoir une empreinte carbone inférieure à la moyenne mondiale. %
Pour éviter que le Plan n'entraîne des transferts des pays à bas ou moyens revenus vers des pays à hauts revenus, une disposition spécifierait que les pays à hauts revenus ne peuvent pas recevoir le revenu de base si leurs émissions par adulte sont inférieures à la moyenne mondiale\footnote{Pour éviter les effets de seuil, le revenu de base perçu par un pays dont le PIB per capita (p.c.) est supérieur à 2 fois la moyenne mondiale ($\overline{y}$) et dont les émissions territoriales p.c. sont inférieures à 1,3 fois la moyenne des pays participants (hors ceux qui activent l'\textit{opt out}) ($\overline{e}$) pourrait être une fonction de ces deux variables, définie de telle sorte qu'un pays neutre en carbone avec un PIB p.c. supérieur à 2,2 fois la moyenne ne percevrait plus le revenu de base. 

En notant $y$ le PIB p.c. d'un pays, $e$ ses émissions p.c., et $B$ le revenu de base non ajusté (c'est-à-dire les recettes totales divisées par la population des pays participants), si $y\geq 2\overline{y}$ et $e \leq 1,3 \overline{e}$, le revenu de base pour ce pays serait ajusté à $\left(\lambda + \left(1-\lambda \right) \frac{e}{1,3\overline{e}} \right) B$, avec $\lambda = \frac{2,2\overline{y}-\min\{y;\;2,2\overline{y}\}}{0,2\overline{y}}$. 
Le revenu de base (dans les autres pays, non concernés par cette disposition) est ensuite ajusté à la hausse en utilisant les recettes libérées.}. 

\paragraph{Sanctions}

Les pays qui n'appliquent pas correctement la tarification carbone ou ne versent pas correctement le revenu de base sur leur territoire ferait face à des sanctions allant jusqu'à l'exclusion de l'union climatique. 
Par ailleurs, si l'organe de gouvernance estime cela approprié pour encourager la participation, il pourrait voter des sanctions à l'encontre des pays non participants, telles que des droits de douane (au-delà de la tarification carbone aux frontières), des confiscations d'actifs ou des restrictions sur les voyages au sein de l'union (ciblant notamment les élites). 

\paragraph{Négociations}

Le Plan a été décrit de façon aussi précise que possible afin d'ancrer les discussions sur une proposition concrète et équitable. Toutefois, certains éléments du Plan peuvent être modifiés sans pour autant l'altérer fondamentalement, comme le seuil d'\textit{opt out} ou l'usage d'une taxe plutôt que d'un système d'échange de permis d'émission. 

Il revient maintenant au public et aux responsables politiques de reprendre cette proposition, de l'amender et de la négocier.



\chapter{Estimation des effets distributifs du Plan
}\label{ch:methodo}

Les estimations des effets distributifs du Plan présentées au Chapitre \ref{ch:effets_distributifs} requièrent des données (notamment issues de travaux de modélisation) et des hypothèses. 
J'ai employé les meilleures données auxquelles j'avais accès et utilisé les hypothèses les plus naturelles. Pour autant, ces estimations restent imparfaites. Je prévois à l'avenir de raffiner ces estimations et d'effectuer un travail de modélisation macroéconomique. Mais il faut reconnaître que des incertitudes subsistent même dans les meilleures projections, notamment concernant le prix du carbone. 
Dans cette annexe, je présente les méthodologies que j'ai employées et leurs limitations.

\section{Effet sur le pouvoir d'achat d'un individu}\label{app:indiv}

Pour évaluer l'effet du Plan sur le pouvoir d'achat d'un individu, mettons de côté les mesures complémentaires au Plan ainsi que les effets du Plan sur le climat (qui limitent les préjudices causés par le changement climatique). Le Plan induit plusieurs effets. Premièrement, le revenu de base augmente le pouvoir d'achat, d'un montant égal au prix du carbone multiplié par l'empreinte carbone moyenne mondiale \textit{ex post}. Deuxièmement, l'<<~effet prix~>> de la tarification carbone réduit le pouvoir d'achat, d'un montant égal au prix du carbone multiplié par l'empreinte individuelle \textit{ex post}. Troisièmement, l'<<~effet volume~>> du prix du carbone induit une réduction de la consommation carbonée (une fois pris en compte la hausse liée au revenu de base), telle qu'un moindre usage de la voiture, qui s'accompagne d'une baisse des taxes payées sur les énergies fossiles (en France, TVA et TICPE). Quatrièmement, l'évolution du mode de vie induit une hausse de la consommation de substituts aux produits carbonés, telle que les vélos, qui réduit le reste à vivre. %

S'il est difficile d'estimer ces quatre effets, on peut établir une borne inférieure. Si l'individu n'ajuste pas sa consommation, les troisième et quatrième effets disparaissent, et le gain net est égal au revenu de base moins son empreinte carbone \textit{ex ante} multipliée par le prix du carbone. Ce montant peut être qualifié de gain net minimal. En effet, si l'individu ajuste sa consommation, on peut faire l'hypothèse que c'est pour améliorer sa situation (par rapport à la situation où il ne l'ajusterait pas), donc la situation non ajustée représente le pire des cas\footnote{Notons que la situation doit s'évaluer tout autant sur le futur que sur le présent. Ainsi, d'importants coûts immédiats  tels qu'une rénovation thermique peuvent constituer un gain grâce aux économies d'énergie fossile à venir%
.}. En sommant les trois premiers effets (mais pas le quatrième), on peut également établir une borne supérieure au gain net, égale au revenu de base moins la différence entre les dépenses de produits carbonés après \textit{versus} avant le Plan. 
Si l'effet prix pousse les dépenses carbonées à la hausse, l'effet volume les pousse à la baisse, si bien que l'effet total est incertain\footnote{D'autant plus qu'on pourrait aussi considérer d'autres effets. D'une part, l'\textit{incidence} du prix, qui n'est peut-être pas entièrement payé par le consommateur, mais potentiellement en partie absorbé par le producteur. D'autre part, les variations des prix des actifs~: la valeur d'un pavillon mal isolé baisserait par rapport à celle d'un appartement en centre-ville. Notons que ce dernier effet, ainsi que les troisième et quatrième effets, ne sont pas spécifiques à la tarification carbone, mais se produiraient dans n'importe quel scénario de décarbonation. Cela peut justifier de se focaliser sur les deux premiers effets.}. 

Devant la difficulté à estimer tous les effets sur le budget d'un individu, les économistes adoptent souvent la mesure du \textit{gain fiscal}, qui se situe entre les deux bornes précédemment décrites. Le gain fiscal est la somme des deux premiers effets~: il correspond au revenu de base touché moins le prix du carbone payé. C'est cette métrique que nous utilisons ci-dessous pour estimer si un individu ou un pays est financièrement gagnant ou perdant suite au Plan. Elle peut s'interpréter comme le gain net par rapport à une situation de référence où l'effort de décarbonation serait identique et où chaque individu ou pays recevrait le prix du carbone qu'il aurait payé. Cette mesure ne tient pas compte de la perte de confort liée aux ajustements de la consommation. Cela dit, incorporer la perte de confort en utilisant une mesure %
telle que le gain net minimal amènerait à exagérer la perte totale (financière et de confort).

\section{Effet sur la distribution mondiale des niveaux de vie}\label{app:revenus}

Pour estimer précisément le gain fiscal d'un individu, il faudrait connaître son empreinte carbone et son revenu. Or, il n'existe pas de jeux de données de la distribution jointe des empreintes carbone et des revenus au niveau mondial. En guise d'approximation, j'utilise une estimation de l'empreinte carbone moyenne par percentile de revenu à l'échelle mondiale, fournie gracieusement par Lucas Chancel\footnote{Plus précisément, il s'agit du revenu pré-taxe en \textit{\texteuro{}} 2019 PPA et des émissions de gaz à effet de serre hors LULUCF%
. Ces données de \textit{revenus} sont plus fiables sur le haut de la distribution et les agrégats que les données de \textit{consommation} de la \textit{Poverty and Inequality Platform}, meilleures pour mesurer la pauvreté.%
}. Je calibre ensuite le Plan avec %
un prix du carbone à 10~\$/tCO$_\text{2}$ et des réductions d'émissions de 9~\% par rapport à 2025%
\footnote{Pour répartir ces réductions d'émissions au niveau individuel, je tiens compte de l'effet rebond~: en particulier, les individus à bas revenus gagnant significativement en pouvoir d'achat émettent davantage suite au Plan.}, qui correspondent aux estimations détaillées ci-après pour 2030. Ces deux valeurs impliquent un revenu de base cohérent avec le modèle ci-dessous, à 42~\euro{} par mois. %

Notons que ces calculs sous-estiment la hausse de niveaux de vie pour les plus pauvres, du fait que les données de revenus sont en parité de pouvoir d'achat.

\section{Proportion de gagnants par pays}

J'estime la proportion de gagnants conformément à la métrique de gain fiscal précédemment décrite~: un individu est considéré financièrement gagnant si son empreinte carbone \textit{ex post} est inférieure à l'empreinte carbone moyenne mondiale \textit{ex post}. J'utilise les données sur l'empreinte carbone moyenne par percentile de revenu dans chaque pays en 2019, construites par la World Inequality Database. En l'absence de données \textit{ex post}, je fais l'hypothèse que dans chaque pays, la part d'individus dont l'empreinte carbone est inférieure à la moyenne mondiale ne varie pas suite au Plan.

\section{Gains nets par pays}\label{app:pays} 

Malheureusement, je n'ai pas encore accès à un modèle qui permet de simuler l'effet d'une tarification uniforme au niveau mondial sur les émissions de chaque pays. J'utilise donc deux jeux de données~: l'un pour le prix du carbone et l'autre pour les trajectoires nationales. Pour les trajectoires nationales d'émissions, de population et de PIB par habitant, j'utilise le scénario SSP2-2.6 du modèle couplé énergie-climat MESSAGE-GLOBIOM\footnote{En effet, c'est le modèle de référence pour ce scénario, cf. \cite{fricko_marker_2017}.}, fourni par \cite{gutschow_country-resolved_2021}. Ce scénario correspond au sentier socio-économique central utilisé par le GIEC (SSP 2), couplé avec un réchauffement limité à +1,8\textdegree{}C (la trajectoire RCP 2.6, avec des émissions de 934~GtCO$_\text{2}$ entre 2025 et net-zéro en 2079). Pour calculer la population adulte, je multiplie la trajectoire de population par la proportion de personnes âgées de 15 ans ou plus d'après les projections de l'ONU. Enfin, je complète les données manquantes manuellement (notamment pour Taïwan et la Corée du Nord). Les données précédentes ne contenant pas de prix du carbone, j'utilise le scénario de prix du modèle IMAGE (fourni par l'IIASA)\footnote{J'utilise le scénario du modèle IMAGE plutôt que de MESSAGE car le prix est environ quatre fois plus faible dans ce dernier, or je préfère une approche conservatrice qui surestime potentiellement les coûts. 
} correspondant au scénario SSP2-2.6. J'utilise le même prix du carbone pour les scénarios avec participation non universelle. 

Le gain net d'un pays correspond à la moyenne des gains fiscaux de sa population~: il s'agit des recettes reçues (a priori en revenu de base) moins le prix du carbone payé. Ce gain net monétaire ne tient pas compte des dégâts du changement climatique subis ou évités ni d'effets macroéconomiques (sur la croissance, les prix ou les revenus du pétrole, etc.). Il correspond au gain net par rapport à une situation sans transferts internationaux mais avec un prix du carbone identique (ainsi que des émissions et des PIB identiques)\footnote{Je calcule le gain net pour chaque année entre 2025 et 2100. 
Je commence par calculer un taux de mutualisation des recettes par pays, qui part du scénario de participation et tient compte du mécanisme d'\textit{opt out}. Ensuite, je calcule la valeur du revenu de base qui découle de la non-participation (ou de la participation partielle) de certains pays. J'itère ces deux étapes jusqu'à ce que l'ensemble de pays exerçant leur droit d'\textit{opt out} converge. 
Le cas échéant, je réduis voire enlève le revenu de base pour certains pays à hauts revenus selon le mécanisme qui les empêche d'être bénéficiaire net à une quelconque période. Puis je calcule la valeur ajustée du revenu de base, et enfin le gain net par pays. Le calcul est reproductible sur \href{https://github.com/bixiou/plan_mondial_climat/blob/main/code_plan/GCP_gain_by_country.R}{github.com/bixiou/plan\_mondial\_climat/blob/main/code\_plan/GCP\_gain\_by\_country.R}.}. 

Les données utilisées souffrent de plusieurs limitations, inhérentes à tous les modèles existants. Premièrement, les données de PIB sont en parité de pouvoir d'achat, ce qui conduit à grandement sous-estimer les gains rapportés au PIB pour les pays à bas revenus. Deuxièmement, les émissions sont données en termes d'émissions territoriales plutôt que d'empreintes carbone, ce qui conduit à légèrement sous-estimer les gains pour les pays exportateurs (comme la Chine) et à sous-estimer les pertes pour les régions importatrices (comme l'Europe). 
Troisièmement, la modélisation ne tient pas compte de l'effet du Plan sur l'activité économique\footnote{Néanmoins, cette limitation est mitigée par des trajectoires de PIB optimistes dans les pays à bas revenus. Par exemple, le modèle sous-jacent projette une croissance du PIB par habitant de 8~\% par an entre 2020 et 2030 en R.D.C.} (qui devrait stimuler la croissance des pays à bas revenus) ni des interactions géostratégiques entre prix du carbone et prix des hydrocarbures. Des chercheurs travaillent à raffiner les modèles pour dépasser ces limitations, et je participerai moi-même à ce travail de modélisation prochainement. Ainsi, nous aurons des estimations plus précises des effets de politiques climatiques mondiales dans quelques années. %

\section{Texte, chiffrages et code en libre accès}

Ce livre est disponible gratuitement en version PDF à l'adresse~: \href{https://global-redistribution-advocates.org/fr/un-plan-mondial-pour-le-climat-et-contre-lextreme-pauvrete/}{bit.ly/gra\_pmc}. %

Les chiffres, tableaux et graphiques de l'ouvrage sont entièrement reproductibles, à l'aide du fichier suivant~:\\ \href{https://github.com/bixiou/plan_mondial_climat/blob/main/code_plan/book.R}{github.com/bixiou/plan\_mondial\_climat/blob/main/code\_plan/book.R}. Ce fichier appelle d'autres fichiers du même répertoire, qui préparent les données et font tourner le modèle. Les données utilisées sont ouvertes.

Pour toute suggestion, vous pouvez m'écrire à adrien.fabre@cnrs.fr.

\renewcommand{\url}[1]{\href{#1}{Lien}} %
{\small 
\bibliographystyle{plainnaturl_clean} %
\bibliography{global_tax_attitudes}
}

\clearpage
\section*{Remerciements}\label{sec:merci}\addcontentsline{toc}{section}{\nameref{sec:merci}} 
Je tiens d'abord à remercier les bénévoles de \textit{Global Redistribution Advocates}, et en particulier Samuel Haddad, Inès Ragot et Julieta Toffoli. Je suis extrêmement reconnaissant envers Gabriel Zucman, pour sa préface généreuse et son combat efficace pour la justice fiscale internationale. Je remercie chaleureusement les personnes qui ont relu ce livre et m'ont donné de précieux conseils pour l'améliorer~: Thomas Douenne, Xavier Fabre, Anne Guillemot, Robert Philippe, Inès Ragot, Isabelle Tallec. Je suis reconnaissant envers les personnes et organisations qui m'ont permis de présenter les travaux à l'origine de ce livre~: Christian Gollier (CEPR/EAERE), Rohini Pande (J-PAL), Thomas Piketty (PSE), Olivier Truffinet (Pour un réveil écologique), ainsi qu'envers les nombreuses audiences qui ont réagi à ces travaux lors de séminaires ou conférences. Mes travaux sont le fruit de collaborations avec de précieux co-auteurs, notamment Antoine Dechezleprêtre, Thomas Douenne, Linus Mattauch, Bluebery Planterose, Ana Sanchez Chico et Stefanie Stantcheva. 
Ils ont été grandement facilités par des universitaires qui m'ont communiqué des données et ont répondu à mes questions, notamment Stefano Battiston, Lucas Chancel, Robert Dellink, Matthew Gidden, Gabrielle du Marais, Xie Jung. %
Je remercie les innombrables personnes qui ont enrichi mes réflexions lors de discussions, notamment Laura Bannister, Ottmar Edenhofer, Camille Étienne, Marc Fleurbaey, Jayati Ghosh, Chris Gong, Jean-Charles Hourcade, Franck Lecocq, Philippe Quirion, Narasimha Rao, Partha Sen, Rick van der Ploeg, Gernot Wagner, Caroline Whyte, ainsi que les responsables politiques et autres personnes rencontrées dans le cadre de mes activités de plaidoyer, qui se reconnaîtront. Je suis reconnaissant envers les associations qui soutiennent certaines de nos propositions, 
en particulier CASCA, CCL France, CCL Europe, Equal Right, Feasta, l'Institut Rousseau, Oxfam et les Young World Federalists. Je salue Opal Ocean, dont le concert a été source d'inspiration pour la postface. Enfin, je remercie les abonné\textperiodcentered{}e\textperiodcentered{}s de ma chaîne YouTube \textit{la chaîne humaine} (\href{https://www.youtube.com/@chaine_humaine}{youtube.com/@chaine\_humaine}), de mon compte Twitter (\href{https://twitter.com/adrien_fabre}{twitter.com/adrien\_fabre}), ainsi que les personnes qui soutiennent \textit{Global Redistribution Advocates}.
