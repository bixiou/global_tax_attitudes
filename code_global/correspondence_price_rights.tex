% Options for packages loaded elsewhere
\PassOptionsToPackage{unicode}{hyperref}
\PassOptionsToPackage{hyphens}{url}
\PassOptionsToPackage{dvipsnames,svgnames,x11names}{xcolor}
%
\documentclass[
  letterpaper,
  DIV=11,
  numbers=noendperiod]{scrartcl}

\usepackage{amsmath,amssymb}
\usepackage{iftex}
\ifPDFTeX
  \usepackage[T1]{fontenc}
  \usepackage[utf8]{inputenc}
  \usepackage{textcomp} % provide euro and other symbols
\else % if luatex or xetex
  \usepackage{unicode-math}
  \defaultfontfeatures{Scale=MatchLowercase}
  \defaultfontfeatures[\rmfamily]{Ligatures=TeX,Scale=1}
\fi
\usepackage{lmodern}
\ifPDFTeX\else  
    % xetex/luatex font selection
\fi
% Use upquote if available, for straight quotes in verbatim environments
\IfFileExists{upquote.sty}{\usepackage{upquote}}{}
\IfFileExists{microtype.sty}{% use microtype if available
  \usepackage[]{microtype}
  \UseMicrotypeSet[protrusion]{basicmath} % disable protrusion for tt fonts
}{}
\makeatletter
\@ifundefined{KOMAClassName}{% if non-KOMA class
  \IfFileExists{parskip.sty}{%
    \usepackage{parskip}
  }{% else
    \setlength{\parindent}{0pt}
    \setlength{\parskip}{6pt plus 2pt minus 1pt}}
}{% if KOMA class
  \KOMAoptions{parskip=half}}
\makeatother
\usepackage{xcolor}
\setlength{\emergencystretch}{3em} % prevent overfull lines
\setcounter{secnumdepth}{-\maxdimen} % remove section numbering
% Make \paragraph and \subparagraph free-standing
\makeatletter
\ifx\paragraph\undefined\else
  \let\oldparagraph\paragraph
  \renewcommand{\paragraph}{
    \@ifstar
      \xxxParagraphStar
      \xxxParagraphNoStar
  }
  \newcommand{\xxxParagraphStar}[1]{\oldparagraph*{#1}\mbox{}}
  \newcommand{\xxxParagraphNoStar}[1]{\oldparagraph{#1}\mbox{}}
\fi
\ifx\subparagraph\undefined\else
  \let\oldsubparagraph\subparagraph
  \renewcommand{\subparagraph}{
    \@ifstar
      \xxxSubParagraphStar
      \xxxSubParagraphNoStar
  }
  \newcommand{\xxxSubParagraphStar}[1]{\oldsubparagraph*{#1}\mbox{}}
  \newcommand{\xxxSubParagraphNoStar}[1]{\oldsubparagraph{#1}\mbox{}}
\fi
\makeatother


\providecommand{\tightlist}{%
  \setlength{\itemsep}{0pt}\setlength{\parskip}{0pt}}\usepackage{longtable,booktabs,array}
\usepackage{calc} % for calculating minipage widths
% Correct order of tables after \paragraph or \subparagraph
\usepackage{etoolbox}
\makeatletter
\patchcmd\longtable{\par}{\if@noskipsec\mbox{}\fi\par}{}{}
\makeatother
% Allow footnotes in longtable head/foot
\IfFileExists{footnotehyper.sty}{\usepackage{footnotehyper}}{\usepackage{footnote}}
\makesavenoteenv{longtable}
\usepackage{graphicx}
\makeatletter
\newsavebox\pandoc@box
\newcommand*\pandocbounded[1]{% scales image to fit in text height/width
  \sbox\pandoc@box{#1}%
  \Gscale@div\@tempa{\textheight}{\dimexpr\ht\pandoc@box+\dp\pandoc@box\relax}%
  \Gscale@div\@tempb{\linewidth}{\wd\pandoc@box}%
  \ifdim\@tempb\p@<\@tempa\p@\let\@tempa\@tempb\fi% select the smaller of both
  \ifdim\@tempa\p@<\p@\scalebox{\@tempa}{\usebox\pandoc@box}%
  \else\usebox{\pandoc@box}%
  \fi%
}
% Set default figure placement to htbp
\def\fps@figure{htbp}
\makeatother

\KOMAoption{captions}{tableheading}
\makeatletter
\@ifpackageloaded{caption}{}{\usepackage{caption}}
\AtBeginDocument{%
\ifdefined\contentsname
  \renewcommand*\contentsname{Table of contents}
\else
  \newcommand\contentsname{Table of contents}
\fi
\ifdefined\listfigurename
  \renewcommand*\listfigurename{List of Figures}
\else
  \newcommand\listfigurename{List of Figures}
\fi
\ifdefined\listtablename
  \renewcommand*\listtablename{List of Tables}
\else
  \newcommand\listtablename{List of Tables}
\fi
\ifdefined\figurename
  \renewcommand*\figurename{Figure}
\else
  \newcommand\figurename{Figure}
\fi
\ifdefined\tablename
  \renewcommand*\tablename{Table}
\else
  \newcommand\tablename{Table}
\fi
}
\@ifpackageloaded{float}{}{\usepackage{float}}
\floatstyle{ruled}
\@ifundefined{c@chapter}{\newfloat{codelisting}{h}{lop}}{\newfloat{codelisting}{h}{lop}[chapter]}
\floatname{codelisting}{Listing}
\newcommand*\listoflistings{\listof{codelisting}{List of Listings}}
\makeatother
\makeatletter
\makeatother
\makeatletter
\@ifpackageloaded{caption}{}{\usepackage{caption}}
\@ifpackageloaded{subcaption}{}{\usepackage{subcaption}}
\makeatother

\usepackage{bookmark}

\IfFileExists{xurl.sty}{\usepackage{xurl}}{} % add URL line breaks if available
\urlstyle{same} % disable monospaced font for URLs
\hypersetup{
  pdftitle={correspondence\_price\_rights},
  colorlinks=true,
  linkcolor={blue},
  filecolor={Maroon},
  citecolor={Blue},
  urlcolor={Blue},
  pdfcreator={LaTeX via pandoc}}


\title{correspondence\_price\_rights}
\author{}
\date{2025-04-27}

\begin{document}
\maketitle


\section{Correspondence between transfers in a cap-and-trade and
differentiated carbon
prices}\label{correspondence-between-transfers-in-a-cap-and-trade-and-differentiated-carbon-prices}

\subsection{The static case}\label{the-static-case}

\subsubsection{Setting}\label{setting}

Decarbonization is costly. When a country \(i\) faces a carbon price
\(p_i\), it curtails consumption by investing in low-carbon equipment
and reorienting consumption to less preferred, less carbon-intensive
products. These costs to consumption are called abatements costs and
denoted \(a_i\). Taking into account the possibility of transfers
\(t_i\) from the rest of the world, expressing everything in per capita
terms, abstracting from exogenous investments, country i's consumption
is related to its potential output \(y_i\) as follows:
\[c_i = y_i + t_i - a_i(p_i)\]

Assume that the welfare of country \(i\) depends on its consumption
\(c_i\) and global emissions \(E=\sum_j e_j n_j\), where \(e_j\) is
country's \(j\) per capita emissions and \(n_j\) its population size.
For now, take global emissions as fixed at their optimal level \(E^*\).
Let \(p^*\) be the uniform carbon price required to attain \(E^*\),
i.e.~\(E^*=\sum_j e_j(p^*) n_j\).

\textbf{A situation is fully specified by the set of transfers
\((t_j)_j\) and carbon prices \((p_j)_j\) for each country}. For country
\(i\), given that global emissions are taken as fixed, the situation can
be summarized as \(s=(t_i, p_i)\), and (with a slight abuse of notation)
we may express \(i\)'s welfare accordingly:
\(u_i(c_i;E^*)=u_i(t_i,p_i)\).

\subsubsection{Equivalent variation from the benchmark in the absence of
transfers}\label{equivalent-variation-from-the-benchmark-in-the-absence-of-transfers}

To analyze an arbitrary situation \(s\), it will be useful to relate it
to two special cases. First, the coordinated situation \(c*=(0,p^*)\),
without transfer and with a uniform carbon price, which we take as a
benchmark. Second, the autarchy situation \(a=(0, p_i)\), without
transfer and with differentiated prices, which will be a useful
intermediary to analyze an arbitrary situation. We denote \(V_i(p_i)\)
the equivalent variation between situation \(a\) and \(c*\), i.e.~the
transfer required at \(c*\) to get \(a\)'s welfare: \[\begin{aligned}
u_i(V_i(p_i), p^*)&=u^a_i=u_i(0, p_i) \\
c^{c*}_i + V_i(p_i) &= c^a_i
\end{aligned}\] In terms of welfare: \[\begin{aligned}
u_i(V_i(p_i), p^*)&=u^a_i=u_i(0, p_i)=u_i(y_i-a_i(p_i))\sim u^{c*}_i - (p_i - p^*) \partial a_i \partial u_{i1} \\
u_i(V_i(p_i), p^*)&= u^{c*}_i + V_i(p_i)\partial u_{i1}
\end{aligned}\]

where \textbf{\(\partial a_i\) is the marginal abatement cost at
\(p^*\)}. Therefore, at the first order,
\[V_i(p_i) = (p^* - p_i) \partial a_i\]

\(V_i\) is positive iff \(p^* > p_i\) as country i ``needs'\,' a
positive compensation for consumption to get the same welfare as in (a)
with a higher price \(p^*\) (and lower emissions) compared to \(c*\).

\subsubsection{Equivalent transfer for an arbitrary
situation}\label{equivalent-transfer-for-an-arbitrary-situation}

Let \(g^s_i\) be country \(i\)'s net gain (or ``equivalent transfer'\,')
in an arbitrary situation \(s=(t_i, p_i)\) compared to the benchmark
situation \(c*=(0,p^*)\): \[\begin{aligned}
g^s_i &= (c^s_i - c^a_i) + (c^a_i - c^{c*}_i) \\
 &= t_i + V_i(p_i)
\end{aligned}\] Note that \textbf{we can call \(g^s_i\) the equivalent
transfer as it is the transfer at zero price that makes \(i\)
indifferent with \(s\), i.e.~the transfer such that \((g^s_i, 0)\) is
welfare-equivalent to \(s=(t_i, p_i)\)}. The equivalent transfer is a
very useful metric to compare the welfare of a country in different
situations.

\subsubsection{Correspondence with an equivalent
price}\label{correspondence-with-an-equivalent-price}

Now, we have all the tools to study the correspondence between
transfers, differentiated emission rights, and differentiated prices.
Let \(p^*_i\) be the ``equivalent price'\,' such that \(s=(t_i, p_i)\)
is welfare-equivalent to \(s*=(0, p^*_i)\). \[\begin{aligned}
g^s_i &= g^{s*}_i \\
t_i + V_i(p_i) &= V_i(p^*_i) \\
t_i + (p^* - p_i) \partial a_i &= (p^*- p^*_i) \partial a_i \\
\end{aligned}\]

Therefore, the equivalent price is:
\[p^*_i = p_i - \frac{t_i}{\partial a_i} \]

The equivalent price is negative when \(p_i \partial a_i < t_i\). As a
negative price would require a transfer from the rest of the world, a
regime with differentiated (positive) prices is restrictive compared to
a system that allows for international transfers. Therefore, \textbf{we
prefer to use the concept of equivalent transfer \(g^s_i\), rather than
its dual the equivalent price}. From the above, the equivalent transfer
can be expressed as:

\[g^s_i = t_i + (p^* - p_i) \partial a_i\]

\subsubsection{Correspondence between differentiated prices and
rights}\label{correspondence-between-differentiated-prices-and-rights}

Imagine that countries establish a uniform carbon price \(p^*\),
e.g.~through a cap-and-trade. Say each person in country \(i\) has an
emission right \(r_i\), so they are entitled a claim \(r_i p^*\) on
global carbon revenue. Therefore, the net monetary transfer to \(i\) is
\[t^*_i = r_i p^* - e^*_i p^*\]

In the static case, \textbf{there is a correspondence between emission
rights \(r_i\) in a cap-and-trade and differentiated prices \(p_i\) in
autarchy (without transfers)}: \[\begin{aligned}
g^a_i &= V_i(p_i) = (p^* - p_i) \partial a_i \\
g^*_i &= t^*_i = (r_i - e^*_i) p^* \\
g^*_i = g^a_i &\Leftrightarrow (r_i - e^*_i) p^* = (p^* - p_i) \partial a_i
\end{aligned}\]

The left-hand side is the cap-and-trade equivalent transfer, which
corresponds to the net monetary transfer received in the uniform
cap-and-trade. The equivalence relation states that it is negatively
related to the difference between the autarchy price and the global
cap-and-trade price. The coefficient of this linear relation is the
marginal abatement cost: the costlier abatement is, the larger the
effect of the autarchy carbon price on the cap-and-trade equivalent
transfer.

One can easily isolate cap-and-trade emission rights or autarchy price
in function of the other variables: \[\begin{aligned}
g^*_i = g^a_i &\Leftrightarrow r_i = e_i^* + (1 - \frac{p_i}{p^*}) \partial a_i  \\
&\Leftrightarrow p_i = p^* (1 + \frac{e^*_i - r_i}{\partial a_i})
\end{aligned}\]

\subsubsection{Using the status quo as
benchmark}\label{using-the-status-quo-as-benchmark}

If, instead of using \(c*=(0, p^*)\) as the counterfactual, governments
use the status quo \(c^- = (0, p^-)\), with \(p^-<p^*\). Then, the net
gain they consider is \[\begin{aligned}
\gamma_i &= (y_i + t_i - a_i(p_i)) - (y_i - a_i(p^-)) + (E^* - E^-) \partial u_{i2}/\partial u_{i1} \\
 &= t_i -a_i(p_i) + a_i(p^-) + (E^* - E^-) \partial u_{i2}/\partial u_{i1} \\
 &\sim t_i + (p^- - p_i) \partial a_i + (p^* - p^-) \partial E \partial u_{i2}/\partial u_{i1} \\
 &= g_i - V_i(p^-) + (E^* - E^-) \partial u_{i2}/\partial u_{i1} \\
\end{aligned}\] where \(\partial u_{i2}\) corresponds to the welfare
benefits of lower world emissions.

The gain relative to the status quo \(\gamma_i\) is larger than the gain
relative to \(c*\) iff the price \(p^*\) would improve welfare compared
to the status quo price \(p^-\), i.e.~iff the welfare benefit from lower
emissions exceeds the cost of decarbonization:
\[\gamma_i > g_i \Leftrightarrow (E^* - E^-) \partial u_{i2}/\partial u_{i1} > V_i(p^-) \]

\subsubsection{Abatement cost required as transfer by those neglecting
climate
costs}\label{abatement-cost-required-as-transfer-by-those-neglecting-climate-costs}

From the point of view of short-sighted governments, we are likely in
the other case \(\gamma_i < g_i\), and these governments likely use
\(\gamma_i\) as a counterfactual to assess welfare. In the worst case
where a government doesn't attach any value to a stable climate,
accepting a price increase \(\Delta p_i = p_i - p^-\) requires a
transfer larger than the abatement costs:
\[t_i > V_i(p^-) = \Delta p_i \partial a_i\]

Conceptually, abatement costs can be divided in two parts: first, the
extra investments needed to build and deploy low-carbon infrastructure;
second, the curtailment of consumption (of energy, meat, flights\ldots)
entailed by a higher carbon price. The former can be estimated with a
bottom-up energy model; the latter using the short (or medium) term
price-elasticity of emissions
\(\epsilon = \frac{\partial e}{e}/\frac{\partial p}{p}\) and the crude
approximation that emissions represent one-twentieth of the consumption
(corresponding to about half the energy share). Assuming
\(\epsilon = .4\), the effect on consumption of an increase in price by
a factor \(\delta p\) is therefore:
\(\delta c = \frac{\partial c/c}{\partial e/e} \epsilon \delta p = .05 \cdot .4 \cdot \delta p = .02 \cdot \delta p\).
So, as a first approximation, a doubling of the carbon price should be
compensated with a transfer of at least 2\% of GDP.

\subsection{The dynamic case}\label{the-dynamic-case}

\subsubsection{Setting}\label{setting-1}

In this section, we will show that the correspondence between transfers
and differentiated carbon prices imperfectly extends to the dynamic
case.

Let us keep the previous notations and add an index \(t\) to denote the
value of a variable at time \(t\). Let \(\beta_t\) be the discount
factor between the initial period and \(t\) (containing the pure rate of
time preference and the welfare discounting from growth). Intertemporal
values are denoted with uppercase letters. Monetary variables are
discounted while physical ones are not, e.g.~ \(i\)'s intertemporal net
gain is \(G_i = \sum_t g_{it} \beta_t\) and its intertemporal emission
right is \(R_i = \sum_t r_{it}\).

A situation \(S\) is now given by the transfers and prices trajectories:
\(S=((t_{it})_t, (p_{it})_t)\). Taking \(y_{it}\) as exogenous as a
first approximation, for country \(i\) a situation \(S\) is
welfare-equivalent to a situation \(S'\) with the same temperature
trajectory iff:
\(\sum_t t_{it} - a_{it}(p_{it}) = \sum_t t'_{it} - a_{it}(p'_{it})\).

\subsubsection{The uniform price without transfer case as
benchmark}\label{the-uniform-price-without-transfer-case-as-benchmark}

Consider the situation \(C*=(0,(p^*_t)_t)\) of a uniform carbon price
trajectory \((p^*_t)_t\) and within-country revenue recycling
(i.e.~without transfers). The carbon price trajectory may be fixed in
advance or emerge from the temporal allocation of the global
intertemporal carbon budget (or emission right) \(R\) in a cap-and-trade
system: \((r_{t})_t\). Let us define the net gain for \(i\) at \(t\) in
a situation \(S\) (with the same emissions trajectory as \(C*\)) in
relation to \(C*\). As above: \(g_{it} = t_{it} + V_{it}(p_{it})\),
where \(V_{it}(p_{it}) = (p^*_t - p_{it}) \partial a_{it}\).

\subsubsection{A more limited correspondence between differentiated
prices and
rights}\label{a-more-limited-correspondence-between-differentiated-prices-and-rights}

As in the static case, let us look for a correspondence between an
autarchy situation \(A\) with differentiated prices and a cap-and-trade
system \(U\) with uniform price. In the general case,
\(A=(0, (p_{it})_t)\) while
\(U=((r_{it}p^*_t-e^*_{it}p^*_t)_t, (p^*_t)_t)\) is indirectly
determined by \(r=(r_{it})_t\). The welfare equivalence for \(i\)
between \(A\) and \(U\) can be derived as follows: \[\begin{aligned}
G^A_i &= \sum_t (p^*_t - p_{it}) \partial a_{it} \beta_t \\
G^U_i &= \sum_t (r_{it} - e^*_{it}) p^*_t \beta_t \\
G^U_i = G^A_i &\Leftrightarrow \sum_t r_{it} \beta_t p^*_t  = \sum_t (\partial a_{it} + e^*_{it}) p^*_t \beta_t - \sum_t p_{it} \partial a_{it} \beta_t  
\end{aligned}\]

\textbf{Contrary to the static case, there is no direct equivalence
between an intertemporal aggregate of the emission rights and an
intertemporal aggregate of differentiated prices}, since the welfare
effect of emission rights depends on the cap-and-trade price trajectory
\(p^*_t\).

The same reason also complicates the comparison between two possible
allocations \(r\) and \(r'\) of the intertemporal carbon budget in a
cap-and-trade, even when the price trajectories coincide. Indeed, the
condition
\(G^U_i > G^{U'}_i \Leftrightarrow \sum_t r_{it} \beta_t p^*_t > \sum_t r'_{it} \beta_t p^*_t\)
does \emph{not} imply that
\(R_i = \sum_t r_{it} > \sum_t r'_{it} = R'_i\). In other words,
\textbf{there is no immediate relation between \(i\)'s intertemporal
carbon budget and its intertemporal welfare, as the price trajectory
matters}.

Let us study whether we can find a correspondence in simpler cases. For
the autarchy situation \(A\), consider a homothetic carbon price
vis-à-vis the world average: \(p_{it}=\pi_ip_t\). For the cap-and-trade
situation \(U\), consider that \(i\) is granted a fixed share of the
world's carbon budget: \(r_{it}=r_ir_t\). Now, the equivalence between
\(A\) and \(U\) rewrites:

\[\begin{aligned}
G^U_i = G^A_i &\Leftrightarrow r_i \sum_t r_{t} \beta_t p^*_t  = \sum_t (\partial a_{it} + e^*_{it}) p^*_t \beta_t - \pi_i \sum_t p_{t} \partial a_{it} \beta_t  
\end{aligned}\]

Provided that the price trajectories in autarchy are homothetic to the
price trajectory in the cap-and-trade (without loss of generality):
\(p_{it}=\pi_ip^*_t\); the equivalence simplifies further:

\[G^U_i = G^A_i \Leftrightarrow r_i \mathscr{B} = \mathscr{E} + (1 - \pi_i) \mathscr{\partial A}\]
where \(\mathscr{B} = \sum_t r_{t} \beta_t p^*_t\),
\(\mathscr{E} = \sum_t e^*_{it} p^*_t \beta_t\), and
\(\mathscr{\partial A} = \sum_t \partial a_{it} p^*_t \beta_t\).

\textbf{Although we do find a correspondence in this special case}, the
formula faces practical limitations, as \textbf{it requires knowing the
future price trajectory \((p^*_t)_t\), which can only be imprecisely
estimated} in the cap-and-trade situation. In any case, although one can
find a trajectory of differentiated prices welfare equivalent to a given
allocation of emissions rights, and vice versa, the derivation cannot be
done analytically, \textbf{one has to resort to an integrated assessment
model}.

However, \textbf{in the Hotelling case} (with a fixed carbon budget,
perfect foresight, and efficient capital markets), the formula simplify
further. Indeed, the Hotelling price grows at the discount rate, so that
the discounted price is constant: \(\beta_t p^*_t = p_0\). Only in this
very special case, \textbf{the formula simplifies to a formula analogous
to the static case one}:

\[G^U_i = G^A_i \Leftrightarrow R_i - E^*_i = (1 - \pi_i) \sum_t{\partial a_{it}}\]




\end{document}
