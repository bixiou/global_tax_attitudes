\documentclass[aspectratio=169,xcolor=dvipsnames, 11pt]{beamer} 

%\documentclass[xcolor=dvipsnames,mathserif]{beamer} % this option has curvier math
%\documentclass[xcolor=dvipsnames,11pt]{beamer}
% Note: the color structure needs to be added here in the title. Now it recognizes all beamer % colors.

%%%%%%%%  PRESENTATION LAYOUT:
\usepackage{appendixnumberbeamer} % this package does not count the appendix pages. /!\ Inconstant behavior, sometimes bug.
\mode<presentation>
{
  \usetheme{Boadilla}
  \usecolortheme{lily} % lily is nice or orchid but not for definition
  \setbeamercovered{invisible}
  \setbeamertemplate{footline}{\raggedleft\insertframenumber~/~\inserttotalframenumber\hspace*{3pt}\vskip3pt} %this command shows frame number, not page number at bottom (means if using overlays, % frame number does not change)
%\setbeamertemplate{footline}[page number]  % this puts only page number on bottom
 \setbeamertemplate{navigation symbols}{}  % this erases navigation symbols.
 \setbeamersize{text margin left=0.2cm, text margin right=0.1cm}
 \setbeamertemplate{frametitle}[default][center]
 \setbeamercolor{frametitle}{fg=black}
 \setbeamerfont{frametitle}{size=\large, series=\bfseries} % Modifies frame title font.
 \setbeamercolor{button}{fg=blue, bg=white}
 \setbeamertemplate{itemize item}[circle]
 \setbeamercolor{itemize item}{fg=black}
 \setbeamercolor{itemize/enumerate body}{fg=black}
 \setbeamerfont{framesubtitle}{series=\mdseries}

}

\renewcommand{\familydefault}{\rmdefault} %Options here are \ttdefault \ssdefault or \rmdefault.

% This defines actual color palette; 
\definecolor{blue}{RGB}{0,114,178}
\definecolor{red}{RGB}{213,94,0}
\definecolor{yellow}{RGB}{240,228,66}
\definecolor{green}{RGB}{0,158,115}
\definecolor{orange}{RGB}{230,159,0}

\hypersetup{
  colorlinks = false,
  linkbordercolor = {white},
  linkcolor = {blue}
}

%%% Color customization: where these colors are used. 
\colorlet{cwords}{blue} %this defines a color, stored under the name cwords that will be used and recognized in document.
\colorlet{cwordsc}{red} %color for contrast with some other word.
\colorlet{cwords2}{green} %2nd color for contrast with some other word.
\colorlet{cmath}{blue} %color for math in text.
%\everymath{\color{blue}} % This in conjunction with the everysel package sets color of math
\everydisplay{\color{blue}}


%%%%%%%%  PACKAGES USED:
\usepackage{amsmath}
\usepackage{setspace} % Only needed for spacing
\usepackage{changepage} % Only needed for local margin setting
\usepackage{mathpazo}% font, is overwritten by times
%\usepackage[hypertexnames=false]{hyperref} %This makes hyperref ``dumber'', and, hence, more robust! (otherwise sometimes the appendix links don't work).
\usepackage{hyperref}
\usepackage{multimedia}
\usepackage[english]{babel}
\usepackage{graphicx}
\usepackage{caption}
%\usepackage{subfig}
\usepackage{subfloat}
\usepackage[en-US]{datetime2}
\usepackage{tabulary}
\usepackage{tabularx}
\usepackage{array,booktabs} % Needed for esttab tables according to the "inequality survey."
\newcommand{\sym}[1]{{#1}} % for symbols in Table
\usepackage[T1]{fontenc}
\usepackage[utf8]{inputenc}
\usepackage{times} % This is a different font.
\usepackage[overlay,absolute]{textpos}
%%%\usepackage{animate} % Animate graphs BUG
\setlength{\TPHorizModule}{1cm}
\setlength{\TPVertModule}{1cm}
\captionsetup[figure]{labelformat=empty} % removes caption prefix figure
\setlength{\itemsep}{\fill} % this is supposed to stretch items across full frame.
%\setlength{\parskip}{0.8\baselineskip} % This affects spacing between normal lines (not itemized).
\usepackage{colortbl} % For cell colors
\usepackage[final]{pdfpages}
\usepackage{caption}
\usepackage{subcaption}
%\captionsetup{justification   = raggedright,
%              singlelinecheck = false}
%\usepackage{adjustbox} % to use resizebox for tables size.
\usepackage[export]{adjustbox} % to use resizebox for tables size.
\usepackage{eurosym}
\usepackage{gensymb}

%% TIKZ
\usepackage{tikz}
\usetikzlibrary{er, positioning,decorations.pathmorphing,calc}
\usepackage{tikzscale}

\tikzset{every entity/.style={draw=black, fill=white}}
\tikzset{comment/.style={draw=white, fill=white}}
\tikzset{
	invisible/.style={opacity=0},
	visible on/.style={alt=#1{}{invisible}},
	alt/.code args={<#1>#2#3}{%
		\alt<#1>{\pgfkeysalso{#2}}{\pgfkeysalso{#3}} % \pgfkeysalso doesn't change the path
	},
}

% Commands from template
\newcommand{\alrt}[1]{{\color{alert} #1}}
\newcommand{\alrtl}[1]{{\color{alert}\large #1}}
\newcommand{\alrtL}[1]{{\color{alert}\Large #1}}
\newcommand{\struc}[1]{{\color{structure} #1}}
\newcommand{\strucL}[1]{{\color{structure}\Large #1}}
\newcommand{\strucl}[1]{{\color{structure}\large #1}}
\newcommand{\dred}[1]{{\color{darkred} #1}}
\newcommand{\dredl}[1]{{\color{darkred}\large #1}}
\newcommand{\dredL}[1]{{\color{darkred}\Large #1}}
\newcommand{\altc}[1]{{\color{darkgreen}\textbf{#1}}}
\newcommand{\altcl}[1]{{\color{darkgreen}\textbf{\large #1}}}
\newcommand{\altcL}[1]{{\color{darkgreen}\textbf{\Large #1}}}
\newcommand{\hush}{\hushit}
\newcommand{\hushalrt}[1]{\hushit{{\color{alert} #1}}}
\newcommand{\hushalrtl}[1]{\hushit{{\large\color{alert} #1}}}
\newcommand{\hushalrtL}[1]{\hushit{{\Large\color{alert} #1}}}
\newcommand{\hushstruc}[1]{\hushit{{\color{structure} #1}}}
\newcommand{\hushstrucl}[1]{\hushit{{\large\color{structure} #1}}}
\newcommand{\hushstrucL}[1]{\hushit{{\Large\color{structure} #1}}}


\setbeamertemplate{caption}[numbered]

%%%%%%%%%SECTION TITLES DISPLAYED ON FULL PAGE %%%%%%%%%%
%%%%%%%%%%%%%%%%%%%%%%%%%%%%%%%%%%%%%%%%%%%
\AtBeginSection[]{
  \begin{frame}
  \vfill
  \centering
  \begin{beamercolorbox}[sep=8pt,center,shadow=true,rounded=true]{title}
    \usebeamerfont{title}{\huge \color{red} \insertsectionhead} \par
  \end{beamercolorbox}
  \vfill
  \end{frame}
}

\AtBeginSubsection[]{
  \begin{frame}
  \vfill
  \centering
  \begin{beamercolorbox}[sep=8pt,center,shadow=true,rounded=true]{title}
    \usebeamerfont{title}{\huge \color{blue} \insertsubsectionhead} \par
  \end{beamercolorbox}
  \vfill
  \end{frame}
}

%
% Custom font for a frame.
%
\usepackage{environ}
\newcommand{\customframefont}[1]{
\setbeamertemplate{itemize/enumerate body begin}{#1}
\setbeamertemplate{itemize/enumerate subbody begin}{#1}
}

\NewEnviron{framefont}[1]{
\customframefont{#1} % for itemize/enumerate
{#1 % For the text outside itemize/enumerate
\BODY
}
\customframefont{\normalsize}
}

%%%%%%%%%%%%%%%%%%%%%%%%%%%%%%%%%%%%%%%%%%%%%%%%%%%%%%%
%%%%%% OTHER PIECES OF TEMPLATE FILE
%%%%%% DELETE ONCE CLEAR THAT NOTHING IS MISSING
%%%%%%%%%%%%%%%%%%%%%%%%%%%%%%%%%%%%%%%%%%%%%%%%%%%%%%%


%\documentclass[aspectratio=169]{beamer} % wide
%\usepackage{amsmath,amsthm,fancyhdr,setspace,graphicx,booktabs,pdflscape}
%\usepackage{geometry}
%\usepackage{etex}
%\usepackage{xcolor,colortbl}
%\usepackage{beamerprosper}
%\usepackage{url}
%\usepackage{enumerate}
%\usepackage{graphicx}
%\usepackage{hyperref}
%\usepackage{multicol}
%\usepackage{caption}
%\usepackage{beamerprosper}
%\usepackage{pgfpages, pdfpages}
%\usepackage{tikz-cd}
%
%\usepackage{tabularx}
%\usepackage{anyfontsize}
%\usepackage{multicol,tabto}
%
%\usepackage{float}
%\usepackage{soul}
%\usepackage{grffile}
%\usepackage{changepage}
%\usepackage{sansmathaccent}
%\pdfmapfile{+sansmathaccent.map}
%
%
%\usetikzlibrary{er,positioning,calc,decorations.pathreplacing}
%
%\mode<presentation>
%\usefonttheme{structuresmallcapsserif}
%\setbeamertemplate{footline}[frame number]{}
%\setbeamertemplate{navigation symbols}{}
%
%%change font
%\usefonttheme{default}
%\setbeamertemplate{footline}[frame number]{}
%\setbeamertemplate{navigation symbols}{}
%
%\newcommand{\fig}[3]{\begin{frame}\frametitle{#2}\centerline{\includegraphics[width=#3in]{#1}}\end{frame}}
%
%\newcommand{\blackslide}[1]{\beamersetaveragebackground{black}\begin{frame}\frametitle{}\end{frame}\beamersetaveragebackground{white}}
%
%% pause commands
%\newcommand{\m}[2]{\begin{frame}\frametitle{#1}{ #2}\end{frame}}
%\newcommand{\mm}[3]{\begin{frame}\frametitle{#1}\uncover<1->{ #2}\uncover<2->{ #3 }\end{frame}}
%\newcommand{\mmm}[4]{\begin{frame}\frametitle{#1}\uncover<1->{ #2}\uncover<2->{ #3 }\uncover<3->{ #4 }\end{frame}}
%\newcommand{\mmmm}[5]{\begin{frame}\frametitle{#1}\uncover<1->{ #2}\uncover<2->{ #3 }\uncover<3->{ #4 }\uncover<4->{ #5 }\end{frame}}
%
%\setlength{\footskip}{24pt}
%
%\newcommand{\ex}{\mathbf{E}}
%\newcommand{\cov}{\mathbb{C}}
%\newcommand{\var}{\mathbb{V}}
%\newcommand{\tu}{\overline{\theta}}
%\newcommand{\vu}{\overline{v}}
%\newcommand{\tl}{\underline{\theta}}
%\newcommand{\ab}{\bar{a}}
%\newcommand{\lb}{\bar{L}}
%\newcommand{\hb}{\bar{H}}
%
%% New colors.
%\definecolor{darkred}{rgb}{0.6,0,0}
%\definecolor{darkblue}{rgb}{.15,.25,.55}
%\definecolor{darkgreen}{rgb}{0,.35,.05}
%\definecolor{ltgreen}{rgb}{0,.05,.8}
%\definecolor{bred}{rgb}{1,0,.05}
%\definecolor{navy}{rgb}{.1,.1,.5}
%
%% from beamer lecture 
%
%\newtheorem{proposition}{Proposition}
%%\newtheorem{theorem}{Theorem}
%
%\newenvironment{changemargin}[2]{%
%    \begin{list}{}{ %
%%        \setlength{\topmargin}{#3}%
%            \setlength{\topsep}{0pt}%
%            \setlength{\leftmargin}{#1}%
%            \setlength{\rightmargin}{#2}%
%            \setlength{\listparindent}{\parindent}%
%            \setlength{\itemindent}{\parindent}%
%            \setlength{\parsep}{\parskip}%
%        }%
%\item[]}{\end{list}}
%
%%Allows us to force a column width in array environment
%\newcolumntype{C}[1]{>{\centering\arraybackslash}p{#1}}
%\newcolumntype{L}[1]{>{\raggedright\arraybackslash}p{#1}}
%\newcolumntype{R}[1]{>{\raggedleft\arraybackslash}p{#1}}

%%SHORTCUTS

%%%%%%%%%%%%%%%%%%%%%%%%%
%% Bullets
%%%%%%%%%%%%%%%%%%%%%%%%%

\newcommand{\p}{\item}
\newcommand{\ip}{\item[]} % Invisible items.
\newcommand{\bb}{\begin{itemize}\itemsep15pt}
\newcommand{\bbs}{\medskip \begin{itemize}\itemsep10pt}
\newcommand{\bbvs}{\medskip \begin{itemize}\itemsep3pt}
\newcommand{\ee}{\end{itemize}}

\newcommand{\ben}{\begin{enumerate}}
\newcommand{\een}{\end{enumerate}}


\newcommand{\can}{\citeasnoun}
\newcommand{\ican}{\iciteasnoun}

\newcommand{\non}{\nonumber}

%%%%%%%%%%%%%%%%%%%%%%%%%
%% Derivatives and partials. 
%%%%%%%%%%%%%%%%%%%%%%%%%
%% Duplicate: two ways to get partials
\newcommand{\pa}[2]{\frac{\partial #1}{\partial #2}} % Stef
\newcommand{\pder}[2]{\frac{\partial #1}{\partial #2}} % Doug

\newcommand{\dneu}{\mbox{d}}
\newcommand{\di}[2]{\frac{\dneu #1}{\dneu #2}}
%% Duplicate: two ways to get derivatives.
\newcommand{\dd}[2]{\frac{d #1}{d #2}} % Stef version
\newcommand{\der}[2]{\frac{d#1}{d#2}} % Doug version

%%%%%%%%%%%%%%%%%%%%%%%%%
%% Brackets and fractions
%%%%%%%%%%%%%%%%%%%%%%%%%


\newcommand{\fr}[2]{\frac{#1}{#2}}
\newcommand{\pfr}[2]{\left(\frac{#1}{#2}\right)}
\newcommand{\bfr}[2]{\left[\frac{#1}{#2}\right]}
\newcommand{\cfr}[2]{\left\{\frac{#1}{#2}\right\}}

\newcommand{\pr}[1]{\left(#1\right)}
\newcommand{\br}[1]{\left[#1\right]}
\newcommand{\cb}[1]{\left\{#1\right\}}
\newcommand{\qand}{\quad\text{and}\quad}


%%%%%%%%%%%%%%%%%%%%%%%%%
%% ARROWS
%%%%%%%%%%%%%%%%%%%%%%%%%

\newcommand{\Ra}{\Rightarrow}
\newcommand{\ra}{\rightarrow}
\newcommand{\Ras}{\ \Rightarrow\ }
\newcommand{\ras}{\ \rightarrow\ }
\newcommand{\Raq}{\quad\Rightarrow\quad}
\newcommand{\raq}{\quad\rightarrow\quad}


%%%%%%%%%%%%%%%%%%%%%%%%%
%% MATH
%%%%%%%%%%%%%%%%%%%%%%%%%

\newcommand{\E}[1]{\mathbb{E}\br{#1}}
\newcommand{\eps}{\varepsilon}

\newcommand{\be}{\begin{equation}}
\newcommand{\eeq}{\end{equation}}

\newcommand{\bea}{\begin{eqnarray}}
\newcommand{\eea}{\end{eqnarray}}

\newcommand{\bean}{\begin{eqnarray*}}
\newcommand{\eean}{\end{eqnarray*}}

\newcommand{\ba}{\begin{array}}
\newcommand{\ea}{\end{array}}


\newcommand{\lb}{\linebreak}
\newcommand{\strich}[2]{\left. #1 \right|_{#2}}
% use \left. before


%\newcommand{\bean}{\begin{multline*}}
%\newcommand{\eean}{\end{multline*}}

\newcommand{\nl}{\newline}
\newcommand{\np}{\newpage}

\newcommand{\rf}[1]{(\ref{#1})}



\newcommand{\ds}{\displaystyle}
\newcommand{\fn}{\footnote}

\newcommand{\var}{\mbox{Var}}
\newcommand{\cov}{\mbox{Cov}}

\newcommand{\il}{\int\limits}

\newcommand{\li}{\left}
\newcommand{\re}{\right}

\newcommand{\s}{\right|_}

\newcommand{\ambigo}{\ba{c}>\\[-3mm]<\ea}
\newcommand{\ambigu}{\ba{c}<\\[-3mm]>\ea}

\newcommand{\ul}{\underline}
\newcommand{\ol}{\overline}

\newcommand{\e}{\mbox{\euro{} }}

\usepackage{multicol}
              
\begin{document}

\begin{frame}
\thispagestyle{empty}
\begin{center}
\begin{LARGE}
\textcolor{blue}{International Attitudes Toward Global Policies}
\end{LARGE}

\vspace{1cm}
\textbf{Adrien Fabre} (CNRS, CIRED), Thomas Douenne (University of Amsterdam), Linus Mattauch (TU Berlin, Oxford)

%\vspace{-0.3cm}
%OECD/CAE \\
\medskip
\DTMlangsetup{showdayofmonth=false}
\textit{February 2023} 
%\textit{\today} 

\end{center}

\bigskip

\end{frame}

\section{Questionnaires}

\begin{frame}{International surveys with a focus on the West\label{questionnaires}}
    \bbvs
    \ip OECD survey (02/2021-02/2022): \textcolor{blue}{20 countries} in all world regions \textcolor{blue}{covering 72\% of global CO$_\text{2}$ emissions}, $~$2,000 respondents per country, median duration: 28 min.
    \ip Complementary surveys (01/2023-ongoing): 
    \bbvs \ip Eu: 3,000 respondents from France, Germany, Spain, UK; 20 min. \textcolor{magenta}{85\% of responses collected}.
    \ip US1: 3,000 respondents from the U.S.; 14 min. \textcolor{magenta}{88\% collected}.
    \ip US2: 2,000 respondents from the U.S. \textcolor{magenta}{Not yet collected}.
    \ee 
    \ee
    \makebox[\textwidth][c]{ \includegraphics[width=.6\textwidth]{"../figures/maps/country_coverage"}}
\end{frame}

\begin{frame}{EU questionnaire}
\vspace{.2cm}
%\makebox[\textwidth][c]{ 
%\begin{itemize}[<+>]
\makebox[\textwidth][c]{ \includegraphics[width=.9\textwidth]{../../oecd_climate/figures/questionnaire/survey_flow_EU.pdf}}
%\end{itemize}
%}
\end{frame}

% \section{Sample quality}

% \begin{frame}{Ensuring data quality\label{data_quality}}
% \bbs
% \ip In each country, $\approx$2,000 respondents selected through quotas that  \textcolor{blue}{ensure representativeness} along: \textcolor{magenta}{gender, age, income, region, urban/rural}. \hyperlink{representativeness}{\beamergotobutton{See table}}
% \ip \textcolor{blue}{All results are re-weighted} along quota variables to increase representativeness even further.
% \ip \textcolor{blue}{Screening question} in the middle of the survey. 
% \ip Data collected between Feb 21 and Feb 22. 
% %\ip Appeal to people's social responsibility by insisting they should answer carefully and honestly, for the sake of science. 
% \ip Warn that ``incoherent and \textcolor{blue}{rushed responses'' (< 11 min) are dismissed} and disqualified for monetary compensation.
% \ip Record time spent on separate questions \& overall survey (median: 28 min).
% \ip Ask for feedback post survey, whether felt survey was biased (\textcolor{magenta}{74\% find it unbiased}). 
% \hyperlink{feedback}{\beamergotobutton{Details}}
% \ip I know you're curious: it cost \~200k, incl. 140k for the sampling (\$8/h per respondent).
% %\ip Check careless response patterns (clicking same ``middle'' answer).
% \ee
% \end{frame}

\begin{framefont}{\small}


\begin{frame}{OECD: Global policies seem strongly supported.\label{global_policies}}
	\vspace{-.3cm}
	\begin{figure}[h!]
		\centering		
		\caption{Share of support (somewhat or strongly) for the main global policies among non-\textit{indifferent}. %\hyperlink{detail_global}{\beamergotobutton{Detailed results}}
        }
		\includegraphics[width=.86\textwidth]{../figures/OECD/Heatplot_global_tax_attitudes_share.pdf} % burden_share_all_share_countries
		\end{figure}
\end{frame}


\begin{frame}{\label{}}
\begin{figure}
	\centering 
	\caption{  %\hyperlink{}{\beamergotobutton{See more}}
    }
	\includegraphics[height=.8\textheight]{../../oecd_climate/figures/} 
\end{figure}
\end{frame}


\begin{frame}{\label{}}
    \begin{figure}
        \centering 
        \caption{  %\hyperlink{}{\beamergotobutton{See more}}
        }
        \includegraphics[height=.8\textheight]{../../oecd_climate/figures/} 
    \end{figure}
\end{frame}

\section{Conclusion}

\begin{frame}{Key take-home messages for policy-making}
\begin{itemize}
\ip 1. \textcolor{blue}{Are people ready for international solidarity?} \textit{Yes, a global ETS with equal right to emit per capita is largely supported, as are other global policies.}
	\begin{itemize}
		\item \textcolor{magenta}{Policy implication:}  
\end{itemize}
\pause
\ip 2. \textcolor{blue}{?} \textit{}
	\begin{itemize}
		\item \textcolor{magenta}{Policy implication:} 
	\end{itemize}
	\pause
	\ip 3. \textcolor{blue}{?} \textit{}
	\begin{itemize}
		\item \textcolor{magenta}{Policy implication:} 
	\end{itemize}
	\pause
	\end{itemize}
\end{itemize}
\end{frame}

\appendix
\section{Appendix}

% \section{Representativeness}

% 	\begin{frame}{Summary statistics\label{representativeness}}	
% 		\begin{table}[h!]
% 			\caption{Summary Statistics -- High-income countries 1 \hyperlink{data_quality}{\beamergotobutton{Go back}}}
% 				\begin{center}
% 					\scalebox{0.6}{\input{"../tables/sample_composition/AU_CA_DK_FR_ageCombined.tex"}}
% 				\end{center}
% 			\end{table}	
% 	\end{frame}

% 	\begin{frame}{Summary statistics}	
% 		\begin{table}[h!]
% 			\caption{Summary Statistics -- High-income countries 2 \hyperlink{data_quality}{\beamergotobutton{Go back}}}
% 				\begin{center}
% 					\scalebox{0.6}{\input{"../tables/sample_composition/DE_IT_JP_PL_ageCombined.tex"}}
% 				\end{center}
% 			\end{table}	
% 	\end{frame}
	
% \begin{frame}{Summary statistics}	
% 	\begin{table}[h!]
% 		\caption{Summary Statistics -- High-income countries 3 \hyperlink{data_quality}{\beamergotobutton{Go back}}}
% 			\begin{center}
% 				\scalebox{0.6}{\input{"../tables/sample_composition/SK_SP_UK_US_ageCombined.tex"}}
% 			\end{center}
% 		\end{table}	
% \end{frame}

\section{Descriptive statistics}

% 
% 
% \subsection{Socio-Demographics}
% \begin{frame}{Education}%\addtocounter{framenumber}{-1} % TODOU
%     \hspace{.2cm}
% \begin{figure}[h!]
% \centering
% \caption{What is the highest level of education you have completed?}
% \includegraphics[width=.78\paperwidth]{../../oecd_climate/figures/all/education_ALL} \\
% \vspace{.3cm}
% \includegraphics[width=.78\paperwidth]{../../oecd_climate/figures/all/diploma_ALL}
% %\caption{What race or ethnicity do you identify with? (Multiple answers are possible)} % TODO
% %\includegraphics[width=.43\paperwidth]{../../oecd_climate/figures/all/race_ALL}\\
% \end{figure}
% \end{frame} % TODO: add socio_mean_countries
% 
% \begin{frame}{Left-right leaning}%\addtocounter{framenumber}{-1}
% \begin{figure}[h!]
% \centering
% \caption{On economic policy matters, where do you see yourself on the liberal/conservative spectrum?}
% \includegraphics[width=.87\paperwidth]{../../oecd_climate/figures/all/left_right_ALL} \\
% \end{figure}
% \end{frame}
% 
% \begin{frame}{Geography}%\addtocounter{framenumber}{-1}
% \begin{figure}[h!]
% \centering
% \caption{Lives in an urban area (town > 20k people), retrieved from zipcode}
% \includegraphics[width=.8\paperwidth]{../../oecd_climate/figures/all/urbanity_ALL.png} \\
% \vspace{.2cm}
% %\caption{Region, retrieved from zipcode} % TODO
% %\includegraphics[width=.43\paperwidth]{../../oecd_climate/figures/all/region_ALL}
% \end{figure}
% \end{frame}

% \begin{frame}{Gender and age}%\addtocounter{framenumber}{-1}
% \begin{figure}[h!]
% \centering
% \caption{What is your gender?}
% \includegraphics[width=.6\paperwidth]{../../oecd_climate/figures/all/gender_ALL} \\
% \centering
% \caption{How old are you?}
% \includegraphics[width=.6\paperwidth]{../../oecd_climate/figures/all/age_ALL}
% \end{figure}
% \end{frame}
\subsection{Household %Composition and Energy 
Characteristics}

\begin{frame}{Income/wealth}%\addtocounter{framenumber}{-1}
\begin{figure}[h!]
\centering
\captionsetup{justification=centering}
\caption{What was the annual income of your household in 2019 (before withholding tax, for you and those who live with you)?}
\includegraphics[width=.43\paperwidth]{../../oecd_climate/figures/all/income_ALL} \\
\caption{\small What is the estimated value of your assets, or the assets of your household if you are married (in [currency])? Include here all your possessions (home, car, savings, etc.) net of debt. For example, if you own a house worth \$300,000 and you have \$100,000 left to repay on your mortgage, your assets are \$200,000.}
\includegraphics[width=.43\paperwidth]{../../oecd_climate/figures/all/wealth_ALL} \\
\end{figure}
\end{frame}


\subsection{Political leaning}

\begin{frame}{Little interest for politics}%\addtocounter{framenumber}{-1}
\vspace{-.5cm}
\begin{figure}[h!]
\caption{To what extent are you interested in politics?}
\includegraphics[width=.52\paperwidth]{../../oecd_climate/figures/all/interested_politics_ALL} \\
%\caption{Could you trust the federal goverment to implement the following policies}
\vspace{.1cm}
\caption{Are you member of an environmental organization?}
\includegraphics[width=.47\paperwidth]{../../oecd_climate/figures/all/member_environmental_orga_ALL}\\
\vspace{.1cm}
\caption{Do you have any relatives who are environmentalists?}
\includegraphics[width=.47\paperwidth]{../../oecd_climate/figures/all/relative_environmentalist_ALL}\\
\end{figure}
\end{frame}

\begin{frame}{Broadly representative political leaning}%\addtocounter{framenumber}{-1}
\vspace{-.5cm}
\begin{figure}[h!]
\caption{Did you vote in the [last Country] election?}
\includegraphics[width=.45\paperwidth]{../../oecd_climate/figures/all/vote_participation_ALL} \\
%\caption{Could you trust the federal goverment to implement the following policies}
\vspace{.1cm}
\caption{Which candidate did you vote / would you have voted for in the last presidential election?}
\includegraphics[width=.7\paperwidth]{../../oecd_climate/figures/all/vote_main_ALL} \\
\caption{On economic policy matters, where do you see yourself on the left/right spectrum?}
\includegraphics[width=.7\paperwidth]{../../oecd_climate/figures/all/left_right_ALL}
% \caption{Which candidate did you vote for in the last presidential election?}
% \includegraphics[width=.52\paperwidth]{../../oecd_climate/figures/all/vote_all_ALL} \\
% \caption{Did you vote in the 2016 [country] presidential election?}
% \includegraphics[width=.47\paperwidth]{../../oecd_climate/figures/all/vote_participation_2016_ALL}\\
\end{figure}
\end{frame}


\subsection{International Burden-Sharing\label{detail_global}}

% \begin{frame}{Quasi-unanimous agreement on need for global policies}%\addtocounter{framenumber}{-1}
% \vspace{-1cm}
% \begin{figure}[h!]
% \centering
% \caption{\small{At which level(s) do you think public policies to tackle climate change need to be put in place? (Multiple answers are possible)}}
% \includegraphics[width=.43\paperwidth]{../../oecd_climate/figures/all/scale_ALL}
% \end{figure}
% \end{frame}

% \begin{frame}{Large support for international transfers}%\addtocounter{framenumber}{-1}
% \begin{figure}[h!]
% \centering
% \caption{To achieve a given reduction of greenhouse gas emissions globally, costly investments are needed.
% Ideally, how should countries bear the costs of fighting climate change?}
% \vspace{2mm}
% % TODOO \includegraphics[width=.95\paperwidth]{../../oecd_climate/figures/all/burden_sharing_ALL}
% %\caption{}
% \end{figure}
% \end{frame}

% \begin{frame}{Large support for a fairer global order}%\addtocounter{framenumber}{-1}
% \begin{figure}[h!]
% \centering
% \caption{Do you support or oppose the following policies?}
% \vspace{2mm}
% \includegraphics[width=.87\paperwidth]{../../oecd_climate/figures/all/global_policies_ALL}
% %\caption{}
% \end{figure}
% \end{frame}

\begin{frame}{} % TODOO
	\begin{figure}[h!]
	\centering
	\caption{Do you agree or disagree with the following statement: ``[country] should take measures to fight climate change.'' \hyperlink{global_policies}{\beamergotobutton{Go back}}
	%Pourcentage de réponses (plutôt ou très favorable) parmi~:	Très opposé$\cdot$e; Plutôt opposé$\cdot$e; Indifférent$\cdot$e; Plutôt favorable; Très favorable
	}
% \item 
% 	\\ \textit{Strongly disagree; Somewhat disagree; Neither agree nor disagree; Somewhat agree; Strongly agree}
	\includegraphics[height=.8\paperheight]{../../oecd_climate/figures/country_comparison/should_fight_CC_countries.pdf}
	\end{figure}
\end{frame}

\begin{frame}{}%\addtocounter{framenumber}{-1}
	\begin{figure}[h!]
	\centering
	\caption{
		At which level(s) do you think public policies to tackle climate change need to be put in place? (Multiple answers are possible) \hyperlink{global_policies}{\beamergotobutton{Go back}}
		%\\ En France, les options sont~: Mondiale; Européenne; Nationale; Locale; 
	}
		% \begin{enumerate}[resume] \item 
% \\ \textit{Global; [Federal / European / ...]; [State / National]; Local}
	\includegraphics[width=\paperwidth]{../../oecd_climate/figures/country_comparison/scale_positive_countries.pdf}
	\end{figure}
\end{frame}

\begin{frame}{}
	\begin{figure}[h!]
	\centering
	\caption{How should [country] climate policies depend on what other countries do? \\
	If other countries do more, [country] should do... \hyperlink{global_policies}{\beamergotobutton{Go back}}
	%Pourcentage de réponse \textit{Plus} ou \textit{Beaucoup plus} parmi~: Beaucoup moins; Moins; À peu près autant; Plus; Beaucoup plus. 
	}
	% \item How should [country] climate policies depend on what other countries do?
% 	\begin{itemize}
% \item If other countries do more, [country] should do…
% \item If other countries do less, [country] should do…
% \end{itemize}
% \textit{Much less; Less; About the same; More; Much more}
	\includegraphics[height=.8\paperheight]{../../oecd_climate/figures/country_comparison/if_other_do_more_countries.pdf}
	\end{figure}
\end{frame}

\begin{frame}{}
	\begin{figure}[h!]
	\centering
	\caption{How should [country] climate policies depend on what other countries do? \\
	If other countries do less, [country] should do... \hyperlink{global_policies}{\beamergotobutton{Go back}}
	%Pourcentage de réponse \textit{Plus} ou \textit{Beaucoup plus} parmi~: Beaucoup moins; Moins; À peu près autant; Plus; Beaucoup plus.
	% Si d'autres pays en font plus, la France devrait en faire...
	}
	% \item How should [country] climate policies depend on what other countries do?
% 	\begin{itemize}
% \item If other countries do more, [country] should do…
% \item If other countries do less, [country] should do…
% \end{itemize}
% \textit{Much less; Less; About the same; More; Much more}
	\includegraphics[height=.8\paperheight]{../../oecd_climate/figures/country_comparison/if_other_do_less_countries.pdf}
	\end{figure}
\end{frame}

% \begin{frame}{}
% 	\begin{figure}[h!]
% 	\centering
% 	\caption{[Question posée seulement aux U.S., au Danemark et en France; ici résultats pour la France] Pour parvenir à une réduction donnée des émissions de gaz à effet de serre au niveau mondial, de coûteux investissements sont nécessaires. \\
% 	Dans l'idéal, comment les pays devraient-ils répartir les coûts de la lutte contre le changement climatique ? \\
% 	Pourcentage de réponses (plutôt ou très favorable) parmi~:	Très opposé$\cdot$e; Plutôt opposé$\cdot$e; Indifférent$\cdot$e; Plutôt favorable; Très favorable
% 	% Les pays devraient payer en proportion de leur richesse
% 	% Les pays devraient payer en proportion de leurs émissions actuelles
% 	% Les pays devraient payer en proportion de leurs émissions passées (à partir de 1990)
% 	% Les pays les plus riches devraient payer davantage, afin que les pays les plus pauvres n'aient pas à payer
% 	% Les pays les plus riches devraient payer beaucoup plus, pour aider les pays vulnérables à faire face aux conséquences néfastes : les pays vulnérables recevraient de l'argent au lieu de payer
% 	}
% % \\ \textit{Strongly oppose; Somewhat oppose; Neither support nor oppose; Somewhat support; Strongly support}
% % \item ~[In all countries but the U.S., Denmark and France] Suppose the above policy is in place. How should the carbon budget be divided among countries?
% % \\ \textit{The emission share of a country should be proportional to its population, so that each human has an equal right to emit.; The emission share of a country should be proportional to its current emissions, so that those who already emit more have more rights to emit.; Countries that have emitted more over the past decades (from 1990 onwards) should receive a lower emission share, because they have already used some of their fair share.; Countries that will be hurt more by climate change should receive a higher emission share, to compensate them for the damages.}
% % \item ~[In the U.S., Denmark, and France only] To achieve a given reduction of greenhouse gas emissions globally, costly investments are needed.
% % Ideally, how should countries bear the costs of fighting climate change?
% % 	\begin{itemize}
% % \item Countries should pay in proportion to their income
% % \item Countries should pay in proportion to their current emissions
% % \item Countries should pay in proportion to their past emissions (from 1990 onwards)
% % \item The richest countries should pay it all, so that the poorest countries do not have to pay anything
% % \item The richest countries should pay even more, to help vulnerable countries face adverse consequences: vulnerable countries would then receive money instead of paying
% % \end{itemize} 
% % \textit{Strongly disagree; Somewhat disagree; Neither agree nor disagree; Somewhat agree; Strongly agree}
% 	\includegraphics[width=\textwidth]{../../oecd_climate/figures/FR/burden_sharing_FR.png}
% 	\end{figure}
% \end{frame}

\begin{frame}{}
	\begin{figure}[h!]
	\centering
	\caption{\scriptsize [Question non posée aux U.S., au Danemark et en France]  All countries have signed the Paris agreement that aims to contain global warming ``well below +2 \textdegree{}C''. To limit global warming to this level, there is a maximum amount of greenhouse gases we can emit globally, called the carbon budget. Each country could aim to emit less than a share of the carbon budget. To respect the global carbon budget, countries that emit more than their national share would pay a fee to countries that emit less than their share. \\ 
	Do you support such a policy? \hyperlink{global_policies}{\beamergotobutton{Go back}}
	}
	\includegraphics[height=.7\paperheight]{../../oecd_climate/figures/country_comparison/global_quota_countries.pdf}
	\end{figure}
\end{frame}

\begin{frame}{}
	\begin{figure}[h!]
	\centering
	\caption{[*Question not asked in the U.S., Denmark and France, answers to a similar question are displayed] \\ Suppose the above policy is in place. How should the carbon budget be divided among countries?
	\\ The emission share of a country should be proportional to its population, so that each human has an equal right to emit.; The emission share of a country should be proportional to its current emissions, so that those who already emit more have more rights to emit.; Countries that have emitted more over the past decades (from 1990 onwards) should receive a lower emission share, because they have already used some of their fair share.; Countries that will be hurt more by climate change should receive a higher emission share, to compensate them for the damages. \\
	Percentage of support (somewhat or strong) among: \textit{Strongly oppose; Somewhat oppose; Neither support nor oppose; Somewhat support; Strongly support} \hyperlink{global_policies}{\beamergotobutton{Go back}}
	}
	% \item Do you support or oppose establishing a global democratic assembly whose role would be to draft international treaties against climate change? Each adult across the world would have one vote to elect members of the assembly.
	\includegraphics[width=\paperwidth]{../../oecd_climate/figures/country_comparison/burden_share_positive_countries.pdf}
	\end{figure}
\end{frame}

\begin{frame}{}
	\begin{figure}[h!]
	\centering
	\caption{ Do you support or oppose establishing a global democratic assembly whose role would be to draft international treaties against climate change? Each adult across the world would have one vote to elect members of the assembly. \hyperlink{global_policies}{\beamergotobutton{Go back}}
	%\\ Pourcentage de réponses (plutôt ou très favorable) parmi~:	Très opposé$\cdot$e; Plutôt opposé$\cdot$e; Indifférent$\cdot$e; Plutôt favorable; Très favorable}
	% \item 
% \\ \textit{Strongly oppose; Somewhat oppose; Neither support nor oppose; Somewhat support; Strongly support
}
	\includegraphics[height=.8\paperheight]{../../oecd_climate/figures/country_comparison/global_assembly_support_countries.pdf}
	\end{figure}
\end{frame}

\begin{frame}{}
	\begin{figure}[h!]
	\centering \vspace{-.3cm}
	\caption{\scriptsize Imagine the following policy: a global tax on greenhouse gas emissions funding a global basic income. 
	Such a policy would progressively raise the price of fossil fuels (for example, the price of gasoline would increase by [40 cents per gallon] in the first years). Higher prices would encourage people and companies to use less fossil fuels, reducing greenhouse gas emissions. Revenues from the tax would be used to finance a basic income of [\$30] per month to each human adult, thereby lifting the 700 million people who earn less than \$2/day out of extreme poverty. 
	The average British person would lose a bit from this policy as they would face [\$130] per month in price increases, which is higher than the [\$30] they would receive.
	\\
	Do you support or oppose such a policy?   \hyperlink{global_policies}{\beamergotobutton{Go back}}
	%Pourcentage de réponses (plutôt ou très favorable) parmi~:	Très opposé$\cdot$e; Plutôt opposé$\cdot$e; Indifférent$\cdot$e; Plutôt favorable; Très favorable
	}
% \\ \textit{Strongly oppose; Somewhat oppose; Neither support nor oppose; Somewhat support; Strongly support}
	\includegraphics[height=.65\paperheight]{../../oecd_climate/figures/country_comparison/global_tax_support_countries.pdf}
	\end{figure}
\end{frame}

\begin{frame}{}
	\begin{figure}[h!]
	\centering
	\caption{\scriptsize o you support or oppose a tax on all millionaires around the world to finance low-income countries that comply with international standards regarding climate action? 
	This would finance infrastructure and public services such as access to drinking water, healthcare, and education.  \hyperlink{global_policies}{\beamergotobutton{Go back}}
	%Pourcentage de réponses (plutôt ou très favorable) parmi~:	Très opposé$\cdot$e; Plutôt opposé$\cdot$e; Indifférent$\cdot$e; Plutôt favorable; Très favorable
	}
	% \item D
% \\ \textit{Strongly oppose; Somewhat oppose; Neither support nor oppose; Somewhat support; Strongly support}
	\includegraphics[height=.8\paperheight]{../../oecd_climate/figures/country_comparison/tax_1p_support_countries.pdf}
	\end{figure}
\end{frame}
	
\begin{frame}{}%\addtocounter{framenumber}{-1}
	\begin{figure}[h!]
	\centering
	\caption{Synthèse~: Pourcentage de réponses positive (e.g. Plutôt/Très favorable). \hyperlink{global_policies}{\beamergotobutton{Go back}}}
	\includegraphics[width=\paperwidth]{../../oecd_climate/figures/country_comparison/burden_share_all_positive_countries.pdf}
	%\caption{}
	\end{figure}
\end{frame}

\begin{frame}{}%\addtocounter{framenumber}{-1}
	\begin{figure}[h!]
	\centering
	\caption{Synthèse~: Pourcentage de réponses positive (e.g. \textit{Plutôt/Très favorable}) parmi les non \textit{indifférents}. \hyperlink{global_policies}{\beamergotobutton{Go back}}}
	\includegraphics[width=\paperwidth]{../../oecd_climate/figures/country_comparison/burden_share_all_share_countries.pdf}
	%\caption{}
	\end{figure}
\end{frame}

	
\begin{frame}{Principales des attitudes sur les politiques mondiales}%\addtocounter{framenumber}{-1}
	\begin{figure}[h!]
	\centering
	\caption{Pourcentage de réponses positive (e.g. Plutôt/Très favorable). \hyperlink{global_policies}{\beamergotobutton{Go back}}}
	\includegraphics[width=\paperwidth]{../../oecd_climate/figures/country_comparison/burden_share_few_positive_countries.pdf}
	%\caption{}
	\end{figure}
\end{frame}

\begin{frame}{Principales attitudes sur les politiques mondiales}%\addtocounter{framenumber}{-1}
	\begin{figure}[h!]
	\centering
	\caption{Pourcentage de réponses positive (e.g. \textit{Plutôt/Très favorable}) parmi les non \textit{indifférents}. \hyperlink{global_policies}{\beamergotobutton{Go back}}}
	\includegraphics[width=\paperwidth]{../../oecd_climate/figures/country_comparison/burden_share_few_share_countries.pdf}
	%\caption{}
	\end{figure}
\end{frame}

\begin{frame}{Principales attitudes sur les politiques mondiales}%\addtocounter{framenumber}{-1}
	\begin{figure}[h!]
	\centering
	\caption{Moyennes des réponses, recodées en [$-$2; +2]. \hyperlink{global_policies}{\beamergotobutton{Go back}}}
	\includegraphics[width=\paperwidth]{../../oecd_climate/figures/country_comparison/burden_share_few_mean_countries.pdf}
	%\caption{}
	\end{figure}
\end{frame}


\end{framefont}


\end{document}