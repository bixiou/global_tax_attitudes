% /!\ No footnote for NCC or NS.

% *. Main Figures
% A. Literature review
% B. Raw results
% C. Questionnaire (global survey)
% D. Questionnaire (complementary surveys)
% E. Net gains from the Global Climate Scheme
% F. Determinants of support
% G. Representativity of the surveys
% H. Attrition analysis
% I. Balance analysis
% J. Placebo tests

% TODO!! resolve "Section ??"
% TODO! more raw results, sources (cf. app.tex)
% TODO! Dechezlepretre as companion paper
% plot maps and compare distributive effects of equal pc, contraction & convergence, greenhouse dvlpt rights, historical respo, and each country retaining its revenues
% improve net gains with SSPs

% app_desc: some more figures (e.g. detailed OECD by country)
% conjoint d: by vote for all countries; merge with r; bug PDF
% TODO appendix sources
% autres to do!: n, additional figures/tables or details
% literature: list experiment, (elite surveys)
% Extra: translate country-specific appendices

% Now: 6200 words (excl. abstract, methods, appendix), incl. 5.2k in Results.
% Current: 5600 words (excl. abstract, methods, appendix), incl. 4.6k in Results.
% Old: 4300 words (excl. abstract, methods, appendix) + 3 medium-large figures (equivalent, I know it's 4) => correct size
% => write a 2-3 pager (1000 words with 2 figures and one-sentence abstract) on Word with Science's template, send it as submission enquiry to Nature (as one can't send full paper); then for Policy forum in Science.
% => write full paper as a 6-page (2500 words*) in LaTeX or Word so it can fit in PNAS, and even for NCC/NS it shouldn't exceed 8-9-page. (It's already too long for Science's max 5 page).
% *6-page is more 5000 than 2500 words. I've checked Steckel et al (NS, 21) and it's 5k words for 7p (excl. abstract and methods, dont 3.8k excl. the 6 medium-large figures/tables) + 2.2k in methods, data availability. Case & Deaton (PNAS, 21) is also 5k words for 5p (excl abstract and biblio, dont 4.7k excl. the 3 medium-large figures). Bruckner et al (NS, 22): 4.3k words for 6p (excl. abstract, methods, biblio, dont 3.8k excl the 5 medium-large figures) + 3.4k in methods. => about 1k per page without figure/table. 
% => Submission order: 1. Nature, 2. Science, 3. NCC, 4. Nature Sust, 5. PNAS, 6. Science Advances, 7. GEC, 8. ERL or JEEM. (or 3. NS (if: 27) and 4. NCC (if: 22)?)
% => Sample sizes should be given (only) on Figures

% Nature "abstract": Major sustainability objectives could be achieved by global approaches to mitigating climate change and inequality    . For instance, a global carbon price funding a global basic income, called the “Global Climate Scheme” (GCS), would be an effective way to jointly combat climate change and poverty. A key condition for the success of global cooperation is the support of citizens in affluent countries for such globally redistributive policies. Yet, few prior attitudinal surveys have examined support for global policies. To explore relevant public attitudes, we survey over 48,000 respondents from 20 high- and middle-income countries. The responses reveal strong support for global redistributive policies, including the GCS and a global wealth tax aimed at financing low-income countries. A list experiment shows no evidence of social desirability bias in survey responses, majorities are willing to sign a real-stake petition, and global redistribution ranks high in the prioritization of policies. Conjoint analyses reveal that a political platform is more likely to be preferred if it contains the GCS or a global tax on millionaires. In sum, our findings indicate that global redistributive policies are genuinely supported by a majority of the population, even in wealthy nations that would bear a significant burden. Public opinion is therefore not the reason that they do not prominently enter political debates. These results could help draw attention to global policies in the public debate and contribute to their increased prominence.

% Nature guidelines:
% Attention to the following details can help expedite publication if we invite a revision after external review.
% A fully referenced ~200 word summary paragraph; main text of 2,500 words and 4 modest display items (figures, tables) for a typical 6 page article and 4300 words and 5-6 modest display items for a typical 8 page article; as a guideline up to 50 references if needed and within the allocated page budget.

%%%%%%%%%%%%%%%%%%%%%%%%%%%%%%%%%%%%%%%%
%%%%% NATURE CLIMATE CHANGE FORMAT %%%%%
%%%%%%%%%%%%%%%%%%%%%%%%%%%%%%%%%%%%%%%%
%% Comment "% WPcomment" lines, uncomment "% NCCcomment" lines as well as the lines below, replace all citet/citep by cite

% \documentclass{nature}
% \usepackage{amsmath}
% \usepackage{amssymb}
% \usepackage{eurosym}
% % The following allows keeping figures within the text (otherwise nature.cls would ignore them)
% \usepackage{graphicx}
% \makeatletter
% \let\saved@includegraphics\includegraphics
% \AtBeginDocument{\let\includegraphics\saved@includegraphics}
% \renewenvironment*{figure}{\@float{figure}}{\end@float}
% \makeatother

% Nature guidelines (not NCC!)
% Sections can only be used in Articles.  Contributions should be organized in the sequence: title, text, methods, references, Supplementary Information line (if any), acknowledgements, interest declaration, corresponding author line, tables, figure legends.

% No subsubsection nor paragraph

% Spelling must be British English (Oxford English Dictionary)

%Each figure legend should begin with a brief title for the whole figure and continue with a short description of each panel and the symbols used. For contributions with methods sections, legends should not contain any details of methods, or exceed 100 words (fewer than 500 words in total for the whole paper). In contributions without methods sections, legends should be fewer than 300 words (800 words or fewer in total for the whole paper).

% Articles are restricted to 50 references,

% In addition, a cover letter needs to be written with the
% following:
% \begin{enumerate}
%  \item A 100 word or less summary indicating on scientific grounds
% why the paper should be considered for a wide-ranging journal like
% \textsl{Nature} instead of a more narrowly focussed journal.
%  \item A 100 word or less summary aimed at a non-scientific audience,
% written at the level of a national newspaper.  It may be used for
% \textsl{Nature}'s press release or other general publicity.
%  \item The cover letter should state clearly what is included as the
% submission, including number of figures, supporting manuscripts
% and any Supplementary Information (specifying number of items and
% format).
%  \item The cover letter should also state the number of
% words of text in the paper; the number of figures and parts of
% figures (for example, 4 figures, comprising 16 separate panels in
% total); a rough estimate of the desired final size of figures in
% terms of number of pages; and a full current postal address,
% telephone and fax numbers, and current e-mail address.
% \end{enumerate}

% See \textsl{Nature}'s website
% (\texttt{http://www.nature.com/nature/submit/gta/index.html}) for
% complete submission guidelines.

%%%%%%%%%%%%%%%%%%%%%%%%%%%%%%%%
%%%%% WORKING PAPER FORMAT %%%%%
%%%%%%%%%%%%%%%%%%%%%%%%%%%%%%%%
%% Comment "% NCCcomment" lines, uncomment "% WPcomment" lines as well as the lines below
\documentclass[12pt,english]{article}
\usepackage[utf8]{inputenc}
\usepackage{tgpagella} % Palatino text only
\usepackage{mathpazo}  % Palatino math & text
\usepackage[left=1.5in,right=1.5in,top=1.5in,bottom=1.5in]{geometry}
% \linespread{1.5}
% \usepackage[super,comma,sort]{natbib} % WPcomment
\usepackage[round,sort&compress]{natbib} % NCCcomment
\usepackage{url} % [hyphens]
\usepackage[hyperpageref]{backref} % back references biblio. Needs latexmk at compilation.
\usepackage[pagebackref]{hyperref}
% \usepackage{multibib} % incompatible with backref
\hypersetup{
  colorlinks=true, % breaklinks=true,
  urlcolor=purple,    % color of external links
  linkcolor=blue,  % color of toc, list of figs etc.
  citecolor=violet,   % color of links to bibliography
}
\usepackage{bm}
\usepackage{indentfirst}
\usepackage{tocbibind}
\setcitestyle{aysep={}} 
\usepackage{amsmath}
\usepackage{tcolorbox}
\usepackage{amssymb}
\usepackage{eurosym}
\usepackage{amsfonts}
\usepackage{enumerate}
\usepackage{babel}
\usepackage{graphicx}
\usepackage{caption}
\usepackage{supertabular}
\usepackage{tabularx}
\usepackage{float}
\usepackage{dsfont}
\usepackage{fancyvrb}
\usepackage{verbatim}
\usepackage{enumitem}
\usepackage{setspace}
\usepackage{comment}
\usepackage{subcaption}
\usepackage{tikz}
\usepackage{gensymb}
\usepackage{textcomp}
\usepackage{lineno}
\linenumbers

\usepackage{tabulary}
\usepackage{tabularx}
\usepackage{booktabs}
\usepackage{fullpage}
\usepackage{morefloats}
\usepackage{makecell}
\usepackage{lscape}
\usepackage{pdflscape}
\usepackage{longtable}
\usepackage{rotating}
\usepackage{fancyhdr}
\usepackage{tocloft}
\usepackage{titletoc}
\usepackage[export]{adjustbox}
\usepackage[anythingbreaks]{breakurl} % for links
\usepackage{multicol}
\newsavebox\ltmcbox % For net gain table over two columns
%\usepackage[nomarkers,figuresonly]{endfloat} % Figures at the end
%\usepackage[section,below]{placeins} % Floats placed in the section they appear in.
\renewcommand{\floatpagefraction}{.99}
\newenvironment{stretchpars}{\par\setlength{\parfillskip}{0pt}}{\par} % to justify a line

% % Getting landscape page and page number/footer on bottom of page (instead of to the left)
% \fancypagestyle{mylandscape}{
% \fancyhf{} %Clears the header/footer
% \fancyfoot{% Footer
% \makebox[\textwidth][r]{% Right
%   \rlap{\hspace{1.5cm}% Push out of margin by \footskip
%     \smash{% Remove vertical height
%       \raisebox{13.6cm}{% Raise vertically
%         \rotatebox{90}{\thepage}}}}}}% Rotate counter-clockwise
% \renewcommand{\headrulewidth}{0pt}% No header rule
% \renewcommand{\footrulewidth}{0pt}% No footer rule
% }

% \fancypagestyle{page_left}{%
% 	\renewcommand{\headrulewidth}{0pt}
%   \fancyhf{}
%   \fancyfoot[OC]{%
%       \begin{tikzpicture}[remember picture,overlay]
%           \node[xshift=1cm] (number) at (current page.west) {\thepage};
%       \end{tikzpicture}
%   }%
% }
% \renewcommand{\thesubfigure}{\Alph{subfigure}}

% \newcites{App}{Appendix References}

% \captionsetup[table]{skip=-10pt}
% \begin{document}

% \maketitle

% \clearpage
% % \startcontents
% % \printcontents{ }{1}{\section{\contentsname}}
% % \clearpage
% \section{Introduction\label{sec:intro}}

% % \clearpage
% \renewcommand{\bibsection}{\section{\refname}}
% \bibliographystyle{naturemag}
% \bibliography{global_tax_attitudes}
% % \stopcontents

% \end{document}


\title{International Attitudes Toward Global Policies \\ Supplementary Material %\\ Addressing Climate Change and Inequality 
} 

% \author{Adrien Fabre$^{1,2}$, Thomas Douenne$^3$ and Linus Mattauch$^{4,5,6}$} % WPcomment
\author{Adrien Fabre\footnote{CNRS, CIRED. E-mail: adrien.fabre@cnrs.fr (corresponding author).}, Thomas Douenne\footnote{University of Amsterdam}\; and Linus Mattauch\footnote{Technical University Berlin, Potsdam Institute for Climate Impact Research -- Member of the Leibniz Association and University of Oxford}}%~~\thanks{The project %is approved by IRB at Harvard University (IRB21-0137), and 
% was preregistered in the Open Science Foundation registry (\href{https://osf.io/fy6gd}{osf.io/fy6gd}). \\ We are grateful for financial support from the University of Amsterdam and TU Berlin. Mattauch also thanks the Robert Bosch Foundation. %We are grateful for financial support from the OECD, the French Ministry of Foreign Affairs, the French Conseil d’Analyse Economique and the Spanish Ministry for the Ecological Transition and Demographic Challenge. We also acknowledge support from the Grantham Foundation for the Protection of the Environment and the Economic and Social Research Council through the Centre for Climate Change Economics and Policy. 
% We thank Antoine Dechezleprêtre, Tobias Kruse, Bluebery Planterose, Ana Sanchez Chico, and Stefanie Stantcheva for their invaluable inputs for the project. We thank Auriane Meilland for feedback. We further thank Jakob Niemann, Laura Schepp, Martín Fernández-Sánchez, Samuel Gervais, Samuel Haddad, and Guadalupe Manzo for assistance. Fabre declares that he also serves as president of Global Redistribution Advocates.}} % NCCcomment

\date{\today} % NCCcomment

\begin{document}

\maketitle

\begin{center}
\end{center}


% WPcomment
% \begin{affiliations}
% \item CNRS
% \item CIRED
% \item University of Amsterdam
% \item Technical University Berlin
% \item Potsdam Institute for Climate Impact Research 
% \item University of Oxford
% \end{affiliations}

% \begin{small} % NCCcomment


\tableofcontents

\onehalfspacing % NCCcomment

\clearpage
\section*{Main figures}\label{sec:figures}
\addcontentsline{toc}{section}{\nameref{sec:figures}}

\begin{figure}[h!]
  % MAJOR figure
  \caption[Relative support for global climate policies]{Relative support for global climate policies. (Reproduced from \cite{dechezlepretre_fighting_2022}, Figure A21.)}  % TODO!? put back?
  \makebox[\textwidth][c]{\includegraphics[width=1.2\textwidth]
  {../figures/OECD/Heatplot_global_tax_attitudes_share.pdf}}\label{fig:oecd} % with dependence on others (absent from OECD): Heatplot_burden_share_all_share_countries
  {\footnotesize \\ $\quad$ \\ Note 1: The numbers represent the share of \textit{Somewhat} or \textit{Strongly support} among non-\textit{indifferent} answers (in percent, $n$ = 40,680). The color blue denotes a relative majority. See Figure \ref{fig:oecd_absolute} for the absolute support. (Questions \ref{q:scale}-\ref{q:millionaire_tax}). %Reproduced from \cite{dechezlepretre_fighting_2022}, Figure A21.) % TODO!? put back?
  \\ Note 2: *In Denmark, France and the U.S., the questions with an asterisk were asked differently, cf. Question F.% \ref{q:burden_sharing_asterisk}. 
  } 
\end{figure}

\begin{figure}
  % MAJOR figure
  \caption[Relative support for further global policies]{Relative support for various global policies (percentage of \textit{somewhat} or \textit{strong support}, after excluding \textit{indifferent} answers). (Questions \ref{q:climate_policies} and \ref{q:other_policies}; See Figure \ref{fig:support_likert_positive} for the absolute support.)% $n_\text{US} = n_\text{Eu} = 3,000,\, n_\text{FR} = 729,\, n_\text{DE} = 929,\, n_\text{ES} = 543,\, n_\text{UK} = 749, n_\text{US, global/national wealth tax} = 2,000$
  }
  \makebox[\textwidth][c]{\includegraphics[width=\textwidth]{../figures/country_comparison/support_likert_share.pdf}}\label{fig:support}
\end{figure} 

\begin{table}[h]
  % MAJOR figure % TODO! same table for NR in appendix
  % TODO table by country
  \caption[List experiment: tacit support for the GCS]{Number of supported policies in the list experiment depending on the presence of the Global Climate Scheme (GCS) in the list.%in function of the composition of the list. GCS stands for the Global Climate Scheme and NR for the National Redistribution Scheme.} % Beware, this question is quite unusual. \\ Among the policies below, how many do you support?  \\ Coal exit, Marriage only for opposite-sex couples 
  }\label{tab:list_exp}
  \makebox[\textwidth][c]{
\begin{tabular}{@{\extracolsep{5pt}}lccc} 
\\[-1.8ex]\hline 
\hline \\[-1.8ex] 
 & \multicolumn{3}{c}{Number of supported policies} \\ 
\cline{2-4} 
\\[-1.8ex] & All & US & Eu \\ 
\hline \\[-1.8ex] 
 List contains: GCS & 0.624$^{***}$ & 0.524$^{***}$ & 0.724$^{***}$ \\ 
  & (0.028) & (0.041) & (0.036) \\ 
\hline  \\[-1.8ex] \textit{Support for GCS} & 0.617  &  0.542  &  0.757 \\
\textit{Social desirability bias} & \textit{$ -0.026 $} & \textit{$ -0.018 $} & \textit{$ -0.033 $}\\
\textit{80\% C.I. for the bias} & \textit{ $[ -0.06 ; 0.01 ]$ } & \textit{ $[ -0.07 ; 0.01 ]$} & \textit{ $[ -0.08 ; 0.01 ]$}\\
 \hline \\[-1.8ex] 
Constant & 1.317 & 1.147 & 1.486 \\ 
Observations & 6,000 & 3,000 & 3,000 \\ 
R$^{2}$ & 0.089 & 0.065 & 0.125 \\ 
\hline 
\hline \\[-1.8ex] 
\textit{Note:}  & \multicolumn{3}{r}{$^{*}$p$<$0.1; $^{**}$p$<$0.05; $^{***}$p$<$0.01} \\ 
\end{tabular} 
  }  
  % {\footnotesize \textit{Note:} $^{*}p<0.1$; $^{**} p<0.05$; $^{***} p<0.01$.}
\end{table}

\begin{table}[h]
  % MAJOR figure
  \caption[Influence of the GCS on electoral prospects]{Preference for a progressive platform depending on whether it includes the GCS or not. (Question \ref{q:conjoint_c}) 
  %Imagine if the [Democratic and Republican presidential candidates in 2024] campaigned with the following policies in their platforms. [Credible Progressive and Conservative platforms] \\ % TODO See More
% Which of these candidates would you vote for? \textit{A; B; None of them} \\
% ~[FR: second round of presidential; DE, ES, UK: two favorite candidates in one's constituency]
} % Beware, this question is quite unusual. \\ Among the policies below, how many do you support?  \\ Coal exit, Marriage only for opposite-sex couples 
  \makebox[\textwidth][c]{
\begin{tabular}{@{\extracolsep{5pt}}lcccccc} 
\\[-1.8ex]\hline 
\hline \\[-1.8ex] 
 & \multicolumn{6}{c}{Prefers the Progressive platform} \\ 
\cline{2-7} 
\\[-1.8ex] & All & United States & France & Germany & UK & Spain \\ 
\hline \\[-1.8ex] 
 GCS in Progressive platform & 0.028$^{*}$ & 0.029 & 0.112$^{***}$ & 0.015 & 0.008 & $-$0.015 \\ 
  & (0.014) & (0.022) & (0.041) & (0.033) & (0.040) & (0.038) \\ 
 \hline \\[-1.8ex] 
Constant & 0.623 & 0.604 & 0.55 & 0.7 & 0.551 & 0.775 \\ 
Observations & 5,202 & 2,619 & 605 & 813 & 661 & 504 \\ 
R$^{2}$ & 0.001 & 0.001 & 0.013 & 0.0003 & 0.0001 & 0.0003 \\ 
\hline 
\hline \\[-1.8ex] 
\end{tabular} 
}\label{tab:conjoint_c}
  {\footnotesize \textit{Note:} The 14\% of \textit{None of them} answers have been excluded from the regression samples. GCS has no significant influence on them. $^{*}p<0.1$; $^{**} p<0.05$; $^{***} p<0.01$. 
  }
\end{table}

\begin{figure}[h!]
  \caption[Support for the Global Climate Scheme]{Support for the GCS, NR and the combination of GCS, NR and C. \\(Questions \ref{q:global_tax}, \ref{q:national_tax}, \ref{q:gcs_support}, \ref{q:nr_support} and \ref{q:crg_support}).%; $n_\text{US} = n_\text{Eu} = 3,000,\, n_\text{FR} = 729,\, n_\text{DE} = 929,\, n_\text{ES} = 543,\, n_\text{UK} = 749$)
  }\label{fig:support_binary}
  \makebox[\textwidth][c]{\includegraphics[width=.9\textwidth]{../figures/country_comparison/support_binary_positive.pdf}} 
\end{figure}


\begin{figure}
  \centering 
  \caption[Preferred share of wealth tax for low-income countries]{Percent of global wealth tax that should finance low-income countries (\textit{mean}). (Question \ref{q:global_tax_global_share})} % TODO? n
  \includegraphics[width=1\textwidth]{../figures/country_comparison/global_tax_global_share_mean.pdf} \label{fig:global_share_mean}
\end{figure}

% \begin{figure}
%   % MAJOR figure
%   \caption[Relative support for further global policies]{Relative support for various global policies (percentage of \textit{somewhat} or \textit{strong support}, after excluding \textit{indifferent} answers). (Questions \ref{q:climate_policies} and \ref{q:other_policies}; See Figure \ref{fig:support_likert_positive} for the absolute support.)% $n_\text{US} = n_\text{Eu} = 3,000,\, n_\text{FR} = 729,\, n_\text{DE} = 929,\, n_\text{ES} = 543,\, n_\text{UK} = 749, n_\text{US, global/national wealth tax} = 2,000$
%   }
%   \makebox[\textwidth][c]{\includegraphics[width=\textwidth]{../figures/country_comparison/support_likert_share.pdf}}\label{fig:support}
% \end{figure} 

\begin{figure}[h!]
\caption[Attitudes on the evolution of foreign aid]{Attitudes regarding the evolution of [own country] foreign aid. (Question \ref{q:foreign_aid_raise_support})}\label{fig:foreign_aid_raise_support}
\makebox[\textwidth][c]{\includegraphics[width=\textwidth]{../figures/country_comparison/foreign_aid_raise_support.pdf}} 
\end{figure}

\begin{figure}[h!]
\caption[Conditions at which foreign aid should be increased]{Conditions at which foreign aid should be increased (in percent). [Asked to those who wish an increase of foreign aid at some conditions.] (Question \ref{q:foreign_aid_condition})}\label{fig:foreign_aid_condition}
\makebox[\textwidth][c]{\includegraphics[width=\textwidth]{../figures/country_comparison/foreign_aid_condition_positive.pdf}} 
\end{figure}

\begin{figure}[h!]
\caption[Reasons why foreign aid should not be increased]{Reasons why foreign aid should not be increased (in percent). [Asked to those who wish a decrease or stability of foreign aid.] (Question \ref{q:foreign_aid_no})}\label{fig:foreign_aid_no}
\makebox[\textwidth][c]{\includegraphics[width=\textwidth]{../figures/country_comparison/foreign_aid_no_positive.pdf}} 
\end{figure}



\begin{figure}[h] 
\caption[Preferences for various policies in political platforms]{Effects of the presence of a policy (rather than none from this domain) in a random platform on the likelihood that it is preferred to another random platform. (See English translations in Figure \ref{fig:ca_r_en}; Question \ref{q:conjoint_r}%; in the U.S., asked only to non-Republicans.
)}\label{fig:ca_r}
\begin{subfigure}{\textwidth}
  \subcaption{U.S. (Asked only to non-Republicans)}
  \includegraphics[width=\textwidth]{../figures/US1/ca_r.png}
\end{subfigure}
\begin{subfigure}{\textwidth}
  \subcaption{France}
  \includegraphics[width=\textwidth]{../figures/FR/ca_r.png}
\end{subfigure}
\end{figure}%
\clearpage
\begin{figure}[h!]\ContinuedFloat % if bugs try b! instead of h!
\begin{subfigure}{\textwidth}
  \subcaption{Germany}
  \includegraphics[width=\textwidth]{../figures/DE/ca_r.png}
\end{subfigure}
\begin{subfigure}{\textwidth}
  \subcaption{Spain}
  \includegraphics[width=\textwidth]{../figures/ES/ca_r.png}
\end{subfigure}
\begin{subfigure}{\textwidth}
  \subcaption{UK}
  \includegraphics[width=\textwidth]{../figures/UK/ca_r.png}
\end{subfigure}
%\makebox[\textwidth][c]{} 
\end{figure}


\begin{figure}[h!]
  \caption[Influence of the GCS on preferred platform]{Influence of the GCS on preferred platform:\\ Preference for a random platform A that contains the Global Climate Scheme rather than a platform B that does not (in percent). (Question \ref{q:conjoint_d}; in the U.S., asked only to non-Republicans.)}\label{fig:conjoint_left_ag_b}
  \makebox[\textwidth][c]{\includegraphics[width=\textwidth]{../figures/country_comparison/conjoint_left_ag_b_binary_positive.pdf}} 
\end{figure}


\begin{figure}[h!]
  \caption[Beliefs about support for the GCS and NR]{Beliefs regarding the support for the GCS and NR. (Questions \ref{q:gcs_belief} and \ref{q:nr_belief})}\label{fig:belief}
  \makebox[\textwidth][c]{\includegraphics[width=.7\textwidth]{../figures/country_comparison/belief_all_mean.pdf}} 
\end{figure}

\appendix % NCCcomment
\renewcommand{\thetable}{A\arabic{table}}
\renewcommand{\thefigure}{A\arabic{figure}}
\setcounter{figure}{0}
\setcounter{table}{0}

\input{literature_review_nature.tex}

% \clearpage
% \section{Raw results% from the complementary surveys
% }\label{app:raw_results}

\input{app_nature} 

\clearpage
\renewcommand{\url}[1]{\href{#1}{Link}} % NCCcomment
\bibliographystyle{plainnaturl_clean} % NCCcomment
\bibliography{global_tax_attitudes}

\clearpage
\listoftables
\listoffigures

\end{document}
