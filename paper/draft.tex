% GCS/NR vs. G/R
% 10-12 days of work:
% decrit(us2$interview) decrit(us1$interview)
% theory: list experiment model
% treat fields and gcs_important
% render: petition comparable sample
% add sociodemos determinants
% plot maps and compare distributive effects of equal pc, contraction & convergence, greenhouse dvlpt rights, historical respo, and each country retaining its revenues
% improve net gains, e.g. Global Energy Assessment
% conjoint d: by vote for all countries; merge with r; bug PDF
% TODO appendix sources
% TODO! aussi in prez: soutien par vote
% autres to do!: n, additional figures/tables or details
% literature: list experiment, world citizenship, elite surveys
% app_desc: some more figures (e.g. detailed OECD by country)
% Extra: translate country-specific appendices

% Current: 4300 words (excl. abstract, methods, appendix) + 3 medium-large figures (equivalent, I know it's 4) => correct size
% => write a 2-3 pager (1000 words with 2 figures) on Word with Science's template, send it as submission enquiry to Nature (as one can't send full paper); then for Policy forum in Science.
% => write full paper as a 6-page (2500 words*) in LaTeX or Word so it can fit in PNAS, and even for NCC/NS it shouldn't exceed 8-9-page. (It's already too long for Science's max 5 page).
% *6-page is more 5000 than 2500 words. I've checked Steckel et al (NS, 21) and it's 5k words for 7p (excl. abstract and methods, dont 3.8k excl. the 6 medium-large figures/tables) + 2.2k in methods, data availability. Case & Deaton (PNAS, 21) is also 5k words for 5p (excl abstract and biblio, dont 4.7k excl. the 3 medium-large figures). Bruckner et al (NS, 22): 4.3k words for 6p (excl. abstract, methods, biblio, dont 3.8k excl the 5 medium-large figures) + 3.4k in methods. => about 1k per page without figure/table. 
% => Submission order: 1. Nature, 2. Science, 3. NCC, 4. Nature Sust, 5. PNAS, 6. Science Advances, 7. GEC, 8. ERL or JEEM. (or 3. NS (if: 27) and 4. NCC (if: 22)?)
% => Sample sizes should be given (only) on Figures

%%%%%%%%%%%%%%%%%%%%%%%%%%%%%%%%%%%%%%%%
%%%%% NATURE CLIMATE CHANGE FORMAT %%%%%
%%%%%%%%%%%%%%%%%%%%%%%%%%%%%%%%%%%%%%%%
%% Comment "% WPcomment" lines, uncomment "% NCCcomment" lines as well as the lines below, replace all citet/citep by cite

% \documentclass{nature}
% \usepackage{amsmath}
% \usepackage{amssymb}
% \usepackage{eurosym}
% % The following allows keeping figures within the text (otherwise nature.cls would ignore them)
% \usepackage{graphicx}
% \makeatletter
% \let\saved@includegraphics\includegraphics
% \AtBeginDocument{\let\includegraphics\saved@includegraphics}
% \renewenvironment*{figure}{\@float{figure}}{\end@float}
% \makeatother

% Nature guidelines (not NCC!)
% Sections can only be used in Articles.  Contributions should be organized in the sequence: title, text, methods, references, Supplementary Information line (if any), acknowledgements, interest declaration, corresponding author line, tables, figure legends.

% No subsubsection nor paragraph

% Spelling must be British English (Oxford English Dictionary)

%Each figure legend should begin with a brief title for the whole figure and continue with a short description of each panel and the symbols used. For contributions with methods sections, legends should not contain any details of methods, or exceed 100 words (fewer than 500 words in total for the whole paper). In contributions without methods sections, legends should be fewer than 300 words (800 words or fewer in total for the whole paper).

% Articles are restricted to 50 references,

% In addition, a cover letter needs to be written with the
% following:
% \begin{enumerate}
%  \item A 100 word or less summary indicating on scientific grounds
% why the paper should be considered for a wide-ranging journal like
% \textsl{Nature} instead of a more narrowly focussed journal.
%  \item A 100 word or less summary aimed at a non-scientific audience,
% written at the level of a national newspaper.  It may be used for
% \textsl{Nature}'s press release or other general publicity.
%  \item The cover letter should state clearly what is included as the
% submission, including number of figures, supporting manuscripts
% and any Supplementary Information (specifying number of items and
% format).
%  \item The cover letter should also state the number of
% words of text in the paper; the number of figures and parts of
% figures (for example, 4 figures, comprising 16 separate panels in
% total); a rough estimate of the desired final size of figures in
% terms of number of pages; and a full current postal address,
% telephone and fax numbers, and current e-mail address.
% \end{enumerate}

% See \textsl{Nature}'s website
% (\texttt{http://www.nature.com/nature/submit/gta/index.html}) for
% complete submission guidelines.

%%%%%%%%%%%%%%%%%%%%%%%%%%%%%%%%
%%%%% WORKING PAPER FORMAT %%%%%
%%%%%%%%%%%%%%%%%%%%%%%%%%%%%%%%
%% Comment "% NCCcomment" lines, uncomment "% WPcomment" lines as well as the lines below
\documentclass[12pt,english]{article}
\usepackage[utf8]{inputenc}
\usepackage{tgpagella} % Palatino text only
\usepackage{mathpazo}  % Palatino math & text
\usepackage[left=1.5in,right=1.5in,top=1.5in,bottom=1.5in]{geometry}
% \linespread{1.5}
% \usepackage[super,comma,sort]{natbib} % WPcomment
\usepackage[round,sort&compress]{natbib} % NCCcomment
\usepackage{url} % [hyphens]
\usepackage[hyperpageref]{backref} % back references biblio. Needs latexmk at compilation.
\usepackage[pagebackref]{hyperref}
% \usepackage{multibib} % incompatible with backref
\hypersetup{
  colorlinks=true, % breaklinks=true,
  urlcolor=purple,    % color of external links
  linkcolor=blue,  % color of toc, list of figs etc.
  citecolor=violet,   % color of links to bibliography
}
\usepackage{bm}
\usepackage{indentfirst}
\usepackage{tocbibind}
\setcitestyle{aysep={}} 
\usepackage{amsmath}
\usepackage{amssymb}
\usepackage{eurosym}
\usepackage{amsfonts}
\usepackage{enumerate}
\usepackage{babel}
\usepackage{graphicx}
\usepackage{caption}
\usepackage{supertabular}
\usepackage{tabularx}
\usepackage{float}
\usepackage{dsfont}
\usepackage{fancyvrb}
\usepackage{verbatim}
\usepackage{enumitem}
\usepackage{setspace}
\usepackage{comment}
\usepackage{subcaption}
\usepackage{tikz}
\usepackage{gensymb}
\usepackage{textcomp}

\usepackage{tabulary}
\usepackage{tabularx}
\usepackage{booktabs}
\usepackage{fullpage}
\usepackage{morefloats}
\usepackage{makecell}
\usepackage{lscape}
\usepackage{pdflscape}
\usepackage{longtable}
\usepackage{rotating}
\usepackage{fancyhdr}
\usepackage{tocloft}
\usepackage{titletoc}
\usepackage[export]{adjustbox}
\usepackage[anythingbreaks]{breakurl} % for links
\usepackage{multicol}
\newsavebox\ltmcbox % For net gain table over two columns
%\usepackage[nomarkers,figuresonly]{endfloat} % Figures at the end
%\usepackage[section,below]{placeins} % Floats placed in the section they appear in.
\renewcommand{\floatpagefraction}{.99}

% % Getting landscape page and page number/footer on bottom of page (instead of to the left)
% \fancypagestyle{mylandscape}{
% \fancyhf{} %Clears the header/footer
% \fancyfoot{% Footer
% \makebox[\textwidth][r]{% Right
%   \rlap{\hspace{1.5cm}% Push out of margin by \footskip
%     \smash{% Remove vertical height
%       \raisebox{13.6cm}{% Raise vertically
%         \rotatebox{90}{\thepage}}}}}}% Rotate counter-clockwise
% \renewcommand{\headrulewidth}{0pt}% No header rule
% \renewcommand{\footrulewidth}{0pt}% No footer rule
% }

% \fancypagestyle{page_left}{%
% 	\renewcommand{\headrulewidth}{0pt}
%   \fancyhf{}
%   \fancyfoot[OC]{%
%       \begin{tikzpicture}[remember picture,overlay]
%           \node[xshift=1cm] (number) at (current page.west) {\thepage};
%       \end{tikzpicture}
%   }%
% }
% \renewcommand{\thesubfigure}{\Alph{subfigure}}

% \newcites{App}{Appendix References}

% \captionsetup[table]{skip=-10pt}
% \begin{document}

% \maketitle

% \clearpage
% % \startcontents
% % \printcontents{ }{1}{\section{\contentsname}}
% % \clearpage
% \section{Introduction\label{sec:intro}}

% % \clearpage
% \renewcommand{\bibsection}{\section{\refname}}
% \bibliographystyle{naturemag}
% \bibliography{global_tax_attitudes}
% % \stopcontents

% \end{document}


\title{International Attitudes Toward Global Policies %\\ Addressing Climate Change and Inequality 
} 

\author{Adrien Fabre$^{1,2}$, Thomas Douenne$^3$ and Linus Mattauch$^{4,5,6}$} % WPcomment
\author{Adrien Fabre\footnote{CNRS, CIRED. E-mail: fabre.adri1@gmail.com (corresponding author).}, Thomas Douenne\footnote{University of Amsterdam}\; and Linus Mattauch\footnote{Technical University Berlin, Potsdam Institute for Climate Impact Research and University of Oxford}~~\thanks{The project %is approved by IRB at Harvard University (IRB21-0137), and 
was preregistered in the Open Science Foundation registry (\href{https://osf.io/fy6gd}{osf.io/fy6gd}). \\ We are grateful for financial support from the University of Amsterdam and TU Berlin. %We are grateful for financial support from the OECD, the French Ministry of Foreign Affairs, the French Conseil d’Analyse Economique and the Spanish Ministry for the Ecological Transition and Demographic Challenge. We also acknowledge support from the Grantham Foundation for the Protection of the Environment and the Economic and Social Research Council through the Centre for Climate Change Economics and Policy. 
We thank Antoine Dechezleprêtre, Tobias Kruse, Bluebery Planterose, Ana Sanchez Chico, and Stefanie Stantcheva for their invaluable inputs for the project. We thank Auriane Meilland for feedback. We thank Laura Schepp, Martín Fernández-Sánchez, Samuel Gervais, Samuel Haddad, and Guadalupe Manzo for assistance in the translation. }} % NCCcomment

\date{\today} % NCCcomment

\begin{document}

\maketitle

\begin{center}
{\textbf{Work in progress --- \href{https://github.com/bixiou/global_tax_attitudes/raw/main/paper/US1.pdf}{Link to most recent version}}}
\end{center}


% WPcomment
% \begin{affiliations}
% \item CNRS
% \item CIRED
% \item University of Amsterdam
% \item Technical University Berlin
% \item Potsdam Institute for Climate Impact Research 
% \item University of Oxford
% \end{affiliations}

% \begin{small} % NCCcomment
\begin{abstract} % 263 words. TODO? mention more the other measures?
  A global carbon price funding a global basic income, called the ``Global Climate Scheme'' (GCS), would be an effective and progressive way to combat climate change and poverty. Yet, such policy is mostly absent from political platforms and the policy debate. 
  Using surveys on 40,680 respondents in 20 countries covering 72\% of global CO$_\text{2}$ emissions, we document majority support for this and other global policies. Using a complementary survey on 8,000 respondents in the U.S., France, Germany, Spain and the UK, we test several hypotheses that could reconcile strong stated support with a lack of salience of these issues. The complementary analyses show that the stated support is mostly sincere, as a list experiment shows no evidence of social desirability bias, majorities are also willing to sign a real-stake petition, and global redistributive policies rank high %(though not highest) 
  in the prioritization of policies. Conjoint analyses reveal that a progressive candidate would not significantly lose voting share by endorsing the GCS in any country, and may even gain 11 p.p. %in voting intention 
  in France. Likewise, a platform is more likely to be preferred if it contains the GCS or a global tax on millionaires. 
  Accurate beliefs about the level of support for the GCS dismisses the hypothesis of pluralistic ignorance of the support. 
  Strong universalistic attitudes are confirmed in more general questions, suggesting that the support cannot be explained away by malleable opinion or experimenter demand. In sum, our findings indicate that global policies are genuinely supported by a majority of the population. Public opinion is therefore not the reason that they do not prominently enter political debates. %Finally, we conclude that, at odds with the absence of global policies in the public debate or political platforms, there is no evidence that most people would reject them. %there is no compelling reason why global policies do not enter the public debate or political platforms, as they seem genuinely supported by a majority of the population.
  % The ``Global Climate Scheme'' (a global carbon price funding a global basic income) would be an effective and progressive way to combat climate change and poverty. Yet, such policy is mostly absent from political platforms and the policy debate. Using surveys on 40,000 respondents in 20 countries covering 72% of global CO2 emissions, we document majority support for this and other global policies. Using a complementary survey on 3,000 U.S. respondents, we test several hypotheses that could reconcile strong stated support with a lack of salience of these issues. The complementary analyses show that the stated support is mostly sincere, although we cannot rule out insincerity for 3% to 9% of the population from the willingness to sign a real-stake petition and a list experiment, respectively. Global redistributive policies rank high (though not highest) in the prioritization of policies. Conjoint analyses reveal that the Democratic party would not significantly lose votes if it endorsed the Global Climate Scheme, while a candidate at the Democratic primary would actually win votes by doing so. Accurate beliefs about the level of support for the scheme dismisses the hypothesis of pluralistic ignorance of the support. Strong universalistic attitudes are confirmed in more general questions, suggesting that the support cannot be explained away by malleable opinion or experimenter demand. In sum, our findings indicate that global policies are genuinely supported by a majority of the population. Public opinion is therefore not the reason that they do not prominently enter political debates.
\end{abstract}
% Conf submissions: AFSE, EAERE, JMA, AFEP, Earth System governance, Philo Éco
% \end{small} % NCCcomment
% TODO add results 66% willing to adopt sustainable behavior under conditions

\textbf{JEL codes:} P48, Q58, H23, Q54 % NCCcomment
% Q54 Climate • Natural Disasters and Their Management • Global Warming
% Q58 Government Policy (Q is Environmental econ)
% D78 Positive Analysis of Policy Formulation and Implementation
% H23 Externalities • Redistributive Effects • Environmental Taxes and Subsidies (H is public econ)
% P48 Political Economy • Legal Institutions • Property Rights • Natural Resources • Energy • Environment • Regional Studies (P4 is Other economic systems)
% H41 Public Goods
% H54 Infrastructures • Other Public Investment and Capital Stock

\textbf{Keywords:} Climate change, global policies, cap-and-trade, attitudes, survey.%, inequality, wealth tax. % NCCcomment

\tableofcontents

\onehalfspacing % NCCcomment

\section{Introduction}  % NCCcomment
% Inequality between countries: https://data.worldbank.org/indicator/NY.GDP.PCAP.PP.CD?contextual=default&end=2021&locations=EU-ZG-XD-XM-1W-IN-US-CD-BI-LU-CN&start=2021&view=bar (PPP current $) / https://data.worldbank.org/indicator/NY.GDP.PCAP.CD?end=2021&locations=EU-ZG-XD-XM-1W-IN-US-CD-BI-LU-CN&start=2021&view=bar (current $) / https://data.worldbank.org/indicator/NY.GDP.PCAP.PP.KD?end=2021&locations=EU-ZG-XD-XM-1W-IN-US-CD-BI-LU-CN&start=2021&view=bar (PPP 2017 $) / Pop high-income 1.2G, low-income 700M https://data.worldbank.org/indicator/SP.POP.TOTL?locations=XD-XM

% TD change the intro, e.g. Global poverty and climate change are among the most critical issues faced by the world today", and then explain that the first could be solved by transfer, the second by capping pollution, hence an effective policy to tackle these two problems is the GCS. Yet, this policy is nowhere to be seen in policy debates. Why? In this paper we provide evidence from surveys showing that people all over the world support this policy. To explain this paradox (people stated support vs absence of the policy), we further investigate the sincerity of these claims and rationales behind the support, etc.

Extreme poverty and climate change are among the most critical issues of our time. The first could be solved by redistributive transfers, the second by capping global emissions. %Indeed, transfers from high- to low-income countries are warranted by widespread ethical theories like utilitarianism,\cite{mill_utilitarianism_1861}.  
A fair and effective policy to tackle these two problems is the ``Global Climate Scheme'' (GCS), which combines these two solutions. The GCS consists of a global cap-and-trade system, where emission rights are auctioned each year to polluting firms, and of a global basic income, funded by the auction revenues. 
% I have added the following lines. TODO? put in a footnote?
Using the price and emissions trajectories from the report by \cite{stern_report_2017}, %Stern-Stiglitz report,\cite{stern_report_2017} 
we estimate that the basic income would amount to \$30 per month for each human above 15 in 2030, enough to lift out of extreme poverty the 700 million people who live with less than PPP \$2 per day. Conversely, assuming a carbon price of \$90/tCO$_\text{2}$ in 2030, high emitters like a typical American (with median U.S. CO$_\text{2}$ emissions) would lose in net \$85 per month, as they would face \$115 per month in price increases (see details in Appendix \ref{app:gain_gcs}). 

Extreme poverty is parallel to global inequality, as the GDP per capita (in 2021, in purchasing power parity) is 26 times larger in high-income countries (home to 1.2 billion people) than in low-income countries (700 million), 59 times larger in the U.S. than in the DRC, and 172 times larger in Luxembourg than in Burundi. A global 2\% tax on individual wealth in excess of \$5 million would collect \$816 billion every year, leaving unaffected 99.9\% of people.\footnote{Figures come from \citet{chancel_world_2022}, the \href{https://wid.world/world-wealth-tax-simulator/}{WID wealth tax simulator}, and the World Bank.} If 35\% of these potential revenues were allocated to low-income countries, their national income would increase by 50\%. % https://data.worldbank.org/indicator/NY.GDP.MKTP.CD?locations=XD-XM

% On top of addressing both global poverty and climate change, we provide evidence from surveys showing that people all over the world support this policy. Yet, the GCS is nowhere to be seen in policy debates. Why? To explain this paradox (absence of the policy despite majority stated support), we further investigate rationales behind the support for the GCS and the sincerity of these claims, as well as attitudes toward other global policies, global redistribution, and universalistic values. % TODO: rework

In this paper, we study attitudes toward global policies that address climate change, global poverty or inequalities, with a focus on the GCS. Using an international survey on climate attitudes, we document majority support for global policies like the GCS and a global millionaire tax in 20 among the largest countriest. Yet, such global policies are nowhere to be seen in policy debates. Why? To explain this paradox (absence of the policy despite majority stated support), we run complementary surveys on 8,000 American and European respondents and test different hypotheses: insincerity of support for the GCS, pluralistic ignorance (i.e. false belief that most do not support it), defavorable electoral outcomes for a candidate that would support it, or low priority given to global issues. Furthermore, we also study attitudes toward other global policies, global redistribution, and universalistic values.

\paragraph{Literature.} The literature review is relegated to Appendix \ref{sec:literature}. It includes references to the few other attitudinal surveys on global policies (Appendix \ref{subsubsec:literature_attitudes_policies}); a critical review of the literature on attitudes toward climate burden sharing (Appendix \ref{subsubsec:literature_attitudes_burden_sharing}); references to the large literature on attitudes toward foreign aid (Appendix \ref{subsubsec:literature_foreign_aid}); and introduction to the literatures on global carbon pricing (Appendix \ref{subsubsec:literature_pricing}), global redistribution (Appendix \ref{subsubsec:literature_redistribution}), basic income (Appendix \ref{subsubsec:literature_basic_income}), and global democracy (Appendix \ref{subsubsec:literature_democracy}). 
% Ethical theories often warrant transfers from high- to low-income people, hence from high- to low-income countries. This is the case of utilitarianism, the benchmark ethical theory used in economics. Utilitarianism assigns the same weight to each person and thus considers that a dollar is better allocated to a low-income person, which has a higher marginal utility than a high-income person.\cite{mill_utilitarianism_1861} 

% Addressing global poverty, inequalities and climate change are at the heart of the universally agreed Sustainable Development Goals (SDG). % 12 out of  17
% It has been pointed out that low-income countries generally do not have enough domestic resources to eliminate the poverty gap in the short run.\cite{bolch_arithmetics_2022} % In other words, it would hardly be possible to achieve the first SDG and end extreme poverty by 2030 without international transfers. => Careful, Bolch use a poverty line above the SDG one.

% Climate change is another issue that calls for a global response and in particular international transfers. Postulating %Assuming
% that each human has an equal right to emit CO$_\text{2}$, low emitters have a legitimate claim \textit{vis-à-vis} high emitters, that can be settled by monetary transfers. Coupling this burden-sharing principle to the carbon budget (remaining emissions that would be compatible with the Paris agreement) naturally defines a global climate policy. We call it the ``Global climate scheme'' (GCS); it consists of a global cap-and-trade system where emission rights are auctioned each year to polluting firms and the revenues finance a global basic income. Using the price and emissions trajectories from the Stern-Stiglitz report,\cite{stern_report_2017} we estimate that the basic income would amount to \$30 per month for each human above 15 in 2030, enough to lift out of extreme poverty the 700 million people who live with less than PPP \$2 per day. Conversely, high emitters like a typical American (with median U.S. CO$_\text{2}$ emissions) would lose in net \$85 per month, as they would face \$115 per month in price increases (assuming a carbon price of \$90/tCO$_\text{2}$ in 2030). % TD Give the numbers in the Results section
% % G default policy for economists, we focus on it; transfers at heart of COP; global wealth tax proposed by Piketty, Saez (fair and effective); democratisation of int'l institutions recurring topic.
% % Few studies on CC burden-sharing, all compatible with G
% % Few studies on global policies, but they show support (Ghassim, Carattini 19)
% % Here, two sets of results. First, twenty countries. Second, dig deeper using complementary survey.

% If high emitters share universalistic ethical values, we expect strong support for the GCS, even in high-income countries. On the contrary, if people defend their own financial interest, we expect low support for the GCS in high-income countries. 

% In this paper, we study attitudes toward global policies that address climate change, global poverty or inequalities, with a focus on the GCS. We measure stated support for different global policies using unpublished results from a survey\cite{dechezlepretre_fighting_2022} on climate attitudes conducted in 2021 on 40,680 respondents from 20 countries covering 72\% of global CO$_\text{2}$ emissions. We then conduct a representative survey on 3,000 U.S. respondents to study in detail the sincerity and rationales behind the support for the GCS, the attitudes toward various global policies, global redistribution, and universalistic values.

\section{Results}
% % 4 most important figures: heatmap OECD, heatmap support, prioritization or conjoint (r), list exp (table)
\subsection{Data}
\textcolor{red}{\textbf{Beware, data collection is still ongoing (we have 80\% of the final sample) so results are partial and not definitive. Please do not cite at this stage.}} \\ 
We measure stated support for different global policies using % TD better way to sell these results?
a survey on climate attitudes conducted in 2021 on 40,680 respondents from 20 countries covering 72\% of global CO$_\text{2}$ emissions (the questions of this survey on national policies are analysed in another paper: \citealp{dechezlepretre_fighting_2022}). We then conduct complementary surveys in the U.S. and four European countries to study in detail the sincerity and rationales behind the support for the GCS, the attitudes toward various global policies, global redistribution, and universalistic values. The U.S. survey has been divided in two waves: \textit{US1} and \textit{US2}, with respectively 3,000 and 2,000 respondents. The European survey, called \textit{Eu}, combines the two U.S. waves (just without one question of US2 that used results from US1). Eu survey comprises 3,000 respondents representative of France, Germany, Spain and the UK, along the dimensions of gender, income, age, highest diploma, country, and degree of urbanisation. The U.S. samples are representative along the same dimensions (with \textit{region} instead of \textit{country}) as well as along ethnicity. Tables \ref{tab:representativeness_waves}-\ref{tab:representativeness_EU} confirm that our samples closely match population frequencies. The questionnaires are given in Appendices \ref{app:questionnaire_oecd} and \ref{app:questionnaire}.

\subsection{International support}
The global survey shows strong support for climate policies enacted at the global level (Figure \ref{fig:oecd}). When asked ``At which level(s) do you think public policies to tackle climate change need to be put in place?'', 70\% (in the U.S.) to 94\% (in Japan) choose the global level. Meanwhile, the European level is chosen by less than half of the European respondents while the federal level is chosen by only 52\% of U.S. respondents. More local levels are generally chosen less than broader ones. This preference for the global level is consistent with (at least) two of the three key motives for supporting climate policies identified in the literature %:
\citep{klenert_making_2018,douenne_yellow_2022,dechezlepretre_fighting_2022}: effectiveness and fairness (the third being self-interest). % GCS more supported in Eu than global T&D. Possible explanations: quota vs. tax, climate club, better understanding / survey fatigue

\begin{figure} % TD have a simpler title (e.g. "Attitudes towards global climate policies"), and a note explaining what the figure exactly does (unit of the numbers, meaning of the colors, exclusion of indifferent people, etc.
  % MAJOR figure
  \caption[Relative support for global climate policies]{Relative support for global climate policies. \\ Share of \textit{Somewhat} or \textit{Strongly support} among non-\textit{indifferent} answers (in percent, $n$ = 40,680). The color blue denotes a relative majority. See Figure \ref{fig:oecd_absolute} for the absolute support. (Questions \ref{q:scale}-\ref{q:millionaire_tax})} 
  \makebox[\textwidth][c]{\includegraphics[width=1.2\textwidth]{../figures/OECD/Heatplot_burden_share_all_share_countries.pdf}}\label{fig:oecd}
  {\footnotesize *In Denmark, France and the U.S., the questions with an asterisk were asked differently, cf. Question \ref{q:burden_sharing_asterisk}. } 
\end{figure}

Several global policies obtain an absolute majority %more than 70\% relative %
support in all countries: ``a tax on all millionaires in dollars around the world to finance low-income countries that comply with international standards regarding climate action [which] would finance infrastructure and public services such as access to drinking water, healthcare, and education'', % TODO TD shorten
``a global democratic assembly whose role would be to draft international treaties against climate change [where] each adult across the world would have one vote to elect members of the assembly'' (though this one receives only 48\% of support in the U.S.), % we haven't yet defined absolute support at this stage
and an international emissions trading system where ``countries that emit more than their national share would pay a fee to countries that emit less than their share''. 
In high-income countries, this global quota obtains 64\% of absolute (i.e. \textit{somewhat} or \textit{strong}) support and 84\% of relative support (i.e. excluding \textit{indifferent} answers). The support is even higher in middle-income countries, though one should interpret the results with caution in middle-income countries as their samples are only representative of the online population (young, graduated and urban people are over-represented). 
After the support for the global quota, we ask how the carbon budget should be divided among countries. 
Consistent with the literature (see Appendix \ref{subsubsec:literature_attitudes_burden_sharing}), the preferred burden-sharing rule is to allocate the rights to emit on an equal per capita basis: this fairness principle secures an absolute majority support in all countries, and a relative majority support never below 84\%. 
Taking into account historical responsibilities or vulnerability to climate damages is also popular, though less consensual, while grand-fathering (i.e. allocating emission shares in proportion to current emissions) comes last everywhere. 
The Global Climate Scheme, i.e. a global quota where emission rights are allocated on an equal per capita basis, has the same distributive effects as a global carbon tax that would fund a global basic income. The support for this policy is also tested, and there the redistributive effects are specified to the respondents: the \$30 per month basic income would lift the 700 million people who earn less than \$2/day out of extreme poverty, and fossil price increases would cost the typical person in their country a certain amount (that is provided).  % The average British person would lose a bit from this policy as they would face £42 per month in price increases, which is higher that the £22 they would receive.
Despite their similarity, the global tax is less supported than the global quota, and it even fails to obtain a relative majority in Anglo-saxon countries. This lower support is likely due to the facts that distributive effects are made salient in the case of the tax, and that people may find a quota more effective than a tax to reduce emissions. This interpretation is consistent with the level of support for the global quota once we make the distributive effects salient, which we do in the complementary surveys. % though we cannot exclude that people find a quota more effective than a tax to reduce emissions. 


\subsection{Stated support for various policies}
% H0: Majority support for each global policies except maximum wealth and debt cancellation
The remainder of the paper analyzes the results from the complementary surveys in the U.S. and Eu.

\subsubsection{Global Climate Scheme} % NCCcomment
In the complementary surveys, we describe the Global Climate Scheme, explain its distributive effects (specifying the amounts at stake, cf. Appendix \ref{app:gain_gcs}), test the understanding that typical people would lose in high-income countries and that the poorest humans would win using an incentivized question, and then give the correct answer. We proceed the same way for a National Redistribution Scheme (NR) that would tax the top 5\% (in the U.S.) or the top 1\% (in Europe) to finance cash transfers to all adults (calibrated to offset the monetary loss of the GCS for the median emitter), expecting people to find out at the comprehension question that the richest would lose and the typical people in their country would win. Then, we summarize both schemes to make sure that the respondents remember them. Right after, we ask a last incentivized question of comprehension, and later give the expected answer that a typical fellow citizen would neither win nor lose with the GCS and NR combined. Finally, we directly ask the support for the GCS and for NR in a simple \textit{Yes}/\textit{No} question. The stated support for the GCS is at 54\% in US1 and 76\% in Eu, and the support for NR is very close, at 56 and 73\%, respectively (Figure \ref{fig:support_binary}).% TD add something if equality remains

\begin{figure}[h!]
    \caption[Support for the Global Climate Scheme]{Support for the GCS, NR and the combination of GCS, NR and C. (Questions \ref{q:gcs_support}, \ref{q:nr_support} and \ref{q:crg_support}; $n_\text{US} = n_\text{Eu} = 3,000,\, n_\text{FR} = 729,\, n_\text{DE} = 929,\, n_\text{ES} = 543,\, n_\text{UK} = 749$)}\label{fig:support_binary}
    \makebox[\textwidth][c]{\includegraphics[width=.9\textwidth]{../figures/country_comparison/support_binary_positive.pdf}} 
\end{figure}

\subsubsection{Global wealth tax} 
Consistent with the global survey's results, a ``tax on millionaires of all countries to finance low-income countries'' receives an absolute majority support of more than 67\% in every country, only 5 p.p. lower than a national millionaires tax overall (Figure \ref{fig:support}). % us2!
To random subsamples, we ask respondents how should the revenues of a global tax on individual wealth in excess of \$5 million be shared with low-income countries, after describing how much revenues such a tax would raise in their country vs. in low-income countries.\footnote{Namely, a 2\% tax on net wealth above \$5 million would raise each year \euro{}5 billion in Spain, \euro{}16 billion in France, £20 billion in the UK, \euro{}44 billion in Germany, \$430 billion in the U.S., and \$1 billion in all low-income countries taken together (28 countries, home to 700 million people).} To some respondents ($n$ = 891), we ask what percentage of the global tax revenues should be pooled to finance low-income countries. In each country, at least 88\% of respondents answer a positive amount, with an overall average of 30\% (Germany) to 36\% (U.S., France) (Figure \ref{fig:global_share_mean}). To some other respondents ($n$ = 859), we ask whether they would prefer that each country retains the revenues it collected or that half of the revenues be pooled to finance low-income countries. About half of the people prefer to channel half of the tax revenues to low-income countries. 

\begin{figure}
    \centering 
    \caption[Preferred share of wealth tax for low-income countries]{Percent of global wealth tax that should finance low-income countries (\textit{mean}). (Question \ref{q:global_tax_global_share})} % TODO! n
    \includegraphics[width=1\textwidth]{../figures/country_comparison/global_tax_global_share_mean.pdf} \label{fig:global_share_mean}
\end{figure}

\subsubsection{Other global policies} % NCCcomment
We also test support for other %more realistic
global policies (Figure \ref{fig:support}). All receive relative majority support in each country except two:  the ``cancellation of low-income countries' public debt'' and ``a maximum wealth limit'' (the latter obtains relative majority support in Eu but not in the U.S., despite the cap being defined as \$10 billion the U.S. instead of \euro{}/£100 million in Eu). Climate-related policies are particularly popular: ``high-income countries funding renewable energy in low-income countries'' obtains absolute majority support everywhere while loss and damages compensation (which was approved at the COP27) receives a relative support of 55\% (U.S.) to 81\% (Spain).

\begin{figure}
  % MAJOR figure
  \caption[Relative support for various global policies]{Relative support for various global policies (percentage of \textit{somewhat} or \textit{strong support}, after excluding \textit{indifferent} answers). (Questions \ref{q:climate_policies} and \ref{q:q:other_policies}; See Figure \ref{fig:support_likert_positive} for the absolute support.) $n_\text{US} = n_\text{Eu} = 3,000,\, n_\text{FR} = 729,\, n_\text{DE} = 929,\, n_\text{ES} = 543,\, n_\text{UK} = 749, n_\text{US, global/national wealth tax} = 2,000$} % ; $n$ = 6,000
  \makebox[\textwidth][c]{\includegraphics[width=\textwidth]{../figures/country_comparison/support_likert_share.pdf}}\label{fig:support}
\end{figure} 

% H0: Foreign aid: less than 20\% want a decrease (because nationalist), median wants increase at some conditions (no diversion, human rights) => GCS mostly addresses these points
\subsubsection{Foreign aid} % NCCcomment
%eu! 
After providing the amount %explaining that ``0.4\% of U.S. government spending (that is, 0.2\% of U.S. GDP) is 
``spent on foreign aid to reduce poverty in low-income countries'' (in proportion to their country's government spending and GDP), less than 16\% state that their country's foreign aid should be reduced while 62\% state that it should be increased, including 17\% who support an unconditional increase (Figure \ref{fig:foreign_aid_raise_support}). To the 45\% who answer that aid should be increased but only if some conditions are respected, we later ask them what condition(s) should be required (Figure \ref{fig:foreign_aid_condition}). The three conditions most chosen are all largely respected by the Global Climate Scheme: ``that we can be sure the aid reaches people in need and money is not diverted'' (chosen by 73\%), ``that recipient countries comply with climate targets and human rights'' (67\%), and ``that other high-income countries also increase their foreign aid'' (48\%). %, we propose different conditions. The most chosen condition (by 74\%) is  and the second most (by 59\%) ``That recipient countries comply with climate targets and human rights''. 
On the other side, not wishing to increase their country's foreign aid is mostly justified by prioritizing one's fellow citizens or viewing each country as responsible for its own fate (Figure \ref{fig:foreign_aid_no}). 

\begin{figure}[h!]
  \caption[Attitudes on the evolution of foreign aid]{Attitudes regarding the evolution of [own country] foreign aid. (Question \ref{q:foreign_aid_raise_support})}\label{fig:foreign_aid_raise_support}
  \makebox[\textwidth][c]{\includegraphics[width=\textwidth]{../figures/country_comparison/foreign_aid_raise_support.pdf}} 
\end{figure}

\begin{figure}[h!]
  \caption[Conditions at which foreign aid should be increased]{Conditions at which foreign aid should be increased (in percent). [Asked to those who wish an increase of foreign aid at some conditions.] (Question \ref{q:foreign_aid_condition})}\label{fig:foreign_aid_condition}
  \makebox[\textwidth][c]{\includegraphics[width=\textwidth]{../figures/country_comparison/foreign_aid_condition_positive.pdf}} 
\end{figure}

\begin{figure}[h!]
  \caption[Reasons why foreign aid should not be increased]{Reasons why foreign aid should not be increased (in percent). [Asked to those who wish a decrease or stability of foreign aid.] (Question \ref{q:foreign_aid_no})}\label{fig:foreign_aid_no}
  \makebox[\textwidth][c]{\includegraphics[width=\textwidth]{../figures/country_comparison/foreign_aid_no_positive.pdf}} 
\end{figure}

\subsection{Sincerity of support}

We use several methods to assess the sincerity of the support for the Global Climate Scheme: a list experiment, a real-stake petition, conjoint analyses, and the prioritization of policies. All methods suggest that the support is either completely sincere, or the share of insincere answers is limited. 

\subsubsection{List experiment}\label{subsubsec:list_exp}  % NCCcomment
% H1: List experiment: There seems to be a 8pp social norm (differential of 3pp with NR). No effect of the number of options. TODO: check literature

By asking \textit{how many} policies within a list respondents support and varying the list among respondents, a list experiment allows identifying the tacit support for a policy of interest. The tacit support is estimated as the difference in the average number of policies supported between two groups, whose list differ only by the inclusion (or not) of that policy \citep{hainmueller_causal_2014}.% respondents who face a list containing the policy, and respondents who face the same list without it. 
List experiments have been used to reveal a social desirability bias that silences racism in Southern U.S. \citep{kuklinski_racial_1997} or the opposition to the invasion of Ukraine in Russia \citep{chapkovski_solid_2022}. % TODO? remove?
As shown in Table \ref{tab:list_exp}, the tacit support for the GCS measured through the list experiment is not significantly lower than the direct stated support. Thus, we cannot reject an absence of social desirability bias in our case.
% The tacit support for the GCS measured through the list experiment is as high as  the direct question in Eu but significantly lower by 5 p.p. in the U.S. This may be the sign of a social norm pushing some Americans to state that they support the GCS although they secretly do not. Still, if there is a social norm in favor of the GCS, there is a similar norm in favor of the National Redistribution Scheme, as the gap between the tacit and direct support for it is comparable (at 6 p.p.). %However, two observations qualify this interpretation. First, the gap between the tacit and direct support for the National Redistribution Scheme is comparable (at 7 p.p.) though we did not expect such a social norm in the case of the national redistribution, as the 95\% who would benefit from it should not feel ashamed to oppose a policy that would benefit them. Second, while we tested the questionnaire on random people in cafés, we noticed that some were confused by the question of the list experiment (asking how many policies from the list they supported), upset with the conservative societal policy (``Marriage only for opposite-sex couples in the U.S.'', ``Death penalty for major crimes'' in Europe), to the point that they did not answer attentively.

\begin{table}[h]
  % MAJOR figure % TODO! same table for NR in appendix
  % TODO table by country
  \caption[List experiment: tacit support for the GCS]{Number of supported policies in the list experiment depending on the presence of the Global Climate Scheme (GCS) in the list.%in function of the composition of the list. GCS stands for the Global Climate Scheme and NR for the National Redistribution Scheme.} % Beware, this question is quite unusual. \\ Among the policies below, how many do you support?  \\ Coal exit, Marriage only for opposite-sex couples 
  }\label{tab:list_exp}
  \makebox[\textwidth][c]{
\begin{tabular}{@{\extracolsep{5pt}}lccc} 
\\[-1.8ex]\hline 
\hline \\[-1.8ex] 
 & \multicolumn{3}{c}{Number of supported policies} \\ 
\cline{2-4} 
\\[-1.8ex] & All & US & Eu \\ 
\hline \\[-1.8ex] 
 List contains: GCS & 0.624$^{***}$ & 0.524$^{***}$ & 0.724$^{***}$ \\ 
  & (0.028) & (0.041) & (0.036) \\ 
\hline  \\[-1.8ex] \textit{Support for GCS} & 0.617  &  0.542  &  0.757 \\
\textit{Social desirability bias} & \textit{$ -0.026 $} & \textit{$ -0.018 $} & \textit{$ -0.033 $}\\
\textit{80\% C.I. for the bias} & \textit{ $[ -0.06 ; 0.01 ]$ } & \textit{ $[ -0.07 ; 0.01 ]$} & \textit{ $[ -0.08 ; 0.01 ]$}\\
 \hline \\[-1.8ex] 
Constant & 1.317 & 1.147 & 1.486 \\ 
Observations & 6,000 & 3,000 & 3,000 \\ 
R$^{2}$ & 0.089 & 0.065 & 0.125 \\ 
\hline 
\hline \\[-1.8ex] 
\textit{Note:}  & \multicolumn{3}{r}{$^{*}$p$<$0.1; $^{**}$p$<$0.05; $^{***}$p$<$0.01} \\ 
\end{tabular} 
  }
  % {\footnotesize \textit{Note:} $^{*}p<0.1$; $^{**} p<0.05$; $^{***} p<0.01$.}
\end{table}

% Donation addresses experimenter demand
\subsubsection{Petition}\label{subsubsec:petition} % Addresses hypothetical bias  % NCCcomment
% H1: Petition: Small effect against GCS: -4pp
We ask the respondents whether they are willing to sign a petition (either for the GCS or NR) and inform them that ``we will send the results to the [head of state]'s office, informing [them] what share of [fellow citizens] are willing to endorse the [Global Climate / National Redistribution] Scheme''. Both policies still obtain majority support when the support is framed as a real-stake petition. In the U.S., we cannot reject equality between the support in the real-stake petitions and the simple questions (GCS: $p=.30$; NR: $p=.76$).\footnote{We run paired weighted \textit{t}-tests, i.e. we test the equality in support for a policy on the subsample of respondents who are questioned about this policy for the petition.} In Eu, the support is significantly lower in the petition, by 7 p.p. for the GCS ($p=10^{-5}$) and 4 p.p. for NR ($p=.008$). Although a few Europeans are not willing to sign a petition for a policy they are supposed to support, this affects both policies tested, and the willingness to sign a real-stake petition is still strong: 69\% for the GCS and 67\% for NR.

% TODO! all '$n$ = '
\subsubsection{Conjoint analyses}\label{subsubsec:conjoint} % Addresses acquiescence bias  % NCCcomment
% H1, H2: Conjoint analysis: G|C+R 56%, G|R 59%, G 48% ~ C (|R), G+C|R 56%, C|R 64%, Left+G - Left = -3pp, A+G vs. B 59%
% => G is supported for itself, rather independently from R or C, with similar support to both, and it doesn't significantly penalize the Left, and would help a Democratic candidate
In our \textit{conjoint analyses}, we ask respondents to make five choices between pairs of political platforms. The first conjoint analysis suggests that the GCS is supported for itself, independently of being complemented by the National Redistribution Scheme and a national climate policy (``Coal exit'' in the U.S., ``Thermal insulation plan'' in Eu, denoted C). Indeed, 54\% of %($n$ = 3,000) 
U.S. respondents and 74\% of %($n$ = 3,000) 
Eu ones prefer the combination of C, NR and the GCS to the combination of C and NR alone, indicating a similar support for the GCS conditional on NR and C than for the GCS alone (Figure \ref{fig:conjoint}). % (as it does not significantly differ from the direct support of 53\%). 
For the second analysis, we split the sample into four random branches. Results from the first branch show that the support for the GCS conditional on NR, at 55\% in the U.S. ($n$ = 757) and 77\% in Eu ($n$ = 746), is not significantly different from the support for the GCS alone. This suggests that rejection to the GCS is not driven by the cost of the policy on oneself. The second branch shows that the support for C conditional on NR is somewhat higher, at 62\% in the U.S. ($n$ = 751) and 84\% in Eu ($n$ = 747). However, the third one shows no significant preference for C compared to GCS (both conditional on NR), neither in Eu, where GCS is preferred by 52\% ($n$ = 741) nor in the U.S., where C is preferred by 53\% ($n$ = 721). The fourth branch shows that 55\% in the U.S. ($n$ = 771) and 77\% in Eu ($n$ = 766) prefer the combination of C, NR and the GCS to NR alone. % TODO? figure?

In other words, there is majority support for the GCS and for a national climate policy C, which are seen as neither complement nor substitute; a few people seem to like a national climate policy and dislike a global one, but as many people prefer a global rather than a national policy; and there is no evidence that a National Redistribution would increase the support for the GCS. % TD rework this paragraph

In the third analysis, we present to two random branches of the sample hypothetical progressive and conservative platforms that differ only by the presence (or not) of the GCS in the progressive platform. Table \ref{tab:conjoint_c} shows that a progressive candidate would not significantly lose voting share by endorsing the GCS in any country, and may even gain 11 p.p. ($p = .005$) in voting intention in France. Although not significant at the 5\% threshold, the effect is also positive at 3 p.p. ($p = .13$) in the U.S. % France holds multiple hypotheses testing

%The third analysis suggests that a progressive candidate would not significantly lose voting share if he or she were to endorse the GCS, and that he or she may even gain 11 p.p. vote intention in France (see Table \ref{tab:conjoint_c}). To estimate this, we present to two random branches of the sample hypothetical progressive and conservative platforms that differ only by the presence (or not) of the GCS in the progressive platform. 

\begin{table}[h]
  % MAJOR figure
  \caption[Influence of the GCS on electoral prospects]{Preference for a progressive platform depending on whether it includes the GCS or not. (Question \ref{q:conjoint_c}) 
    %Imagine if the [Democratic and Republican presidential candidates in 2024] campaigned with the following policies in their platforms. [Credible Progressive and Conservative platforms] \\ % TODO See More
  % Which of these candidates would you vote for? \textit{A; B; None of them} \\
  % ~[FR: second round of presidential; DE, ES, UK: two favorite candidates in one's constituency]
  } % Beware, this question is quite unusual. \\ Among the policies below, how many do you support?  \\ Coal exit, Marriage only for opposite-sex couples 
  \makebox[\textwidth][c]{
\begin{tabular}{@{\extracolsep{5pt}}lcccccc} 
\\[-1.8ex]\hline 
\hline \\[-1.8ex] 
 & \multicolumn{6}{c}{Prefers the Progressive platform} \\ 
\cline{2-7} 
\\[-1.8ex] & All & United States & France & Germany & UK & Spain \\ 
\hline \\[-1.8ex] 
 GCS in Progressive platform & 0.028$^{*}$ & 0.029 & 0.112$^{***}$ & 0.015 & 0.008 & $-$0.015 \\ 
  & (0.014) & (0.022) & (0.041) & (0.033) & (0.040) & (0.038) \\ 
 \hline \\[-1.8ex] 
Constant & 0.623 & 0.604 & 0.55 & 0.7 & 0.551 & 0.775 \\ 
Observations & 5,202 & 2,619 & 605 & 813 & 661 & 504 \\ 
R$^{2}$ & 0.001 & 0.001 & 0.013 & 0.0003 & 0.0001 & 0.0003 \\ 
\hline 
\hline \\[-1.8ex] 
\end{tabular} 
}\label{tab:conjoint_c}
  {\footnotesize \textit{Note:} The 14\% of \textit{None of them} answers have been excluded from the regression samples. GCS has no significant influence on them. $^{*}p<0.1$; $^{**} p<0.05$; $^{***} p<0.01$. 
  }
\end{table}

Our last two analyses (four and five) make people choose between two random platforms. In Eu, respondents are prompted to imagine that a left- or center-left coalition will win the next election and are asked what platform they would prefer that coalition to have campaigned on. In the U.S., the question is framed as a hypothetical duel in a Democratic primary, and asked only to non-Republicans ($n$ = 2,218), i.e. the respondents who choose \textit{Democrat}, \textit{Independent}, \textit{Non-Affiliated} or \textit{Other} for their political affiliation. In the fourth analysis, a policy (or an absence of policy) is randomly drawn for each platform in each of five categories: \textit{economic issues}, \textit{societal issues}, \textit{climate policy}, \textit{tax system}, \textit{foreign policy} (Figure \ref{fig:ca_r}). 
Except for the category \textit{foreign policy}, which features the GCS 42\% of the time, the policies are prominent progressive policies and they are drawn uniformly. % except for tax1: .35 vs. tax2: .4 in EU. 
In the UK, Germany, and France, a platform is about 9 to 13 p.p. more likely to be preferred if it includes the GCS rather than no foreign policy.\footnote{This is the Average Marginal Component Effect computed following \citet{hainmueller_causal_2014}.} This effect is between 1 and 4 p.p. and no longer significant in the U.S. and in Spain. 
Moreover, a platform that includes a global tax on millionaires rather that no foreign policy is 9 to 13 p.p. more likely to be preferred in all countries but Spain (not significant, at +5 p.p.). Likewise, a global democratic assembly on climate change has a significant effect of 8 to 12 p.p. in the U.S., Germany, and France. 
%In each country, a platform is more likely to be preferred if it includes the GCS rather than no foreign policy. This effect is significant in France, Germany and the UK, where a platform is about 10 p.p. more likely to be preferred. 
These effects are large, and not far from the effects of the policies most influential on the platforms, which range between 15 and 18 p.p. in most countries (and 27 p.p. in Spain), and all relate to improved public services (in particular healthcare, housing and education). 

\begin{figure}[h!] 
    \caption[Preferences for various policies in political platforms]{Effects of the presence of a policy (rather than none from this domain) in a random platform on the likelihood that it is preferred to another random platform. (Question \ref{q:conjoint_r}%; in the U.S., asked only to non-Republicans.
    )}\label{fig:ca_r}
    \begin{subfigure}{\textwidth}
      \subcaption{U.S. (Asked only to non-Republicans)}
      \includegraphics[width=\textwidth]{../figures/US1/ca_r.png}
    \end{subfigure}
    \begin{subfigure}{\textwidth}
      \subcaption{France}
      \includegraphics[width=\textwidth]{../figures/FR/ca_r.png}
    \end{subfigure}
\end{figure}%
\clearpage
\begin{figure}[h!]\ContinuedFloat % if bugs try b! instead of h!
    \begin{subfigure}{\textwidth}
      \subcaption{Germany}
      \includegraphics[width=\textwidth]{../figures/DE/ca_r.png}
    \end{subfigure}
    \begin{subfigure}{\textwidth}
      \subcaption{Spain}
      \includegraphics[width=\textwidth]{../figures/ES/ca_r.png}
    \end{subfigure}
    \begin{subfigure}{\textwidth}
      \subcaption{UK}
      \includegraphics[width=\textwidth]{../figures/UK/ca_r.png}
    \end{subfigure}
    %\makebox[\textwidth][c]{} 
\end{figure}
\clearpage

The fifth analysis draws random platforms in a similar ways, except that candidate A's platform always contains the GCS while B's includes no foreign policy. In this case, A is chosen by 60\% in Eu %($n$ = 3,000) 
and 58\% in the U.S. (Figure \ref{fig:conjoint_left_ag_b}). %($n$ = 2,218). 
In the U.S. for example, our conjoint analyses indicate that a candidate at the Democratic primary would have more chances to obtain the nomination by endorsing the GCS, and this endorsement would not penalize her or him at the presidential election. This result reminds the finding that 12\% of Germans shift their voting intention from SPD and CDU/CSU to the Greens and the Left when they are told that the latter parties support global democracy \citep{ghassim_who_2020}.

\begin{figure}[h!]
    \caption[Influence of the GCS on preferred platform]{Influence of the GCS on preferred platform:\\ Preference for a random platform A that contains the Global Climate Scheme rather than a platform B that does not (in percent). (Question \ref{q:conjoint_d}; in the U.S., asked only to non-Republicans.)}\label{fig:conjoint_left_ag_b}
    \makebox[\textwidth][c]{\includegraphics[width=\textwidth]{../figures/country_comparison/conjoint_left_ag_b_binary_positive.pdf}} 
\end{figure}

% \begin{figure}
%   % Imagine that at the 2024 Democratic party presidential primaries, the two main candidates campaign with the following key policies in their platforms. \\ Which of these candidates do you prefer?

%   \caption{Conjoint analysis. Average Marginal Component Effects (relative to the baseline: an absence of policy of that category) of policies in the choice between two platforms, where policies in each platform are randomly drawn ($n$ = 6,000). In Eu, it is framed as two potential platforms of a left-wing coalition that would win the next elections; in the U.S., it is framed as a hypothetical duel in the 2024 Democratic primary and asked only to non-Republicans.}\label{fig:ca_r} % TODO: add ref to Question
%   \makebox[\textwidth][c]{\includegraphics[width=\textwidth]{../figures/all/ca_r.png}}
% \end{figure}

\subsubsection{Prioritization} % Addresses acquiescence bias and social desirability bias
% H1: Prioritization: G has mean only slightly lower than average, makes better than ban of cars and coal exit; global tax on millionaires does as well as wealth tax and almost as good as $15 minimum wage
At the end of the survey, we pick six policies at random (and uniformly) among the policies used in the last conjoint analyses, and ask respondents to allocate 100 points among them (using sliders), with the instruction that ``the more you give points to a policy, the more you support it''. For each policy presented, the average support is thus 16.67 points. %(Figure \ref{fig:points}). % TODO! figures for each country
In each country, the GCS ranks in the middle of all policies or at a higher rank, with an average number of points from 15.4 (U.S.) to 22.9 (Germany). 
In Germany, the the global tax on millionaires is the most prioritized policy and  the GCS the second most. The global tax on millionaires ranks at worst in fifth position in every country, and receives an average number of points from 18.3 (Spain) to 22.9 (Germany). 
This question further reveals the potential mismatch between policies prioritized by the public and those enacted by legislators. Indeed, to ``ban the sale of new combustion-engine cars by 2030'' is one of the three least prioritized policies in each country, with an average of 7.8 (France) to 11.4 (UK), and yet the European Union and California have enacted to phase out %a target of no 
new combustion-engine cars by 2035. 
%It is higher than to ``ban the sale of new combustion-engine cars by 2030'' (13.4) and ``coal exit'' (10.0), but lower than the third climate policy: ``trillion dollar investment in clean transportation infrastructure and building insulation'' (20.3). The support for other globally redistributive policies is variable: ``Doubling foreign aid'' is the least supported policy (8.4), while the ``Global tax on millionaires'' is one of the five policies with more than 20 points (20.2), and the ``global democratic assembly on climate change'' is just below the GCS (14.5). The most supported policies are ``Funding affordable housing'' (28.5), ``\$15 minimum wage'' (23.8), and ``Universal childcare/pre-K'' (22.1). % TODO? share that allocated at least 1

% \begin{figure}[h!]
%   \caption{Prioritization of policies. Each respondent faces six policies taken at random from the ones below and allocates 100 points among them to signal the strength of their support for each one ($n$ = 3,000).} % Imagine you have 100 points that you can allocate to different policies. The more you give points to a policy, the more you support it.  \\  How do you allocate the points among the following policies?  
  
%   \makebox[\textwidth][c]{\includegraphics[width=\textwidth]{../figures/US1/points_us.pdf}}\label{fig:points}
% \end{figure}


\subsection{Second-order beliefs}
% H3 belief: No pluralistic ignorance
To explain a strong support for the GCS despite its absence from political platforms and the public debate, we hypothesized pluralistic ignorance, i.e. that most people and policy-makers wrongly perceive the GCS as unpopular. People would then hide their support for such globally redistributive policies, knowing that advocating for them would be vain. We find limited evidence for pluralistic ignorance in an incentivized question on the perceived support (Figure \ref{fig:belief}). Americans have quite accurate beliefs regarding the level of support for the GCS. Indeed, the mean (resp. quartiles) perceived support is 52\% (resp. 36\%, 52\%, 68\%%, $n$ = 3,000
) vs. an actual support of 53\%. Europeans underestimate the support by 17 p.p., but 65\% of them correctly guess that the GCS obtains a majority (mean of 59\% and quartiles of 43\%, 61\%, 74\% vs. an actual support of 76\%). For the record, the second-order beliefs are equally accurate for the National Redistribution Scheme in the U.S., and equally underestimated in Eu.%, with mean (resp. quartiles) perceived support of 54.7\% (resp. 40, 55, 71\%, $n$ = 3,000) vs. 56\%.

\begin{figure}[h!]
    \caption[Beliefs about support for the GCS and NR]{Beliefs regarding the support for the GCS and NR. (Questions \ref{q:gcs_belief} and \ref{q:nr_belief})}\label{fig:belief}
    \makebox[\textwidth][c]{\includegraphics[width=.7\textwidth]{../figures/country_comparison/belief_all_mean.pdf}} 
\end{figure}

\subsection{Universalistic values}
% H4: A strong majority is universalist/cosmopolitan (TODO: which word?), even a majority for non-Republican
% TD It is not obvious how these answers are informative of malleable opinions. So I don't think we should state the hypothesis and sell this as a test.
%Another hypothesis to explain the discrepancy between the lack of interest for global policies in the public debate despite a strong stated support is that opinions on the topic are weak and malleable. A way to test this is to
We ask broad questions on people's values to see whether their core values are consistent with universalism. When asked what group they defend when they vote% ($n$ = 6,000)
, 20\% choose ``sentient beings (humans and animals)'', 22\% ``humans'', 33\% their fellow citizens (or ``Europeans''), 15\% ``My family and myself'', and the rest (10\%) choose another group (mostly ``My State or region'' or ``People sharing my culture or religion''). The first two categories can be described as universalist, and they represent close to one out of two people. The share of universalist even constitutes a majority of left-wing voters. % TODO check
When asked what should their country's diplomats defend in international climate negotiations, only 11\% prefer their country's ``interests, even if it goes against global justice''; 30\% prefer global justice (mitigated or not by national interests) and the bulk of respondents (38\%) prefer their country's ``interests, to the extent it respects global justice''. % ($n$ = 6,000). 
Furthermore, when asked to judge the extent to which climate change, global poverty, and inequality in their country are an issue, climate change is generally viewed as the biggest problem (with a mean of 0.59 once we recode answers between $-2$ and $2$), followed by global poverty (0.42) and national inequality (0.37). %, $n$ = 6,000). 
% TODO! heatmap with universalist, nationalist, egoist, only country's interest, country's interest respecting global, mostly global justice, problem CC, pov, ineq
Finally, we elicit unversalistic values through a lottery experiment. We automtically enroll the respondents in a lottery with one \$100 prize. Respondents have to choose which share of the prize to keep for themself vs. give to a person living in poverty. The charity donation is destined either for an African or a fellow citizen, depending on the respondent's random branch. We observe no significant variation in the willingness to donate in function of the recipient's origin in Eu, and a donation lower by 3 p.p. for the African in the U.S. (the average donation is 34\%). Moreover, the slightly lower donations to Africans are entirely driven by right-wing voters or non-voters. 
Overall, answers to these broad value questions are consistent with half of Americans and three quarters of Europeans supporting global policies like the GCS: people are almost as much willing to give to poor Africans than to poor fellow citizens, find that global issues are among the biggest problems, almost half of them are universalist when they vote, and most of them wish that their diplomats take into account global justice.


\section{Discussion} % Summary, conclusion
In 20 among the largest countries, we find strong majority support for global climate policies, even in high-income countries that would financially lose from the globally redistributive policies that we test. The complementary surveys in the U.S. and four European countries confirm these results. For example, there is a strong support for global taxes on the wealthiest, and majority support for our main policy of interest, a Global Climate Scheme that would establish both carbon pricing at the global level through an emissions trading system, and a global basic income funded by its revenues. A list experiment and a real-stake petition show that the support for the GCS is mostly sincere. This genuine support is confirmed by the prioritization of this global climate policy above some prominent national climate policies, and consistent with close to half of the population holding universalistic (rather than nationalistic or egoistic) values. Moreover, the conjoint analyses reveals that a progressive candidate should not lose voting shares by endorsing the GCS, and should even get a voting share 11 p.p. higher in France. Likewise, a candidate endorsing the GCS would win votes at a U.S. Democratic primary, while in Europe, a progressive platform including the GCS would be preferred to a platform not including it. %, and would even win votes at the Democratic primary by doing so. 
Besides a potential lack of sincerity or weak opinions, we dismiss another hypothesis that could have explained the scarcity of global policies in the public debate despite a strong support: that people underestimate the support of their fellow citizens. As we ruled out all hypotheses of our registration plan,\footnote{The project was preregistered in the Open Science Foundation registry (\href{https://osf.io/fy6gd}{osf.io/fy6gd}).} we now need to study new explanations. %formulate new hypotheses.

% LM Similarly to Thomas' comment, I had to read this three times before getting it. I had the other idea that it might help to link to the pre-analysis plan here. That would help me understand what is implied. 
We see four potential explanations for the scarce mention of globally redistributive policies in the public debate and political platforms. Among the new hypotheses, the first two are variations of pluralistic ignorance, and the last two represent complementary (rather than substitute) explanations. First, there may be pluralistic ignorance of univeralistic values, of the support for the GCS, or of the electoral advantage of endorsing it \textit{among policy makers}. Second, people or policy makers may believe that globally redistributive policies are politically infeasible in some key (potentially foreign) countries like the U.S.  We intend to test these hypotheses by running a survey on %Congress staffers and 
Members of the European Parliament. 
%Second, there may be a more subtle form of pluralistic ignorance: although most people correctly predict what people would answer to a survey question, they may view globally redistributive policies as unrealistic, perhaps because they have never reflected upon the fact that many people across the world hold univeralistic values and are supportive of global solidarity. Third, most people and perhaps even most policy makers may have simply never heard of the GCS, let alone built their political ideas upon it. 
Third, most institutions are national: the largest scale votes take place at the national level% so political platforms are devised at this level
, most media target a national audience, most commentators frame their discourse from a national perspective and portray relations to foreign countries as conflictual. The prominence of national institutions may create a nationalistic bias in political thoughts, silencing people's univeralistic values. 
% unrealistic, never heard of, technical implementation
Fourth, most people and perhaps even most policy makers may have simply never heard of specific global redistributive policies, %like the GCS, 
let alone built their political ideas upon it. Being unaware of prominent global policy proposals, people or policy makers would cautiously doubt that they are well-specified or technically implementable, and would therefore dismiss them as unrealistic. % TODO? rewrite as a fifth hypothesis that the policies may indeed not be well specified nor implementable?
The ignorance of the GCS itself seems supported by the feedback fields, where the most common answer is a variation upon ``thank you for this interesting, thought-provoking survey''. % doesn't apply to millionaire tax
% TODO! add swing state analysis
% TODO! add reweighted estimate

% Also: vested interest influence on elites (a la Gilens and Page for the US), cf. Boone story
% decline of support the more specific a measure gets / enters the public debate
% people not believing that the policy is technically implementable
% politicians seeing defects or lack of specification of the policy that warrant opposition, while ignorant citizens would naively like it

% LM This are all legtimiate and important points. For a global transfer (rather than say a EU transfer), I am really missing remarks about (lack) of quality of governance, rent creation and capture, corruption etc. To play devil's advocate, as much as I am sympathetic to the idea, I'm not sure I would currently vote in favor of such policies. Why? Not because I dislike the idea, but I am not sure that my carbon tax rent as a German would actually reach the poor population of Indonesia or Nigeria rather than the pockets of cleptocratic elites :) I'd like the draft to reflect that and could make an attempt to pre-empt that objection based on some references later in the project.
%In any case, 
If any (or several) of the remaining hypotheses is confirmed by evidence, we could draw the same conclusion: % TD But what if they are not? I don't think that would invalidate the conclusion stated below.
There is a strong support for global policies that address climate change and global inequality, even in high-income countries, and the frontier of what is considered politically realistic might soon shift on this issue. Uncovering evidence for this might actually contribute itself to garner more attention to global policies in the public debate. % and political platforms. % TD I don't like this because it sounds a bit like asking editors to publish our work. => changed "Publishing" to "Uncovering"

% \begin{methods}  % WPcomment
\begin{small} % NCCcomment
%Put methods in here.  If you are going to subsection it, use \subsection commands.  Methods section should be less than 800 words and if it is less than 200 words, it can be incorporated into the main text.
\section*{\normalsize Methods}\label{sec:methods} % NCCcomment
\addcontentsline{toc}{section}{\nameref{sec:methods}}
% \subsection{Data collection.} % WPcomment
\paragraph{\small Data collection.} % NCCcomment
The paper relies on two different sets of surveys. The first set of surveys was conducted between March 2021 and March 2022 on 40,680 respondents from 20 countries  (between 1,465 and 2,488 respondents per country). The first U.S. complementary, denoted US1, was conducted on 3,000 U.S. respondents between January and March 2023. The Eu complementary survey is conducted on 3,000 respondents between February and March 2023 (data collection is still ongoing on March 23, as $n$ = 2,979). The second U.S. complementary has started in March 2023, with an expected sample size of 2,000. 
We used the survey companies \emph{Dynata} and \emph{Respondi}. Stratified quotas ensure that the samples are representative along the dimensions of gender, age (5 brackets), income (4), region (4), education level (3), as well as ethnicity (3) for the U.S. % TODO and urbanity instead of education for OECD, TODO table representativeness
To correct for small remaining imbalances, we apply survey weights throughout the analysis, constructed using the quotas variables as well as the degree of urbanity, and trimmed between 0.25 and 4. Weights make the results fully representative of the country (or of Eu in the case of results at the Eu level, where different weights are used). %\footnote{We trim weights so that no respondent receives a weight below 0.25 or above 4. Overall, trimming changes the weights for xx\% of the respondents.} 
Appendix \ref{app:representativeness} confirms that our samples are representative of the population.

% \subsection*{\small Data quality.} % WPcomment % TODO attrition analysis
\paragraph{\small Data quality.} % NCCcomment
The median duration is 28 minutes for the global survey and 15 minutes in the US1 survey. To ensure the best possible data quality, we exclude respondents who fail an attention test or rush through the survey (i.e. answer in less than 11.5 minutes in the global survey, 4 minutes in US1 or US2, 6 minutes in Eu). %All the results and analyses use survey weights, defined to make the results fully representative of the country (or of Eu in the case of results at the Eu level) along the quota variables. Weights are trimmed to be between .25 and 4. 
%\textit{Ex post}, we checked that there were only a few careless response patterns (such as choosing the same answer for all items in a matrix of questions; see Appendix \ref{app:data_quality}). At the end of the survey, we ask whether respondents thought that our survey was politically biased and provide some feedback. 67\% of the respondents found the survey unbiased. 25\% found it left-wing biased, and 8\% found it right-wing biased.

% \subsection*{\small Questionnaires and raw results.} % WPcomment
\paragraph{\small Questionnaires and raw results.} % NCCcomment
% Possible confusion in the questionnaire: people confuse GCS with the four policies together (so support for GCS can suffer from dislike of death penalty), although this confusion is mitigating by the fact that we right after ask about NR; people may answer about revenue-use rather than the whole measure for ETS2 support; people may answer that GCS will or will not have the effects proposed rather than these effects being important or not in their attitude towards GCS; they may answer that it's important that others (not them) get more information; the minimum wage could be reduced at 50% of local median wage.
The questionnaire and raw results of the global survey can be found in the Appendix of the companion paper \citep{dechezlepretre_fighting_2022}. %.\cite{dechezlepretre_fighting_2022} 
The raw results are reported in Appendix \ref{app:raw_results}\footnote{Country-specific raw results are also available as supplementary material files:  \href{https://github.com/bixiou/global_tax_attitudes/raw/main/paper/app_desc_stats_US.pdf}{US}, \href{https://github.com/bixiou/global_tax_attitudes/raw/main/paper/app_desc_stats_EU.pdf}{EU}, \href{https://github.com/bixiou/global_tax_attitudes/raw/main/paper/app_desc_stats_FR.pdf}{FR}, \href{https://github.com/bixiou/global_tax_attitudes/raw/main/paper/app_desc_stats_DE.pdf}{DE}, \href{https://github.com/bixiou/global_tax_attitudes/raw/main/paper/app_desc_stats_ES.pdf}{ES}, \href{https://github.com/bixiou/global_tax_attitudes/raw/main/paper/app_desc_stats_UK.pdf}{UK}.} while the surveys' structures and questionnaires are given in Appendices \ref{app:questionnaire_oecd} and \ref{app:questionnaire}. The questionnaires are the same as the ones given \textit{ex ante} in the registration plan (\href{https://osf.io/fy6gd}{osf.io/fy6gd}).

% \subsection*{\small Incentives.} % WPcomment
\paragraph{\small Incentives.} % NCCcomment
To encourage respondents to answer accurately and truthfully, several questions of the US1 survey use incentives. For each of the three comprehension questions that follow the policies' descriptions, we reward three (randomly drawn) respondents with the correct answer with a \$50 gift certificate. For each of the questions asking respondents to guess the share of support for the GCS and NR, we reward three people who are closest to the true value with a \$50 gift certificate. For the donation lottery question, we randomly draw one respondent and split the \$100 prize between the NGO GiveDirectly and the winner according to the winner's choice. In total, our incentives scheme distributes gift certificates (and donation) for a value of \$850. Finally, respondents have an incentive to answer truthfully to the petition question, given that they know that the results to that question (the share of respondents supporting the policy) will be transmitted to the U.S. President's office.

% Here is a description of a specific method used.  Note that the subsection heading ends with a full stop (period) and that the command is \verb|\subsection{}| not \verb|\subsection**{}|.

\section*{\normalsize Data and code availability}

All data and code of the complementary surveys as well as figures of the paper are available on \href{https://github.com/bixiou/global_tax_attitudes}{github.com/bixiou/global\_tax\_attitudes}. Data and code for the global survey will be made public upon publication.

% \end{methods} % WPcomment
\end{small}  % NCCcomment

% \bibliographystyle{naturemag_noURL} % nature class works only with style naturemag or naturemag_noURL, and naturemag bugs if there are certain URLs (like .pdf). Also, nature class only works with \cite, not \citet or \citep.  % WPcomment
\renewcommand{\url}[1]{\href{#1}{Link}} % NCCcomment
\bibliographystyle{plainnaturl_clean} % NCCcomment
\bibliography{global_tax_attitudes}

\appendix % NCCcomment
\renewcommand{\thetable}{A\arabic{table}}
\renewcommand{\thefigure}{A\arabic{figure}}
\setcounter{figure}{0}
\setcounter{table}{0}

\clearpage
\section{Literature review}\label{sec:literature}

\subsection{Attitudes and perceptions}\label{subsec:literature_attitudes}

\subsubsection{Population attitudes on global policies}\label{subsubsec:literature_attitudes_policies}
Our surveys fill gaps in the knowledge of attitudes toward global policies. 
We are not aware of any other survey on a global wealth tax. 
\citet{carattini_how_2019} test the support for different variants of a global carbon tax, but their samples are representative only along gender and age, and as respondents face only one variant, the sample size for a given variant is about 167 respondents per country. They find more than 80\% of support for any variant in India, between 50 and 65\% in Australia, the UK and South Africa, and 43 to 59\% of support in the U.S., depending on the variant. The support for a global carbon tax funding an equal dividend for each human is close to 50\% in high-income countries (e.g. at 44\% in the U.S.), consistently with what we find in the OECD survey (see Figure \ref{fig:oecd}). 
Using a conjoint analysis in the U.S. and Germany, \citet{beiser-mcgrath_could_2019} find that the support for a carbon tax increases by up to 50\% % e.g. in their Fig. 4 the DE support for $70/t jumps from 26 to 39% with extension to all industrialized countries
if it applies to all industralized countries rather than just one's own country. % Variant of carbon tax is 8 (US) - 17 (DE) p.p. more likely to be preferred if tax is extended to all industrialized countries

In surveys in Brazil, Germany, Japan, the UK and the U.S., \citet{ghassim_who_2020} finds 55 to 74\% of support for ``a global democracy including both a global government and a global parliament, directly elected by the world population, to recommend and implement policies on global issues''. % (for example, international peace, world poverty, and climate change)''
Using an experiment, he also finds that, in countries where the government stems from a coalition, voting shares would shift by 8 (Brazil) to 12 p.p. (Germany) from parties who are said to oppose global democracy to parties that supposedly support it. For example, the Greens and the Left gained respectively 9 and 3 p.p. in vote intentions while the SPD and the CDU-CSU each lost 6 p.p., when Germans respondents were told that (only) the former parties support global democracy. 
\citet{ghassim_who_2020} also document survey results which show strong majorities support in each of 18 countries for the direct election of one's country's UN representative. % GlobeScan 2005; also: half/half (majorities or not depend on the country) for “Global Parliament, where votes are based on country population sizes, and the global parliament is able to make binding policies” (Synovate 2007); also: (GlobeScan 22, not from Ghassim) in 31 countries: 77% agree that “Rich countries must pay for poorer countries do deal with the effects of CC”
Similarly, in each of 10 countries, there are clear majorities in favor of ``a new supranational entity [taking] enforceable global decisions in order to solve global risks'' \citep{global_challenges_foundation_attitudes_2018}. Actually, already in 1946, 54\% of Americans agreed (and 24\% disagreed) that ``the UN should be strengthened to make it a world government with the power to control the armed forces of all nations'' \citep{gallup_seventy_1946}. 
In surveys in Argentina, China, India, Russia, Spain, and the U.S., \citet{ghassim_public_2022} find support for UN reform that would make United Nations' decisions binding, give veto powers at the Security Council to a few other major countries, and complement the highest body of the UN with a chamber of directly elected representatives. 
% TODO Schleich 16: international agreements are important but current ones are unsuccessful, people find themselves poorly represented in climate negotiations

These specific questions are in line with the answers to more general questions. In each of 36 countries, \citet{issp_research_group_international_2010} find near consensus that ``for environmental problems, there should be international agreements that [their country] and other countries should be made to follow'' (overall, 82\% agree and 4\% disagree). % No question like this in the next Envi wave in 2022
In each of 29 countries, \citet{issp_international_2019} find near consensus that ``resent economic differences between rich and poor countries are too large'' (overall, 78\% agree and 5\% disagree). 
%* Also in ISSP (19): slight minorities (in rich countries) that “People in wealthy countries should make an additional tax contribution to help people in poor countries.” p. 104, but strong majorities everywhere that “People from poor countries should be allowed to work in wealthy countries.” p. 106
Relatedly, \citet{meilland_international_2023} find that Americans and French people prefer an international settlement of climate justice even if it empedes on sovereignty. In a 2013 survey in China, Germany and the U.S., \citet{schleich_citizens_2016} show that more than 73\% of people find important future international climate agreements, while less than 26\% think that international reached so far are successful. 

%* ISSP (19): Near consensus that “Present economic differences between rich and poor countries are too large.” p. 102, slight minorities (in rich countries) that “People in wealthy countries should make an additional tax contribution to help people in poor countries.” p. 104, but strong majorities everywhere that “People from poor countries should be allowed to work in wealthy countries.” p. 106
%* Ghassim et al. (22): support for stronger UN with more direct elections.
%* Ghassim (20):  in Germany those two parties that supposedly endorse global democracy – the Greens and the Left – benefitted, gaining nine and three percentage points respectively in terms of voting intentions. Meanwhile, the traditional centrist parties – SPD and CDU – each lost six percentage points due to their supposed opposition to global democracy.
%* Beiser-McGrath & Bernauer (19): Conjoint analysis in US, DE. Variant of carbon tax is 8 (US) - 17 (DE) p.p. more likely to be preferred and 50% more likely to be supported if tax is extended to all industrialized countries (Fig 1, 4). (Unfortunately, don't test extension to global level).
%- Çarkoğlu.. (15) International Social Survey Program 2010 data reveal that people in LDCs are less supportive of international agreements forcing their country to take necessary environmental measures than are citizens in the developed world [80% instead of 85%]. (‘for environmental problems, there should be international agreements that [their country] and other countries should be made to follow.’)
%* Carattini et al. (Nature, 19): 1k in US, IA, ZA, AU, UK. Each respondent receives one variant at random of global carbon price of 40/60/80 $/t redistributed as international dividend / national dividend / mitigation in all countries / mitigation in developing countries / domestic mitigation / reduced labour tax. Immense majorities for any scheme in India, small majorities for each elsewhere except US international dividend (44%) or mitigation in developing (43%), and AU mitigation in developing (49,6%). PB: very low sample size (~167) for a given redistribution, even lower (~55) for a given variant (that also specifies the price). Appendix also contains estimation of distributive impacts. Representative only along the two quotas: gender and age. Don't give the representativeness in terms of income (the third socio-demos that they ask) so it's probably bad.

\subsubsection{Population attitudes on climate burden sharing}\label{subsubsec:literature_attitudes_burden_sharing}

Despite their differences in the description of the fairness principles, the surveys burden-sharing rules show consistent attitudes. Or at least, their various results can be made compatible with the following interpretation. 
Concerning emissions reductions, most people want that every country engage in strong decarbonization effort together, with a global quota converging to climate neutrality in the medium run. Concerning the financial effort, most people support high-emitting countries paying and low-income countries receive funding. The most supported rules are those that appear equitable, in particular an equal right to emit per person. 
% When the rankings between rules differ, it can be due to the difference in countries surveyed, but it is most often due to differences in definitions and wording. 

This interpretation helps understanding the apparent differences between articles, which approach burden sharing from different angles: cost sharing (i.e. in money terms), effort sharing (in terms of emissions reductions), or resource sharing (in terms of rights to emit). Most papers adopt the cost sharing or effort sharing approaches and preclude any country being a net receiver of money. Also, by focusing on either the financial or the decarbonization effort, these surveys miss the other half of the picture, which can explain why some papers find strong support for the ability-to-pay principle while others find strong support for grandfathering (defined as emissions reductions being the same in every country). The literature follow these approaches to stick to the terms used by the UNFCCC. Yet, we argue that the resource sharing approach is preferable to uncover attitudes, as it unambiguously describes the distributive implications of each rule while achieving an efficient location of emissions reductions and explicitly allowing for monetary gains for some countries. % TODO? say more simply that the location of emissions reductions is flexible in resource sharing
% TODO? appendix with the definitions for each author, incl. us

Now, let us summarize the different papers' results in the light of this clarification. 
\citet{schleich_citizens_2016} find an identical ranking in the support for the burden-sharing principles in China, Germany, and the U.S.: polluter-pays followed by ability-to-pay, equal emissions per capita, and grandfathering. 
% \footnote{The survey of \citet{schleich_citizens_2016} defines these rules as follows: \\
% \textit{Polluter-pays}: ``Every country has to bear costs according to the emissions it causes (hence countries causing higher emissions have a higher share of the costs).'' \\
% \textit{Ability-to-pay}: ``Every country has to bear costs according to its economic strength (hence richer countries have a higher share of the costs).''
% \textit{Egalitarianism}: ``Every country is allowed to produce the same amount of emissions per capita (hence countries with currently high emissions per capita have higher costs).''
% \textit{Sovereignty} (i.e. grandfathering): ``Every country is allowed to produce the same share of global emissions as in the past (hence the proportional reduction of emissions is the same for every country).''} 
Note that the authors do not allow for emissions trading in their description of equal \textit{emissions per capita}, which may explain its relatively low support. 
Yet, the relative support for egalitarianism also depends on how \textit{the other} rules are described. Indeed, \citet{carlsson_is_2011} find that Swedes prefer that ``all countries are allowed to emit an equal amount per capita'' rather than options where emissions are reduced in relation to current or historical emissions for which it is explicitly written that high-emitting countries ``will continue to emit more than others''. 
\citet{bechtel_mass_2013} find agreement that rich countries should pay more and historical emissions matter, but that rich countries should not be the only one to make the efforts. More precisely, their conjoint analysis in France, Germany, the UK and the U.S. shows that a climate agreement is 15 p.p. more likely to be preferred  (to a random alternative) if it includes 160 countries rather than 20, and 5 p.p. less likely to be preferred if ``only rich countries pay'' comapred other burden-sharing rules: ``rich countries pay more than poor'', ``countries pay proportional to current emissions'' or ``countries pay proportional to historical emissions''. %=> confirms preference for global policies (rather than only partial coverage). Finds that costs is what matters most: preference decreases by 30pp if it’s 2.5\% of GDP compared to 0.5\%.
Using a choice experiment, \citet{carlsson_fair_2013} find that the least preferred option in China and the U.S. is when low-emitting countries are exempted from any effort. Ability-to-pay is appreciated in both countries, though the preferred option in China is another one, which accounts for historical responsibility. %that Americans prefer capacity to pay > current responsibility > historical responsibility > equal emissions per capita while Chinese prefer historical > capacity > current > equal emissions.
%   Capacity to pay: Countries with high income levels must pay a larger share of the costs than countries with low income levels. This option says that countries with greater ability to pay should pay more
%   Current responsibility: Countries with currently high emissions levels must pay a larger share of the costs than countries with currently low emissions levels. This option says that those countries that are currently a larger part of the problem should pay more.
%   Historical responsibility: Countries with a history of high emissions levels must pay a larger share of the costs than countries with a history of lower emissions. This option recognizes that CO2 builds up in the atmosphere over many years. Thus, countries with a history of high emissions should pay more because they caused more of the problem.
%   Equal emissions pc: Countries with emissions per person greater than an agreed amount must pay, and they must pay more the higher their emissions per person are.
% > "equal emissions" is a misnomer as this is about costs (not emissions) and it's just a more progressive version of current responsibility / polluter-pay, where high-emitting pay more and low-emitting don't pay. The result for US is compatible with the other papers as Americans agree that rich countries (or high-emitting, the diff is small) should pay more. The Chinese position could also be reconciliable once we define responsibility from footprint rather than territorial and that there will be transfers from rich to poor countries.
In the U.S. and France, \citet{meilland_international_2023} find that the most favored fairness principle is that ``all countries commit to converge to the same average of total emissions per inhabitant, compatible with a controlled climate change''. Furthermore, in each country, 73\% disagree with grandfathering defined as ``countries which emitted a lot of carbon in the past have a right to continue emitting more than others in the future''. \citet{meilland_international_2023} contain many other results, for example majorities prefers to hold countries accountable for their consumption-based rather than territorial emissions, and the median choice regarding historical responsibility is to hold a country accountable for their post-1990 emissions (rather than post-1850 or just their current emissions). 
% - Meilland et al. (23) find that in US and France, most favored fairness principle is Equality in per capita emissions: "all countries commit to converge to the same average of total emissions per inhabitant, compatible with a controlled climate change" and second-most (which closely follows) is grandfathering: "all countries commit to reduce their emissions by a same proportion". 73% in each disagree with grandfathering when defined as "countries which emitted a lot of carbon in the past have a right to continue emitting more than others in the future". To rationalize these contrasted views with grandfathering, we can interpret them as: equal rights, equal emission reductions, and transfers. 
%   convergence per capita (70%): all countries commit to converge to the same average of total emissions per inhabitant, compatible with a controlled climate change
%   grandfathering (60%): all countries commit to reduce their emissions by a same proportion
%   past emissions (20% choose it among their two favorite): countries which emitted less in the past commit to reduce their emissions less than other countries
%   poor countries (20%): poorer countries commit to reduce their emissions less than richer countries
%   cost-efficiency (20%): countries where reducing emissions is more costly commit to reduce their emissions less than other countries
% - Meilland et al. (23) Other findings: people prefer international settlement on CC even if it empedes on sovereignty, a majority prefers to target footprint rather than territorial emissions, median is that countries should be held accountable for post-1990 emissions, self-serving bias when judging e.g. India vs. EU, no shared understanding of fairness when asked to coordinate between French and Americans
Finally, in each of 28 (among the largest) countries, \citet{dabla-norris_public_2023} find strong majority for ``all countries'' to the question ``Which countries do you think should be paying to reduce carbon emissions?''. Asked to choose between a cost sharing based on \textit{current} vs. \textit{accumulated historic emissions}, a majority prefers \textit{current emissions} in all countries but China and Saudi Arabia (where the two options are close to equally preferred). 


\subsubsection{Population attitudes on foreign aid}\label{subsubsec:literature_foreign_aid}

\subsubsection{Population attitudes on wealth tax}\label{subsubsec:literature_wealth_tax}

\subsubsection{Population attitudes on ethical norms}\label{subsubsec:literature_wealth_tax}
\paragraph{Universalism}
% TODO WVS on world citizenship (e.g. Bayram 15), Reysen and Katzarska-Miller 2018
%- Buntaine & Prather (18), Diedrich & Goeschl (18) Willingness to act for domestic vs. international climate action (lab experiment) READ
\paragraph{Free-riding}

\subsubsection{Second-order beliefs}\label{subsubsec:literature_beliefs}

\subsubsection{Elite attitudes}\label{subsubsec:literature_beliefs}


\subsection{Proposals and analyses of global policy-making}\label{subsec:literature_policies}

\subsubsection{Global carbon pricing}\label{subsubsec:literature_pricing}

\subsubsection{Climate burden sharing}\label{subsubsec:literature_burden_sharing}

\subsubsection{Global redistribution}\label{subsubsec:literature_redistribution}

\subsubsection{Global democracy}\label{subsubsec:literature_democracy}


% Burden-sharing
%- Agarwal & Narain (91) first to defend an equal right to emit per capita (equal to the absorbing capacity of the Earth)
%- Gampfer (14): lab experiment (ultimatum game) to test whether preferences respect fairness principles
%- Chancel & Piketty (15): global progressive carbon tax
% cf. bottom

% Global policies attitudes 
%* ISSP (19): Near consensus that “Present economic differences between rich and poor countries are too large.” p. 102, slight minorities (in rich countries) that “People in wealthy countries should make an additional tax contribution to help people in poor countries.” p. 104, but strong majorities everywhere that “People from poor countries should be allowed to work in wealthy countries.” p. 106
%* Ghassim et al. (22): support for stronger UN with more direct elections.
%* Ghassim (20):  in Germany those two parties that supposedly endorse global democracy – the Greens and the Left – benefitted, gaining nine and three percentage points respectively in terms of voting intentions. Meanwhile, the traditional centrist parties – SPD and CDU – each lost six percentage points due to their supposed opposition to global democracy.
%* Beiser-McGrath & Bernauer (19): Conjoint analysis in US, DE. Variant of carbon tax is 8 (US) - 17 (DE) p.p. more likely to be preferred and 50% more likely to be supported if tax is extended to all industrialized countries (Fig 1, 4). (Unfortunately, don't test extension to global level).
%- Çarkoğlu.. (15) International Social Survey Program 2010 data reveal that people in LDCs are less supportive of international agreements forcing their country to take necessary environmental measures than are citizens in the developed world [80% instead of 85%]. (‘for environmental problems, there should be international agreements that [their country] and other countries should be made to follow.’)
%* Carattini et al. (Nature, 19): 1k in US, IA, ZA, AU, UK. Each respondent receives one variant at random of global carbon price of 40/60/80 $/t redistributed as international dividend / national dividend / mitigation in all countries / mitigation in developing countries / domestic mitigation / reduced labour tax. Immense majorities for any scheme in India, small majorities for each elsewhere except US international dividend (44%) or mitigation in developing (43%), and AU mitigation in developing (49,6%). PB: very low sample size (~167) for a given redistribution, even lower (~55) for a given variant (that also specifies the price). Appendix also contains estimation of distributive impacts. Representative only along the two quotas: gender and age. Don't give the representativeness in terms of income (the third socio-demos that they ask) so it's probably bad.


% Global policies
% Pottier et al (17): A survey of global climate justice 
%* Hickel (17): The Divide: A Brief Guide to Global Inequality and its Solutions
%* Kopczuk et al (EER, 17) Compute optimal linear tax rate for all countries in two ways: decentralized or globally. Shows that the tax rate increases with inequality of skills (calibrated with the gini). The average decentralized rate is 0.41 The global one 0.62, with a global demogrant of 250$/month (higher than 73 countries' GDP). Show that within decentralized/country optimal taxation would not decrease global inequality by much (gini from 0.695 to 0.69, but down to 0.25 with global income tax). Show that USA don't give a damn of poor countries' people. citizens in the US (one of the richest) attach only 1/(2,000\*a) of the weight to the welfare of citizens in poorest countries, where a is the share  of transfer (supposedly) effectively arriving to the recipients. e.g. if half of aid is wasted by corrupt politicians, the weight is 1/1000.
% Carthy & Walsh (Oxfam, 22) propose various sources of funding for damages.
% Piketty (2014) "At what rate would [a global wealth tax] be levied? One might imagine a rate of 0 percent for net assets below 1 million euros, 1 percent between 1 and 5 million, and 2 percent above 5 million. Or one might prefer a much more steeply progressive tax on the largest fortunes (for example, a rate of 5 or 10 percent on assets above 1 billion euros). There might also be advantages to having a minimal rate on modest-to-average wealth (for example, 0.1 percent below 200,000 euros and 0.5 percent between 200,000 and 1 million)" He doesn't explicitly talk about revenue use, but implicitly they would be retained by each collecting country: "Le rôle principal de l'impôt sur le capital n'est pas de financer l'État social, mais de réguler le capitalisme.", "En principe, chaque pays de l'Union européenne pourrait obtenir des recettes du même ordre en appliquant seul un tel système."

% Global carbon pricing TODO! find current advocates of GCS
%* Grubb (90), Betram (92) advocate for global market with equal pc right
%* Bergh et al. (20) call for a "dual-track transition to global carbon pricing": an expanding climate club, and "a reorientation of UNFCCC negotiations creates room for talking seriously about a global carbon price schedule, including redistribution-of-revenues rules." They don't specify which equity rules to use.
%* Jamieson (01) advocates of equal pc burden-sharing (after the precursors Agarwal & Narain (91))
%* Bear et al (Science, 00), Bear (02), Athanasiou & Baer (02) advocate for equal pc burden-sharing (although weirdly, Bear & Athanasiou then change mind and advocate for the Greenhouse Development Rights, accounting for capacity and responsibility)
%* Cramton et al (17): Livre de pontes. Tout le monde est d'accord : un prix mondial du carbone est requis, il ne peut être obtenu que par la réciprocité des engagements (style climate club), et il faut quelques transferts des riches vers les pauvres ainsi que des sanctions commerciales pour aligner les incitations. Ch 4 (also Cramton et al 15) propose la formule suivante de transfert (positif ou négatif) à un fonds climat : générosité*émissions en excès (par rapport à la cible)*prix du carbone. On demanderait aux États autour de la moyenne d'émission de fixer ce paramètre de générosité, pour qu'il soit fixé de sorte à maximiser le prix, puis on fixerait le prix comme le prix minimum proposé (après avoir éjecté qqs pays récalcitrants des négos). Puis, sanctions commerciales pour ceux qui ne respectent pas le prix. Ch Gollier & Tirole proposent une formule aussi simple que l'autre : quota global*((1-g)*part des émissions à t=0 + g*part de la population), où g joue le même rôle de paramètre de générosité/éthique (que je voudrais mettre à 1, mais qu'ils disent tous de mettre < 1 pour que les pays riches acceptent. Le livre argumente bcp sur prix vs. quantité (TLM préfère prix sauf Gollier & Tirole), l'argument le plus convaincant en faveur du prix c'est qu'avec la procédure proposée le prix négocié serait le plus élevé possible, alors qu'avec la quantité c'est le budget carbone qui serait le point focal et ça aboutirait à une impasse (objectif trop ambitieux).
%* MacKay et al (Nature, 15) summarizes the above
%* Weitzman (17) advocates for a World Climate Assembly, choosing the price level with the median voter, and each country retaining the revenues.
% Fleurbaey & Zuber (13): The discount rate converges to the worst-off (affected by the measure) to the worst-off (beneficiary of the measure) discount rate, which depends on the growth between both agents. Applied to real data, we can consider that the worst-off affected by a global tax on CO_2 is the average-earner on earth (around 75% centile i.e. ~1000€/month, cf. Chancel & Piketty, Lakner & Milanovic, Chakravorty) while the worst-off beneficiary is the worst-off person in the future (among those less affected by CC thanks to the measure), probably below 1000€/month => negative discount rate.
% Stanton (11): Negishi weights obviate the IAMs’ equalization of income. 4 ways to solve this problem: 1. be more transparent, 2. stop weighting, 3. take linear utility (i.e. maximize global GDP), 4. stop optimizing. 
% Hoel (91): Shows that an international tax can be designed so that it is both efficient and satisfies whatever distributional objectives one might have.
% IMF (2019): global pricing (with either differentiated prices or international transfers) or, as a first step, a carbon price floor. 25% of revenues should be rebated to the bottom 40%, the rest used to reduce distortionary taxes or for green investments. Estimate that $75/t is needed in 2030 for 2°C.
% Parry et al (21): Proposal for an International Carbon Price Floor Among Large Emitters. Acknowledges that transfers could be necessary to induce climate action in low/middle-income countries, talks about transferring 1% of carbon revenues.
%- Sager: distributive effects of global pricing without int'l transfers.
%- Budolfson et al. (incl. Fleurbaey, Méjean, Zuber, Dennig) (21): global carbon price with within-country per capita dividend. Acknowledge that "The overall benefits to society are even greater if total carbon tax revenues are returned on an equal per capita basis globally, which directs more of the revenues towards the poorest populations in the world (rather than the poorest within each country or region)." Very short (3p, no appendix, no suppl. info)

% Foreign aid TODO: find more recent, check .lyx for already written paragraph
%* Kaufmann et al (12) Shows the level of perceived and desired aid in 26 countries between 2005 and 2008 (cf. Table 1). In most countries (incl. UK, DE, FR, ES but not U.S.) desired aid is larger than perceived. Argue that this is due to political influence efforts/possibilities of the rich, as they prefer less aid due to vested interests (support this by a theoretical model + correlations between level of lobbying and actual aid level, controling for desired aid). In most countries the gap between the two is small, except in the U.S. where perceived is 7.5% of GDP and preferred is 3%.Use WVS and Gallup (like Chong & Gradstein, Paxton & Knack) but have more waves and the others don't use the question on perceived aid. Shows that richer want less aid ("those in the top income quintile favour ODA (as a share of GNI) that is 0.13 percentage points lower than the preferred share for individuals in the bottom 40\% of the income distribution" after controling for perceived aid - our regression results are sensibly the same.). from 0 to higher than 25%: threshold at 0.05; 0.15; 0.35; 0.75; 1.5; 2.5; 4; 7.5; 17.5; 25, i.e. same number of thresholds but small than ours below 2.5 and higher above. 
%*? Milner & Tingley (13): (highly cited but no original data, don't think we need to cite it) In 2008, 44% of American wanted foreign aid cut (american elections study, 08). fraction of federal budget going to foreign aid (mean: 27%, median: 25%) / should go (mean: 13%, median: 10%) (WorldPublicOpinion, 10)
% PIPA (01): Overwhelming majorities support a multilateral effort to cut hunger in half by the year 2015 and say that they would be willing to pay for the costs of such a program. However, most do not think that the average American would be as willing to pay the necessary costs. when PIPA asked respondents to estimate how much of the federal budget was devoted to foreign aid, the median estimate was 15% -- 15 times the actual amount, which was just under 1%. More dramatically, when asked what an appropriate percentage would be, the median response was 5% -- 5 times the actual amount. And when asked to imagine that they heard the real amount was only 1%, only 18% of respondents said they thought that would be too much--as compared to the 75% who had initially said that the US was spending too much. what percentage of their "tax dollars that go to help poor people at home and abroad...should go to help poor people in other countries." The mean response was 16% (down a bit from 22% in response to this question in a 1996 PIPA poll). Strikingly, this turns out to be a far higher percentage than is currently given. In 1999, a bit less than 4% of the total spent on the poor went to the poor abroad. Sixty percent of respondents proposed a percentage that was higher than 4%.
%- DFID (10): Priorities: 1 NHS, 2 education, 3 support to poor countries, 4 police, 5 defence (p. 19). Show majority support for increased aid until 07, then median is to support stable aid (due to crisis?). It seems they don't give the info on actual amount though.
%* PIPA (08): Across 20 countries, 81% support that "developed countries have a moral responsibility to help reduce hunger ansevere poverty in poor countries (majority in every country). “the World Bank (Shantayanan et al, 2002) has estimated that it will require an extra US$39-54 billion per year to meet Millennium Development Goal 1 (MDG1). (…) The per person cost of meeting MDG1 came to £25 for the UK, $56 for the US, €27 for Germany, and so on. On average 77 per cent of respondents are in favour of contributing towards meeting the goal (provided that all others do too). To take the US example, 75 per cent of people supported paying an extra $56 per year to meet MDG1. What is significant about this figure is that it is only slightly below the support for the ‘cost free’ question as to whether the US should be willing to share a  small portion of its wealth with those who are in great need (79%).” Hudson & van Heerde (12)
%* Hudson & van Heerde (12):Reviews literature on foreign aid and criticizes it on a number of points (e.g. not uncovering the determinants, and not asking well the questions). Shows strong support for poverty alleviation, (at least partly) out of intrinsic altruism. Use 4 main sources: PIPA (01, 08) UK DIDP, Eurobarometer; cf. Table 1 for all surveys on foreign aid / Public support for development has been famously described as a mile wide and and inch deep (Smillie, 1996: ref impossible to find). Hard times at home have meant that public support appears to have turned against international development efforts (Henson and Lindstrom, 2010). / Monitor public support: (Fransman and Solignac Lacomte, 2004; McDonnell et al, 2003), Paxton and Knack, 2008; Chong & Gradstein 2006. Review surveys on aid. / ~75% support aid in developed countries (stable) but ‘84 per cent agreed with the assertion that ‘taking care of problems at home is more important than giving aid to foreign countries’ (PIPA, 2001:9).” / References on covariates of aid support / PIPA 2001, "On average, Americans thought just under 25 per cent of the US budget was allocated to foreign aid, and government should allocate less than 14 per cent of the national budget. However, when told that US spends approximately 1 per cent of the federal budget on foreign aid, 37 per cent of respondents thought this was too little, 44 per cent thought it was about right, and 13 per cent thought it too much."  Think that only 23% of aid really goes to the poor / “The 2009 UK survey, Public Attitudes towards Development, reports ‘public support for overseas aid’ at 72 per cent (DFID, 2009); while in the US support was a comparable 79 per cent (PIPA, 2001); and average support across the EU trends slightly higher than in the US and UK with 91 per cent saying it was either very (53%) or fairly (38%) important to provide aid to poor countries (Eurobarometer, 2005).” / “DFID has now begun asking questions that provide relative measures of the salience of development aid vis-à-vis other competing policy issues (DFID, 2009; IDC, 2009). / "high proportion (61%) of US citizens who felt that the US spends too much on foreign aid. [from another source]” / “The distinction between foreign aid, which includes military spending, and development aid/assistance is an important one” / “81 per cent of respondents believed that developed countries do have a moral responsibility to work towards reducing hunger and severe poverty (WorldPublicOpinion.org, 2008). (…) there are a good number of people who support aid despite the fact they do not think it works. What this suggests – but cannot show in any detail – is that people have nonutilitarian motives for supporting aid.” / “support for development assistance is highly contingent on respondents’ perceptions of the effectiveness of aid, especially with regard to corruption (Henson et al, 2010). For example, in the UK, 47 per cent of respondents thought that aid was wasted, with sizable majorities citing corruption and poor management and/or delivery as primary factors (DFID, 2008). More disconcertingly, US respondents thought that only 23 per cent of US aid money that goes to poor countries ends up helping the people who really need it and 54 per cent of US aid money that goes to poor countries ends up in the pockets of corrupt government officials (PIPA, 2001). (…) international charities and NGOs are deemed best suited/most effective compared to donor countries” / UK ‘MyAid’ plan – where the public gets to vote on how a pot of money should be distributed – / "public engagement should be about ‘opening up the political and wider societal space to the possibility of deeper change’ (Darnton and Kirk, 2011:14).”
%* Gilens (01) 17% fewer American with high political knowledge want to cut foreign aid when we provide them specific information about aid amount.
%- Chong & Gradstein (16): from WVS 95-99, 58% want that their country give more foreign aid (but misperceptions are not taken into account)
%* Bauhr et al (13): Support for aid is reduced by perception of corruption in recipient countries. However, this effect is reduced by the aid-corruption paradox (and other things): most corrupt countries need more help.
%- Nair (18): (lack of) Aid support in US driven by information on global distribution, because people underestimate their rank by 27 centiles and overestimate global median income by a factor 10.
%- Williamson (19): Public Ignorance or Elitist Jargon? Reconsidering Americans’ Overestimates of Government Waste and Foreign Aid. "Foreign aid" encompasses military spending, in the mind of American.
%- McDonnell et al (03) Public Opinion and the Fight against Poverty
%- Nair (16): preferences driven by worldviews rather than self-interest
%- Bodenstein & Faust (17): Determinants of support for aid conditionality. They are: perceived corruption in donor country, right-wing.
%- Scotto et al (17): We Spend How Much? Misperceptions, Innumeracy, and Support for the Foreign Aid in the United States and Great Britain. Less American and British want aid cut when information on current aid is given in % of GDP rather than in $.
%* Paxton & Knack (12): Majorities want more aid, and main determinants are trust, ideology, interest in politics, and female (all positive). Gallup 02: in US 45% want more aid (rather than stable) vs. 68-91 in DE-UK-ES. Like Chong & Gradstein, find that desired aid increases with income, contrary to Kaufmann et al. but the latter contains more datasets.
%- Wood (15): Determinants for aid support in Australia. Wood (18) Examine Australian support for aid: although there is support to help foreign poor, people back recent aid cuts.
%- Bayram (17): Aid support associated with trust, i.e. seeing integrity and trustworthiness in others.
%- Cheng & Smyth (16): Why Give it Away When You Need it Yourself? Understanding Public Support for Foreign Aid in China. Political ideology and patriotism main explaining variables for aid support. People in poorer provinces less supportive.
%- Milner & Tingley (10) theory + empirics: who supports aid and why. owners of capital in donor countries tend to gain from aid and thus are more likely to support giving aid
%- Easterly (JEP, 03) Can Foreign Aid Buy Growth? No (disproves Hansen & Tarp).
%- Hansen & Tarp (01) Aid increases growth (empirical evidence)
%- Tresch et al. (22): 66% of Swiss people want to increase their foreign aid
%- Harris (17): majority of French want to decrease foreign aid


% Universalism
%- Enke et al. (Manag. Science, 23): measures universalism by asking to split donation to domestic and foreigner of same absolute income (US).
%- Enke et al. (ReStud, 23): unviersalism more correlated to policy attitudes than income, education, religiosity or beliefs about government efficiency (West).
%- Cappelen et al. (NBER, 22): how unviversalism (as measured above) varies across countries. Comparable in Europe and US (lower in China, higher in Africa)
%- Cherry et al (17) show in the lab that some people prefer policies detrimental to them due to their worldview.


% Free-riding
%- Mildenberg (2019): people are not free riders
%- McGrath & Bernauer (17): review paper. people are not free riders. Preferences concerning climate policy tend to be driven primarily by a range of personal predispositions and cost considerations, which existing research has already explored quite extensively, rather than by considerations of what other countries do
%- Bernauer & Gampfer (15): US and IA people are not free riders. They each overestimate their country's emissions at one third of global total.


% Social norms
%- Bursztyn et al. (AER, 20): social norms can change following new public information such as unexpected election outcome. After Trump election, people express more xenophobic views and judge less severely those who do.
%- Farrow et al. (17): review of effect of social norm intervention on environmental attitudes

% Incentive compatibility
%- Danz et al


% Second-order beliefs
%* Mildenberg & Tingley (19): survey elites (Congress staffers, scholars) and public in U.S. and China and show pluralistic ignorance of pro-climate attitudes, egocentric bias, and increasing support after beliefs are updated.
%- Bursztyn & Yang (21): Review of the field. Misperceptions about others are widespread, asymmetric, much larger when about out-group members, and positively associated with one’s own attitudes.
%- Drews et al. (22): in Spain, supporters (resp. opponents) of carbon tax overestimate (resp. underestimate) support. Providing information doesn't change the overall support.
%* Falk et al. (21): Respondents vastly underestimate the prevalence of climate- friendly behaviors and norms among their fellow citizens. Providing respondents with correct information causally raises individual willingness to fight climate change as well as individual support for climate policies. The effects are strongest for individuals who are skeptical about the existence and threat of global warming.
%- Di Tella et al. (AER, 15): The results of the lab experiment favor the hypothesis that people avoid altruistic actions by distorting beliefs about others' altruism
%- Allport (1924): first book on pluralistic ignorance
%- Allport (40): function of poll is to correct pluralistic ignorance
%- Studies on pluralistic ignorance: business (Buckley et al. 00), against affirmative action (Van Boven 00), political correctness (Braghieri, AER 21), alcohol (Suls & Green, 03), white support for racial segregation (O'Gorman 75), CC (Geiger & Swim 16), hooking up (Lambert et al 03, cf. note for paragraph of pluralistic ignorance), women working outside home in Saudi Arabia (Bursztyn et al. 20)
%- Geiger & Swim (16) Shows that pluralistic ignorance of others' concern about CC leads people to talk less about CC and self-silence themselves.
%- Miller & MacFarland (87) Shows that pluralistic ignorance emerges because individuals believe that fear of embarrassment is a sufficient cause for their own behavior but not for the behavior of others.


% Elite surveys TODO find more
%* Mildenberg & Tingley (19): Congress staffers, cf. second-order beliefs
%- Hertel-Fernandez et al. (2019): Survey on US Congress staffers (not on climate)
%- Milner & Tingley (10) (not sure it's a survey) owners of capital in donor countries tend to gain from aid and thus are more likely to support giving aid
%- Lange et al. (Energy Econ, 2007): climate negotiators
%- Lange et al. (EER, 2010): same data as Lange et al. (10)
%- Dannenberg et al. (ERE, 2010): elicit climate negotiators’ equity preferences using Fehr & Schmidt (99) method => regional differences in addressing climate change are driven more by national interests than by different equity concerns
%- Kesternich et al. (EEPS, 2020): survey on climate negotiators about their preferred burden-sharing rules: we observe tendencies for a more harmonized view among key groups towards the ability-to-pay rule in a setting of weighted burden sharing rules
%- Lange & Schwirplies (ERE, 2017): combines Lange et al. (10) and Schleich et al.
%* Hjerpe et al. (2011): Delegates at COP2009. The results indicate that voluntary contribution, indicated as willingness to contribute, was the least preferred principle among both negotiators and observers. Three of the four principles for allocating mitigation commitments were recognized widely across the major geographical regions: historic 1990, capacity to pay, and equal per capita emissions. The difference was never below 25 percentage units, and the opponent share never exceeded 16%.
%- Scholte et al. (2020)
%- Bayram (17): cosmopolitanism of German politicians and their respect of international law


% Global poverty gap
%* Bolch et al. (22)
%- Zhang (16) estimates the poverty gap in each country. Global one is at $80G/year.


% Basic income TODO find more
%* Egger et al. (19): positive gen eq effects. We provided one-time cash transfers of about USD 1000 to over 10,500 poor households across 653 randomized villages in rural Kenya. The implied fiscal shock was over 15 percent of local GDP. We find large impacts on consumption and assets for recipients. Importantly, we document large positive spillovers on non-recipient households and firms, and minimal price inflation.
%* Haushofer & Shapiro (16): The Short-term Impact of Unconditional Cash Transfers to the Poor: Experimental Evidence from Kenya. Monthly transfers are more likely than lump-sum transfers to improve food security


% Unequal exchange / embodided labour
%- Reyes et al (17)
%- Sakai et al (17)
%- Alsamawi et al. 2014


% NDCs assessments or burden-sharing computations. TODO check the contraction & convergence scheme proposed by France
%- Bourban (18): Soutient un marché du carbone avec droits en proportion des émissions cumulées depuis 1990. Et des “mesures volontaires de contrôle de la population mondiale”.
%- Raupach et al (NCC, 14) 
%- Grasso (2012)
%- van den Berg et al (20)
%- Meyer (04) Contraction and Convergence (i.e. grandfathering converging to equal pc, within an ETS)
%- >Baer et al (08)< (cite this one, others don't give more info), Baer (13), Athanasiou et al (22), Holz et al (19) https://calculator.climateequityreference.org/ Athanasiou, Greenhouse Development Rights, EcoEquity calculator, US fair share. Effort-sharing approach based on splitting emissions reductions in function of capacity to pay (~ share of global income in top 30%) and responsibility (share of emissions since 1950), weighted equally. Corresponds to UNFCCC wording. Pb of this method (applying to any choice of parameters): A country with relatively low incomes (e.g. equal distribution slightly above the p70) and that has few historical responsibility would have a relatively low effort. Even more problematic, the **poorest countries would have virtually 0% of the effort, hence they would be allowed to emit following the baseline trajectory… but this baseline is not fair; it amounts to grandfathering**. It is computed as the “product of the projected GDP and CO2 emission intensity”. ([https://climateequityreference.org/calculator-information/gdp-and-emissions-baselines/](https://climateequityreference.org/calculator-information/gdp-and-emissions-baselines/)), and give for example 0.8tCO2e/cap for RDC in 2030 (16% more than in 2020, but lot lower than the objective of ~4t). => Compared to an equal right to emit pc, this method favors countries like China (allowed to remain stable over 2020-30 vs. reduced by 35-40%) and penalizes countries like the U.S. and Africa. 
%  in Athanasiou et al (22) Justification of Greenhouse Development Rights instead of Equal per capita right is on p. 36. It is weak, and basically that historical responsibility should be taken into account. Conversely, justification against historical resp. is that the latter doesn’t take into account capacity to pay (it is not said like this, but we can think of ex-USSR).
%- Pachauri et al. (Science, 2022) 
%- Robiou du Pont et al. (NCC, 2016)
%- Robiou du Pont et al. (ERL, 2016)
%- Höhne et al. (Climate Policy, 2014): review of 40 papers
%- Gao et al. (FEM, 2019)
%- Gignac & Matthews (ERL, 15)
%- Matthews (16) Quantifying carbon debts among nations
%- https://climateequitymonitor.in/ computes carbon debt based on equal per capita cumulative emissions. contact@climateequitymonitor.in https://twitter.com/equity4climate


% Mismatch between preferences and climate action
%- McCright & Dunlap (03) show that it's an organized conservative movement that succeeded in the U.S. not ratifying Kyoto, through lobbying and disinformation.


% Wealth tax attitudes
% look for surveys on global tax => I've found no result with survey or attitudes + "global tax" or "global wealth tax" in google scholar
% Fisman et al (17): Americans want a 3% tax on inherited wealth
%- Christensen et al. (Oxfam, 23) p. 32 gives references on rich tax attitudes, with always strong majority support:
%* OECD (19): 52-80% of absolute support for "government tax the rich more than they currently do in order to support the poor" in 21 OECD countries
%* Isbell (22): 34 African countries
%- Patriotic Millionaires (22), UK
%- Americans for Tax Fairness (21), US
%- Gallup (22), US
%- Fight Inequality Alliance India (22), IA

% Different framing of burden-sharing, depending on what should be split:
% - mitigation costs: this is the most used as it is easiest to explain. The issue is that it is not specified how agents pay (or if some agents receive payments) and implicitly, there is no negative costs (transfers exceeding the costs) and the carbon price is not uniform. Used in .
% - emission: this one is vague as it doesn't state at which date emissions pc converge (if they do) and whether there are side payments.
% - emission rights: this one is the most accurate as there is no need of a BAU scenario to compute the mitigation needed and its cost.

% Different fairness principles:
% - equal emission right per capita: using this as a baseline, we can call 'grandfathering' any principle that is more regressive and 'historical responsibility' any principle that is more progressive
% - equal emission reduction (in share of current emission) per capita: grandfathering
% - emission rights proportional to current emissions: grandfathering
% - costs proportional to current emissions: polluter-pay principle
% - costs proportional to cumulative emissions: so-called historical responsibility but may actually have a grandfathering component

% Surveys of population:
% - Schleich et al. (Climate Policy, 16) ask for ranking (TODO check) and find an identical ranking of fairness principles in China, Germany, and the US: accountability (costs according to emissions) followed by capability (according to economic strength), egalitarianism (equal emission per capita), and sovereignty (constant share of global emission) (see Lange & Schiwplies (17) for the computations). 
%   Polluter-pays: Every country has to bear costs according to the emissions it causes (hence countries causing higher emissions have a higher share of the costs).
%   Ability-to-pay: Every country has to bear costs according to its economic strength (hence richer countries have a higher share of the costs).
%   Egalitarian: Every country is allowed to produce the same amount of emissions per capita (hence countries with currently high emissions per capita have higher costs).
%   Sovereignty: Every country is allowed to produce the same share of global emissions as in the past (hence the proportional reduction of emissions is the same for every country).
% other findings: international agreements are important but current ones are unsuccessful, people find themselves poorly represented in climate negotiations
% - Bechtel & Scheve (PNAS, 13) find with a conjoint analysis on FR, DE, UK, US that a climate agreement is 5 p.p. less likely to be preferred (to a random alternative) if only rich countries pay (other burden-sharing are: pay prop. to current emissions / historical emissions / rich countries pay more than poor countries) [TODO: check SI that these are the verbatim] and 15 p.p. more likely to be preferred if it includes 160 (out of 192) countries rather than 20 => confirms preference for global policies (rather than only partial coverage). Finds that costs is what matters most: preference decreases by 30pp if it’s 2.5% of GDP compared to 0.5%.
% - Carlsson et al. (REE, 13) find using a 09 choice experiment that Americans prefer capacity to pay > current responsibility > historical responsibility > equal emissions per capita while Chinese prefer historical > capacity > current > equal emissions.
%   Capacity to pay: Countries with high income levels must pay a larger share of the costs than countries with low income levels. This option says that countries with greater ability to pay should pay more
%   Current responsibility: Countries with currently high emissions levels must pay a larger share of the costs than countries with currently low emissions levels. This option says that those countries that are currently a larger part of the problem should pay more.
%   Historical responsibility: Countries with a history of high emissions levels must pay a larger share of the costs than countries with a history of lower emissions. This option recognizes that CO2 builds up in the atmosphere over many years. Thus, countries with a history of high emissions should pay more because they caused more of the problem.
%   Equal emissions pc: Countries with emissions per person greater than an agreed amount must pay, and they must pay more the higher their emissions per person area.
% > "equal emissions" is a misnomer as this is about costs (not emissions) and it's just a more progressive version of current responsibility / polluter-pay, where high-emitting pay more and low-emitting don't pay. The result for US is compatible with the other papers as Americans agree that rich countries (or high-emitting, the diff is small) should pay more. The Chinese position could also be reconciliable once we define responsibility from footprint rather than territorial and that there will be transfers from rich to poor countries.
% - Carlsson et al. (Ecol Eco, 11) find that Swedes prefer that "all countries are allowed to emit an equal amount per capita" rather than options where emissions reduce in relation to current or historical emissions and continue to be higher in high-emitting countries. 
% - Meilland et al. (23) find that in US and France, most favored fairness principle is Equality in per capita emissions: "all countries commit to converge to the same average of total emissions per inhabitant, compatible with a controlled climate change" and second-most (which closely follows) is grandfathering: "all countries commit to reduce their emissions by a same proportion". 73% in each disagree with grandfathering when defined as "countries which emitted a lot of carbon in the past have a right to continue emitting more than others in the future". To rationalize these contrasted views with grandfathering, we can interpret them as: equal rights, equal emission reductions, and transfers. 
%   convergence per capita (70%): all countries commit to converge to the same average of total emissions per inhabitant, compatible with a controlled climate change
%   grandfathering (60%): all countries commit to reduce their emissions by a same proportion
%   past emissions (20% choose it among their two favorite): countries which emitted less in the past commit to reduce their emissions less than other countries
%   poor countries (20%): poorer countries commit to reduce their emissions less than richer countries
%   cost-efficiency (20%): countries where reducing emissions is more costly commit to reduce their emissions less than other countries
% Other findings: people prefer international settlement on CC even if it empedes on sovereignty, a majority prefers to target footprint rather than territorial emissions, median is that countries should be held accountable for post-1990 emissions, self-serving bias when judging e.g. India vs. EU, no shared understanding of fairness when asked to coordinate between French and Americans
% - Dechezleprêtre et al. (WP, 22) find that equal per capita right > historical responsability, capabilities > grandfathering; that global CC policies are needed; 50% support for global T&D; strong support for global tax on millionaires; no free-riding. TODO: check FR, US wording
% - Dabla-Norris et al. (WP, 23) find strong majority for “all countries” everywhere in “Which countries do you think should be paying to reduce carbon emissions?”, and majority for current rather than historical in all countries but China and Saudi Arabia in “Should countries be paying to reduce carbon emissions based on their current or accumulated historic levels of emissions?”

% > Position making all this compatible: people want that every country engage in strong decarbonization effort together, with a global quota, converging to climate neutrality in the medium run, based on an equal right to emit per person, implying that rich countries pay and low-emitting countries receive funding. Where the rankings differ, it is likely because the definitions or wordings are different, and also because it involves different countries (Sweden != US != China).
% - Schleich find support for costs according to emissions and against immediate equalization of emissions (but nothing against convergence to equal emissions per capita).
% - This is just in contradiction with Carlsson (11) which finds that Swedes prefer the equalization (with a similar wording) to other reduction options. TODO: check wording of the latter.
% - Bechtel find agreement that rich countries should pay more and historical emissions matter, but just that they should not be the only one to make the efforts. 
% - Carlsson (13) find that the least preferred option in China and US is when low-emitting countries don't participate to the effort. Ability to pay is liked in both countries.
% - Meilland find that convergence is the most preferred, followed by emission reductions of same proportion, disagreement with grandfathering expressed in terms of emission rights.
% - Dechezleprêtre find support for equal right is strongest, although historical responsibility and capabilities are also supported. The quota system is strongly supported.

% Surveys of negotiators:
% - Hjerpe et al. (WP, 11)
% - Dannenberg et al. (ERE, 10): measuring negotiators' equity preferences, regional differences in addressing climate change are driven more by national interests than by different equity concerns.
% - Lange et al. (Energy Econ, 07): Mix of self-serving bias and support for egalitarian principle.
% - Kesternich et al. (EEPS, 21): kind of convergence on ability-to-pay.

% Other papers:
% - Lange & Schwirplies (ERE, 17) develop a theoretical model (building on Buchholz et al. (05)), supported by data, justifying that climate negotiators (chosen by the citizens) have lower environmental preferences than their citizens and equity views more aligned with the other negotiators. 
\clearpage
\section{Raw results% from the complementary surveys
}\label{app:raw_results}
% /!\ Do not replace by app_desc_stats_US1 as the latter also contains figures that are already in the main text
% TODO? add country-specific prioritization? No, it's in (separate) country appendices.
% TODO! add share who click on info or reminder
% TODO! Appendix Sources or at least clean up specificities.xlsx

Country-specific raw results are also available as supplementary material files:  \href{https://github.com/bixiou/global_tax_attitudes/raw/main/paper/app_desc_stats_US.pdf}{US}, \href{https://github.com/bixiou/global_tax_attitudes/raw/main/paper/app_desc_stats_EU.pdf}{EU}, \href{https://github.com/bixiou/global_tax_attitudes/raw/main/paper/app_desc_stats_FR.pdf}{FR}, \href{https://github.com/bixiou/global_tax_attitudes/raw/main/paper/app_desc_stats_DE.pdf}{DE}, \href{https://github.com/bixiou/global_tax_attitudes/raw/main/paper/app_desc_stats_ES.pdf}{ES}, \href{https://github.com/bixiou/global_tax_attitudes/raw/main/paper/app_desc_stats_UK.pdf}{UK}.

\begin{figure}[h!]
    \caption[Absolute support for global climate policies]{Absolute support for global climate policies. \\ Share of \textit{Somewhat} or \textit{Strongly support} (in percent, $n$ = 40,680). The color blue denotes an absolute majority. See Figure \ref{fig:oecd} for the relative support. (Questions \ref{q:scale}-\ref{q:millionaire_tax} of the global survey. Reproduced from \citealp{dechezlepretre_fighting_2022}, Figure A20.)} 
    \makebox[\textwidth][c]{\includegraphics[width=1.2\textwidth]{../figures/OECD/Heatplot_global_tax_attitudes_positive.pdf}}\label{fig:oecd_absolute}% with dependence on others (absent from OECD): Heatplot_burden_share_all_positive_countries
    {\footnotesize *In Denmark, France and the U.S., the questions with an asterisk were asked differently, cf. Question \ref{q:burden_sharing_asterisk}. } 
\end{figure}

\begin{figure}[h!]
    \caption[Comprehension]{Correct answers to comprehension questions (in percent). (Questions \ref{q:understood_gcs}-\ref{q:understood_both})}\label{fig:understood_each}
    \makebox[\textwidth][c]{\includegraphics[width=\textwidth]{../figures/country_comparison/understood_each_positive.pdf}} 
\end{figure}

\begin{figure}[h!]
    \caption[Comprehension score]{Number of correct answers to comprehension questions (mean). (Questions \ref{q:understood_gcs}-\ref{q:understood_both})}\label{fig:understood_score}
    \makebox[\textwidth][c]{\includegraphics[width=\textwidth]{../figures/country_comparison/understood_score_mean.pdf}} 
\end{figure}

% \begin{figure}[h!]
%     \caption[Support for the Global Climate Scheme]{Support for the GCS, NR and the combination of GCS, NR and C. (Questions \ref{q:gcs_support}, \ref{q:nr_support} and \ref{q:crg_support})}\label{fig:support_binary}
%     \makebox[\textwidth][c]{\includegraphics[width=.9\textwidth]{../figures/country_comparison/support_binary.pdf}} 
% \end{figure}

% \begin{figure}[h!]
%     \caption[Beliefs about support for the GCS and NR]{Beliefs regarding the support for the GCS and NR. (Questions \ref{q:gcs_belief} and \ref{q:nr_belief})}\label{fig:belief}
%     \makebox[\textwidth][c]{\includegraphics[width=.8\textwidth]{../figures/country_comparison/belief.pdf}} 
% \end{figure}

\begin{figure}[h!]
    \caption[List experiment]{List experiment: mean number of supported policies. (Section \ref{subsubsec:list_exp}, Question \ref{q:list_exp})}\label{fig:list_exp}
    \makebox[\textwidth][c]{\includegraphics[width=.7\textwidth]{../figures/country_comparison/list_exp_mean.pdf}} 
\end{figure}

\begin{figure}[h!]
    \caption[Conjoint analyses 1 and 2]{Conjoint analyses 1 and 2. (Questions \ref{q:conjoint_a}-\ref{q:conjoint_b}, Back to Section \ref{subsubsec:conjoint})}\label{fig:conjoint}
    \makebox[\textwidth][c]{\includegraphics[width=.8\textwidth]{../figures/country_comparison/conjoint_ab_all_positive.pdf}} 
\end{figure}

% \begin{figure}[h!] % already in text
%     \caption{[Asked only to non-Republicans] Conjoint analysis n°4: random programs at the Democratic primary. (Question \ref{q:conjoint_r})}\label{fig:ca_r}
%     \makebox[\textwidth][c]{\includegraphics[width=\textwidth]{../figures/country_comparison/ca_r.png}} 
% \end{figure}

% \begin{figure}[h!]
%     \caption[Influence of the GCS on preferred platform]{Influence of the GCS on preferred platform:\\ Preference for a random platform A that contains the Global Climate Scheme rather than a platform B that does not (in percent). (Question \ref{q:conjoint_d}; in the U.S., asked only to non-Republicans.)}\label{fig:conjoint_left_ag_b}
%     \makebox[\textwidth][c]{\includegraphics[width=\textwidth]{../figures/country_comparison/conjoint_left_ag_b_binary_positive.pdf}} 
% \end{figure}

\begin{figure}[h!]
    \caption[Perceptions of the GCS]{Perceptions of the GCS. Elements seen as important for supporting the GCS in a 4-Likert scale (in percent). (Question \ref{q:gcs_important})  \hfill (Back~to~Section~\ref{subsubsec:pros_cons})}\label{fig:gcs_important}
    \makebox[\textwidth][c]{\includegraphics[width=\textwidth]{../figures/country_comparison/gcs_important_positive.pdf}} 
\end{figure}

\begin{figure}[h!]
    \caption[Classification of open-ended field on the GCS]{Perceptions of the GCS. Elements found in the open-ended field on the GCS (manually recoded, in percent). (Question \ref{q:gcs_field}) \hfill (Back~to~Section~\ref{subsubsec:pros_cons})}\label{fig:gcs_field}
    \makebox[\textwidth][c]{\includegraphics[width=.75\textwidth]{../figures/country_comparison/gcs_field_positive.pdf}} 
\end{figure}

\begin{figure}[h!]
    \caption[Topics of open-ended field on the GCS]{Perceptions of the GCS. Keywords found in the open-ended field on the GCS (automatic search ignoring case, in percent). (Question \ref{q:gcs_field}) \hfill (Back~to~Section~\ref{subsubsec:pros_cons})}\label{fig:gcs_field_contains}
    \makebox[\textwidth][c]{\includegraphics[width=\textwidth]{../figures/country_comparison/gcs_field_contains_positive.pdf}} 
\end{figure}

\begin{table}[h]
    \caption[Campaign and bandwagon effects on the support for the GCS.]{Effects on the support for the GCS of a question on its pros and cons and on information about the actual support, in the U.S. (See Section \ref{subsec:questionnaire_perceptions} in the US2 Questionnaire)  \hfill (Back~to~Section~\ref{subsubsec:pros_cons})} \label{tab:branch_gcs}
    \makebox[\textwidth][c]{
        
\begin{tabular}{@{\extracolsep{5pt}}lcccc} 
\\[-1.8ex]\hline 
\hline \\[-1.8ex] 
 & \multicolumn{4}{c}{Support} \\ 
\cline{2-5} 
\\[-1.8ex] & \multicolumn{2}{c}{Global Climate Scheme} & \multicolumn{2}{c}{National Redistribution} \\ 
\\[-1.8ex] & (1) & (2) & (3) & (4)\\ 
\hline \\[-1.8ex] 
Control group mean & 0.557 & 0.557 & 0.569 & 0.569  \\ \hline \\[-1.8ex]
 Treatment: Open\mbox{-}ended field on GCS pros \& cons & $-$0.073$^{**}$ & $-$0.073$^{**}$ & $-$0.035 & $-$0.031 \\ 
  & (0.035) & (0.031) & (0.035) & (0.032) \\ 
  Treatment: Closed questions on GCS pros \& cons & $-$0.109$^{***}$ & $-$0.096$^{***}$ & $-$0.065$^{*}$ & $-$0.062$^{**}$ \\ 
  & (0.034) & (0.031) & (0.034) & (0.031) \\ 
  Treatment: Info on actual support for GCS and NR & $-$0.021 & $-$0.017 & 0.048 & 0.054$^{*}$ \\ 
  & (0.034) & (0.031) & (0.033) & (0.031) \\ 
 \hline \\[-1.8ex] 
Includes controls &  & \checkmark &  & \checkmark \\

Observations & 2,000 & 1,995 & 2,000 & 1,995 \\ 
R$^{2}$ & 0.007 & 0.169 & 0.007 & 0.153 \\ 
\hline 
\hline \\[-1.8ex] 
\end{tabular} 
    }
    {\footnotesize %\textit{Note}: 
    }
\end{table}

\begin{figure}[h!]
    \caption[Donation to Africa vs. own country]{Donation in case of lottery win, depending on the recipient's (randomly drawn) nationality (mean). (Question \ref{q:donation})\hfill (Back~to~Section~\ref{subsec:universalistic})}\label{fig:donation}
    \makebox[\textwidth][c]{\includegraphics[width=.8\textwidth]{../figures/country_comparison/donation_mean.pdf}} 
\end{figure}

\begin{table}[h]
    \caption[Donation to Africa vs. own country]{Donation in case of lottery win, depending on the recipient's (randomly drawn) nationality. (Question \ref{q:donation})\hfill (Back~to~Section~\ref{subsec:universalistic})} \label{tab:donation}
    \makebox[\textwidth][c]{\input{../tables/continents/donation_interaction.tex}}
\end{table}

\begin{figure}[h!]
    \caption[Support for a global wealth tax]{Support for a global wealth tax. \\
    ``Do you support or oppose a tax on millionaires of all countries to finance low-
    income countries? \\
    Such tax would finance infrastructure and public services such as access to drinking water, healthcare, and education.'' (Question \ref{q:global_tax})}\label{fig:global_tax}
    \makebox[\textwidth][c]{\includegraphics[width=\textwidth]{../figures/country_comparison/global_tax_support.pdf}} 
\end{figure}

\begin{figure}[h!]
    \caption[Support for a national wealth tax]{Support for a national wealth tax financing public services like healthcare, education, and social housing. (Question \ref{q:national_tax})}\label{fig:national_tax}
    \makebox[\textwidth][c]{\includegraphics[width=\textwidth]{../figures/country_comparison/national_tax_support.pdf}} 
\end{figure}

\begin{figure}[h!]
    \caption[Preferred share of global tax for low-income countries]{Preferred share of global wealth tax revenues that should be pooled to finance low-income countries. (Question \ref{q:global_tax_global_share})}\label{fig:global_tax_global_share}
    \makebox[\textwidth][c]{\includegraphics[width=\textwidth]{../figures/country_comparison/global_tax_global_share.pdf}} 
\end{figure}

\begin{figure}[h!]
    \caption[Support for sharing half of global tax revenues with low-income countries]{Support for sharing half of global tax revenues with low-income countries, rather that each country retaining all the revenues it collects (in percent). (Question \ref{q:global_tax_sharing})}\label{fig:global_tax_sharing}
    \makebox[\textwidth][c]{\includegraphics[width=\textwidth]{../figures/country_comparison/global_tax_sharing_positive.pdf}} 
\end{figure}

\begin{figure} 
    \caption[Actual, perceived and preferred amount of foreign aid (mean)]{Actual, perceived and preferred amount of foreign aid, with random info (or not) on actual amount. (\textit{Mean}, Questions \ref{q:foreign_aid_belief}, \ref{q:foreign_aid_preferred})  \hfill (Back~to~Section~\ref{subsubsec:support_foreign_aid})}\label{fig:foreign_aid_amount}
    \makebox[\textwidth][c]{\includegraphics[width=.9\textwidth]{../figures/country_comparison/foreign_aid_amount_mean.pdf} } 
\end{figure}

% \begin{figure} 
%     \caption{Actual, perceived and preferred amount of foreign aid, with random info (or not) on actual amount. (\textit{Median}, Questions \ref{q:foreign_aid_belief}, \ref{q:foreign_aid_preferred})}\label{fig:foreign_aid_amount}
%     \makebox[\textwidth][c]{\includegraphics[width=.9\textwidth]{../figures/country_comparison/foreign_aid_amount_median.pdf} } % TODO? add? not necessary as the info on median can be deduced from below figures
% \end{figure}

\begin{figure} 
    % \caption{Support for increased foreign aid (vs. reduced or stable): from previous question, and directly asked (with info).}\vspace{-.2cm}
    % \includegraphics[height=.32\textheight]{../figures/country_comparison/foreign_aid_more_positive.pdf} 
    \caption[Preferred foreign aid (summary)]{Preferred foreign aid (after info or after perception). (Questions \ref{q:foreign_aid_belief} and \ref{q:foreign_aid_preferred})}\label{fig:foreign_aid_no_less_all}
    \makebox[\textwidth][c]{\includegraphics[width=\textwidth]{../figures/country_comparison/foreign_aid_no_less_all_positive.pdf} }
\end{figure} 

% \begin{figure}
%     \centering 
%     \caption{Your previous answer shows that you would like to increase [UK] foreign aid.\\How would you like to finance such increase in foreign aid? (Multiple answers possible)}
%     \includegraphics[width=\columnwidth]{../figures/all/foreign_aid_raise.pdf} 
% \end{figure}		
% \begin{figure}
%     \centering 
%     \caption{Your previous answer shows that you would like to reduce [UK] foreign aid.\\How would you like to use the freed budget? (Multiple answers possible)}
%     \includegraphics[width=\columnwidth]{../figures/all/foreign_aid_reduce.pdf} 
% \end{figure}

\begin{figure}[h!]
    \caption[Perceived foreign aid]{Perceived foreign aid. ``From your best guess, what percentage of [own country] government spending is allocated to foreign aid (that is, to reduce poverty in low-income countries)?'' (Question \ref{q:foreign_aid_belief})  \hfill (Back~to~Section~\ref{subsubsec:support_foreign_aid}) \\ Actual values: France: 0.8\%; Germany: 1.3\%; Spain: 0.5\%; UK: 1.7\%; U.S.: 0.4\%.}\label{fig:foreign_aid_belief}
    \makebox[\textwidth][c]{\includegraphics[width=\textwidth]{../figures/country_comparison/foreign_aid_belief_agg.pdf}} 
\end{figure}

\begin{figure}[h!]
    \caption[Preferred foreign aid (without info on actual amount)]{Preferred foreign aid (without info on actual amount). \\ ``If you could choose the government spending, what percentage would you allocate
    to foreign aid?'' (Question \ref{q:foreign_aid_preferred})  \hfill (Back~to~Section~\ref{subsubsec:support_foreign_aid})}\label{fig:foreign_aid_preferred_no_info}
    \makebox[\textwidth][c]{\includegraphics[width=\textwidth]{../figures/country_comparison/foreign_aid_preferred_no_info_agg.pdf}} 
\end{figure}

\begin{figure}[h!]
    \caption[Preferred foreign aid (after info on actual amount)]{Preferred foreign aid (after info on actual amount). \\ ``Actually,
    [US1: 0.4\%; FR: 0.8\%; DE: 1.3\%; ES: 0.5\%; UK: 1.7\%] of [own country] government spending is allocated to foreign aid. \\
    If you could choose the government spending, what percentage would you allocate
    to foreign aid?'' (Question \ref{q:foreign_aid_preferred})  \hfill (Back~to~Section~\ref{subsubsec:support_foreign_aid})}\label{fig:foreign_aid_preferred_info}
    \makebox[\textwidth][c]{\includegraphics[width=\textwidth]{../figures/country_comparison/foreign_aid_preferred_info_agg.pdf}} 
\end{figure}

\begin{figure}[h!]
    \caption[Preferences for funding increased foreign aid]{Preferences for funding increased foreign aid. [Asked iff preferred foreign aid is strictly greater than [Info: actual; No info: perceived] foreign aid] \\ ``How would you like to finance such increase in foreign aid? (Multiple answers possible)'' (in percent) (Question \ref{q:foreign_aid_raise_how})  \hfill (Back~to~Section~\ref{subsubsec:support_foreign_aid})}\label{fig:foreign_aid_raise_how}
    \makebox[\textwidth][c]{\includegraphics[width=.75\textwidth]{../figures/country_comparison/foreign_aid_raise_positive.pdf}} 
\end{figure}

\begin{figure}[h!]
    \caption[Preferences of spending following reduced foreign aid]{Preferences of spending following reduced foreign aid. [Asked iff preferred foreign aid is strictly lower than [Info: actual; No info: perceived] foreign aid] \\ ``How would you like to use the freed budget? (Multiple answers possible)'' (in percent) (Question \ref{q:foreign_aid_reduce_how})  \hfill (Back~to~Section~\ref{subsubsec:support_foreign_aid})}\label{fig:foreign_aid_reduce_how}
    \makebox[\textwidth][c]{\includegraphics[width=.75\textwidth]{../figures/country_comparison/foreign_aid_reduce_positive.pdf}} 
\end{figure}

% \begin{figure}[h!]
%     \caption[Attitudes on the evolution of foreign aid]{Attitudes regarding the evolution of [own country] foreign aid. (Question \ref{q:foreign_aid_raise_support})}\label{fig:foreign_aid_raise_support}
%     \makebox[\textwidth][c]{\includegraphics[width=\textwidth]{../figures/country_comparison/foreign_aid_raise_support.pdf}} 
% \end{figure}

% \begin{figure}[h!]
%     \caption[Conditions at which foreign aid should be increased]{Conditions at which foreign aid should be increased (in percent). [Asked to those who wish an increase of foreign aid at some conditions.] (Question \ref{q:foreign_aid_condition})}\label{fig:foreign_aid_condition}
%     \makebox[\textwidth][c]{\includegraphics[width=\textwidth]{../figures/country_comparison/foreign_aid_condition_positive.pdf}} 
% \end{figure}

% \begin{figure}[h!]
%     \caption[Reasons why foreign aid should not be increased]{Reasons why foreign aid should not be increased (in percent). [Asked to those who wish a decrease or stability of foreign aid.] (Question \ref{q:foreign_aid_no})}\label{fig:foreign_aid_no}
%     \makebox[\textwidth][c]{\includegraphics[width=\textwidth]{../figures/country_comparison/foreign_aid_no_positive.pdf}} 
% \end{figure}

% \begin{figure}[h!]
%     \caption[Willingness to sign a real-stake petition]{Willingness to sign real-stake petition for the Global Climate Scheme or National Redistribution. (Question \ref{q:petition})}\label{fig:petition}
%     \makebox[\textwidth][c]{\includegraphics[width=.8\textwidth]{../figures/country_comparison/petition_only_positive.pdf}} 
% \end{figure}

\begin{figure}[h!]
    \caption[Willingness to sign a real-stake petition]{Willingness to sign real-stake petition for the Global Climate Scheme or National Redistribution, compared to stated support in corresponding subsamples (e.g. support for the GCS in the branch where the petition was about the GCS). (Question \ref{q:petition})}\label{fig:petition}
    \makebox[\textwidth][c]{\includegraphics[width=.8\textwidth]{../figures/country_comparison/petition_comparable_positive.pdf}} 
\end{figure}

\begin{figure}[h!] % TODO? More details?
    \caption[Absolute support for various global policies]{Absolute support for various global policies (Percent of (\textit{somewhat} or \textit{strong}) support). (Questions \ref{q:climate_policies} and \ref{q:other_policies}. See Figure \ref{fig:support} for the relative support.)}\label{fig:support_likert_positive}
    \makebox[\textwidth][c]{\includegraphics[width=\textwidth]{../figures/country_comparison/support_likert_positive.pdf}} 
\end{figure}

% \begin{figure}[h!]
%     \caption{label}\label{fig:climate_policies}
%     \makebox[\textwidth][c]{\includegraphics[width=\textwidth]{../figures/country_comparison/climate_policies.pdf}} 
% \end{figure}

% \begin{figure}[h!]
%     \caption{label}\label{fig:global_policies}
%     \makebox[\textwidth][c]{\includegraphics[width=\textwidth]{../figures/country_comparison/global_policies.pdf}} 
% \end{figure}

\begin{figure}[h!]
    \caption[Preferred approach for international climate negotiations]{Preferred approach of diplomats at international climate negotiations. \\ In international climate negotiations, would you prefer [U.S.] diplomats to defend [own country] interests or global justice? (Question \ref{q:negotiation})}\label{fig:negotiation}
    \makebox[\textwidth][c]{\includegraphics[width=\textwidth]{../figures/country_comparison/negotiation.pdf}} 
\end{figure}

\begin{figure}[h!]
    \caption[Importance of selected issues]{Percent of selected issues viewed as important.\\ ``To what extent do you think the following issues are a problem?'' (Question \ref{q:problem})}\label{fig:problem}
    \makebox[\textwidth][c]{\includegraphics[width=.75\textwidth]{../figures/country_comparison/problem_positive.pdf}} 
\end{figure}

\begin{figure}[h!]
    \caption[Group defended when voting]{Group defended when voting. \\ ``What group do you defend when you vote?'' (Question \ref{q:group_defended})}\label{fig:group_defended}
    \makebox[\textwidth][c]{\includegraphics[width=\textwidth]{../figures/country_comparison/group_defended_agg2.pdf}} 
\end{figure}

% \begin{figure}[h!]
%     \caption{label}\label{fig:group_defended}
%     \makebox[\textwidth][c]{\includegraphics[width=\textwidth]{../figures/country_comparison/group_defended.pdf}} 
% \end{figure}

\begin{figure}[h!] 
    \caption[Mean prioritization of policies]{Mean prioritization of policies. \\Mean number of points allocated policies to express intensity of support (among six policies chosen at random). Blue color means that the policy has been awarded more points than the average policy. (Question \ref{q:points})}\label{fig:points}
    \makebox[\textwidth][c]{\includegraphics[width=\textwidth]{../figures/country_comparison/points_mean.pdf}} 
\end{figure}

\begin{figure}[h!] 
    \caption[Positive prioritization of policies]{Positive prioritization of policies. \\ Percent of people allocating a positive number of points to policies, expressing their support (among six policies chosen at random). (Question \ref{q:points})}\label{fig:points_positive}
    \makebox[\textwidth][c]{\includegraphics[width=\textwidth]{../figures/country_comparison/points_positive.pdf}} 
\end{figure}

\begin{figure}[h!]
    \caption[Charity donation]{Charity donation. \\ ``How much did you give to charities in 2022?'' (Question \ref{q:donation_charities})}\label{fig:donation_charities}
    \makebox[\textwidth][c]{\includegraphics[width=.8\textwidth]{../figures/country_comparison/donation_charities.pdf}} 
\end{figure}

\begin{figure}[h!] 
    \caption[Interest in politics]{Interest in politics. \\ ``To what extent are you interested in politics?'' (Question \ref{q:interested_politics})}\label{fig:interested_politics}
    \makebox[\textwidth][c]{\includegraphics[width=.8\textwidth]{../figures/country_comparison/interested_politics.pdf}} 
\end{figure}

\begin{figure}[h!] 
    \caption[Desired involvement of government]{Desired involvement of government (from 1 to 5). (Question \ref{q:involvement_govt})}\label{fig:involvement_govt}
    \makebox[\textwidth][c]{\includegraphics[width=.9\textwidth]{../figures/country_comparison/involvement_govt.pdf}} 
\end{figure}

\begin{figure}[h!] 
    \caption[Political leaning]{Political leaning on economics (from 1: Left to 5: Right). (Question \ref{q:left_right})}\label{fig:left_right}
    \makebox[\textwidth][c]{\includegraphics[width=.8\textwidth]{../figures/country_comparison/left_right.pdf}} 
\end{figure}

\begin{figure}[h!] 
    \caption[Voted in last election]{Voted in last election. (Question \ref{q:vote_participation})}\label{fig:vote_participation}
    \makebox[\textwidth][c]{\includegraphics[width=.8\textwidth]{../figures/country_comparison/vote_participation.pdf}} 
\end{figure}

\begin{figure}[h!] 
    \caption[Vote in last election]{Vote in last election (aggregated). \textit{PNR} includes people who did not vote or prefer not to answer. (Question \ref{q:vote})}\label{fig:vote}
    \makebox[\textwidth][c]{\includegraphics[width=.75\textwidth]{../figures/country_comparison/vote.pdf}} 
\end{figure}

\begin{figure}[h!] 
    \caption[Perception that survey was biased]{Perception that survey was biased. \\ ``Do you feel that this survey was politically biased?'' (Question \ref{q:survey_biased})}\label{fig:survey_biased}
    \makebox[\textwidth][c]{\includegraphics[width=.7\textwidth]{../figures/country_comparison/survey_biased.pdf}} 
\end{figure}

% \begin{columns}
% \begin{column}{.5\textwidth}
% \begin{multicols}{2}
    \begin{figure}[h!]
        \caption[Classification of open-ended field on extreme poverty]{Opinion on the fight against extreme poverty. \\ ``According to you, what should high-income countries do to fight extreme poverty in low-income countries?'' (Question \ref{q:poverty_field})  \hfill (Back~to~Section~\ref{subsubsec:support_foreign_aid})}\label{fig:poverty_field}
    \begin{subfigure}{.34\textwidth}
        \caption{Elements found in the open-ended field on the question (manually recoded, in percent)}.
        \includegraphics[width=\textwidth]{../figures/country_comparison/poverty_field_positive.pdf}        
    \end{subfigure}
    \hspace{.02\textwidth}
    \begin{subfigure}{.64\textwidth}
        \caption{Keywords found in the open-ended field on the GCS (automatic search ignoring case, in percent).}
        \includegraphics[width=\textwidth]{../figures/country_comparison/poverty_field_contains_positive.pdf}    
    \end{subfigure}
    \end{figure}
% \end{column}
% \begin{column}{.5\textwidth}
    % \begin{figure}[h!]
    %     \caption[Topics of open-ended field on extreme poverty]{Opinion on the fight against extreme poverty. \\ ``According to you, what should high-income countries do to fight extreme poverty in low-income countries?'' \\ Keywords found in the open-ended field on the GCS (automatic search ignoring case, in percent). (Question \ref{q:poverty_field})}\label{fig:poverty_field_contains}
    %     \makebox[\textwidth][c]{\includegraphics[width=\columnwidth]{../figures/country_comparison/poverty_field_contains_positive.pdf}} 
    % \end{figure}
% \end{multicols}
% \end{column}
% \end{columns}


\begin{figure}[h!] 
    \caption[Main attitudes by vote]{Main attitudes by vote (``Right'' spans from Center-right to Far right). \\ (Relative support in percent in Questions \ref{q:gcs_support}, \ref{q:global_tax}, \ref{q:other_policies}, \ref{q:foreign_aid_raise_support}, \ref{q:negotiation}) \hfill (Back~to~Section~\ref{subsec:universalistic})}\label{fig:main_by_vote}
    \makebox[\textwidth][c]{\includegraphics[width=\textwidth]{../figures/country_comparison/main_all_by_vote_share.pdf}} 
\end{figure}

% \begin{figure}[h!] 
%     \caption[Interested to be interviewed]{Interested to be interviewed by a researcher for 30 min through videoconference. (Question \ref{q:interview})}\label{fig:interview}
%     \makebox[\textwidth][c]{\includegraphics[width=\textwidth]{../figures/country_comparison/interview.pdf}} 
% \end{figure}    

% \begin{figure}[h!]
%     \caption{label}\label{fig:share_policies_supported}
%     \makebox[\textwidth][c]{\includegraphics[width=\textwidth]{../figures/country_comparison/share_policies_supported.pdf}} 
% \end{figure} % TODO? uncomment?

% \begin{figure}[h!]
%     \caption{label}\label{fig:vars}
%     \makebox[\textwidth][c]{\includegraphics[width=\textwidth]{../figures/country_comparison/vars.pdf}} 
% \end{figure}

% In Denmark, France and the U.S., the questions with an asterisk were asked differently, asking ``To achieve a given reduction of greenhouse gas emissions globally, costly investments are needed. Ideally, how should countries bear the costs of fighting climate change?''. Instead of the equal right per capita, the item was ``Countries should pay in proportion to their current emissions'', historical responsibilities was worded as ``Countries should pay in proportion to their past emissions (from 1990 onwards)'', then there was an item ``The richest countries should pay it all'', and compensating vulnerable countries was worded as ``The richest countries should pay even more, to help vulnerable countries face adverse consequences: vulnerable countries would then receive money instead of paying''.

\clearpage 
\section{Questionnaire of the global survey (section on global policies)}\label{app:questionnaire_oecd}
%\subsection*{International burden-sharing}
\renewcommand{\theenumi}{\Alph{enumi}}
\begin{enumerate} \item \label{q:scale} At which level(s) do you think public policies to tackle climate change need to be put in place? (Multiple answers are possible) [\textit{Figures \ref{fig:oecd} and \ref{fig:oecd_absolute}}]
\\ \textit{Global; [Federal / European / ...]; [State / National]; Local}
\item Do you agree or disagree with the following statement: ``[country] should take measures to fight climate change.''% TODO! figure
	\\ \textit{Strongly disagree; Somewhat disagree; Neither agree nor disagree; Somewhat agree; Strongly agree}
\item How should [country] climate policies depend on what other countries do?% TODO! figure
 \begin{itemize}
\item If other countries do more, [country] should do...
\item If other countries do less, [country] should do...
\end{itemize}
\textit{Much less; Less; About the same; More; Much more}
\item ~[In all countries but the U.S., Denmark and France]  All countries have signed the Paris agreement that aims to contain global warming ``well below +2 \textdegree{}C\''. To limit global warming to this level, there is a maximum amount of greenhouse gases we can emit globally, called the carbon budget. Each country could aim to emit less than a share of the carbon budget. To respect the global carbon budget, countries that emit more than their national share would pay a fee to countries that emit less than their share. \\ 
Do you support such a policy? [\textit{Figures \ref{fig:oecd} and \ref{fig:oecd_absolute}}]
\\ \textit{Strongly oppose; Somewhat oppose; Neither support nor oppose; Somewhat support; Strongly support}
\item ~[In all countries but the U.S., Denmark and France] Suppose the above policy is in place. How should the carbon budget be divided among countries? [\textit{Figures \ref{fig:oecd} and \ref{fig:oecd_absolute}}]
\\ \textit{The emission share of a country should be proportional to its population, so that each human has an equal right to emit.; The emission share of a country should be proportional to its current emissions, so that those who already emit more have more rights to emit.; Countries that have emitted more over the past decades (from 1990 onwards) should receive a lower emission share, because they have already used some of their fair share.; Countries that will be hurt more by climate change should receive a higher emission share, to compensate them for the damages.}
\item \label{q:burden_sharing_asterisk} ~[In the U.S., Denmark, and France only] To achieve a given reduction of greenhouse gas emissions globally, costly investments are needed. % TODO! figure
Ideally, how should countries bear the costs of fighting climate change?
 \begin{itemize}
\item Countries should pay in proportion to their income
\item Countries should pay in proportion to their current emissions [Used as a substitute to the equal right per capita in Figure \ref{fig:oecd}]
\item Countries should pay in proportion to their past emissions (from 1990 onwards) [Used as a substitute to historical responsibilities in Figure \ref{fig:oecd}]
\item The richest countries should pay it all, so that the poorest countries do not have to pay anything
\item The richest countries should pay even more, to help vulnerable countries face adverse consequences: vulnerable countries would then receive money instead of paying [Used as a substitute to compensating vulnerable countries in Figures \ref{fig:oecd} and \ref{fig:oecd_absolute}]
\end{itemize} 
\textit{Strongly disagree; Somewhat disagree; Neither agree nor disagree; Somewhat agree; Strongly agree}
\item Do you support or oppose establishing a global democratic assembly whose role would be to draft international treaties against climate change? Each adult across the world would have one vote to elect members of the assembly. [\textit{Figures \ref{fig:oecd} and \ref{fig:oecd_absolute}}]
\\ \textit{Strongly oppose; Somewhat oppose; Neither support nor oppose; Somewhat support; Strongly support}
\item Imagine the following policy: a global tax on greenhouse gas emissions funding a global basic income. 
Such a policy would progressively raise the price of fossil fuels (for example, the price of gasoline would increase by [40 cents per gallon] in the first years). Higher prices would encourage people and companies to use less fossil fuels, reducing greenhouse gas emissions. Revenues from the tax would be used to finance a basic income of [\$30] per month to each human adult, thereby lifting the 700 million people who earn less than \$2/day out of extreme poverty. 
The average British person would lose a bit from this policy as they would face [\$130] per month in price increases, which is higher than the [\$30] they would receive.

Do you support or oppose such a policy?  [\textit{Figures \ref{fig:oecd} and \ref{fig:oecd_absolute}}]
\\ \textit{Strongly oppose; Somewhat oppose; Neither support nor oppose; Somewhat support; Strongly support}
\item \label{q:millionaire_tax} Do you support or oppose a tax on all millionaires around the world to finance low-income countries that comply with international standards regarding climate action? 
This would finance infrastructure and public services such as access to drinking water, healthcare, and education. [\textit{Figures \ref{fig:oecd} and \ref{fig:oecd_absolute}}]
\\ \textit{Strongly oppose; Somewhat oppose; Neither support nor oppose; Somewhat support; Strongly support}
\end{enumerate}

% \clearpage
% \section{Questionnaire of US1 %the first U.S. complementary 
% survey}\label{app:questionnaire_US1}

% \begin{figure}[h!]
%     \caption{US1 survey structure}\label{fig:flow_US1}
%     \makebox[\textwidth][c]{\includegraphics[width=\textwidth]{../questionnaire/survey_flow_US1.pdf}} 
% \end{figure}

\renewcommand{\theenumi}{\arabic{enumi}}
\clearpage
\section{Questionnaire of the complementary surveys}\label{app:questionnaire}
\input{app_questionnaire}


\clearpage
\section{Net gains from the Global Climate Scheme}\label{app:gain_gcs}

To specify the GCS, we use the IEA's 2DS scenario \citep{iea_energy_2017}, which is consistent with limiting the global average temperature increase to 2\textdegree{}C with a probability of at least 50\%. The paper by \citet{hood_input_2017} contributing to the Report of the High-Level Commission on Carbon Prices \citep{stern_report_2017} presents a price corridor compatible with this emissions scenario, from which we take the midpoint. The product of these two series provides an estimate of the revenues expected from a global carbon price. We then use the UN median scenario of future population aged over 15 years (\textit{adults}, for short). We derive the basic income that could be paid to all adults by recycling the revenues from the global carbon price: evolving between \$20 and \$30 per month, with a peak in 2030. Accounting for the lower price levels in low-income countries, an additional income of \$30 per month would allow \href{https://data.worldbank.org/indicator/SI.POV.DDAY}{670 million people} to escape extreme poverty, defined with the threshold of \$2.15 per day in purchasing power parity.\footnote{By taking the \href{https://data.worldbank.org/indicator/PA.NUS.PPPC.RF}{ratio} of the World Bank series relating the GDP per capita of Sub-Saharan Africa in \href{https://data.worldbank.org/indicator/NY.GDP.PCAP.PP.KD?locations=ZG&year_high_desc=true}{PPP} and \href{https://data.worldbank.org/indicator/NY.GDP.PCAP.KD?locations=ZG&year_high_desc=true}{nominal}, we obtain the purchasing power of \$1 in this region: \$2.4 in 2019. %See also the price level ratio of PPP conversion factor to market exchange rate.
} 

To estimate the increase in fossil fuel expenditures (or ``cost'') in each country by 2030, we make a key assumption concerning the evolution of the carbon footprints per adult: that they will decrease by the same proportion %$\rho$ 
in each country. We use data from the Global Carbon Project \citep{peters_synthesis_2012}. 
% Noting $e_c$ (resp. $e_c^b$) the carbon footprint per adult of a country $c$ in 2030 (resp. in baseline year $b$), we have $e_c = \rho e_c^b$. Noting $a_c$ (resp. $a_c^b$) the adult population of a country $c$ in 2030 (resp. in baseline year $b$) and $E = \sum_c e_c a_c$ global emissions in 2030, we find $\rho = \frac{E}{\sum_c e_c^b a_c}$. Finally, the average cost per adult in year $y$ is $p \cdot e_c \frac{a_c}{a^y_c}$. %Multiplying country $c$'s carbon footprint per capita with the carbon price $p$ yields its average cost per adult: $p \cdot e_c$. %$\frac{s_c^y}{p^y_c} R$. 
In 2030, the average carbon footprint of a country $c$, $e_c$, evolves from baseline year $b$ proportionally to the evolution of its adult population $\Delta p_c = p^{2030}_c/p^b_c$. Thus, the global share of country $c$'s carbon footprint, $s_c$, is proportional to $\sigma_c = e_c \Delta p_c$, and as countries' shares sum to 1, $s_c = \frac{\sigma_c}{\sum_k \sigma_k}$. Multiplying country $c$'s emission share with global revenues in 2030, $R$, and dividing by $c$'s adult population in year $y$, yields its average cost per adult: $R \cdot s_c / p^y_c$. %$\frac{s_c^y}{p^y_c} R$. 
Using findings from \citet{ivanova_unequal_2020} for Europe and \citet{fremstad_impact_2019} for the U.S., we approximate the median cost as 90\% of the average cost. Finally, the net gain is given by the basic income (\$30 per month) minus the cost. We provided consistent estimates of net gains in all surveys (using $y = b = 2015$), though in the global survey we gave the average net gains vs. the median ones in the complementary surveys. The latter are shown in Figure \ref{fig:median_gain_2015}. 
For the record, Table \ref{tab:gain_gcs.tex} also provides an estimate of \textit{average} net gains (computed with $b = 2019$ and $y = 2030$).\footnote{2015 was the last year of data available when the global questionnaire was conceived (\href{https://stats.oecd.org/Index.aspx?DataSetCode=IO_GHG_2019}{OECD data} was then used -- it does not cover all countries but give identical rounded estimates than those recomputed from the Global Carbon Project data for our complementary surveys). 2030 was chosen as the reference year as it is the date at which global carbon price revenues are expected to peak (and the GCS redistributive effects would be largest), and the GCS could not realistically enter into force before that date. In the surveys, we chose $y = b = 2015$ rather than $b = 2019$ and $y = 2030$ to get more conservative estimates of the monthly cost in the U.S. (\$20 higher than the other option) and in Europe (\euro{5} or £10 higher).}% TODO? remove footnote?
%  ((e/E)*(f/a)*A/F)*R/a

Estimates of the net gains from the Global Climate Scheme are necessarily imprecise, given the uncertainties surrounding the carbon price required to achieve emissions reductions as well as each country's trajectory in terms of emissions and population. These values are highly dependent on future (non-price) climate policies, technical progress, and economic growth of each country, which are only partially known. Integrated Assessment Models have been used to derive a Global Energy Assessment \citep{johansson_global_2012}, a 100\% renewable scenario \citep{greenpeace_energy_2015} as well as Shared Socioeconomic Pathways (SSPs), which include consistent trajectories of population, emissions, and carbon price \citep{riahi_shared_2017,bauer_shared_2017,van_vuuren_energy_2017,fricko_marker_2017}. Instead of using some of these modelling trajectories, we relied on a simple and transparent formula, for a number of reasons. First and foremost, those trajectories describe territorial emissions while we need consumption-based emissions to compute the incidence of the GCS. Second, the carbon price is relatively low in trajectories of SSPs that contain global warming below 2\textdegree{}C (less than \$35/tCO$_\text{2}$ in 2030), so we conservatively chose a method yielding a higher carbon price (\$90 in 2030). Third, modelling results are available only for a few macro regions, while we wanted country by country estimates. Finally, we have checked that the emissions per capita given by our method are broadly in line with alternative methods, even if it tends to overestimate net gains in countries which will decarbonize less rapidly than average.\footnote{Computations with alternative methods can be found on \href{https://github.com/bixiou/global_tax_attitudes/blob/main/code_global/map_GCS_incidence.R}{our public repository}.} For example, although countries' decarbonization plans should realign with the GCS in place, India might still decarbonize less quickly than the European Union, so India's gain and the EU's loss might be overestimated in our computations. For a more sophisticated version of the Global Climate Scheme which includes participation mechanisms preventing middle-income countries (like China) to lose from it and estimations of the Net Present Value by country, see \citet{fabre_global_2023}.  \hfill (Back~to~Section~\ref{box:GCS})

\begin{figure}[h!]
    \caption{Net gains from the Global Climate Scheme.}\label{fig:median_gain_2015}
    \makebox[\textwidth][c]{\includegraphics[width=\textwidth]{../figures/maps/median_gain_2015.pdf}} 
\end{figure}

% \begin{table}[h]\label{tab:gain_gcs}
%     \caption{Net gains from the Global Climate Scheme.} 
%     \makebox[\textwidth][c]{
        % \resizebox*{!}{.7\textheight}{
\clearpage
\begin{multicols}{2}
    \setbox\ltmcbox\vbox{
    \makeatletter\col@number\@ne
        
\begin{longtable}[t]{lrr}
\caption{\label{tab:gain_gcs.tex}Estimated net gain from the GCS in 2030 and carbon footprint by country.}\\
\toprule
  & \makecell{Mean\\net gain\\from\\the GCS\\(\$/month)} & \makecell{CO$_\text{2}$\\footprint\\per adult\\in 2019\\(tCO$_\text{2}$/y)}\\
\midrule
Saudi Arabia & -93 & 24.0\\
United States & -77 & 21.0\\
Australia & -60 & 17.6\\
Canada & -56 & 16.7\\
South Korea & -50 & 15.6\\
Germany & -30 & 11.7\\
Russia & -29 & 11.5\\
Japan & -28 & 11.3\\
Malaysia & -21 & 10.0\\
Iran & -19 & 9.5\\
Poland & -19 & 9.5\\
United Kingdom & -18 & 9.4\\
China & -14 & 8.6\\
Italy & -13 & 8.4\\
South Africa & -11 & 8.0\\
France & -10 & 7.8\\
Iraq* & -8 & 7.4\\
Spain & -6 & 7.0\\
Turkey & -2 & 6.2\\
Algeria* & -1 & 6.0\\
Mexico & 2 & 5.6\\
Ukraine & 2 & 5.6\\
Uzbekistan* & 4 & 5.1\\
Argentina & 5 & 4.9\\
Thailand & 6 & 4.6\\
Egypt & 12 & 3.6\\
Indonesia & 13 & 3.3\\
Colombia & 15 & 3.0\\
Brazil & 15 & 2.9\\
Vietnam & 15 & 2.9\\
Peru & 16 & 2.8\\
Morocco & 16 & 2.7\\
North Korea* & 17 & 2.5\\
India & 18 & 2.4\\
Philippines & 18 & 2.3\\
Pakistan & 22 & 1.6\\
Bangladesh & 24 & 1.1\\
Nigeria & 25 & 1.0\\
Kenya & 25 & 0.9\\
Myanmar* & 26 & 0.9\\
Sudan* & 26 & 0.9\\
Tanzania & 27 & 0.5\\
Afghanistan* & 27 & 0.5\\
Uganda & 28 & 0.4\\
Ethiopia & 28 & 0.3\\
Venezuela & 29 & 0.3\\
DRC* & 30 & 0.1\\
\bottomrule
\end{longtable}
    \unskip
    \unpenalty
    \unpenalty}
    \unvbox\ltmcbox
\end{multicols}
        % }
%     }
    {\footnotesize \textit{Note}: %Emission data is from \cite{peters_synthesis_2012}. 
    Asterisks denote countries where footprint is missing and territorial emissions is used instead. %Estimation of net gains is described in the text. 
    Values differ from Figure \ref{fig:median_gain_2015} as this table present estimates of \textit{mean} net gain per adult in \textit{2030}, not at the present. Only the countries with more than 20 million adults (covering 87\% of the global total) are shown. 
    }
% \end{table}

% \clearpage
% \section{Sources}\label{app:sources}

\clearpage
\section{Determinants of support}\label{app:determinants}

\begin{table}[h]\label{tab:gcs_determinant}
    \caption[Determinants of support for the GCS]{Determinants of support for the Global Climate Scheme. (Back to \ref{subsubsec:support_gcs})} 
    \makebox[\textwidth][c]{
\resizebox*{!}{.73\textheight}{ % 73 is the max when there is a title
        
\begin{tabular}{@{\extracolsep{5pt}}lccccccc} 
\\[-1.8ex]\hline 
\hline \\[-1.8ex] 
 & \multicolumn{7}{c}{\makecell{Supports the Global Climate Scheme}} \\ 
\cline{2-8} 
\\[-1.8ex] & All & United States & Europe & France & Germany & Spain & United Kingdom \\ 
\hline \\[-1.8ex] 
 Country: Germany & $-$0.157$^{***}$ &  & $-$0.144$^{***}$ &  &  &  &  \\ 
  & (0.022) &  & (0.022) &  &  &  &  \\ 
  Country: Spain & $-$0.044$^{*}$ &  & $-$0.026 &  &  &  &  \\ 
  & (0.024) &  & (0.024) &  &  &  &  \\ 
  Country: United Kingdom & $-$0.079$^{***}$ &  & $-$0.104$^{***}$ &  &  &  &  \\ 
  & (0.024) &  & (0.023) &  &  &  &  \\ 
  Country: United States & $-$0.375$^{***}$ &  &  &  &  &  &  \\ 
  & (0.019) &  &  &  &  &  &  \\ 
  Income quartile: 2 & 0.037$^{**}$ & 0.031 & 0.038 & 0.047 & 0.058 & 0.013 & 0.023 \\ 
  & (0.017) & (0.022) & (0.023) & (0.043) & (0.049) & (0.053) & (0.043) \\ 
  Income quartile: 3 & 0.042$^{**}$ & 0.033 & 0.049$^{**}$ & 0.080$^{**}$ & 0.059 & 0.074 & $-$0.052 \\ 
  & (0.017) & (0.024) & (0.024) & (0.040) & (0.052) & (0.056) & (0.052) \\ 
  Income quartile: 4 & 0.056$^{***}$ & 0.063$^{**}$ & 0.010 & 0.018 & $-$0.015 & $-$0.001 & $-$0.005 \\ 
  & (0.018) & (0.026) & (0.026) & (0.047) & (0.055) & (0.056) & (0.057) \\ 
  Diploma: Post secondary & 0.023$^{*}$ & 0.033$^{*}$ & 0.010 & 0.007 & 0.045 & 0.007 & $-$0.010 \\ 
  & (0.012) & (0.017) & (0.018) & (0.029) & (0.039) & (0.039) & (0.039) \\ 
  Age: 25-34 & $-$0.076$^{***}$ & $-$0.083$^{***}$ & $-$0.044 & $-$0.031 & $-$0.077 & $-$0.050 & $-$0.103 \\ 
  & (0.025) & (0.031) & (0.035) & (0.057) & (0.083) & (0.066) & (0.091) \\ 
  Age: 35-49 & $-$0.101$^{***}$ & $-$0.108$^{***}$ & $-$0.069$^{**}$ & $-$0.094$^{*}$ & $-$0.009 & $-$0.168$^{**}$ & $-$0.050 \\ 
  & (0.024) & (0.030) & (0.034) & (0.055) & (0.077) & (0.070) & (0.090) \\ 
  Age: 50-64 & $-$0.137$^{***}$ & $-$0.164$^{***}$ & $-$0.038 & $-$0.039 & $-$0.020 & $-$0.146$^{**}$ & $-$0.017 \\ 
  & (0.024) & (0.030) & (0.035) & (0.056) & (0.082) & (0.067) & (0.087) \\ 
  Age: 65+ & $-$0.116$^{***}$ & $-$0.140$^{***}$ & $-$0.056 & 0.003 & $-$0.045 & $-$0.258$^{***}$ & 0.011 \\ 
  & (0.028) & (0.034) & (0.044) & (0.076) & (0.094) & (0.091) & (0.105) \\ 
  Gender: Man & 0.019$^{*}$ & 0.023 & $-$0.010 & $-$0.014 & $-$0.018 & 0.042 & $-$0.005 \\ 
  & (0.011) & (0.015) & (0.016) & (0.029) & (0.033) & (0.038) & (0.034) \\ 
  Lives with partner & 0.029$^{**}$ & 0.022 & 0.058$^{***}$ & 0.070$^{**}$ & 0.082$^{**}$ & 0.017 & 0.040 \\ 
  & (0.013) & (0.017) & (0.018) & (0.033) & (0.038) & (0.038) & (0.039) \\ 
  Employment status: Retired & $-$0.020 & $-$0.047 & 0.056 & 0.087 & 0.096 & 0.040 & 0.001 \\ 
  & (0.024) & (0.030) & (0.038) & (0.081) & (0.075) & (0.082) & (0.073) \\ 
  Employment status: Student & 0.045 & 0.063 & 0.101$^{**}$ & 0.165$^{*}$ & 0.192$^{**}$ & 0.116 & $-$0.021 \\ 
  & (0.033) & (0.048) & (0.044) & (0.085) & (0.087) & (0.074) & (0.107) \\ 
  Employment status: Working & $-$0.016 & $-$0.021 & 0.011 & 0.082 & 0.006 & $-$0.050 & 0.036 \\ 
  & (0.019) & (0.024) & (0.028) & (0.064) & (0.056) & (0.056) & (0.051) \\ 
  Vote: Center-right or Right & $-$0.331$^{***}$ & $-$0.435$^{***}$ & $-$0.106$^{***}$ & $-$0.131$^{***}$ & $-$0.004 & $-$0.114$^{***}$ & $-$0.081$^{**}$ \\ 
  & (0.013) & (0.017) & (0.019) & (0.035) & (0.044) & (0.038) & (0.041) \\ 
  Vote: PNR/Non-voter & $-$0.184$^{***}$ & $-$0.198$^{***}$ & $-$0.136$^{***}$ & $-$0.196$^{***}$ & $-$0.034 & $-$0.116$^{**}$ & $-$0.108$^{***}$ \\ 
  & (0.016) & (0.022) & (0.021) & (0.039) & (0.043) & (0.046) & (0.040) \\ 
  Vote: Far right & $-$0.396$^{***}$ &  & $-$0.308$^{***}$ & $-$0.493$^{***}$ & $-$0.168$^{***}$ & $-$0.130 & $-$0.314$^{***}$ \\ 
  & (0.032) &  & (0.033) & (0.064) & (0.051) & (0.102) & (0.080) \\ 
  Urban & 0.049$^{***}$ & 0.074$^{***}$ & 0.006 & $-$0.002 & 0.019 & $-$0.014 & 0.017 \\ 
  & (0.012) & (0.018) & (0.016) & (0.029) & (0.032) & (0.036) & (0.033) \\ 
  Race: White &  & $-$0.030 &  &  &  &  &  \\ 
  &  & (0.019) &  &  &  &  &  \\ 
  Region: Northeast &  & 0.009 &  &  &  &  &  \\ 
  &  & (0.023) &  &  &  &  &  \\ 
  Region: South &  & 0.011 &  &  &  &  &  \\ 
  &  & (0.020) &  &  &  &  &  \\ 
  Region: West &  & 0.011 &  &  &  &  &  \\ 
  &  & (0.022) &  &  &  &  &  \\ 
  Swing State &  & $-$0.019 &  &  &  &  &  \\ 
  &  & (0.017) &  &  &  &  &  \\ 
 \hline \\[-1.8ex] 
Constant & 1.048 & 0.729 & 0.89 & 0.7 & 0.732 & 0.935 & 0.886 \\ 
Observations & 7,986 & 4,992 & 2,994 & 977 & 727 & 748 & 542 \\ 
R$^{2}$ & 0.160 & 0.180 & 0.064 & 0.116 & 0.067 & 0.043 & 0.063 \\ 
\hline 
\hline \\[-1.8ex] 
\textit{Note:}  & \multicolumn{7}{r}{$^{*}$p$<$0.1; $^{**}$p$<$0.05; $^{***}$p$<$0.01} \\ 
\end{tabular} 
        }
    }
    {\footnotesize %\textit{Note}: 
    }
\end{table}


\clearpage
\section{Representativeness of the surveys}\label{app:representativeness}


\begin{table}[h!]
    \caption[Sample representativeness of US1, US2, Eu]{Sample representativeness of the complementary surveys. (Back to \ref{par:surveys}) } \label{tab:representativeness_waves}
    \makebox[\textwidth][c]{
        \resizebox*{!}{.80\textheight}{% 73 without notes cf. https://tex.stackexchange.com/questions/13809/resizing-a-table-by-textheight 
        
\begin{tabular}[t]{llllllllll}
\toprule
\multicolumn{1}{c}{} & \multicolumn{3}{c}{US1} & \multicolumn{3}{c}{US2} & \multicolumn{3}{c}{EU} \\
\cmidrule(l{3pt}r{3pt}){2-4} \cmidrule(l{3pt}r{3pt}){5-7} \cmidrule(l{3pt}r{3pt}){8-10}
  & Pop. & Sample & \makecell{Weighted\\sample} & Pop. & Sample & \makecell{Weighted\\sample} & Pop. & Sample & \makecell{Weighted\\sample}\\
\midrule
Sample size &  & 3,000 & 3,000 &  & 678 & 678 &  & 3,000 & 3,000\\
\addlinespace
Gender: Woman & 0.51 & 0.52 & 0.51 & 0.51 & 0.67 & 0.57 & 0.51 & 0.49 & 0.51\\
Gender: Man & 0.49 & 0.47 & 0.49 & 0.49 & 0.32 & 0.43 & 0.49 & 0.51 & 0.49\\
\addlinespace
Income\_quartile: 1 & 0.25 & 0.27 & 0.25 & 0.25 & 0.55 & 0.34 & 0.25 & 0.28 & 0.25\\
Income\_quartile: 2 & 0.25 & 0.24 & 0.25 & 0.25 & 0.29 & 0.32 & 0.25 & 0.23 & 0.25\\
Income\_quartile: 3 & 0.25 & 0.25 & 0.25 & 0.25 & 0.12 & 0.23 & 0.25 & 0.25 & 0.25\\
Income\_quartile: 4 & 0.25 & 0.23 & 0.25 & 0.25 & 0.04 & 0.12 & 0.25 & 0.24 & 0.25\\
\addlinespace
Age: 18-24 & 0.12 & 0.12 & 0.12 & 0.12 & 0.14 & 0.12 & 0.10 & 0.11 & 0.10\\
Age: 25-34 & 0.18 & 0.15 & 0.18 & 0.18 & 0.16 & 0.17 & 0.15 & 0.17 & 0.15\\
Age: 35-49 & 0.24 & 0.25 & 0.24 & 0.24 & 0.25 & 0.25 & 0.24 & 0.25 & 0.24\\
Age: 50-64 & 0.25 & 0.27 & 0.25 & 0.25 & 0.22 & 0.24 & 0.26 & 0.24 & 0.26\\
Age: 65+ & 0.21 & 0.21 & 0.21 & 0.21 & 0.22 & 0.22 & 0.25 & 0.23 & 0.25\\
\addlinespace
Diploma\_25\_64: Below upper secondary & 0.06 & 0.02 & 0.05 & 0.06 & 0.08 & 0.07 & 0.13 & 0.14 & 0.13\\
Diploma\_25\_64: Upper secondary & 0.28 & 0.25 & 0.28 & 0.28 & 0.33 & 0.30 & 0.23 & 0.19 & 0.23\\
Diploma\_25\_64: Post secondary & 0.34 & 0.40 & 0.34 & 0.34 & 0.23 & 0.28 & 0.29 & 0.33 & 0.29\\
\addlinespace
Race: White only & 0.60 & 0.67 & 0.61 & 0.60 & 0.20 & 0.46 &  &  & \\
Race: Hispanic & 0.18 & 0.15 & 0.19 & 0.18 & 0.41 & 0.27 &  &  & \\
Race: Black & 0.13 & 0.16 & 0.14 & 0.13 & 0.36 & 0.20 &  &  & \\
\addlinespace
Region: Northeast & 0.17 & 0.20 & 0.17 & 0.17 & 0.15 & 0.16 &  &  & \\
Region: Midwest & 0.21 & 0.22 & 0.21 & 0.21 & 0.15 & 0.20 &  &  & \\
Region: South & 0.38 & 0.39 & 0.38 & 0.38 & 0.50 & 0.45 &  &  & \\
Region: West & 0.24 & 0.20 & 0.24 & 0.24 & 0.20 & 0.20 &  &  & \\
\addlinespace
Urban: TRUE & 0.73 & 0.78 & 0.74 & 0.73 & 0.73 & 0.69 &  &  & \\
\addlinespace
Employment\_18\_64: Inactive & 0.20 & 0.16 & 0.16 & 0.20 & 0.18 & 0.15 & 0.17 & 0.15 & 0.15\\
Employment\_18\_64: Unemployed & 0.02 & 0.07 & 0.08 & 0.02 & 0.15 & 0.11 & 0.03 & 0.06 & 0.05\\
\addlinespace
Vote: Left & 0.32 & 0.47 & 0.45 & 0.32 & 0.48 & 0.42 & 0.30 & 0.32 & 0.32\\
Vote: Center-right or Right & 0.30 & 0.31 & 0.31 & 0.30 & 0.15 & 0.24 & 0.28 & 0.32 & 0.32\\
Vote: Far right &  &  &  &  &  &  & 0.10 & 0.10 & 0.10\\
\addlinespace
Country: FR &  &  &  &  &  &  & 0.24 & 0.24 & 0.24\\
Country: DE &  &  &  &  &  &  & 0.33 & 0.33 & 0.33\\
Country: ES &  &  &  &  &  &  & 0.18 & 0.18 & 0.18\\
Country: UK &  &  &  &  &  &  & 0.25 & 0.25 & 0.25\\
\addlinespace
Urbanity: Cities &  &  &  &  &  &  & 0.43 & 0.49 & 0.43\\
Urbanity: Towns and suburbs &  &  &  &  &  &  & 0.33 & 0.32 & 0.33\\
Urbanity: Rural &  &  &  &  &  &  & 0.25 & 0.20 & 0.25\\
\bottomrule
\end{tabular}
        }
    }
    {\footnotesize \textit{Note}: This table displays summary statistics of the samples alongside actual population frequencies. %For \textit{Vote}, we regroup candidates or parties into three broad categories and we take abstention into account (but omit this category). 
    %For \textit{Inactivity rate (15-64)}, the sample statistics include the share of respondents aged between 15 and 64 years old who indicated being either ``\textit{Inactive (not searching for a job)},'' a ``\textit{Student},'' or ``\textit{Retired}.'' For \textit{Unemployment rate (15-64)}, the sample statistics include the share of respondents aged between 15 and 64 years old who indicated being ``\textit{Unemployed (searching for a job)}'', (`\textit{Unemployed (searching for a job)},'' ``\textit{Full-time employed},'' ``\textit{Part-time employed},'' or ``\textit{Self-employed}''). For	\textit{Employment rate (15-64)}, the sample statistics include the share of respondents aged between 15 and 64 years old who indicated being either ``\textit{Full-time employed},'' ``\textit{Part-time employed},'' or ``\textit{Self-employed}.'' 
    Detailed sources for each variable and country population frequencies, as well as the definitions of regions, diploma, urbanity, employment, and vote are available in \href{https://github.com/bixiou/global_tax_attitudes/raw/main/questionnaire/specificities.xlsx}{this spreadsheet}. % TODO! Appendix \ref{app:sources}.
    } % TODO add hline before Urbanity, move Country/Urbanity above and add in Notes that quotas are those above the line
\end{table}

\begin{table}[h]
    \caption[Sample representativeness of each European country]{Sample representativeness for each European country. (Back to \ref{par:surveys})} \label{tab:representativeness_EU}
    \makebox[\textwidth][c]{
        \resizebox*{!}{.50\textheight}{% 73 without notes cf. https://tex.stackexchange.com/questions/13809/resizing-a-table-by-textheight 
        
\begin{tabular}[t]{lllllllllllll}
\toprule
\multicolumn{1}{c}{} & \multicolumn{3}{c}{FR} & \multicolumn{3}{c}{DE} & \multicolumn{3}{c}{ES} & \multicolumn{3}{c}{UK} \\
\cmidrule(l{3pt}r{3pt}){2-4} \cmidrule(l{3pt}r{3pt}){5-7} \cmidrule(l{3pt}r{3pt}){8-10} \cmidrule(l{3pt}r{3pt}){11-13}
  & Pop. & Sample & \makecell{Weighted\\sample} & Pop. & Sample & \makecell{Weighted\\sample} & Pop. & Sample & \makecell{Weighted\\sample} & Pop. & Sample & \makecell{Weighted\\sample}\\
\midrule
Sample size &  & 620 & 620 &  & 757 & 757 &  & 543 & 543 &  & 644 & 644\\
\addlinespace
Gender: Woman & 0.52 & 0.49 & 0.54 & 0.51 & 0.53 & 0.58 & 0.51 & 0.55 & 0.60 & 0.50 & 0.26 & 0.32\\
Gender: Man & 0.48 & 0.51 & 0.46 & 0.49 & 0.47 & 0.42 & 0.49 & 0.45 & 0.40 & 0.50 & 0.74 & 0.68\\
\addlinespace
Income\_quartile: 1 & 0.25 & 0.30 & 0.27 & 0.25 & 0.28 & 0.23 & 0.25 & 0.27 & 0.23 & 0.25 & 0.32 & 0.28\\
Income\_quartile: 2 & 0.25 & 0.17 & 0.17 & 0.25 & 0.25 & 0.24 & 0.25 & 0.32 & 0.33 & 0.25 & 0.29 & 0.28\\
Income\_quartile: 3 & 0.25 & 0.22 & 0.22 & 0.25 & 0.29 & 0.30 & 0.25 & 0.25 & 0.25 & 0.25 & 0.20 & 0.21\\
Income\_quartile: 4 & 0.25 & 0.32 & 0.34 & 0.25 & 0.18 & 0.23 & 0.25 & 0.15 & 0.19 & 0.25 & 0.19 & 0.23\\
\addlinespace
Age: 18-24 & 0.12 & 0.08 & 0.06 & 0.09 & 0.18 & 0.15 & 0.08 & 0.17 & 0.15 & 0.10 & 0.02 & 0.02\\
Age: 25-34 & 0.15 & 0.17 & 0.16 & 0.15 & 0.21 & 0.20 & 0.12 & 0.15 & 0.14 & 0.17 & 0.10 & 0.09\\
Age: 35-49 & 0.24 & 0.33 & 0.37 & 0.22 & 0.20 & 0.22 & 0.28 & 0.23 & 0.26 & 0.24 & 0.12 & 0.15\\
Age: 50-64 & 0.24 & 0.20 & 0.19 & 0.28 & 0.23 & 0.26 & 0.27 & 0.25 & 0.27 & 0.25 & 0.28 & 0.33\\
Age: 65+ & 0.25 & 0.23 & 0.22 & 0.26 & 0.18 & 0.18 & 0.25 & 0.19 & 0.19 & 0.24 & 0.48 & 0.42\\
\addlinespace
Urbanity: Cities & 0.47 & 0.51 & 0.43 & 0.37 & 0.47 & 0.40 & 0.52 & 0.67 & 0.62 & 0.40 & 0.37 & 0.31\\
Urbanity: Towns and suburbs & 0.19 & 0.18 & 0.18 & 0.40 & 0.34 & 0.34 & 0.22 & 0.27 & 0.29 & 0.42 & 0.46 & 0.47\\
Urbanity: Rural & 0.34 & 0.30 & 0.39 & 0.23 & 0.18 & 0.25 & 0.26 & 0.06 & 0.08 & 0.18 & 0.17 & 0.22\\
\addlinespace
Diploma\_25\_64: Below upper secondary & 0.11 & 0.22 & 0.18 & 0.10 & 0.17 & 0.16 & 0.24 & 0.10 & 0.09 & 0.12 & 0.10 & 0.08\\
Diploma\_25\_64: Upper secondary & 0.26 & 0.15 & 0.24 & 0.27 & 0.11 & 0.18 & 0.16 & 0.15 & 0.23 & 0.21 & 0.18 & 0.29\\
Diploma\_25\_64: Post secondary & 0.26 & 0.33 & 0.30 & 0.29 & 0.36 & 0.33 & 0.28 & 0.38 & 0.33 & 0.33 & 0.23 & 0.20\\
\addlinespace
Employment\_18\_64: Inactive & 0.20 & 0.18 & 0.16 & 0.15 & 0.16 & 0.14 & 0.20 & 0.16 & 0.15 & 0.16 & 0.14 & 0.15\\
Employment\_18\_64: Unemployed & 0.04 & 0.05 & 0.05 & 0.02 & 0.04 & 0.04 & 0.07 & 0.10 & 0.10 & 0.02 & 0.03 & 0.03\\
\addlinespace
Vote: Left & 0.23 & 0.18 & 0.17 & 0.37 & 0.42 & 0.42 & 0.33 & 0.37 & 0.38 & 0.25 & 0.27 & 0.27\\
Vote: Center-right or Right & 0.26 & 0.31 & 0.32 & 0.28 & 0.26 & 0.27 & 0.18 & 0.22 & 0.22 & 0.36 & 0.50 & 0.50\\
Vote: Far right & 0.23 & 0.23 & 0.24 & 0.08 & 0.07 & 0.08 & 0.09 & 0.08 & 0.07 & 0.01 & 0.03 & 0.04\\
\bottomrule
\end{tabular}
        }
    }
    % TODO add explanatory note
    {\footnotesize \textit{Note}: This table displays summary statistics of the samples alongside actual population frequencies. In this Table, weights are defined at the country level.  %For \textit{Vote}, we regroup candidates or parties into three broad categories and we take abstention into account (but omit this category). 
    %For \textit{Inactivity rate (15-64)}, the sample statistics include the share of respondents aged between 15 and 64 years old who indicated being either ``\textit{Inactive (not searching for a job)},'' a ``\textit{Student},'' or ``\textit{Retired}.'' For \textit{Unemployment rate (15-64)}, the sample statistics include the share of respondents aged between 15 and 64 years old who indicated being ``\textit{Unemployed (searching for a job)}'', (`\textit{Unemployed (searching for a job)},'' ``\textit{Full-time employed},'' ``\textit{Part-time employed},'' or ``\textit{Self-employed}''). For	\textit{Employment rate (15-64)}, the sample statistics include the share of respondents aged between 15 and 64 years old who indicated being either ``\textit{Full-time employed},'' ``\textit{Part-time employed},'' or ``\textit{Self-employed}.'' 
    Detailed sources for each variable and country population frequencies, as well as the definitions of regions, diploma, urbanity, employment, and vote are available in \href{https://github.com/bixiou/global_tax_attitudes/raw/main/questionnaire/specificities.xlsx}{this spreadsheet}. % TODO Appendix \ref{app:sources}.
    }
\end{table}

Similar tables for the global surveys can be found in \citet{dechezlepretre_fighting_2022}.

\clearpage
\section{Attrition analysis}\label{app:attrition}

\begin{table}[h]\label{tab:attrition_US1}
    \caption[Attrition analysis: US1]{Attrition analysis for the US1 survey.} 
    \makebox[\textwidth][c]{
\resizebox*{!}{.73\textheight}{ % 73 is the max when there is a title
        
\begin{tabular}{@{\extracolsep{5pt}}lccccc} 
\\[-1.8ex]\hline 
\hline \\[-1.8ex] 
\\[-1.8ex] & \makecell{Dropped out} & \makecell{Dropped out\\after\\socio-eco} & \makecell{Failed\\attention test} & \makecell{Duration\\(in min)} & \makecell{Duration\\below\\4 min} \\ 
\\[-1.8ex] & (1) & (2) & (3) & (4) & (5)\\ 
\hline \\[-1.8ex] 
Mean & 0.08 & 0.059 & 0.082 & 21.198 & 0.016  \\ \hline \\[-1.8ex]
 Income quartile: 3 & 0.001 & 0.001 & $-$0.022$^{*}$ & $-$0.770 & $-$0.009 \\ 
  & (0.010) & (0.010) & (0.012) & (3.203) & (0.006) \\ 
  Income quartile: 4 & 0.004 & 0.004 & $-$0.029$^{**}$ & 0.775 & $-$0.004 \\ 
  & (0.012) & (0.012) & (0.012) & (2.737) & (0.007) \\ 
  Diploma: Post secondary & $-$0.012 & $-$0.012 & 0.011 & $-$4.141 & $-$0.004 \\ 
  & (0.012) & (0.012) & (0.014) & (2.803) & (0.007) \\ 
  Age: 25-34 & 0.006 & 0.006 & 0.001 & 1.004 & 0.004 \\ 
  & (0.009) & (0.009) & (0.009) & (2.509) & (0.005) \\ 
  Age: 35-49 & $-$0.058$^{***}$ & $-$0.058$^{***}$ & 0.001 & $-$0.859 & $-$0.032$^{**}$ \\ 
  & (0.015) & (0.015) & (0.019) & (2.503) & (0.013) \\ 
  Age: 50-64 & $-$0.053$^{***}$ & $-$0.053$^{***}$ & 0.001 & 4.431 & $-$0.033$^{***}$ \\ 
  & (0.015) & (0.015) & (0.017) & (2.945) & (0.013) \\ 
  Age: 65+ & $-$0.031$^{**}$ & $-$0.031$^{**}$ & $-$0.055$^{***}$ & 5.358$^{**}$ & $-$0.041$^{***}$ \\ 
  & (0.015) & (0.015) & (0.016) & (2.556) & (0.012) \\ 
  Race: Black & 0.034$^{*}$ & 0.034$^{*}$ & $-$0.061$^{***}$ & 8.417$^{**}$ & $-$0.050$^{***}$ \\ 
  & (0.018) & (0.018) & (0.016) & (4.117) & (0.012) \\ 
  Race: Hispanic & 0.026$^{**}$ & 0.026$^{**}$ & 0.017 & 7.964$^{***}$ & 0.003 \\ 
  & (0.010) & (0.010) & (0.014) & (2.759) & (0.008) \\ 
  Gender: Man & 0.007 & 0.007 & 0.120$^{**}$ & $-$2.808 & 0.031 \\ 
  & (0.024) & (0.024) & (0.047) & (1.804) & (0.029) \\ 
  Region: Northeast & $-$0.049$^{***}$ & $-$0.049$^{***}$ & 0.020$^{**}$ & $-$0.344 & 0.00003 \\ 
  & (0.007) & (0.007) & (0.009) & (2.339) & (0.005) \\ 
  Region: South & 0.0002 & 0.0002 & 0.010 & $-$4.919 & $-$0.004 \\ 
  & (0.011) & (0.011) & (0.013) & (4.796) & (0.007) \\ 
  Region: West & $-$0.004 & $-$0.004 & 0.009 & $-$0.945 & $-$0.004 \\ 
  & (0.009) & (0.009) & (0.011) & (4.520) & (0.006) \\ 
  Urban & 0.005 & 0.005 & $-$0.020 & $-$4.232 & $-$0.004 \\ 
  & (0.011) & (0.011) & (0.013) & (4.485) & (0.007) \\ 
  urban & 0.001 & 0.001 & 0.008 & 4.599$^{**}$ & $-$0.005 \\ 
  & (0.009) & (0.009) & (0.010) & (2.221) & (0.006) \\ 
 \hline \\[-1.8ex] 

Observations & 5,719 & 5,719 & 3,252 & 3,044 & 3,044 \\ 
R$^{2}$ & 0.023 & 0.023 & 0.030 & 0.006 & 0.016 \\ 
\hline 
\hline \\[-1.8ex] 
\end{tabular} 
        }
    }
    {\footnotesize %\textit{Note}: 
    }
\end{table}

\begin{table}[h]\label{tab:attrition_US2}
    \caption[Attrition analysis: US2]{Attrition analysis for the US2 survey.} 
    \makebox[\textwidth][c]{
\resizebox*{!}{.73\textheight}{ % 73 is the max when there is a title
        
\begin{tabular}{@{\extracolsep{5pt}}lccccc} 
\\[-1.8ex]\hline 
\hline \\[-1.8ex] 
\\[-1.8ex] & \makecell{Dropped out} & \makecell{Dropped out\\after\\socio-eco} & \makecell{Failed\\attention test} & \makecell{Duration\\(in min)} & \makecell{Duration\\below\\4 min} \\ 
\\[-1.8ex] & (1) & (2) & (3) & (4) & (5)\\ 
\hline \\[-1.8ex] 
Mean & 0.105 & 0.08 & 0.112 & 21.78 & 0.041  \\ \hline \\[-1.8ex]
 Income quartile: 2 & 0.007 & 0.007 & $-$0.053$^{***}$ & 1.441 & $-$0.043$^{***}$ \\ 
  & (0.022) & (0.022) & (0.020) & (3.244) & (0.015) \\ 
  Income quartile: 3 & 0.020 & 0.020 & $-$0.011 & 45.106 & $-$0.033 \\ 
  & (0.030) & (0.030) & (0.034) & (46.289) & (0.025) \\ 
  Income quartile: 4 & $-$0.002 & $-$0.002 & $-$0.003 & 1.041 & $-$0.079$^{***}$ \\ 
  & (0.043) & (0.043) & (0.061) & (10.058) & (0.019) \\ 
  Diploma: Post secondary & $-$0.043$^{**}$ & $-$0.043$^{**}$ & $-$0.043$^{**}$ & 9.394 & 0.026 \\ 
  & (0.021) & (0.021) & (0.020) & (9.764) & (0.016) \\ 
  Age: 25-34 & 0.053$^{*}$ & 0.053$^{*}$ & $-$0.045 & $-$7.393 & 0.017 \\ 
  & (0.030) & (0.030) & (0.042) & (6.961) & (0.033) \\ 
  Age: 35-49 & 0.052$^{**}$ & 0.052$^{**}$ & $-$0.042 & 17.468 & 0.006 \\ 
  & (0.026) & (0.026) & (0.039) & (16.385) & (0.029) \\ 
  Age: 50-64 & 0.066$^{**}$ & 0.066$^{**}$ & $-$0.071$^{*}$ & $-$7.421 & $-$0.042$^{*}$ \\ 
  & (0.029) & (0.029) & (0.040) & (9.109) & (0.025) \\ 
  Age: 65+ & 0.057$^{*}$ & 0.057$^{*}$ & $-$0.107$^{***}$ & $-$1.734 & $-$0.052$^{**}$ \\ 
  & (0.030) & (0.030) & (0.037) & (9.343) & (0.025) \\ 
  Race: Black & 0.100$^{***}$ & 0.100$^{***}$ & $-$0.011 & 20.168 & $-$0.016 \\ 
  & (0.021) & (0.021) & (0.033) & (14.147) & (0.023) \\ 
  Race: Hispanic & 0.062$^{***}$ & 0.062$^{***}$ & $-$0.054 & $-$4.035 & $-$0.028 \\ 
  & (0.019) & (0.019) & (0.033) & (7.283) & (0.023) \\ 
  Gender: Man & $-$0.050$^{***}$ & $-$0.050$^{***}$ & 0.015 & 13.563 & 0.017 \\ 
  & (0.018) & (0.018) & (0.023) & (16.255) & (0.017) \\ 
  Region: Northeast & $-$0.018 & $-$0.018 & 0.030 & $-$4.964 & 0.014 \\ 
  & (0.030) & (0.030) & (0.043) & (4.837) & (0.029) \\ 
  Region: South & 0.013 & 0.013 & $-$0.029 & 10.628 & 0.007 \\ 
  & (0.024) & (0.024) & (0.034) & (13.411) & (0.022) \\ 
  Region: West & 0.006 & 0.006 & $-$0.023 & 0.452 & 0.010 \\ 
  & (0.029) & (0.029) & (0.038) & (5.076) & (0.027) \\ 
  Urban & 0.050$^{**}$ & 0.050$^{**}$ & 0.007 & 8.278 & 0.001 \\ 
  & (0.019) & (0.019) & (0.026) & (6.513) & (0.018) \\ 
 \hline \\[-1.8ex] 

Observations & 946 & 946 & 777 & 706 & 706 \\ 
R$^{2}$ & 0.042 & 0.042 & 0.046 & 0.023 & 0.043 \\ 
\hline 
\hline \\[-1.8ex] 
\end{tabular} 
        }
    }
    {\footnotesize %\textit{Note}: 
    }
\end{table}

\begin{table}[h]\label{tab:attrition_EU}
    \caption[Attrition analysis: Eu]{Attrition analysis for the Eu survey.} 
    \makebox[\textwidth][c]{
\resizebox*{!}{.73\textheight}{ % 73 is the max when there is a title
        
\begin{tabular}{@{\extracolsep{5pt}}lccccc} 
\\[-1.8ex]\hline 
\hline \\[-1.8ex] 
\\[-1.8ex] & \makecell{Dropped out} & \makecell{Dropped out\\after\\socio-eco} & \makecell{Failed\\attention test} & \makecell{Duration\\(in min)} & \makecell{Duration\\below\\6 min} \\ 
\\[-1.8ex] & (1) & (2) & (3) & (4) & (5)\\ 
\hline \\[-1.8ex] 
Mean & 0.067 & 0.044 & 0.151 & 54.602 & 0.039  \\ \hline \\[-1.8ex]
 Income quartile: 3 & 0.001 & $-$0.001 & $-$0.031$^{**}$ & 27.825 & $-$0.015 \\ 
  & (0.013) & (0.012) & (0.013) & (20.371) & (0.010) \\ 
  Income quartile: 4 & 0.002 & 0.001 & $-$0.061$^{***}$ & 0.612 & $-$0.022$^{**}$ \\ 
  & (0.014) & (0.013) & (0.011) & (11.887) & (0.010) \\ 
  Diploma: Post secondary & $-$0.022 & $-$0.024$^{*}$ & $-$0.042$^{***}$ & 13.029 & $-$0.019$^{*}$ \\ 
  & (0.014) & (0.014) & (0.013) & (19.608) & (0.010) \\ 
  Age: 25-34 & $-$0.006 & $-$0.005 & $-$0.033$^{***}$ & 5.978 & $-$0.008 \\ 
  & (0.011) & (0.010) & (0.009) & (12.265) & (0.007) \\ 
  Age: 35-49 & 0.028$^{**}$ & 0.025$^{**}$ & 0.033$^{*}$ & 33.335 & $-$0.004 \\ 
  & (0.013) & (0.013) & (0.018) & (20.624) & (0.018) \\ 
  Age: 50-64 & 0.048$^{***}$ & 0.047$^{***}$ & $-$0.006 & 32.456$^{**}$ & $-$0.013 \\ 
  & (0.013) & (0.012) & (0.016) & (14.803) & (0.016) \\ 
  Age: 65+ & 0.074$^{***}$ & 0.073$^{***}$ & $-$0.010 & 41.300$^{**}$ & $-$0.063$^{***}$ \\ 
  & (0.014) & (0.014) & (0.017) & (20.533) & (0.015) \\ 
  Gender: Man & 0.142$^{***}$ & 0.140$^{***}$ & $-$0.011 & 26.513$^{**}$ & $-$0.063$^{***}$ \\ 
  & (0.016) & (0.016) & (0.017) & (12.755) & (0.015) \\ 
  Urban & $-$0.031$^{***}$ & $-$0.031$^{***}$ & 0.013 & $-$24.850$^{*}$ & 0.010 \\ 
  & (0.009) & (0.009) & (0.009) & (14.378) & (0.007) \\ 
  urban & $-$0.010 & $-$0.009 & 0.016$^{*}$ & 13.704 & $-$0.005 \\ 
  & (0.009) & (0.009) & (0.008) & (15.465) & (0.007) \\ 
 \hline \\[-1.8ex] 

Observations & 3,963 & 3,963 & 3,326 & 3,115 & 3,115 \\ 
R$^{2}$ & 0.026 & 0.026 & 0.021 & 0.003 & 0.024 \\ 
\hline 
\hline \\[-1.8ex] 
\end{tabular} 
        }
    }
    {\footnotesize %\textit{Note}: 
    }
\end{table} 

\clearpage
\listoftables
\listoffigures
% WPcomment
%% Here is the endmatter stuff: Supplementary Info, etc.
%% Use \item's to separate, default label is "Acknowledgements"
% \begin{addendum} % 177 words
%  \item We are grateful for financial support from the University of Amsterdam and TU Berlin. We are grateful for financial support from the OECD, the French Ministry of Foreign Affairs, the French Conseil d’Analyse Economique and the Spanish Ministry for the Ecological Transition and Demographic Challenge. We also acknowledge support from the Grantham Foundation for the Protection of the Environment and the Economic and Social Research Council through the Centre for Climate Change Economics and Policy. We thank Antoine Dechezleprêtre, Tobias Kruse, Bluebery Planterose, % particularly Bluebery and Ana who helped for the figures
% Ana Sanchez Chico, and Stefanie Stantcheva for their invaluable inputs for the project. We thank Auriane Meilland for feedback. We thank Laura Schepp, Martín Fernández-Sánchez, Samuel Gervais, Samuel Haddad, and Guadalupe Manzo for assistance in the translation. 
%  \item[Registration] The project %is approved by IRB at Harvard University (IRB21-0137), and 
%  was preregistered in the Open Science Foundation registry (osf.io/fy6gd).
%  \item[Competing Interests] The authors declare that they have no
% competing interests.
% \item[JEL codes] P48, Q58, H23, Q54.
% \item[Keywords] Climate change, global policies, cap-and-trade, perceptions, survey, inequality, wealth tax.
%  \item[Correspondence] Correspondence and requests for materials
% should be addressed to Adrien Fabre~(email: fabre.adri1@gmail.com).
% \end{addendum}

%%
%% TABLES
%%
%% If there are any tables, put them here.
%%

% \begin{table}
% \centering
% \caption{This is a table with scientific results.}
% \medskip
% \begin{tabular}{ccccc}
% \hline
% 1 & 2 & 3 & 4 & 5\\
% \hline
% aaa & bbb & ccc & ddd & eee\\
% aaaa & bbbb & cccc & dddd & eeee\\
% aaaaa & bbbbb & ccccc & ddddd & eeeee\\
% aaaaaa & bbbbbb & cccccc & dddddd & eeeeee\\
% 1.000 & 2.000 & 3.000 & 4.000 & 5.000\\
% \hline
% \end{tabular}
% \end{table}

\end{document}
