\clearpage
\section{Raw results% from the complementary surveys
}\label{app:raw_results}
% /!\ Do not replace by app_desc_stats_US1 as the latter also contains figures that are already in the main text
% TODO? add country-specific prioritization? No, it's in (separate) country appendices.
% TODO! add share who click on info or reminder
% TODO! Appendix Sources or at least clean up specificities.xlsx

Country-specific raw results are also available as supplementary material files:  \href{https://github.com/bixiou/international_attitudes_toward_global_policies/raw/main/paper/app_desc_stats_US.pdf}{US}, \href{https://github.com/bixiou/international_attitudes_toward_global_policies/raw/main/paper/app_desc_stats_EU.pdf}{EU}, \href{https://github.com/bixiou/international_attitudes_toward_global_policies/raw/main/paper/app_desc_stats_FR.pdf}{FR}, \href{https://github.com/bixiou/international_attitudes_toward_global_policies/raw/main/paper/app_desc_stats_DE.pdf}{DE}, \href{https://github.com/bixiou/international_attitudes_toward_global_policies/raw/main/paper/app_desc_stats_ES.pdf}{ES}, \href{https://github.com/bixiou/international_attitudes_toward_global_policies/raw/main/paper/app_desc_stats_UK.pdf}{UK}.

\begin{figure}[h!]
    \caption[Absolute support for global climate policies]{Absolute support for global climate policies. \\ Share of \textit{Somewhat} or \textit{Strongly support} (in percent, $n$ = 40,680). The color blue denotes an absolute majority. See Figure \ref{fig:oecd} for the relative support. (Questions \ref{q:scale}-\ref{q:millionaire_tax} of the global survey.)% Reproduced from \citealp{dechezlepretre_fighting_2022}, Figure A20.)
    } 
    \makebox[\textwidth][c]{\includegraphics[width=1.2\textwidth]{../figures/OECD/Heatplot_global_tax_attitudes_positive.pdf}}\label{fig:oecd_absolute}% with dependence on others (absent from OECD): Heatplot_burden_share_all_positive_countries
    {\footnotesize \\ *In Denmark, France and the U.S., the questions with an asterisk were asked differently, cf. Question \ref{q:burden_sharing_asterisk}. } 
\end{figure}

\begin{figure}[h!]
    \caption[Comprehension]{Correct answers to comprehension questions (in percent). (Questions \ref{q:understood_gcs}-\ref{q:understood_both})}\label{fig:understood_each}
    \makebox[\textwidth][c]{\includegraphics[width=\textwidth]{../figures/country_comparison/understood_each_positive.pdf}} 
\end{figure}

\begin{figure}[h!]
    \caption[Comprehension score]{Number of correct answers to comprehension questions (mean). (Section \ref{subsec:gcs_stated_support}, Questions \ref{q:understood_gcs}-\ref{q:understood_both})}\label{fig:understood_score}
    \makebox[\textwidth][c]{\includegraphics[width=\textwidth]{../figures/country_comparison/understood_score_mean.pdf}} 
\end{figure}

% \begin{figure}[h!]
%     \caption[Support for the Global Climate Scheme]{Support for the GCS, NR and the combination of GCS, NR and C. (Questions \ref{q:gcs_support}, \ref{q:nr_support} and \ref{q:crg_support})}\label{fig:support_binary}
%     \makebox[\textwidth][c]{\includegraphics[width=.9\textwidth]{../figures/country_comparison/support_binary.pdf}} 
% \end{figure}

% \begin{figure}[h!]
%     \caption[Beliefs about support for the GCS and NR]{Beliefs regarding the support for the GCS and NR. (Questions \ref{q:gcs_belief} and \ref{q:nr_belief})}\label{fig:belief}
%     \makebox[\textwidth][c]{\includegraphics[width=.8\textwidth]{../figures/country_comparison/belief.pdf}} 
% \end{figure}

\begin{figure}[h!]
    \caption[List experiment]{List experiment: mean number of supported policies. (Section \ref{subsubsec:list_exp}, Question \ref{q:list_exp})}\label{fig:list_exp}
    \makebox[\textwidth][c]{\includegraphics[width=.7\textwidth]{../figures/country_comparison/list_exp_mean.pdf}} 
\end{figure}

\begin{figure}[h!]
    \caption[Conjoint analyses 1 and 2]{Conjoint analyses 1 and 2. (Questions \ref{q:conjoint_a}-\ref{q:conjoint_b}, Back to Section \ref{subsubsec:conjoint})}\label{fig:conjoint}
    \makebox[\textwidth][c]{\includegraphics[width=.8\textwidth]{../figures/country_comparison/conjoint_ab_all_positive.pdf}} 
\end{figure}

% \begin{figure}[h!] % already in text
%     \caption{[Asked only to non-Republicans] Conjoint analysis n°4: random programs at the Democratic primary. (Question \ref{q:conjoint_r})}\label{fig:ca_r}
%     \makebox[\textwidth][c]{\includegraphics[width=\textwidth]{../figures/country_comparison/ca_r.png}} 
% \end{figure}

% \begin{figure}[h!]
%     \caption[Influence of the GCS on preferred platform]{Influence of the GCS on preferred platform:\\ Preference for a random platform A that contains the Global Climate Scheme rather than a platform B that does not (in percent). (Question \ref{q:conjoint_d}; in the U.S., asked only to non-Republicans.)}\label{fig:conjoint_left_ag_b}
%     \makebox[\textwidth][c]{\includegraphics[width=\textwidth]{../figures/country_comparison/conjoint_left_ag_b_binary_positive.pdf}} 
% \end{figure}

\begin{figure}[h] 
  \caption[Preferences for various policies in political platforms (original)]{Effects of the presence of a policy (rather than none from this domain) in a random platform on the likelihood that it is preferred to another random platform. (See English translations in Figure \ref{fig:ca_r}; Question \ref{q:conjoint_r}%; in the U.S., asked only to non-Republicans.
  )}\label{fig:ca_r_en}
    \begin{subfigure}{.97\textwidth}
      \subcaption{Germany}
      \includegraphics[width=.97\textwidth]{../figures/DE/ca_r.png}
    \end{subfigure}
    \begin{subfigure}{.98\textwidth}
      \subcaption{France}
      \includegraphics[width=.98\textwidth]{../figures/FR/ca_r.png}
    \end{subfigure}
    \begin{subfigure}{.98\textwidth}
      \subcaption{Spain}
      \includegraphics[width=.98\textwidth]{../figures/ES/ca_r.png}
    \end{subfigure}
\end{figure}

\begin{figure}[h!]
    \caption[Perceptions of the GCS]{Perceptions of the GCS. Elements seen as important for supporting the GCS in a 4-Likert scale (in percent). (Question \ref{q:gcs_important})  \hfill (Back~to~Section~\ref{subsubsec:pros_cons})}\label{fig:gcs_important}
    \makebox[\textwidth][c]{\includegraphics[width=\textwidth]{../figures/country_comparison/gcs_important_positive.pdf}} 
\end{figure}

\begin{figure}[h!]
    \caption[Classification of open-ended field on the GCS]{Perceptions of the GCS. Elements found in the open-ended field on the GCS (manually recoded, in percent). \\ ``When thinking about the Global climate scheme, what comes to
    your mind?
    \\ Please list pros and cons of the Global climate scheme.'' (Question \ref{q:gcs_field}) \hfill (Back~to~Section~\ref{subsubsec:pros_cons})}\label{fig:gcs_field}
    \makebox[\textwidth][c]{\includegraphics[width=.75\textwidth]{../figures/country_comparison/gcs_field_positive.pdf}} 
\end{figure}

\begin{figure}[h!]
    \caption[Topics of open-ended field on the GCS]{Perceptions of the GCS. Keywords found in the open-ended field on the GCS (automatic search ignoring case, in percent). \\ ``When thinking about the Global climate scheme, what comes to
    your mind?
    \\ Please list pros and cons of the Global climate scheme.'' (Question \ref{q:gcs_field}) \hfill (Back~to~Section~\ref{subsubsec:pros_cons})}\label{fig:gcs_field_contains}
    \makebox[\textwidth][c]{\includegraphics[width=\textwidth]{../figures/country_comparison/gcs_field_contains_positive.pdf}} 
\end{figure}

\begin{table}[h]
    \caption[Campaign and bandwagon effects on the support for the GCS.]{Effects on the support for the GCS of a question on its pros and cons (either in open-ended of closed format) and on information about the actual support, in the U.S. (See Section \ref{subsec:questionnaire_perceptions} in the \textit{US2} Questionnaire)  \hfill (Back~to~Section~\ref{subsubsec:pros_cons})} \label{tab:branch_gcs}
    \makebox[\textwidth][c]{
        
\begin{tabular}{@{\extracolsep{5pt}}lcccc} 
\\[-1.8ex]\hline 
\hline \\[-1.8ex] 
 & \multicolumn{4}{c}{Support} \\ 
\cline{2-5} 
\\[-1.8ex] & \multicolumn{2}{c}{Global Climate Scheme} & \multicolumn{2}{c}{National Redistribution} \\ 
\\[-1.8ex] & (1) & (2) & (3) & (4)\\ 
\hline \\[-1.8ex] 
Control group mean & 0.557 & 0.557 & 0.569 & 0.569  \\ \hline \\[-1.8ex]
 Treatment: Open\mbox{-}ended field on GCS pros \& cons & $-$0.073$^{**}$ & $-$0.073$^{**}$ & $-$0.035 & $-$0.031 \\ 
  & (0.035) & (0.031) & (0.035) & (0.032) \\ 
  Treatment: Closed questions on GCS pros \& cons & $-$0.109$^{***}$ & $-$0.096$^{***}$ & $-$0.065$^{*}$ & $-$0.062$^{**}$ \\ 
  & (0.034) & (0.031) & (0.034) & (0.031) \\ 
  Treatment: Info on actual support for GCS and NR & $-$0.021 & $-$0.017 & 0.048 & 0.054$^{*}$ \\ 
  & (0.034) & (0.031) & (0.033) & (0.031) \\ 
 \hline \\[-1.8ex] 
Includes controls &  & \checkmark &  & \checkmark \\

Observations & 2,000 & 1,995 & 2,000 & 1,995 \\ 
R$^{2}$ & 0.007 & 0.169 & 0.007 & 0.153 \\ 
\hline 
\hline \\[-1.8ex] 
\end{tabular} 
    }
    {\footnotesize %\textit{Note}: 
    }
\end{table}

\begin{figure}[h!]
    \caption[Donation to Africa vs. own country]{Donation in case of lottery win, depending on the recipient's (randomly drawn) nationality (mean). (Question \ref{q:donation})\hfill (Back~to~Section~\ref{subsec:universalistic})}\label{fig:donation}
    \makebox[\textwidth][c]{\includegraphics[width=.8\textwidth]{../figures/country_comparison/donation_mean.pdf}} 
\end{figure}

\begin{table}[h]
    \caption[Donation to Africa vs. own country]{Donation in case of lottery win, depending on the recipient's (randomly drawn) nationality. (Question \ref{q:donation})\hfill (Back~to~Section~\ref{subsec:universalistic})} \label{tab:donation}
    \makebox[\textwidth][c]{\input{../tables/continents/donation_interaction.tex}}
\end{table}

\begin{figure}[h!]
    \caption[Support for a global wealth tax]{Support for a global wealth tax. \\
    ``Do you support or oppose a tax on millionaires of all countries to finance low-
    income countries? \\
    Such tax would finance infrastructure and public services such as access to drinking water, healthcare, and education.'' (Question \ref{q:global_tax})}\label{fig:global_tax}
    \makebox[\textwidth][c]{\includegraphics[width=\textwidth]{../figures/country_comparison/global_tax_support.pdf}} 
\end{figure}

\begin{figure}[h!]
    \caption[Support for a national wealth tax]{Support for a national wealth tax. \\ ``Do you support or oppose a tax on millionaires in [the U.S.] to finance [\textit{US2}: affordable housing and universal childcare/pre-K; \textit{Eu}: finance government hospitals and schools]?'' %financing public services like healthcare, education, and social housing. 
    (Question \ref{q:national_tax})}\label{fig:national_tax}
    \makebox[\textwidth][c]{\includegraphics[width=\textwidth]{../figures/country_comparison/national_tax_support.pdf}} 
\end{figure}

\begin{figure}[h!]
    \caption[Preferred share of global tax for low-income countries]{Preferred share of global wealth tax revenues that should be pooled to finance low-income countries. (Question \ref{q:global_tax_global_share})}\label{fig:global_tax_global_share}
    \makebox[\textwidth][c]{\includegraphics[width=\textwidth]{../figures/country_comparison/global_tax_global_share.pdf}} 
\end{figure}

\begin{figure}[h!]
    \caption[Support for sharing half of global tax revenues with low-income countries]{Support for sharing half of global tax revenues with low-income countries, rather that each country retaining all the revenues it collects (in percent). (Question \ref{q:global_tax_sharing})}\label{fig:global_tax_sharing}
    \makebox[\textwidth][c]{\includegraphics[width=\textwidth]{../figures/country_comparison/global_tax_sharing_positive.pdf}} 
\end{figure}

% \begin{figure}
%     \centering 
%     \caption{Your previous answer shows that you would like to increase [UK] foreign aid.\\How would you like to finance such increase in foreign aid? (Multiple answers possible)}
%     \includegraphics[width=\columnwidth]{../figures/all/foreign_aid_raise.pdf} 
% \end{figure}		
% \begin{figure}
%     \centering 
%     \caption{Your previous answer shows that you would like to reduce [UK] foreign aid.\\How would you like to use the freed budget? (Multiple answers possible)}
%     \includegraphics[width=\columnwidth]{../figures/all/foreign_aid_reduce.pdf} 
% \end{figure}

\begin{figure}[h!]
    \caption[Perceived foreign aid]{Perceived foreign aid. ``From your best guess, what percentage of [own country] government spending is allocated to foreign aid (that is, to reduce poverty in low-income countries)?'' (Question \ref{q:foreign_aid_belief})  \hfill (Back~to~Section~\ref{subsubsec:support_foreign_aid}) \\ Actual values: France: 0.8\%; Germany: 1.3\%; Spain: 0.5\%; UK: 1.7\%; U.S.: 0.4\%.}\label{fig:foreign_aid_belief}
    \makebox[\textwidth][c]{\includegraphics[width=\textwidth]{../figures/country_comparison/foreign_aid_belief_agg.pdf}} 
\end{figure}

\begin{figure}[h!]
    \caption[Preferred foreign aid (without info on actual amount)]{Preferred foreign aid (without info on actual amount). \\ ``If you could choose the government spending, what percentage would you allocate
    to foreign aid?'' (Question \ref{q:foreign_aid_preferred})  \hfill (Back~to~Section~\ref{subsubsec:support_foreign_aid})}\label{fig:foreign_aid_preferred_no_info}
    \makebox[\textwidth][c]{\includegraphics[width=\textwidth]{../figures/country_comparison/foreign_aid_preferred_no_info_agg.pdf}} 
\end{figure}

\begin{figure}[h!]
    \caption[Preferred foreign aid (after info on actual amount)]{Preferred foreign aid (after info on actual amount). \\ ``Actually,
    [US1: 0.4\%; FR: 0.8\%; DE: 1.3\%; ES: 0.5\%; UK: 1.7\%] of [own country] government spending is allocated to foreign aid. \\
    If you could choose the government spending, what percentage would you allocate
    to foreign aid?'' (Question \ref{q:foreign_aid_preferred})  \hfill (Back~to~Section~\ref{subsubsec:support_foreign_aid})}\label{fig:foreign_aid_preferred_info}
    \makebox[\textwidth][c]{\includegraphics[width=\textwidth]{../figures/country_comparison/foreign_aid_preferred_info_agg.pdf}} 
\end{figure}

\begin{figure} 
    \caption[Actual, perceived and preferred amount of foreign aid (mean)]{Actual, perceived and preferred amount of foreign aid, with random info (or not) on actual amount. (\textit{Mean} in percent of public spending, Questions \ref{q:foreign_aid_belief}, \ref{q:foreign_aid_preferred})  \hfill (Back~to~Section~\ref{subsubsec:support_foreign_aid})}\label{fig:foreign_aid_amount}
    \makebox[\textwidth][c]{\includegraphics[width=.9\textwidth]{../figures/country_comparison/foreign_aid_amount_mean.pdf} } 
\end{figure}

% \begin{figure} 
%     \caption{Actual, perceived and preferred amount of foreign aid, with random info (or not) on actual amount. (\textit{Median}, Questions \ref{q:foreign_aid_belief}, \ref{q:foreign_aid_preferred})}\label{fig:foreign_aid_amount}
%     \makebox[\textwidth][c]{\includegraphics[width=.9\textwidth]{../figures/country_comparison/foreign_aid_amount_median.pdf} } % TODO? add? not necessary as the info on median can be deduced from below figures
% \end{figure}

\begin{figure} 
    % \caption{Support for increased foreign aid (vs. reduced or stable): from previous question, and directly asked (with info).}\vspace{-.2cm}
    % \includegraphics[height=.32\textheight]{../figures/country_comparison/foreign_aid_more_positive.pdf} 
    \caption[Preferred foreign aid (summary)]{Preferred foreign aid (after info or after perception). (Questions \ref{q:foreign_aid_belief} and \ref{q:foreign_aid_preferred})}\label{fig:foreign_aid_no_less_all}
    \makebox[\textwidth][c]{\includegraphics[width=\textwidth]{../figures/country_comparison/foreign_aid_no_less_all_positive.pdf} }
\end{figure} 

\begin{figure}[h!]
    \caption[Preferences for funding increased foreign aid]{Preferences for funding increased foreign aid. [Asked iff preferred foreign aid is strictly greater than [Info: actual; No info: perceived] foreign aid] \\ ``How would you like to finance such increase in foreign aid? (Multiple answers possible)'' (in percent) (Question \ref{q:foreign_aid_raise_how})  \hfill (Back~to~Section~\ref{subsubsec:support_foreign_aid})}\label{fig:foreign_aid_raise_how}
    \makebox[\textwidth][c]{\includegraphics[width=.75\textwidth]{../figures/country_comparison/foreign_aid_raise_positive.pdf}} 
\end{figure}

\begin{figure}[h!]
    \caption[Preferences of spending following reduced foreign aid]{Preferences of spending following reduced foreign aid. [Asked iff preferred foreign aid is strictly lower than [Info: actual; No info: perceived] foreign aid] \\ ``How would you like to use the freed budget? (Multiple answers possible)'' (in percent) (Question \ref{q:foreign_aid_reduce_how})  \hfill (Back~to~Section~\ref{subsubsec:support_foreign_aid})}\label{fig:foreign_aid_reduce_how}
    \makebox[\textwidth][c]{\includegraphics[width=.75\textwidth]{../figures/country_comparison/foreign_aid_reduce_positive.pdf}} 
\end{figure}

% \begin{figure}[h!]
%     \caption[Attitudes on the evolution of foreign aid]{Attitudes regarding the evolution of [own country] foreign aid. (Question \ref{q:foreign_aid_raise_support})}\label{fig:foreign_aid_raise_support}
%     \makebox[\textwidth][c]{\includegraphics[width=\textwidth]{../figures/country_comparison/foreign_aid_raise_support.pdf}} 
% \end{figure}

% \begin{figure}[h!]
%     \caption[Conditions at which foreign aid should be increased]{Conditions at which foreign aid should be increased (in percent). [Asked to those who wish an increase of foreign aid at some conditions.] (Question \ref{q:foreign_aid_condition})}\label{fig:foreign_aid_condition}
%     \makebox[\textwidth][c]{\includegraphics[width=\textwidth]{../figures/country_comparison/foreign_aid_condition_positive.pdf}} 
% \end{figure}

% \begin{figure}[h!]
%     \caption[Reasons why foreign aid should not be increased]{Reasons why foreign aid should not be increased (in percent). [Asked to those who wish a decrease or stability of foreign aid.] (Question \ref{q:foreign_aid_no})}\label{fig:foreign_aid_no}
%     \makebox[\textwidth][c]{\includegraphics[width=\textwidth]{../figures/country_comparison/foreign_aid_no_positive.pdf}} 
% \end{figure}

% \begin{figure}[h!]
%     \caption[Willingness to sign a real-stake petition]{Willingness to sign real-stake petition for the Global Climate Scheme or National Redistribution. (Question \ref{q:petition})}\label{fig:petition}
%     \makebox[\textwidth][c]{\includegraphics[width=.8\textwidth]{../figures/country_comparison/petition_only_positive.pdf}} 
% \end{figure}

\begin{figure}[h!]
    \caption[Willingness to sign a real-stake petition]{Willingness to sign real-stake petition for the Global Climate Scheme or National Redistribution, compared to stated support in corresponding subsamples (e.g. support for the GCS in the branch where the petition was about the GCS). (Question \ref{q:petition})}\label{fig:petition}
    \makebox[\textwidth][c]{\includegraphics[width=.8\textwidth]{../figures/country_comparison/petition_comparable_positive.pdf}} 
\end{figure}

\begin{figure}[h!] % TODO? More details?
    \caption[Absolute support for various global policies]{Absolute support for various global policies (Percent of (\textit{somewhat} or \textit{strong}) support). (Questions \ref{q:climate_policies} and \ref{q:other_policies}. See Figure \ref{fig:support} for the relative support.)}\label{fig:support_likert_positive}
    \makebox[\textwidth][c]{\includegraphics[width=\textwidth]{../figures/country_comparison/support_likert_positive.pdf}} 
\end{figure}

% \begin{figure}[h!]
%     \caption{label}\label{fig:climate_policies}
%     \makebox[\textwidth][c]{\includegraphics[width=\textwidth]{../figures/country_comparison/climate_policies.pdf}} 
% \end{figure}

% \begin{figure}[h!]
%     \caption{label}\label{fig:global_policies}
%     \makebox[\textwidth][c]{\includegraphics[width=\textwidth]{../figures/country_comparison/global_policies.pdf}} 
% \end{figure}

\begin{figure}[h!]
    \caption[Preferred approach for international climate negotiations]{Preferred approach of diplomats at international climate negotiations. \\ In international climate negotiations, would you prefer [U.S.] diplomats to defend [own country] interests or global justice? (Question \ref{q:negotiation})}\label{fig:negotiation}
    \makebox[\textwidth][c]{\includegraphics[width=\textwidth]{../figures/country_comparison/negotiation.pdf}} 
\end{figure}

\begin{figure}[h!]
    \caption[Importance of selected issues]{Percent of selected issues viewed as important.\\ ``To what extent do you think the following issues are a problem?'' (Question \ref{q:problem})}\label{fig:problem}
    \makebox[\textwidth][c]{\includegraphics[width=.75\textwidth]{../figures/country_comparison/problem_positive.pdf}} 
\end{figure}

\begin{figure}[h!]
    \caption[Group defended when voting]{Group defended when voting. \\ ``What group do you defend when you vote?'' (Question \ref{q:group_defended})}\label{fig:group_defended}
    \makebox[\textwidth][c]{\includegraphics[width=\textwidth]{../figures/country_comparison/group_defended_agg2.pdf}} 
\end{figure}

% \begin{figure}[h!]
%     \caption{label}\label{fig:group_defended}
%     \makebox[\textwidth][c]{\includegraphics[width=\textwidth]{../figures/country_comparison/group_defended.pdf}} 
% \end{figure}

\begin{figure}[h!] 
    \caption[Mean prioritization of policies]{Mean prioritization of policies. \\Mean number of points allocated policies to express intensity of support (among six policies chosen at random). Blue color means that the policy has been awarded more points than the average policy. \\ ``In this question, you have 100 points that you can allocate to different policies. The more you give points to a policy, the more you support it. \\ How do you allocate the points among the following policies?'' (Question \ref{q:points})}\label{fig:points}
    \makebox[\textwidth][c]{\includegraphics[width=\textwidth]{../figures/country_comparison/points_mean.pdf}} 
\end{figure}

\begin{figure}[h!] 
    \caption[Positive prioritization of policies]{Positive prioritization of policies. \\ Percent of people allocating a positive number of points to policies, expressing their support (among six policies chosen at random). \\ ``In this question, you have 100 points that you can allocate to different policies. The more you give points to a policy, the more you support it. \\ How do you allocate the points among the following policies?'' (Question \ref{q:points})}\label{fig:points_positive}
    \makebox[\textwidth][c]{\includegraphics[width=\textwidth]{../figures/country_comparison/points_positive.pdf}} 
\end{figure}

\begin{figure}[h!]
    \caption[Charity donation]{Charity donation. \\ ``How much did you give to charities in 2022?'' (Question \ref{q:donation_charities})}\label{fig:donation_charities}
    \makebox[\textwidth][c]{\includegraphics[width=.8\textwidth]{../figures/country_comparison/donation_charities.pdf}} 
\end{figure}

\begin{figure}[h!] 
    \caption[Interest in politics]{Interest in politics. \\ ``To what extent are you interested in politics?'' (Question \ref{q:interested_politics})}\label{fig:interested_politics}
    \makebox[\textwidth][c]{\includegraphics[width=.8\textwidth]{../figures/country_comparison/interested_politics.pdf}} 
\end{figure}

\begin{figure}[h!] 
    \caption[Desired involvement of government]{Desired involvement of government (from 1 to 5). (Question \ref{q:involvement_govt})}\label{fig:involvement_govt}
    \makebox[\textwidth][c]{\includegraphics[width=.9\textwidth]{../figures/country_comparison/involvement_govt.pdf}} 
\end{figure}

\begin{figure}[h!] 
    \caption[Political leaning]{Political leaning on economics (from 1: Left to 5: Right). (Question \ref{q:left_right})}\label{fig:left_right}
    \makebox[\textwidth][c]{\includegraphics[width=.8\textwidth]{../figures/country_comparison/left_right.pdf}} 
\end{figure}

\begin{figure}[h!] 
    \caption[Voted in last election]{Voted in last election. (Question \ref{q:vote_participation})}\label{fig:vote_participation}
    \makebox[\textwidth][c]{\includegraphics[width=.8\textwidth]{../figures/country_comparison/vote_participation.pdf}} 
\end{figure}

\begin{figure}[h!] 
    \caption[Vote in last election]{Vote in last election (aggregated). \textit{PNR} includes people who did not vote or prefer not to answer. (Question \ref{q:vote})}\label{fig:vote}
    \makebox[\textwidth][c]{\includegraphics[width=.75\textwidth]{../figures/country_comparison/vote.pdf}} 
\end{figure}

\begin{figure}[h!] 
    \caption[Perception that survey was biased]{Perception that survey was biased. \\ ``Do you feel that this survey was politically biased?'' (Question \ref{q:survey_biased})}\label{fig:survey_biased}
    \makebox[\textwidth][c]{\includegraphics[width=.7\textwidth]{../figures/country_comparison/survey_biased.pdf}} 
\end{figure}

% \begin{columns}
% \begin{column}{.5\textwidth}
% \begin{multicols}{2}
    \begin{figure}[h!]
        \caption[Classification of open-ended field on extreme poverty]{Opinion on the fight against extreme poverty. \\ ``According to you, what should high-income countries do to fight extreme poverty in low-income countries?'' (Question \ref{q:poverty_field})  \hfill (Back~to~Section~\ref{subsubsec:support_foreign_aid})}\label{fig:poverty_field}
    \begin{subfigure}{.34\textwidth}
        \caption{Elements found in the open-ended field on the question (manually recoded, in percent)}.
        \includegraphics[width=\textwidth]{../figures/country_comparison/poverty_field_positive.pdf}        
    \end{subfigure}
    \hspace{.02\textwidth}
    \begin{subfigure}{.64\textwidth}
        \caption{Keywords found in the open-ended field on the GCS (automatic search ignoring case, in percent).}
        \includegraphics[width=\textwidth]{../figures/country_comparison/poverty_field_contains_positive.pdf}    
    \end{subfigure}
    \end{figure}
% \end{column}
% \begin{column}{.5\textwidth}
    % \begin{figure}[h!]
    %     \caption[Topics of open-ended field on extreme poverty]{Opinion on the fight against extreme poverty. \\ ``According to you, what should high-income countries do to fight extreme poverty in low-income countries?'' \\ Keywords found in the open-ended field on the GCS (automatic search ignoring case, in percent). (Question \ref{q:poverty_field})}\label{fig:poverty_field_contains}
    %     \makebox[\textwidth][c]{\includegraphics[width=\columnwidth]{../figures/country_comparison/poverty_field_contains_positive.pdf}} 
    % \end{figure}
% \end{multicols}
% \end{column}
% \end{columns}


\begin{figure}[h!] 
    \caption[Main attitudes by vote]{Main attitudes by vote (``Right'' spans from Center-right to Far right). \\ (Relative support in percent in Questions \ref{q:gcs_support}, \ref{q:global_tax}, \ref{q:other_policies}, \ref{q:foreign_aid_raise_support}, \ref{q:negotiation}) \hfill (Back~to~Section~\ref{subsec:universalistic})}\label{fig:main_by_vote}
    \makebox[\textwidth][c]{\includegraphics[width=\textwidth]{../figures/country_comparison/main_all_by_vote_share.pdf}} 
\end{figure}

% \begin{figure}[h!] 
%     \caption[Interested to be interviewed]{Interested to be interviewed by a researcher for 30 min through videoconference. (Question \ref{q:interview})}\label{fig:interview}
%     \makebox[\textwidth][c]{\includegraphics[width=\textwidth]{../figures/country_comparison/interview.pdf}} 
% \end{figure}    

% \begin{figure}[h!]
%     \caption{label}\label{fig:share_policies_supported}
%     \makebox[\textwidth][c]{\includegraphics[width=\textwidth]{../figures/country_comparison/share_policies_supported.pdf}} 
% \end{figure} % TODO? uncomment?

% \begin{figure}[h!]
%     \caption{label}\label{fig:vars}
%     \makebox[\textwidth][c]{\includegraphics[width=\textwidth]{../figures/country_comparison/vars.pdf}} 
% \end{figure}

% In Denmark, France and the U.S., the questions with an asterisk were asked differently, asking ``To achieve a given reduction of greenhouse gas emissions globally, costly investments are needed. Ideally, how should countries bear the costs of fighting climate change?''. Instead of the equal right per capita, the item was ``Countries should pay in proportion to their current emissions'', historical responsibilities was worded as ``Countries should pay in proportion to their past emissions (from 1990 onwards)'', then there was an item ``The richest countries should pay it all'', and compensating vulnerable countries was worded as ``The richest countries should pay even more, to help vulnerable countries face adverse consequences: vulnerable countries would then receive money instead of paying''.

\clearpage 
\section{Questionnaire of the global survey (section on global policies)}\label{app:questionnaire_oecd}
%\subsection*{International burden-sharing}
\renewcommand{\theenumi}{\Alph{enumi}}
\begin{enumerate} \item \label{q:scale} At which level(s) do you think public policies to tackle climate change need to be put in place? (Multiple answers are possible) [\textit{Figures \ref{fig:oecd} and \ref{fig:oecd_absolute}}]
\\ \textit{Global; [Federal / European / ...]; [State / National]; Local}
\item Do you agree or disagree with the following statement: ``[country] should take measures to fight climate change.''% TODO! figure
	\\ \textit{Strongly disagree; Somewhat disagree; Neither agree nor disagree; Somewhat agree; Strongly agree}
\item How should [country] climate policies depend on what other countries do?% TODO! figure
 \begin{itemize}
\item If other countries do more, [country] should do...
\item If other countries do less, [country] should do...
\end{itemize}
\textit{Much less; Less; About the same; More; Much more}
\item ~[In all countries but the U.S., Denmark and France]  All countries have signed the Paris agreement that aims to contain global warming ``well below +2 \textdegree{}C\''. To limit global warming to this level, there is a maximum amount of greenhouse gases we can emit globally, called the carbon budget. Each country could aim to emit less than a share of the carbon budget. To respect the global carbon budget, countries that emit more than their national share would pay a fee to countries that emit less than their share. \\ 
Do you support such a policy? [\textit{Figures \ref{fig:oecd} and \ref{fig:oecd_absolute}}] 
\\ \textit{Strongly oppose; Somewhat oppose; Neither support nor oppose; Somewhat support; Strongly support}
\item ~[In all countries but the U.S., Denmark and France] Suppose the above policy is in place. How should the carbon budget be divided among countries? [\textit{Figures \ref{fig:oecd} and \ref{fig:oecd_absolute}}]
\\ \textit{The emission share of a country should be proportional to its population, so that each human has an equal right to emit.; The emission share of a country should be proportional to its current emissions, so that those who already emit more have more rights to emit.; Countries that have emitted more over the past decades (from 1990 onwards) should receive a lower emission share, because they have already used some of their fair share.; Countries that will be hurt more by climate change should receive a higher emission share, to compensate them for the damages.}
\item \label{q:burden_sharing_asterisk} ~[In the U.S., Denmark, and France only] To achieve a given reduction of greenhouse gas emissions globally, costly investments are needed. % TODO! figure
Ideally, how should countries bear the costs of fighting climate change?
 \begin{itemize}
\item Countries should pay in proportion to their income
\item Countries should pay in proportion to their current emissions [Used as a substitute to the equal right per capita in Figure \ref{fig:oecd}]
\item Countries should pay in proportion to their past emissions (from 1990 onwards) [Used as a substitute to historical responsibilities in Figure \ref{fig:oecd}]
\item The richest countries should pay it all, so that the poorest countries do not have to pay anything
\item The richest countries should pay even more, to help vulnerable countries face adverse consequences: vulnerable countries would then receive money instead of paying [Used as a substitute to compensating vulnerable countries in Figures \ref{fig:oecd} and \ref{fig:oecd_absolute}]
\end{itemize} 
\textit{Strongly disagree; Somewhat disagree; Neither agree nor disagree; Somewhat agree; Strongly agree}
\item Do you support or oppose establishing a global democratic assembly whose role would be to draft international treaties against climate change? Each adult across the world would have one vote to elect members of the assembly. [\textit{Figures \ref{fig:oecd} and \ref{fig:oecd_absolute}}]
\\ \textit{Strongly oppose; Somewhat oppose; Neither support nor oppose; Somewhat support; Strongly support}
\item \label{q:global_tax_dividend} Imagine the following policy: a global tax on greenhouse gas emissions funding a global basic income. 
Such a policy would progressively raise the price of fossil fuels (for example, the price of gasoline would increase by [40 cents per gallon] in the first years). Higher prices would encourage people and companies to use less fossil fuels, reducing greenhouse gas emissions. Revenues from the tax would be used to finance a basic income of [\$30] per month to each human adult, thereby lifting the 700 million people who earn less than \$2/day out of extreme poverty. 
The average [American] person would lose a bit from this policy as they would face [\$130] per month in price increases, which is higher than the [\$30] they would receive.

Do you support or oppose such a policy?  [\textit{Figures \ref{fig:oecd} and \ref{fig:oecd_absolute}}]
\\ \textit{Strongly oppose; Somewhat oppose; Neither support nor oppose; Somewhat support; Strongly support}
\item \label{q:millionaire_tax} Do you support or oppose a tax on all millionaires around the world to finance low-income countries that comply with international standards regarding climate action? 
This would finance infrastructure and public services such as access to drinking water, healthcare, and education. [\textit{Figures \ref{fig:oecd} and \ref{fig:oecd_absolute}}]
\\ \textit{Strongly oppose; Somewhat oppose; Neither support nor oppose; Somewhat support; Strongly support}
\end{enumerate}

% \clearpage
% \section{Questionnaire of US1 %the first U.S. complementary 
% survey}\label{app:questionnaire_US1}

% \begin{figure}[h!]
%     \caption{US1 survey structure}\label{fig:flow_US1}
%     \makebox[\textwidth][c]{\includegraphics[width=\textwidth]{../questionnaire/survey_flow_US1.pdf}} 
% \end{figure}

\renewcommand{\theenumi}{\arabic{enumi}}
\clearpage
\section{Questionnaire of the complementary surveys}\label{app:questionnaire}
\input{app_questionnaire}


\clearpage
\section{Net gains from the Global Climate Scheme}\label{app:gain_gcs}

To specify the GCS, we use the IEA's 2DS scenario \citep{iea_energy_2017}, which is consistent with limiting the global average temperature increase to 2\textdegree{}C with a probability of at least 50\%. The paper by \citet{hood_input_2017} contributing to the Report of the High-Level Commission on Carbon Prices \citep{stern_report_2017} presents a price corridor compatible with this emissions scenario, from which we take the midpoint. The product of these two series provides an estimate of the revenues expected from a global carbon price. We then use the UN median scenario of future population aged over 15 years (\textit{adults}, for short). We derive the basic income that could be paid to all adults by recycling the revenues from the global carbon price: evolving between \$20 and \$30 per month, with a peak in 2030. Accounting for the lower price levels in low-income countries, an additional income of \$30 per month would allow \href{https://data.worldbank.org/indicator/SI.POV.DDAY}{670 million people} to escape extreme poverty, defined with the threshold of \$2.15 per day in purchasing power parity.\footnote{The average carbon footprint of Sub-Saharan Africa is 0.75tCO$_{\text{2}}$ per capita (\href{https://www.macrotrends.net/countries/SSF/sub-saharan-africa-/carbon-co2-emissions}{World Bank}), and it is even lower for people living in extreme poverty, under one tenth of the world average of about 5tCO$_{\text{2}}$ per capita \citep{chancel_carbon_2015}. Under a GCS with a cash transfer of \$30 per person, if one's emission is one tenth of the world average, their net gain would be \$27 per month in nominal terms. In regions with extreme poverty like Sub-Saharan Africa (excluding high-income countries), the conversion factor from Market Exchange Rate to Purchasing Power Parity (PPP) is 2.4 (computed as the \href{https://data.worldbank.org/indicator/PA.NUS.PPPC.RF}{ratio} of the World Bank series relating the GDP per capita of Sub-Saharan Africa in \href{https://data.worldbank.org/indicator/NY.GDP.PCAP.PP.KD?locations=ZG&year_high_desc=true}{PPP} and \href{https://data.worldbank.org/indicator/NY.GDP.PCAP.KD?locations=ZG&year_high_desc=true}{nominal}). Therefore, the net gain for the extreme poor is \$65 per month (or \$2.13 a day) in PPP, enough to lift them out of extreme poverty.%, defined with the threshold of \$2 a day in PPP.
% By taking the \href{https://data.worldbank.org/indicator/PA.NUS.PPPC.RF}{ratio} of the World Bank series relating the GDP per capita of Sub-Saharan Africa in \href{https://data.worldbank.org/indicator/NY.GDP.PCAP.PP.KD?locations=ZG&year_high_desc=true}{PPP} and \href{https://data.worldbank.org/indicator/NY.GDP.PCAP.KD?locations=ZG&year_high_desc=true}{nominal}, we obtain the purchasing power of \$1 in this region: \$2.4 in 2019. %See also the price level ratio of PPP conversion factor to market exchange rate.
} 

To estimate the increase in fossil fuel expenditures (or ``cost'') in each country by 2030, we make a key assumption concerning the evolution of the carbon footprints per adult: that they will decrease by the same proportion %$\rho$ 
in each country. We use data from the Global Carbon Project \citep{peters_synthesis_2012}. 
% Noting $e_c$ (resp. $e_c^b$) the carbon footprint per adult of a country $c$ in 2030 (resp. in baseline year $b$), we have $e_c = \rho e_c^b$. Noting $a_c$ (resp. $a_c^b$) the adult population of a country $c$ in 2030 (resp. in baseline year $b$) and $E = \sum_c e_c a_c$ global emissions in 2030, we find $\rho = \frac{E}{\sum_c e_c^b a_c}$. Finally, the average cost per adult in year $y$ is $p \cdot e_c \frac{a_c}{a^y_c}$. %Multiplying country $c$'s carbon footprint per capita with the carbon price $p$ yields its average cost per adult: $p \cdot e_c$. %$\frac{s_c^y}{p^y_c} R$. 
In 2030, the average carbon footprint of a country $c$, $e_c$, evolves from baseline year $b$ proportionally to the evolution of its adult population $\Delta p_c = p^{2030}_c/p^b_c$. Thus, the global share of country $c$'s carbon footprint, $s_c$, is proportional to $\sigma_c = e_c \Delta p_c$, and as countries' shares sum to 1, $s_c = \frac{\sigma_c}{\sum_k \sigma_k}$. Multiplying country $c$'s emission share with global revenues in 2030, $R$, and dividing by $c$'s adult population in year $y$, yields its average cost per adult: $R \cdot s_c / p^y_c$. %$\frac{s_c^y}{p^y_c} R$. 
Using findings from \citet{ivanova_unequal_2020} for Europe and \citet{fremstad_impact_2019} for the U.S., we approximate the median cost as 90\% of the average cost. Finally, the net gain is given by the basic income (\$30 per month) minus the cost. We provided consistent estimates of net gains in all surveys (using $y = b = 2015$), though in the global survey we gave the average net gains vs. the median ones in the complementary surveys. The latter are shown in Figure \ref{fig:median_gain_2015}. 
For the record, Table \ref{tab:gain_gcs.tex} also provides an estimate of \textit{average} net gains (computed with $b = 2019$ and $y = 2030$).\footnote{2015 was the last year of data available when the global questionnaire was conceived (\href{https://stats.oecd.org/Index.aspx?DataSetCode=IO_GHG_2019}{OECD data} was then used -- it does not cover all countries but give identical rounded estimates than those recomputed from the Global Carbon Project data for our complementary surveys). 2030 was chosen as the reference year as it is the date at which global carbon price revenues are expected to peak (and the GCS redistributive effects would be largest), and the GCS could not realistically enter into force before that date. In the surveys, we chose $y = b = 2015$ rather than $b = 2019$ and $y = 2030$ to get more conservative estimates of the monthly cost in the U.S. (\$20 higher than the other option) and in Europe (\euro{5} or £10 higher).}% TODO? remove footnote?
%  ((e/E)*(f/a)*A/F)*R/a

Estimates of the net gains from the Global Climate Scheme are necessarily imprecise, given the uncertainties surrounding the carbon price required to achieve emissions reductions as well as each country's trajectory in terms of emissions and population. These values are highly dependent on future (non-price) climate policies, technical progress, and economic growth of each country, which are only partially known. Integrated Assessment Models have been used to derive a Global Energy Assessment \citep{johansson_global_2012}, a 100\% renewable scenario \citep{greenpeace_energy_2015} as well as Shared Socioeconomic Pathways (SSPs), which include consistent trajectories of population, emissions, and carbon price \citep{riahi_shared_2017,bauer_shared_2017,van_vuuren_energy_2017,fricko_marker_2017}. Instead of using some of these modelling trajectories, we relied on a simple and transparent formula, for a number of reasons. First and foremost, those trajectories describe territorial emissions while we need consumption-based emissions to compute the incidence of the GCS. Second, the carbon price is relatively low in trajectories of SSPs that contain global warming below 2\textdegree{}C (less than \$35/tCO$_\text{2}$ in 2030), so we conservatively chose a method yielding a higher carbon price (\$90 in 2030). Third, modelling results are available only for a few macro regions, while we wanted country by country estimates. Finally, we have checked that the emissions per capita given by our method are broadly in line with alternative methods, even if it tends to overestimate net gains in countries which will decarbonize less rapidly than average.\footnote{Computations with alternative methods can be found on \href{https://github.com/bixiou/international_attitudes_toward_global_policies/blob/main/code_global/map_GCS_incidence.R}{our public repository}.} For example, although countries' decarbonization plans should realign with the GCS in place, India might still decarbonize less quickly than the European Union, so India's gain and the EU's loss might be overestimated in our computations. For a more sophisticated version of the Global Climate Scheme which includes participation mechanisms preventing middle-income countries (like China) to lose from it and estimations of the Net Present Value by country, see \citet{fabre_global_2023}.  \hfill (Back~to~Section~\ref{box:GCS})

\begin{figure}[h!]
    \caption{Net gains from the Global Climate Scheme.}\label{fig:median_gain_2015}
    \makebox[\textwidth][c]{\includegraphics[width=\textwidth]{../figures/maps/median_gain_2015.pdf}} 
\end{figure}

% \begin{table}[h]\label{tab:gain_gcs}
%     \caption{Net gains from the Global Climate Scheme.} 
%     \makebox[\textwidth][c]{
        % \resizebox*{!}{.7\textheight}{
\clearpage
\begin{multicols}{2}
    \setbox\ltmcbox\vbox{
    \makeatletter\col@number\@ne
        
\begin{longtable}[t]{lrr}
\caption{\label{tab:gain_gcs.tex}Estimated net gain from the GCS in 2030 and carbon footprint by country.}\\
\toprule
  & \makecell{Mean\\net gain\\from\\the GCS\\(\$/month)} & \makecell{CO$_\text{2}$\\footprint\\per adult\\in 2019\\(tCO$_\text{2}$/y)}\\
\midrule
Saudi Arabia & -93 & 24.0\\
United States & -77 & 21.0\\
Australia & -60 & 17.6\\
Canada & -56 & 16.7\\
South Korea & -50 & 15.6\\
Germany & -30 & 11.7\\
Russia & -29 & 11.5\\
Japan & -28 & 11.3\\
Malaysia & -21 & 10.0\\
Iran & -19 & 9.5\\
Poland & -19 & 9.5\\
United Kingdom & -18 & 9.4\\
China & -14 & 8.6\\
Italy & -13 & 8.4\\
South Africa & -11 & 8.0\\
France & -10 & 7.8\\
Iraq* & -8 & 7.4\\
Spain & -6 & 7.0\\
Turkey & -2 & 6.2\\
Algeria* & -1 & 6.0\\
Mexico & 2 & 5.6\\
Ukraine & 2 & 5.6\\
Uzbekistan* & 4 & 5.1\\
Argentina & 5 & 4.9\\
Thailand & 6 & 4.6\\
Egypt & 12 & 3.6\\
Indonesia & 13 & 3.3\\
Colombia & 15 & 3.0\\
Brazil & 15 & 2.9\\
Vietnam & 15 & 2.9\\
Peru & 16 & 2.8\\
Morocco & 16 & 2.7\\
North Korea* & 17 & 2.5\\
India & 18 & 2.4\\
Philippines & 18 & 2.3\\
Pakistan & 22 & 1.6\\
Bangladesh & 24 & 1.1\\
Nigeria & 25 & 1.0\\
Kenya & 25 & 0.9\\
Myanmar* & 26 & 0.9\\
Sudan* & 26 & 0.9\\
Tanzania & 27 & 0.5\\
Afghanistan* & 27 & 0.5\\
Uganda & 28 & 0.4\\
Ethiopia & 28 & 0.3\\
Venezuela & 29 & 0.3\\
DRC* & 30 & 0.1\\
\bottomrule
\end{longtable}
    \unskip
    \unpenalty
    \unpenalty}
    \unvbox\ltmcbox
\end{multicols}
        % }
%     }
    {\footnotesize \textit{Note}: %Emission data is from \cite{peters_synthesis_2012}. 
    Asterisks denote countries where footprint is missing and territorial emissions is used instead. %Estimation of net gains is described in the text. 
    Values differ from Figure \ref{fig:median_gain_2015} as this table present estimates of \textit{mean} net gain per adult in \textit{2030}, not at the present. Only the countries with more than 20 million adults (covering 87\% of the global total) are shown. 
    }
% \end{table}

% \clearpage
% \section{Sources}\label{app:sources}

\clearpage
\section{Determinants of support}\label{app:determinants}

\begin{table}[h]\label{tab:gcs_determinant}
    \caption[Determinants of support for the GCS]{Determinants of support for the Global Climate Scheme. (Back to \ref{subsec:gcs_stated_support})} 
    \makebox[\textwidth][c]{
\resizebox*{!}{.73\textheight}{ % 73 is the max when there is a title
        
\begin{tabular}{@{\extracolsep{5pt}}lccccccc} 
\\[-1.8ex]\hline 
\hline \\[-1.8ex] 
 & \multicolumn{7}{c}{\makecell{Supports the Global Climate Scheme}} \\ 
\cline{2-8} 
\\[-1.8ex] & All & United States & Europe & France & Germany & Spain & United Kingdom \\ 
\hline \\[-1.8ex] 
 Country: Germany & $-$0.157$^{***}$ &  & $-$0.144$^{***}$ &  &  &  &  \\ 
  & (0.022) &  & (0.022) &  &  &  &  \\ 
  Country: Spain & $-$0.044$^{*}$ &  & $-$0.026 &  &  &  &  \\ 
  & (0.024) &  & (0.024) &  &  &  &  \\ 
  Country: United Kingdom & $-$0.079$^{***}$ &  & $-$0.104$^{***}$ &  &  &  &  \\ 
  & (0.024) &  & (0.023) &  &  &  &  \\ 
  Country: United States & $-$0.375$^{***}$ &  &  &  &  &  &  \\ 
  & (0.019) &  &  &  &  &  &  \\ 
  Income quartile: 2 & 0.037$^{**}$ & 0.031 & 0.038 & 0.047 & 0.058 & 0.013 & 0.023 \\ 
  & (0.017) & (0.022) & (0.023) & (0.043) & (0.049) & (0.053) & (0.043) \\ 
  Income quartile: 3 & 0.042$^{**}$ & 0.033 & 0.049$^{**}$ & 0.080$^{**}$ & 0.059 & 0.074 & $-$0.052 \\ 
  & (0.017) & (0.024) & (0.024) & (0.040) & (0.052) & (0.056) & (0.052) \\ 
  Income quartile: 4 & 0.056$^{***}$ & 0.063$^{**}$ & 0.010 & 0.018 & $-$0.015 & $-$0.001 & $-$0.005 \\ 
  & (0.018) & (0.026) & (0.026) & (0.047) & (0.055) & (0.056) & (0.057) \\ 
  Diploma: Post secondary & 0.023$^{*}$ & 0.033$^{*}$ & 0.010 & 0.007 & 0.045 & 0.007 & $-$0.010 \\ 
  & (0.012) & (0.017) & (0.018) & (0.029) & (0.039) & (0.039) & (0.039) \\ 
  Age: 25-34 & $-$0.076$^{***}$ & $-$0.083$^{***}$ & $-$0.044 & $-$0.031 & $-$0.077 & $-$0.050 & $-$0.103 \\ 
  & (0.025) & (0.031) & (0.035) & (0.057) & (0.083) & (0.066) & (0.091) \\ 
  Age: 35-49 & $-$0.101$^{***}$ & $-$0.108$^{***}$ & $-$0.069$^{**}$ & $-$0.094$^{*}$ & $-$0.009 & $-$0.168$^{**}$ & $-$0.050 \\ 
  & (0.024) & (0.030) & (0.034) & (0.055) & (0.077) & (0.070) & (0.090) \\ 
  Age: 50-64 & $-$0.137$^{***}$ & $-$0.164$^{***}$ & $-$0.038 & $-$0.039 & $-$0.020 & $-$0.146$^{**}$ & $-$0.017 \\ 
  & (0.024) & (0.030) & (0.035) & (0.056) & (0.082) & (0.067) & (0.087) \\ 
  Age: 65+ & $-$0.116$^{***}$ & $-$0.140$^{***}$ & $-$0.056 & 0.003 & $-$0.045 & $-$0.258$^{***}$ & 0.011 \\ 
  & (0.028) & (0.034) & (0.044) & (0.076) & (0.094) & (0.091) & (0.105) \\ 
  Gender: Man & 0.019$^{*}$ & 0.023 & $-$0.010 & $-$0.014 & $-$0.018 & 0.042 & $-$0.005 \\ 
  & (0.011) & (0.015) & (0.016) & (0.029) & (0.033) & (0.038) & (0.034) \\ 
  Lives with partner & 0.029$^{**}$ & 0.022 & 0.058$^{***}$ & 0.070$^{**}$ & 0.082$^{**}$ & 0.017 & 0.040 \\ 
  & (0.013) & (0.017) & (0.018) & (0.033) & (0.038) & (0.038) & (0.039) \\ 
  Employment status: Retired & $-$0.020 & $-$0.047 & 0.056 & 0.087 & 0.096 & 0.040 & 0.001 \\ 
  & (0.024) & (0.030) & (0.038) & (0.081) & (0.075) & (0.082) & (0.073) \\ 
  Employment status: Student & 0.045 & 0.063 & 0.101$^{**}$ & 0.165$^{*}$ & 0.192$^{**}$ & 0.116 & $-$0.021 \\ 
  & (0.033) & (0.048) & (0.044) & (0.085) & (0.087) & (0.074) & (0.107) \\ 
  Employment status: Working & $-$0.016 & $-$0.021 & 0.011 & 0.082 & 0.006 & $-$0.050 & 0.036 \\ 
  & (0.019) & (0.024) & (0.028) & (0.064) & (0.056) & (0.056) & (0.051) \\ 
  Vote: Center-right or Right & $-$0.331$^{***}$ & $-$0.435$^{***}$ & $-$0.106$^{***}$ & $-$0.131$^{***}$ & $-$0.004 & $-$0.114$^{***}$ & $-$0.081$^{**}$ \\ 
  & (0.013) & (0.017) & (0.019) & (0.035) & (0.044) & (0.038) & (0.041) \\ 
  Vote: PNR/Non-voter & $-$0.184$^{***}$ & $-$0.198$^{***}$ & $-$0.136$^{***}$ & $-$0.196$^{***}$ & $-$0.034 & $-$0.116$^{**}$ & $-$0.108$^{***}$ \\ 
  & (0.016) & (0.022) & (0.021) & (0.039) & (0.043) & (0.046) & (0.040) \\ 
  Vote: Far right & $-$0.396$^{***}$ &  & $-$0.308$^{***}$ & $-$0.493$^{***}$ & $-$0.168$^{***}$ & $-$0.130 & $-$0.314$^{***}$ \\ 
  & (0.032) &  & (0.033) & (0.064) & (0.051) & (0.102) & (0.080) \\ 
  Urban & 0.049$^{***}$ & 0.074$^{***}$ & 0.006 & $-$0.002 & 0.019 & $-$0.014 & 0.017 \\ 
  & (0.012) & (0.018) & (0.016) & (0.029) & (0.032) & (0.036) & (0.033) \\ 
  Race: White &  & $-$0.030 &  &  &  &  &  \\ 
  &  & (0.019) &  &  &  &  &  \\ 
  Region: Northeast &  & 0.009 &  &  &  &  &  \\ 
  &  & (0.023) &  &  &  &  &  \\ 
  Region: South &  & 0.011 &  &  &  &  &  \\ 
  &  & (0.020) &  &  &  &  &  \\ 
  Region: West &  & 0.011 &  &  &  &  &  \\ 
  &  & (0.022) &  &  &  &  &  \\ 
  Swing State &  & $-$0.019 &  &  &  &  &  \\ 
  &  & (0.017) &  &  &  &  &  \\ 
 \hline \\[-1.8ex] 
Constant & 1.048 & 0.729 & 0.89 & 0.7 & 0.732 & 0.935 & 0.886 \\ 
Observations & 7,986 & 4,992 & 2,994 & 977 & 727 & 748 & 542 \\ 
R$^{2}$ & 0.160 & 0.180 & 0.064 & 0.116 & 0.067 & 0.043 & 0.063 \\ 
\hline 
\hline \\[-1.8ex] 
\textit{Note:}  & \multicolumn{7}{r}{$^{*}$p$<$0.1; $^{**}$p$<$0.05; $^{***}$p$<$0.01} \\ 
\end{tabular} 
        }
    }
    {\footnotesize %\textit{Note}: 
    }
\end{table}

\begin{table}[h]\label{tab:gcs_support_understood}
    \caption[Support for the GCS regressed on comprehension]{Support for the GCS regressed on comprehension of its financial effects.} 
    \makebox[\textwidth][c]{
% \resizebox*{!}{.73\textheight}{ % 73 is the max when there is a title
        
\begin{tabular}{@{\extracolsep{5pt}}lcccccc} 
\\[-1.8ex]\hline 
\hline \\[-1.8ex] 
 & \multicolumn{6}{c}{Supports the Global Climate Scheme} \\ 
\cline{2-7} 
\\[-1.8ex] & All & United States & France & Germany & United Kingdom & Spain \\ 
\hline \\[-1.8ex] 
 \makecell{With GCS, typical\\~[country] people lose\\and poorest humans win} & 0.029$^{**}$ & $-$0.004 & 0.043 & 0.051 & 0.040 & 0.038 \\ 
  & (0.012) & (0.016) & (0.033) & (0.033) & (0.036) & (0.037) \\ 
 \hline \\[-1.8ex] 
Constant & 0.596 & 0.53 & 0.764 & 0.677 & 0.707 & 0.796 \\ 
Observations & 8,000 & 5,000 & 729 & 979 & 749 & 543 \\ 
R$^{2}$ & 0.001 & 0.00001 & 0.003 & 0.003 & 0.002 & 0.002 \\ 
\hline 
\hline \\[-1.8ex] 
\end{tabular} 
        }
    % }
    {\footnotesize %\textit{Note}: 
    }
\end{table}

\begin{table}[h]\label{tab:tax_determinant_hi}
    \caption[Beliefs correlated with support for a global GHG tax and dividend (HICs)]{Correlation between (\textit{Somewhat} or \textit{Strong}) \textit{support for a global tax on GHG financing a global basic income} (Question \ref{q:global_tax_dividend}) and beliefs in high-income countries.}\vspace{.5cm}
    \makebox[\textwidth][c]{
        \scalebox{0.5}{
        \input{"../tables/global_tax_support_pos_AtC_keepC_hi"}
        }
    }            
        {\footnotesize \textit{Note}: The table shows the results of regressions on standardized variables measuring respondents' beliefs and perceptions. Treatment indicators and individual socioeconomic characteristics are included but not displayed. Robust standard errors are in parentheses; $^{*}$p$<$0.1; $^{**}$p$<$0.05; $^{***}$p$<$0.01. See Appendix A-1 of \cite{dechezlepretre_fighting_2022} for variable definitions. }
\end{table}

\begin{table}[h]\label{tab:tax_determinant_mi}
    \caption[Beliefs correlated with support for a global GHG tax and dividend (MICs)]{Correlation between (\textit{Somewhat} or \textit{Strong}) \textit{support for a global tax on GHG financing a global basic income} (Question \ref{q:global_tax_dividend}) and beliefs in middle-income countries.}\vspace{.5cm}
    \makebox[\textwidth][c]{
        \scalebox{0.5}{
        \input{"../tables/global_tax_support_pos_AtC_keepC_mi"}
        }
    }            
        {\footnotesize \textit{Note}: The table shows the results of regressions on standardized variables measuring respondents' beliefs and perceptions. Treatment indicators and individual socioeconomic characteristics are included but not displayed. Robust standard errors are in parentheses; $^{*}$p$<$0.1; $^{**}$p$<$0.05; $^{***}$p$<$0.01. See Appendix A-1 of \cite{dechezlepretre_fighting_2022} for variable definitions. }
\end{table}

\clearpage
\section{Representativeness of the surveys}\label{app:representativeness}


\begin{table}[h!]
    \caption[Sample representativeness of \textit{US1}, \textit{US2}, \textit{Eu}]{Sample representativeness of the complementary surveys. (Back to \ref{par:surveys}) } \label{tab:representativeness_waves}
    \makebox[\textwidth][c]{
        \resizebox*{!}{.80\textheight}{% 73 without notes cf. https://tex.stackexchange.com/questions/13809/resizing-a-table-by-textheight 
        
\begin{tabular}[t]{llllllllll}
\toprule
\multicolumn{1}{c}{} & \multicolumn{3}{c}{US1} & \multicolumn{3}{c}{US2} & \multicolumn{3}{c}{EU} \\
\cmidrule(l{3pt}r{3pt}){2-4} \cmidrule(l{3pt}r{3pt}){5-7} \cmidrule(l{3pt}r{3pt}){8-10}
  & Pop. & Sample & \makecell{Weighted\\sample} & Pop. & Sample & \makecell{Weighted\\sample} & Pop. & Sample & \makecell{Weighted\\sample}\\
\midrule
Sample size &  & 3,000 & 3,000 &  & 678 & 678 &  & 3,000 & 3,000\\
\addlinespace
Gender: Woman & 0.51 & 0.52 & 0.51 & 0.51 & 0.67 & 0.57 & 0.51 & 0.49 & 0.51\\
Gender: Man & 0.49 & 0.47 & 0.49 & 0.49 & 0.32 & 0.43 & 0.49 & 0.51 & 0.49\\
\addlinespace
Income\_quartile: 1 & 0.25 & 0.27 & 0.25 & 0.25 & 0.55 & 0.34 & 0.25 & 0.28 & 0.25\\
Income\_quartile: 2 & 0.25 & 0.24 & 0.25 & 0.25 & 0.29 & 0.32 & 0.25 & 0.23 & 0.25\\
Income\_quartile: 3 & 0.25 & 0.25 & 0.25 & 0.25 & 0.12 & 0.23 & 0.25 & 0.25 & 0.25\\
Income\_quartile: 4 & 0.25 & 0.23 & 0.25 & 0.25 & 0.04 & 0.12 & 0.25 & 0.24 & 0.25\\
\addlinespace
Age: 18-24 & 0.12 & 0.12 & 0.12 & 0.12 & 0.14 & 0.12 & 0.10 & 0.11 & 0.10\\
Age: 25-34 & 0.18 & 0.15 & 0.18 & 0.18 & 0.16 & 0.17 & 0.15 & 0.17 & 0.15\\
Age: 35-49 & 0.24 & 0.25 & 0.24 & 0.24 & 0.25 & 0.25 & 0.24 & 0.25 & 0.24\\
Age: 50-64 & 0.25 & 0.27 & 0.25 & 0.25 & 0.22 & 0.24 & 0.26 & 0.24 & 0.26\\
Age: 65+ & 0.21 & 0.21 & 0.21 & 0.21 & 0.22 & 0.22 & 0.25 & 0.23 & 0.25\\
\addlinespace
Diploma\_25\_64: Below upper secondary & 0.06 & 0.02 & 0.05 & 0.06 & 0.08 & 0.07 & 0.13 & 0.14 & 0.13\\
Diploma\_25\_64: Upper secondary & 0.28 & 0.25 & 0.28 & 0.28 & 0.33 & 0.30 & 0.23 & 0.19 & 0.23\\
Diploma\_25\_64: Post secondary & 0.34 & 0.40 & 0.34 & 0.34 & 0.23 & 0.28 & 0.29 & 0.33 & 0.29\\
\addlinespace
Race: White only & 0.60 & 0.67 & 0.61 & 0.60 & 0.20 & 0.46 &  &  & \\
Race: Hispanic & 0.18 & 0.15 & 0.19 & 0.18 & 0.41 & 0.27 &  &  & \\
Race: Black & 0.13 & 0.16 & 0.14 & 0.13 & 0.36 & 0.20 &  &  & \\
\addlinespace
Region: Northeast & 0.17 & 0.20 & 0.17 & 0.17 & 0.15 & 0.16 &  &  & \\
Region: Midwest & 0.21 & 0.22 & 0.21 & 0.21 & 0.15 & 0.20 &  &  & \\
Region: South & 0.38 & 0.39 & 0.38 & 0.38 & 0.50 & 0.45 &  &  & \\
Region: West & 0.24 & 0.20 & 0.24 & 0.24 & 0.20 & 0.20 &  &  & \\
\addlinespace
Urban: TRUE & 0.73 & 0.78 & 0.74 & 0.73 & 0.73 & 0.69 &  &  & \\
\addlinespace
Employment\_18\_64: Inactive & 0.20 & 0.16 & 0.16 & 0.20 & 0.18 & 0.15 & 0.17 & 0.15 & 0.15\\
Employment\_18\_64: Unemployed & 0.02 & 0.07 & 0.08 & 0.02 & 0.15 & 0.11 & 0.03 & 0.06 & 0.05\\
\addlinespace
Vote: Left & 0.32 & 0.47 & 0.45 & 0.32 & 0.48 & 0.42 & 0.30 & 0.32 & 0.32\\
Vote: Center-right or Right & 0.30 & 0.31 & 0.31 & 0.30 & 0.15 & 0.24 & 0.28 & 0.32 & 0.32\\
Vote: Far right &  &  &  &  &  &  & 0.10 & 0.10 & 0.10\\
\addlinespace
Country: FR &  &  &  &  &  &  & 0.24 & 0.24 & 0.24\\
Country: DE &  &  &  &  &  &  & 0.33 & 0.33 & 0.33\\
Country: ES &  &  &  &  &  &  & 0.18 & 0.18 & 0.18\\
Country: UK &  &  &  &  &  &  & 0.25 & 0.25 & 0.25\\
\addlinespace
Urbanity: Cities &  &  &  &  &  &  & 0.43 & 0.49 & 0.43\\
Urbanity: Towns and suburbs &  &  &  &  &  &  & 0.33 & 0.32 & 0.33\\
Urbanity: Rural &  &  &  &  &  &  & 0.25 & 0.20 & 0.25\\
\bottomrule
\end{tabular}
        }
    }
    {\footnotesize \textit{Note}: This table displays summary statistics of the samples alongside actual population frequencies. %For \textit{Vote}, we regroup candidates or parties into three broad categories and we take abstention into account (but omit this category). 
    %For \textit{Inactivity rate (15-64)}, the sample statistics include the share of respondents aged between 15 and 64 years old who indicated being either ``\textit{Inactive (not searching for a job)},'' a ``\textit{Student},'' or ``\textit{Retired}.'' For \textit{Unemployment rate (15-64)}, the sample statistics include the share of respondents aged between 15 and 64 years old who indicated being ``\textit{Unemployed (searching for a job)}'', (`\textit{Unemployed (searching for a job)},'' ``\textit{Full-time employed},'' ``\textit{Part-time employed},'' or ``\textit{Self-employed}''). For	\textit{Employment rate (15-64)}, the sample statistics include the share of respondents aged between 15 and 64 years old who indicated being either ``\textit{Full-time employed},'' ``\textit{Part-time employed},'' or ``\textit{Self-employed}.'' 
    Detailed sources for each variable and country population frequencies, as well as the definitions of regions, diploma, urbanity, employment, and vote are available in \href{https://github.com/bixiou/international_attitudes_toward_global_policies/raw/main/questionnaire/specificities.xlsx}{this spreadsheet}. % TODO! Appendix \ref{app:sources}.
    } % TODO add hline before Urbanity, move Country/Urbanity above and add in Notes that quotas are those above the line
\end{table}

\begin{table}[h]
    \caption[Sample representativeness of each European country]{Sample representativeness for each European country. (Back to \ref{par:surveys})} \label{tab:representativeness_EU}
    \makebox[\textwidth][c]{
        \resizebox*{!}{.50\textheight}{% 73 without notes cf. https://tex.stackexchange.com/questions/13809/resizing-a-table-by-textheight 
        
\begin{tabular}[t]{lllllllllllll}
\toprule
\multicolumn{1}{c}{} & \multicolumn{3}{c}{FR} & \multicolumn{3}{c}{DE} & \multicolumn{3}{c}{ES} & \multicolumn{3}{c}{UK} \\
\cmidrule(l{3pt}r{3pt}){2-4} \cmidrule(l{3pt}r{3pt}){5-7} \cmidrule(l{3pt}r{3pt}){8-10} \cmidrule(l{3pt}r{3pt}){11-13}
  & Pop. & Sample & \makecell{Weighted\\sample} & Pop. & Sample & \makecell{Weighted\\sample} & Pop. & Sample & \makecell{Weighted\\sample} & Pop. & Sample & \makecell{Weighted\\sample}\\
\midrule
Sample size &  & 620 & 620 &  & 757 & 757 &  & 543 & 543 &  & 644 & 644\\
\addlinespace
Gender: Woman & 0.52 & 0.49 & 0.54 & 0.51 & 0.53 & 0.58 & 0.51 & 0.55 & 0.60 & 0.50 & 0.26 & 0.32\\
Gender: Man & 0.48 & 0.51 & 0.46 & 0.49 & 0.47 & 0.42 & 0.49 & 0.45 & 0.40 & 0.50 & 0.74 & 0.68\\
\addlinespace
Income\_quartile: 1 & 0.25 & 0.30 & 0.27 & 0.25 & 0.28 & 0.23 & 0.25 & 0.27 & 0.23 & 0.25 & 0.32 & 0.28\\
Income\_quartile: 2 & 0.25 & 0.17 & 0.17 & 0.25 & 0.25 & 0.24 & 0.25 & 0.32 & 0.33 & 0.25 & 0.29 & 0.28\\
Income\_quartile: 3 & 0.25 & 0.22 & 0.22 & 0.25 & 0.29 & 0.30 & 0.25 & 0.25 & 0.25 & 0.25 & 0.20 & 0.21\\
Income\_quartile: 4 & 0.25 & 0.32 & 0.34 & 0.25 & 0.18 & 0.23 & 0.25 & 0.15 & 0.19 & 0.25 & 0.19 & 0.23\\
\addlinespace
Age: 18-24 & 0.12 & 0.08 & 0.06 & 0.09 & 0.18 & 0.15 & 0.08 & 0.17 & 0.15 & 0.10 & 0.02 & 0.02\\
Age: 25-34 & 0.15 & 0.17 & 0.16 & 0.15 & 0.21 & 0.20 & 0.12 & 0.15 & 0.14 & 0.17 & 0.10 & 0.09\\
Age: 35-49 & 0.24 & 0.33 & 0.37 & 0.22 & 0.20 & 0.22 & 0.28 & 0.23 & 0.26 & 0.24 & 0.12 & 0.15\\
Age: 50-64 & 0.24 & 0.20 & 0.19 & 0.28 & 0.23 & 0.26 & 0.27 & 0.25 & 0.27 & 0.25 & 0.28 & 0.33\\
Age: 65+ & 0.25 & 0.23 & 0.22 & 0.26 & 0.18 & 0.18 & 0.25 & 0.19 & 0.19 & 0.24 & 0.48 & 0.42\\
\addlinespace
Urbanity: Cities & 0.47 & 0.51 & 0.43 & 0.37 & 0.47 & 0.40 & 0.52 & 0.67 & 0.62 & 0.40 & 0.37 & 0.31\\
Urbanity: Towns and suburbs & 0.19 & 0.18 & 0.18 & 0.40 & 0.34 & 0.34 & 0.22 & 0.27 & 0.29 & 0.42 & 0.46 & 0.47\\
Urbanity: Rural & 0.34 & 0.30 & 0.39 & 0.23 & 0.18 & 0.25 & 0.26 & 0.06 & 0.08 & 0.18 & 0.17 & 0.22\\
\addlinespace
Diploma\_25\_64: Below upper secondary & 0.11 & 0.22 & 0.18 & 0.10 & 0.17 & 0.16 & 0.24 & 0.10 & 0.09 & 0.12 & 0.10 & 0.08\\
Diploma\_25\_64: Upper secondary & 0.26 & 0.15 & 0.24 & 0.27 & 0.11 & 0.18 & 0.16 & 0.15 & 0.23 & 0.21 & 0.18 & 0.29\\
Diploma\_25\_64: Post secondary & 0.26 & 0.33 & 0.30 & 0.29 & 0.36 & 0.33 & 0.28 & 0.38 & 0.33 & 0.33 & 0.23 & 0.20\\
\addlinespace
Employment\_18\_64: Inactive & 0.20 & 0.18 & 0.16 & 0.15 & 0.16 & 0.14 & 0.20 & 0.16 & 0.15 & 0.16 & 0.14 & 0.15\\
Employment\_18\_64: Unemployed & 0.04 & 0.05 & 0.05 & 0.02 & 0.04 & 0.04 & 0.07 & 0.10 & 0.10 & 0.02 & 0.03 & 0.03\\
\addlinespace
Vote: Left & 0.23 & 0.18 & 0.17 & 0.37 & 0.42 & 0.42 & 0.33 & 0.37 & 0.38 & 0.25 & 0.27 & 0.27\\
Vote: Center-right or Right & 0.26 & 0.31 & 0.32 & 0.28 & 0.26 & 0.27 & 0.18 & 0.22 & 0.22 & 0.36 & 0.50 & 0.50\\
Vote: Far right & 0.23 & 0.23 & 0.24 & 0.08 & 0.07 & 0.08 & 0.09 & 0.08 & 0.07 & 0.01 & 0.03 & 0.04\\
\bottomrule
\end{tabular}
        }
    }
    % TODO add explanatory note
    {\footnotesize \textit{Note}: This table displays summary statistics of the samples alongside actual population frequencies. In this Table, weights are defined at the country level.  %For \textit{Vote}, we regroup candidates or parties into three broad categories and we take abstention into account (but omit this category). 
    %For \textit{Inactivity rate (15-64)}, the sample statistics include the share of respondents aged between 15 and 64 years old who indicated being either ``\textit{Inactive (not searching for a job)},'' a ``\textit{Student},'' or ``\textit{Retired}.'' For \textit{Unemployment rate (15-64)}, the sample statistics include the share of respondents aged between 15 and 64 years old who indicated being ``\textit{Unemployed (searching for a job)}'', (`\textit{Unemployed (searching for a job)},'' ``\textit{Full-time employed},'' ``\textit{Part-time employed},'' or ``\textit{Self-employed}''). For	\textit{Employment rate (15-64)}, the sample statistics include the share of respondents aged between 15 and 64 years old who indicated being either ``\textit{Full-time employed},'' ``\textit{Part-time employed},'' or ``\textit{Self-employed}.'' 
    Detailed sources for each variable and country population frequencies, as well as the definitions of regions, diploma, urbanity, employment, and vote are available in \href{https://github.com/bixiou/international_attitudes_toward_global_policies/raw/main/questionnaire/specificities.xlsx}{this spreadsheet}. % TODO Appendix \ref{app:sources}.
    }
\end{table}

Similar tables for the global surveys can be found in \citet{dechezlepretre_fighting_2022}.

\clearpage
\section{Attrition analysis}\label{app:attrition}

\begin{table}[h]\label{tab:attrition_US1}
    \caption[Attrition analysis: \textit{US1}]{Attrition analysis for the \textit{US1} survey.} 
    \makebox[\textwidth][c]{
\resizebox*{!}{.73\textheight}{ % 73 is the max when there is a title
        
\begin{tabular}{@{\extracolsep{5pt}}lccccc} 
\\[-1.8ex]\hline 
\hline \\[-1.8ex] 
\\[-1.8ex] & \makecell{Dropped out} & \makecell{Dropped out\\after\\socio-eco} & \makecell{Failed\\attention test} & \makecell{Duration\\(in min)} & \makecell{Duration\\below\\4 min} \\ 
\\[-1.8ex] & (1) & (2) & (3) & (4) & (5)\\ 
\hline \\[-1.8ex] 
Mean & 0.08 & 0.059 & 0.082 & 21.198 & 0.016  \\ \hline \\[-1.8ex]
 Income quartile: 2 & 0.025$^{***}$ & 0.025$^{***}$ & 0.000 & $-$0.740 & $-$0.009 \\ 
  & (0.010) & (0.010) & (0.000) & (3.064) & (0.006) \\ 
  Income quartile: 3 & 0.062$^{***}$ & 0.062$^{***}$ & 0.000$^{***}$ & 0.754 & $-$0.004 \\ 
  & (0.012) & (0.012) & (0.000) & (2.813) & (0.007) \\ 
  Income quartile: 4 & 0.035$^{***}$ & 0.035$^{***}$ & $-$0.000$^{***}$ & $-$3.917 & $-$0.003 \\ 
  & (0.011) & (0.011) & (0.000) & (2.798) & (0.007) \\ 
  Diploma: Post secondary & 0.039$^{***}$ & 0.039$^{***}$ & $-$0.000$^{***}$ & 1.544 & 0.006 \\ 
  & (0.009) & (0.009) & (0.000) & (2.665) & (0.006) \\ 
  Age: 25\mbox{-}34 & $-$0.094$^{***}$ & $-$0.094$^{***}$ & $-$0.000$^{***}$ & $-$0.597 & $-$0.031$^{**}$ \\ 
  & (0.015) & (0.015) & (0.000) & (2.604) & (0.013) \\ 
  Age: 35\mbox{-}49 & $-$0.100$^{***}$ & $-$0.100$^{***}$ & $-$0.000$^{***}$ & 4.824 & $-$0.032$^{**}$ \\ 
  & (0.015) & (0.015) & (0.000) & (3.176) & (0.013) \\ 
  Age: 50\mbox{-}64 & $-$0.060$^{***}$ & $-$0.060$^{***}$ & 0.000$^{***}$ & 5.723$^{**}$ & $-$0.039$^{***}$ \\ 
  & (0.015) & (0.015) & (0.000) & (2.763) & (0.012) \\ 
  Age: 65+ & 0.048$^{***}$ & 0.048$^{***}$ & 0.000$^{**}$ & 8.952$^{**}$ & $-$0.047$^{***}$ \\ 
  & (0.017) & (0.017) & (0.000) & (4.267) & (0.012) \\ 
  Gender: Man & $-$0.039$^{***}$ & $-$0.039$^{***}$ & $-$0.000$^{*}$ & $-$0.451 & $-$0.0001 \\ 
  & (0.007) & (0.007) & (0.000) & (2.210) & (0.005) \\ 
  Urban & 0.006 & 0.006 & $-$0.000$^{***}$ & 4.888$^{**}$ & $-$0.004 \\ 
  & (0.008) & (0.008) & (0.000) & (2.443) & (0.006) \\ 
  Race: Black & 0.020$^{**}$ & 0.020$^{**}$ & $-$0.000$^{***}$ & 8.554$^{***}$ & 0.004 \\ 
  & (0.010) & (0.010) & (0.000) & (2.600) & (0.007) \\ 
  Race: Hispanic & 0.021$^{**}$ & 0.021$^{**}$ & $-$0.000$^{***}$ & 4.119$^{*}$ & $-$0.002 \\ 
  & (0.010) & (0.010) & (0.000) & (2.293) & (0.007) \\ 
  Region: Northeast & $-$0.005 & $-$0.005 & $-$0.000$^{***}$ & $-$4.862 & $-$0.004 \\ 
  & (0.011) & (0.011) & (0.000) & (4.782) & (0.007) \\ 
  Region: South & $-$0.009 & $-$0.009 & $-$0.000 & $-$1.151 & $-$0.004 \\ 
  & (0.009) & (0.009) & (0.000) & (4.710) & (0.006) \\ 
  Region: West & 0.006 & 0.006 & 0.000$^{***}$ & $-$4.000 & $-$0.003 \\ 
  & (0.011) & (0.011) & (0.000) & (4.305) & (0.007) \\ 
  Vote: Biden & $-$0.048$^{***}$ & $-$0.048$^{***}$ & 0.000$^{***}$ & $-$2.901 & $-$0.009 \\ 
  & (0.008) & (0.008) & (0.000) & (2.379) & (0.007) \\ 
  Vote: Trump & $-$0.043$^{***}$ & $-$0.043$^{***}$ & $-$0.000 & 0.145 & $-$0.005 \\ 
  & (0.009) & (0.009) & (0.000) & (2.878) & (0.008) \\ 
 \hline \\[-1.8ex] 

Observations & 5,719 & 5,719 & 3,252 & 3,044 & 3,044 \\ 
R$^{2}$ & 0.127 & 0.127 & 1.000 & 0.006 & 0.017 \\ 
\hline 
\hline \\[-1.8ex] 
\end{tabular} 
        }
    }
    {\footnotesize %\textit{Note}: 
    }
\end{table}

\begin{table}[h]\label{tab:attrition_US2}
    \caption[Attrition analysis: \textit{US2}]{Attrition analysis for the \textit{US2} survey.} 
    \makebox[\textwidth][c]{
\resizebox*{!}{.73\textheight}{ % 73 is the max when there is a title
        
\begin{tabular}{@{\extracolsep{5pt}}lccccc} 
\\[-1.8ex]\hline 
\hline \\[-1.8ex] 
\\[-1.8ex] & \makecell{Dropped out} & \makecell{Dropped out\\after\\socio-eco} & \makecell{Failed\\attention test} & \makecell{Duration\\(in min)} & \makecell{Duration\\below\\4 min} \\ 
\\[-1.8ex] & (1) & (2) & (3) & (4) & (5)\\ 
\hline \\[-1.8ex] 
Mean & 0.095 & 0.074 & 0.092 & 16.338 & 0.052  \\ \hline \\[-1.8ex]
 Income quartile: 2 & 0.023$^{*}$ & 0.023$^{*}$ & $-$0.000$^{**}$ & 1.352 & $-$0.029$^{**}$ \\ 
  & (0.013) & (0.013) & (0.000) & (1.601) & (0.014) \\ 
  Income quartile: 3 & 0.054$^{***}$ & 0.054$^{***}$ & $-$0.000 & 8.502 & $-$0.009 \\ 
  & (0.014) & (0.014) & (0.000) & (9.649) & (0.016) \\ 
  Income quartile: 4 & 0.060$^{***}$ & 0.060$^{***}$ & $-$0.000 & 5.254 & 0.0003 \\ 
  & (0.016) & (0.016) & (0.000) & (3.376) & (0.017) \\ 
  Diploma: Post secondary & $-$0.033$^{***}$ & $-$0.033$^{***}$ & 0.000 & 1.601 & 0.012 \\ 
  & (0.011) & (0.011) & (0.000) & (2.630) & (0.011) \\ 
  Age: 25\mbox{-}34 & $-$0.004 & $-$0.004 & 0.000 & $-$0.929 & $-$0.032 \\ 
  & (0.015) & (0.015) & (0.000) & (1.535) & (0.024) \\ 
  Age: 35\mbox{-}49 & 0.012 & 0.012 & 0.000$^{***}$ & 9.076 & $-$0.047$^{**}$ \\ 
  & (0.014) & (0.014) & (0.000) & (6.651) & (0.022) \\ 
  Age: 50\mbox{-}64 & 0.040$^{***}$ & 0.040$^{***}$ & $-$0.000$^{***}$ & 0.364 & $-$0.079$^{***}$ \\ 
  & (0.014) & (0.014) & (0.000) & (1.565) & (0.022) \\ 
  Age: 65+ & 0.115$^{***}$ & 0.115$^{***}$ & $-$0.000$^{***}$ & 2.619 & $-$0.095$^{***}$ \\ 
  & (0.017) & (0.017) & (0.000) & (3.150) & (0.022) \\ 
  Gender: Man & $-$0.073$^{***}$ & $-$0.073$^{***}$ & 0.000 & 4.707 & 0.010 \\ 
  & (0.009) & (0.009) & (0.000) & (6.037) & (0.010) \\ 
  Urban & 0.019$^{*}$ & 0.019$^{*}$ & 0.000$^{***}$ & 1.766 & 0.005 \\ 
  & (0.011) & (0.011) & (0.000) & (1.135) & (0.012) \\ 
  Race: Black & 0.060$^{***}$ & 0.060$^{***}$ & 0.000$^{***}$ & 18.673 & $-$0.010 \\ 
  & (0.015) & (0.015) & (0.000) & (13.328) & (0.015) \\ 
  Race: Hispanic & 0.079$^{***}$ & 0.079$^{***}$ & $-$0.000 & 2.930 & $-$0.027$^{**}$ \\ 
  & (0.014) & (0.014) & (0.000) & (1.813) & (0.012) \\ 
  Region: Northeast & $-$0.026$^{*}$ & $-$0.026$^{*}$ & 0.000 & $-$0.837 & $-$0.011 \\ 
  & (0.014) & (0.014) & (0.000) & (2.855) & (0.015) \\ 
  Region: South & $-$0.006 & $-$0.006 & $-$0.000 & 3.220 & 0.009 \\ 
  & (0.012) & (0.012) & (0.000) & (5.002) & (0.014) \\ 
  Region: West & $-$0.010 & $-$0.010 & 0.000 & $-$1.759 & $-$0.009 \\ 
  & (0.013) & (0.013) & (0.000) & (1.942) & (0.015) \\ 
  Vote: Biden & $-$0.049$^{***}$ & $-$0.049$^{***}$ & $-$0.000$^{***}$ & 3.230 & $-$0.006 \\ 
  & (0.008) & (0.008) & (0.000) & (2.731) & (0.014) \\ 
  Vote: Trump & $-$0.026$^{***}$ & $-$0.026$^{***}$ & $-$0.000 & $-$0.554 & 0.007 \\ 
  & (0.009) & (0.009) & (0.000) & (1.272) & (0.016) \\ 
 \hline \\[-1.8ex] 

Observations & 2,973 & 2,973 & 2,280 & 2,103 & 2,103 \\ 
R$^{2}$ & 0.241 & 0.241 & 1.000 & 0.010 & 0.031 \\ 
\hline 
\hline \\[-1.8ex] 
\end{tabular} 
        }
    }
    {\footnotesize %\textit{Note}: 
    }
\end{table}

\begin{table}[h]\label{tab:attrition_EU}
    \caption[Attrition analysis: \textit{Eu}]{Attrition analysis for the \textit{Eu} survey.} 
    \makebox[\textwidth][c]{
\resizebox*{!}{.73\textheight}{ % 73 is the max when there is a title
        
\begin{tabular}{@{\extracolsep{5pt}}lccccc} 
\\[-1.8ex]\hline 
\hline \\[-1.8ex] 
\\[-1.8ex] & \makecell{Dropped out} & \makecell{Dropped out\\after\\socio-eco} & \makecell{Failed\\attention test} & \makecell{Duration\\(in min)} & \makecell{Duration\\below\\6 min} \\ 
\\[-1.8ex] & (1) & (2) & (3) & (4) & (5)\\ 
\hline \\[-1.8ex] 
Mean & 0.067 & 0.044 & 0.151 & 54.602 & 0.039  \\ \hline \\[-1.8ex]
 Country: Germany & 0.035$^{***}$ & 0.031$^{***}$ & $-$0.014 & 9.533 & 0.019$^{*}$ \\ 
  & (0.012) & (0.012) & (0.011) & (18.906) & (0.010) \\ 
  Country: Spain & $-$0.059$^{***}$ & $-$0.057$^{***}$ & $-$0.028$^{**}$ & $-$29.136$^{*}$ & 0.010 \\ 
  & (0.012) & (0.012) & (0.013) & (15.948) & (0.010) \\ 
  Country: United Kingdom & 0.062$^{***}$ & 0.062$^{***}$ & 0.002 & $-$7.458 & 0.010 \\ 
  & (0.013) & (0.013) & (0.013) & (18.046) & (0.010) \\ 
  Income quartile: 2 & 0.029$^{**}$ & 0.027$^{**}$ & $-$0.006 & 32.749$^{*}$ & $-$0.015 \\ 
  & (0.012) & (0.012) & (0.012) & (19.771) & (0.010) \\ 
  Income quartile: 3 & 0.042$^{***}$ & 0.040$^{***}$ & $-$0.027$^{**}$ & 6.130 & $-$0.021$^{**}$ \\ 
  & (0.013) & (0.012) & (0.011) & (11.734) & (0.010) \\ 
  Income quartile: 4 & 0.018 & 0.016 & $-$0.013 & 18.659 & $-$0.018$^{*}$ \\ 
  & (0.013) & (0.013) & (0.012) & (19.955) & (0.011) \\ 
  Diploma: Post secondary & 0.020$^{**}$ & 0.020$^{**}$ & $-$0.017$^{**}$ & 10.647 & $-$0.007 \\ 
  & (0.010) & (0.010) & (0.008) & (12.959) & (0.007) \\ 
  Age: 25\mbox{-}34 & 0.029$^{**}$ & 0.026$^{*}$ & 0.032$^{*}$ & 36.132 & $-$0.005 \\ 
  & (0.014) & (0.013) & (0.018) & (22.285) & (0.018) \\ 
  Age: 35\mbox{-}49 & 0.073$^{***}$ & 0.071$^{***}$ & 0.014 & 37.159$^{**}$ & $-$0.013 \\ 
  & (0.013) & (0.013) & (0.016) & (17.190) & (0.016) \\ 
  Age: 50\mbox{-}64 & 0.091$^{***}$ & 0.088$^{***}$ & 0.008 & 48.363$^{**}$ & $-$0.063$^{***}$ \\ 
  & (0.014) & (0.014) & (0.016) & (22.526) & (0.015) \\ 
  Age: 65+ & 0.159$^{***}$ & 0.156$^{***}$ & 0.021 & 36.351$^{**}$ & $-$0.061$^{***}$ \\ 
  & (0.015) & (0.015) & (0.016) & (14.226) & (0.015) \\ 
  Gender: Man & $-$0.021$^{**}$ & $-$0.021$^{**}$ & 0.020$^{**}$ & $-$22.980 & 0.009 \\ 
  & (0.009) & (0.009) & (0.008) & (14.093) & (0.007) \\ 
  Degree of urbanization: Towns and suburbs & $-$0.004 & $-$0.006 & $-$0.024$^{***}$ & $-$16.736 & 0.004 \\ 
  & (0.010) & (0.010) & (0.009) & (17.256) & (0.008) \\ 
  Degree of urbanization: Rural & 0.004 & 0.004 & $-$0.018$^{*}$ & $-$14.593 & $-$0.001 \\ 
  & (0.012) & (0.012) & (0.011) & (19.733) & (0.009) \\ 
  Vote: Center\mbox{-}right or Right & $-$0.025$^{***}$ & $-$0.025$^{***}$ & $-$0.004 & $-$17.558 & 0.019$^{**}$ \\ 
  & (0.005) & (0.005) & (0.003) & (13.143) & (0.008) \\ 
  Vote: Far right & 0.015$^{**}$ & 0.014$^{**}$ & $-$0.008 & 15.838 & 0.029$^{**}$ \\ 
  & (0.007) & (0.007) & (0.005) & (32.281) & (0.014) \\ 
  Vote: PNR\slash Non\mbox{-}voter & 0.227$^{***}$ & 0.221$^{***}$ & 0.197$^{***}$ & 24.631 & 0.030$^{***}$ \\ 
  & (0.011) & (0.011) & (0.012) & (19.824) & (0.010) \\ 
 \hline \\[-1.8ex] 

Observations & 3,963 & 3,963 & 3,326 & 3,115 & 3,115 \\ 
R$^{2}$ & 0.188 & 0.183 & 0.157 & 0.006 & 0.028 \\ 
\hline 
\hline \\[-1.8ex] 
\end{tabular} 
        }
    }
    {\footnotesize %\textit{Note}: 
    }
\end{table}

\clearpage
\section{Balance analysis}\label{app:balance}

\begin{table}[h]\label{tab:balance}
    \caption[Balance analysis]{Balance analysis.} 
    \makebox[\textwidth][c]{
\resizebox*{!}{.72\textheight}{ % 73 is the max when there is a title
        
\begin{tabular}{@{\extracolsep{5pt}}lcccc} 
\\[-1.8ex]\hline 
\hline \\[-1.8ex] 
\\[-1.8ex] & \makecell{List contains: G} & \makecell{Branch petition: NR} & \makecell{Branch donation: Own nation} & \makecell{Branch conjoint 3: with GCS} \\ 
\\[-1.8ex] & (1) & (2) & (3) & (4)\\ 
\hline \\[-1.8ex] 
Mean & 0.496 & 0.493 & 0.5 & 0.499  \\ \hline \\[-1.8ex]
 Country: Germany & $-$0.026 & 0.017 & 0.020 & 0.005 \\ 
  & (0.026) & (0.026) & (0.026) & (0.026) \\ 
  Country: Spain & 0.025 & 0.026 & 0.026 & 0.043 \\ 
  & (0.030) & (0.030) & (0.030) & (0.030) \\ 
  Country: United Kingdom & 0.002 & 0.018 & 0.037 & 0.063$^{**}$ \\ 
  & (0.028) & (0.028) & (0.028) & (0.028) \\ 
  Country: United States & $-$0.001 & 0.019 & 0.007 & 0.023 \\ 
  & (0.024) & (0.024) & (0.024) & (0.024) \\ 
  Income quartile: 2 & $-$0.013 & $-$0.024 & 0.012 & $-$0.010 \\ 
  & (0.021) & (0.021) & (0.021) & (0.021) \\ 
  Income quartile: 3 & 0.021 & $-$0.005 & 0.011 & $-$0.004 \\ 
  & (0.022) & (0.022) & (0.022) & (0.022) \\ 
  Income quartile: 4 & $-$0.001 & $-$0.017 & $-$0.013 & 0.0001 \\ 
  & (0.023) & (0.023) & (0.023) & (0.023) \\ 
  Diploma: Post secondary & 0.008 & 0.014 & $-$0.010 & $-$0.001 \\ 
  & (0.016) & (0.016) & (0.016) & (0.016) \\ 
  Age: 25\mbox{-}34 & 0.023 & $-$0.049 & $-$0.003 & $-$0.009 \\ 
  & (0.031) & (0.031) & (0.031) & (0.031) \\ 
  Age: 35\mbox{-}49 & 0.032 & $-$0.002 & $-$0.014 & $-$0.016 \\ 
  & (0.030) & (0.030) & (0.030) & (0.030) \\ 
  Age: 50\mbox{-}64 & 0.030 & $-$0.005 & $-$0.016 & $-$0.020 \\ 
  & (0.030) & (0.030) & (0.030) & (0.030) \\ 
  Age: 65+ & 0.029 & $-$0.037 & $-$0.015 & $-$0.012 \\ 
  & (0.037) & (0.037) & (0.037) & (0.037) \\ 
  Gender: Man & 0.024 & 0.012 & 0.002 & $-$0.016 \\ 
  & (0.015) & (0.015) & (0.015) & (0.015) \\ 
  Degree of urbanization: Towns and suburbs & $-$0.010 & $-$0.0005 & $-$0.010 & $-$0.011 \\ 
  & (0.017) & (0.017) & (0.017) & (0.017) \\ 
  Degree of urbanization: Rural & 0.013 & 0.017 & $-$0.004 & 0.027 \\ 
  & (0.024) & (0.024) & (0.024) & (0.024) \\ 
  Employment status: Retired & $-$0.005 & $-$0.031 & $-$0.034 & $-$0.016 \\ 
  & (0.032) & (0.032) & (0.032) & (0.032) \\ 
  Employment status: Student & 0.005 & $-$0.023 & $-$0.033 & $-$0.025 \\ 
  & (0.044) & (0.044) & (0.044) & (0.044) \\ 
  Employment status: Working & 0.010 & $-$0.027 & $-$0.033 & $-$0.012 \\ 
  & (0.024) & (0.024) & (0.024) & (0.024) \\ 
  Vote: Center\mbox{-}right or Right & $-$0.004 & 0.003 & 0.010 & 0.002 \\ 
  & (0.017) & (0.017) & (0.017) & (0.017) \\ 
  Vote: PNR\slash Non\mbox{-}voter & 0.001 & 0.014 & $-$0.005 & $-$0.012 \\ 
  & (0.019) & (0.019) & (0.019) & (0.019) \\ 
  Vote: Far right & 0.009 & 0.030 & 0.023 & 0.038 \\ 
  & (0.034) & (0.034) & (0.035) & (0.034) \\ 
 \hline \\[-1.8ex] 

Observations & 5,991 & 5,991 & 5,991 & 5,991 \\ 
R$^{2}$ & 0.003 & 0.003 & 0.002 & 0.003 \\ 
\hline 
\hline \\[-1.8ex] 
\end{tabular} 
        }
    }
    {\footnotesize \textit{Note}: Standard errors are reported in parentheses.
    }
\end{table}
\clearpage

\section{Placebo tests}\label{app:placebo}

\begin{table}[h]\label{tab:placebo}
    \caption[Placebo tests]{Placebo tests.} 
    \makebox[\textwidth][c]{
% \resizebox*{!}{.73\textheight}{ % 73 is the max when there is a title
        
\begin{tabular}{@{\extracolsep{5pt}}lccccc} 
\\[-1.8ex]\hline 
\hline \\[-1.8ex] 
\\[-1.8ex] & \makecell{G+R+C\\preferred to\\R+C} & \makecell{Supports\\G+R+C} & \makecell{Signs\\petition} & \makecell{Share of\\policies\\supported} & \makecell{Conjoint 5\\A+CGS\\preferred to B} \\ 
\\[-1.8ex] & (1) & (2) & (3) & (4) & (5)\\ 
\hline \\[-1.8ex] 
Mean & 0.642 & 0.63 & 0.612 & 0.532 & 0.591  \\ \hline \\[-1.8ex]
 Branch of list experiment: l & $-$0.013 & $-$0.024 & $-$0.019 & $-$0.013 & $-$0.018 \\ 
  & (0.019) & (0.019) & (0.019) & (0.012) & (0.021) \\ 
  Branch of list experiment: rgl & 0.005 & 0.006 & $-$0.002 & 0.001 & 0.010 \\ 
  & (0.019) & (0.019) & (0.019) & (0.012) & (0.021) \\ 
  Branch of list experiment: rl & $-$0.009 & $-$0.005 & 0.022 & 0.007 & 0.007 \\ 
  & (0.019) & (0.019) & (0.019) & (0.012) & (0.021) \\ 
  Branch of petition: nr & 0.011 & 0.006 & 0.022 & 0.003 & $-$0.006 \\ 
  & (0.014) & (0.014) & (0.014) & (0.009) & (0.015) \\ 
  Poor is in own country & $-$0.002 & $-$0.003 & 0.015 & 0.003 & $-$0.020 \\ 
  & (0.014) & (0.014) & (0.014) & (0.009) & (0.015) \\ 
 \hline \\[-1.8ex] 

Observations & 6,000 & 6,000 & 6,000 & 6,000 & 5,218 \\ 
R$^{2}$ & 0.0004 & 0.001 & 0.002 & 0.001 & 0.001 \\ 
\hline 
\hline \\[-1.8ex] 
\end{tabular} 
        }
    % }
    {\footnotesize \textit{Note}: Standard errors are reported in parentheses.
    }
\end{table}

\section{Main results on the extended sample}\label{app:extended}

As a robustness check, we reproduce our main results on the extended sample that includes the 14\% respondents who failed the attention check ($n = 9,318$). These results are non-weighted. They closely match the results in our main specification. For example, the support for the GCS is 54\% in the U.S. and 75\% in Europe, while the same coefficients are significant for the list experiment. % and the conjoint analyses. 

\begin{figure}[h!] 
    \caption[(Extended sample) Main attitudes by vote]{[Extended sample] Main attitudes by vote (``Right'' spans from Center-right to Far right). \\ (Relative support in percent in Questions \ref{q:gcs_support}, \ref{q:global_tax}, \ref{q:other_policies}, \ref{q:foreign_aid_raise_support}, \ref{q:negotiation}) \hfill (Back~to~Section~\ref{subsec:universalistic})}\label{fig:main_by_vote_alla}
    \makebox[\textwidth][c]{\includegraphics[width=\textwidth]{../figures/country_comparison/main_alla_share.pdf}} 
\end{figure}

\begin{figure}[h!]
    \caption[(Extended sample) Influence of the GCS on preferred platform]{[Extended sample] Influence of the GCS on preferred platform:\\ Preference for a random platform A that contains the Global Climate Scheme rather than a platform B that does not (in percent). (Question \ref{q:conjoint_d}; in the U.S., asked only to non-Republicans.)}\label{fig:conjoint_left_ag_b_alla}
    \makebox[\textwidth][c]{\includegraphics[width=\textwidth]{../figures/country_comparison/conjoint_left_ag_b_binary_alla_positive.pdf}} 
\end{figure}

  
\begin{table}[h]
  \caption[(Extended sample) List experiment: tacit support for the GCS]{[Extended sample] Number of supported policies in the list experiment depending on the presence of the Global Climate Scheme (GCS) in the list. 
   The tacit support for the GCS is estimated by regressing the number of supported policies on the presence of the GCS in the list of policies. The social desirability is estimated as the difference between the tacit and stated support, and it is not significantly different from zero even at a 20\% threshold (see \nameref{sec:methods}).
  }\label{tab:list_exp_alla}
  \makebox[\textwidth][c]{\input{../tables/continents/reg_list_exp_g_alla.tex}
  }  
\end{table}


\begin{table}[h]
    \caption[(Extended sample) Influence of the GCS on electoral prospects]{[Extended sample] Preference for a progressive platform depending on whether it includes the GCS or not. (Question \ref{q:conjoint_c}) 
  } 
    \makebox[\textwidth][c]{\input{../tables/country_comparison/conjoint_c_wo_none_alla.tex}}\label{tab:_alla}
    {\footnotesize \textit{Note:} Simple OLS model. \textit{None of them} answers have been excluded from the regression samples. $^{*}p<0.1$; $^{**} p<0.05$; $^{***} p<0.01$. 
    }
\end{table}

\clearpage
\section{Effect of questionnaire framing}\label{app:ordering_us}

When comparing the samples \textit{US1} and \textit{US2}, we observe no effect of questionnaire framing (or block sequence) on the question ``What group do you defend when you vote?'', common to all waves and placed close to the end of the questionnaire (Table \ref{tab:ordering_us}).

\begin{table}[h]
    \caption[Effect of the wave on group defend when voting]{Effect of the wave (\textit{US1} vs. \textit{US2}) on the group defend when voting. \\ ``What group do you defend when you vote?'' (Question \ref{q:group_defended}) 
  } 
    \makebox[\textwidth][c]{
\begin{tabular}{@{\extracolsep{5pt}}lccc} 
\\[-1.8ex]\hline 
\hline \\[-1.8ex] 
 & \multicolumn{3}{c}{Group defended when voting} \\ 
\cline{2-4} 
\\[-1.8ex] & Humans \textit{or} Sentient beings & Fellow citizens & Family and self \\ 
\\[-1.8ex] & (1) & (2) & (3)\\ 
\hline \\[-1.8ex] 
 Wave: \textit{US2} & $-$0.009 & 0.009 & 0.010 \\ 
  & (0.014) & (0.014) & (0.011) \\ 
 \hline \\[-1.8ex] 
Mean & 0.432 & 0.335 & 0.156 \\ 
Observations & 5,000 & 5,000 & 5,000 \\ 
R$^{2}$ & 0.0001 & 0.0001 & 0.0002 \\ 
\hline 
\hline \\[-1.8ex] 
\end{tabular} }\label{tab:ordering_us}
    {\footnotesize \textit{Note:} Simple OLS model. $^{*}p<0.1$; $^{**} p<0.05$; $^{***} p<0.01$. 
    }
\end{table}