\documentclass[12pt,english]{article}
\usepackage[utf8]{inputenc}
\usepackage{tgpagella} % Palatino text only
\usepackage{mathpazo}  % Palatino math & text
\usepackage[left=1.5in,right=1.5in,top=1.5in,bottom=1.5in]{geometry}
% \linespread{1.5}
\usepackage[super,comma,sort]{natbib}
% \usepackage[round,sort&compress]{natbib}
\usepackage{url} % [hyphens]
\usepackage[hyperpageref]{backref} % back references biblio. Needs latexmk at compilation.
\usepackage[pagebackref]{hyperref}
% \usepackage{multibib} % incompatible with backref
\hypersetup{
  colorlinks=true, % breaklinks=true,
  urlcolor=purple,    % color of external links
  linkcolor=blue,  % color of toc, list of figs etc.
  citecolor=violet,   % color of links to bibliography
}
\usepackage{bm}
\usepackage{indentfirst}
\usepackage{tocbibind}
\setcitestyle{aysep={}} 
\usepackage{amsmath}
\usepackage{amssymb}
\usepackage{eurosym}
\usepackage{amsfonts}
\usepackage{enumerate}
\usepackage{babel}
\usepackage{caption}
\usepackage{supertabular}
\usepackage{tabularx}
\usepackage{float}
\usepackage{dsfont}
\usepackage{fancyvrb}
\usepackage{verbatim}
\usepackage{enumitem}
\usepackage{setspace}
\usepackage{comment}
\usepackage{subcaption}
\usepackage{graphicx}
\usepackage{tikz}
\usepackage{gensymb}
\usepackage{textcomp}

\usepackage{tabulary}
\usepackage{tabularx}
\usepackage{booktabs}
\usepackage{fullpage}
\usepackage{morefloats}
\usepackage{makecell}
\usepackage{lscape}
\usepackage{pdflscape}
\usepackage{longtable}
\usepackage{rotating}
\usepackage{fancyhdr}
\usepackage{tocloft}
\usepackage{titletoc}
\usepackage[export]{adjustbox}
\usepackage[anythingbreaks]{breakurl} % for links
\usepackage{multicol}
\newsavebox\ltmcbox % For net gain table over two columns
%\usepackage[nomarkers,figuresonly]{endfloat} % Figures at the end
%\usepackage[section,below]{placeins} % Floats placed in the section they appear in.
\renewcommand{\floatpagefraction}{.9}

\title{Global Solidarity Advocates -- Position Paper
} 

\author{Adrien Fabre\footnote{CNRS, CIRED. E-mail: fabre.adri1@gmail.com (corresponding author).}} 

\date{\today} 

\begin{document}

\maketitle

\begin{center}
{\textbf{\href{https://github.com/bixiou/global_tax_attitudes/raw/main/paper/position_paper.pdf}{Link to most recent version}}}
\end{center}

% Figure showing inequalities between countries
% Our policies may seem utopian but they are the minimum we should do, a first step towards world fiscal federation, cf. Kopzcuk
% Compute transfer from HIC to LIC by GCS and by millionaire tax, hoping it's 1% of global GDP.
% redistribution would lower interest rates in LIC
% complementarity between policies, e.g. opt out GCS under condition of wealth tax + wealth tax funding carbon removal
% need for other climate policies and for climate finance


% en vrac:
% On calcule la taxe sur la fortune de sorte que ça compense le gain qu'ils obtiennent les recettes en fonction de leurs émissions territoriales plutôt que de leur empreinte. =$>$ calculer le montant nécessaire -$>$ Environ .1\% du PIB mondial pour la Chine (cf. policy\_brief\_tax). Si 1/3 est reversé aux pays à bas revenus, il faut une taxe qui rapporte .3\% du PIB mondial en Chine, soit $\approx 2\%$ du PIB mondial en tout. Ce qui est faisable en ne s'attaquant qu'aux fortunes $>$5M et sans même les réduire (taux max de 7\%). Ça opérerait un transfert de $\approx.6\%$ du PIB mondial, du même ordre que le $\approx$.75\% du GCS. À comparer au .85T\$ (surestimé, cf. calcul dans map\_GCS\_incidence.R) nécessaire pour résorber le poverty gap à 3.65\$ (il est de 4T pour le pg à 6.85\$/day) % TODO! faire ces calculs dans le position paper (ou en off)
% Finally, financial actions are needed to de-risk low carbon projects (especially in low-income countries) and unleash green investments (\href{https://www.foreign.gov.bb/the-2022-barbados-agenda/}{Bridgetown Initiative}, \citealp{hourcade_accelerating_2021}).
% il faut adhérer au club GCS pour toucher les recettes (le mettre dans position paper)

\renewcommand{\url}[1]{\href{#1}{Link}} % NCCcomment
\bibliographystyle{plainnaturl_clean} % NCCcomment
\bibliography{global_tax_attitudes}

\end{document}