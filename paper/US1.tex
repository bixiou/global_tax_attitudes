\documentclass{nature}
% \documentclass[12pt,english]{article}
% \usepackage[utf8]{inputenc}
% \usepackage[left=1.5in,right=1.5in,top=1.5in,bottom=1.5in]{geometry}
% \usepackage{bm}
\usepackage{amsmath}
\usepackage{amssymb}
% \usepackage{indentfirst}
% \usepackage[hyperpageref]{backref} % back references biblio
% \usepackage{tocbibind}
% \usepackage[round,sort&compress]{natbib}
% \setcitestyle{aysep={}} 
% \usepackage{amsfonts}
% \usepackage{enumerate}
% \usepackage{babel}
% \usepackage{caption}
% \usepackage{supertabular}
% \usepackage{tabularx}
% \usepackage{float}
% \usepackage{dsfont}
% \usepackage{fancyvrb}
% \usepackage{verbatim}
% \usepackage[hyphens]{url}
% \usepackage{hyperref}
% \usepackage{enumitem}
% \usepackage{setspace}
% \usepackage{comment}
% \usepackage{subcaption}
% \usepackage{graphicx}
% \usepackage{tikz}
% \usepackage{gensymb}
\usepackage{eurosym}
% \usepackage{textcomp}

% \usepackage{tabulary}
% \usepackage{tabularx}
% \usepackage{booktabs}
% \usepackage{fullpage}
% \usepackage{morefloats}
% % \usepackage[utf8]{inputenc}
% % \usepackage{bm}
% % \usepackage{indentfirst}
% % \usepackage{tocbibind}
% % \usepackage{enumerate}
% \usepackage{makecell}
% \usepackage{lscape}
% \usepackage{pdflscape}
% \usepackage{longtable}
% \usepackage{rotating}
% \usepackage{fancyhdr}
% \usepackage{tocloft}
% \usepackage{multibib}
% \usepackage{titletoc}
% \usepackage[export]{adjustbox}
% \usepackage[anythingbreaks]{breakurl} % for links
% %\usepackage[nomarkers,figuresonly]{endfloat} % Figures at the end
% \hypersetup{
%   colorlinks=true, % breaklinks=true,
%   urlcolor=purple,    % color of external links
%   linkcolor=blue,  % color of toc, list of figs etc.
%   citecolor=violet,   % color of links to bibliography
% }
% %\usepackage[section,below]{placeins} % Floats placed in the section they appear in.

% % Getting landscape page and page number/footer on bottom of page (instead of to the left)
% \fancypagestyle{mylandscape}{
% \fancyhf{} %Clears the header/footer
% \fancyfoot{% Footer
% \makebox[\textwidth][r]{% Right
%   \rlap{\hspace{1.5cm}% Push out of margin by \footskip
%     \smash{% Remove vertical height
%       \raisebox{13.6cm}{% Raise vertically
%         \rotatebox{90}{\thepage}}}}}}% Rotate counter-clockwise
% \renewcommand{\headrulewidth}{0pt}% No header rule
% \renewcommand{\footrulewidth}{0pt}% No footer rule
% }

% \fancypagestyle{page_left}{%
% 	\renewcommand{\headrulewidth}{0pt}
%   \fancyhf{}
%   \fancyfoot[OC]{%
%       \begin{tikzpicture}[remember picture,overlay]
%           \node[xshift=1cm] (number) at (current page.west) {\thepage};
%       \end{tikzpicture}
%   }%
% }
% \renewcommand{\thesubfigure}{\Alph{subfigure}}

% \newcites{App}{Appendix References}

% \captionsetup[table]{skip=-10pt}

% \title{International Attitudes Toward Global Policies \\ Addressing Climate Change and Inequality\thanks{Corresponding author: Fabre: CNRS, CIRED (fabre.adri1@gmail.com).
% Thomas Douenne: University of Amsterdam (t.r.g.r.douenne@uva.nl). Linus Mattauch: TU Berlin (linus.mattauch@tu-berlin.de). We are grateful for financial support from the University of Amsterdam and TU Berlin. We are grateful for financial support from the OECD, the French Ministry of Foreign Affairs, the French Conseil d’Analyse Economique and the Spanish Ministry for the Ecological Transition and Demographic Challenge. We also acknowledge support from the Grantham Foundation for the Protection of the Environment and the Economic and Social Research Council through the Centre for Climate Change Economics and Policy. We thank Antoine Dechezleprêtre, Tobias Kruse, Bluebery Planterose, Ana Sanchez Chico, and Stefanie Stantcheva for their invaluable inputs for the project. We thank Auriane Meilland for feedback. We thank Laura Schepp, Martín Fernández-Sánchez, Samuel Gervais, Samuel Haddad, and Guadalupe Manzo for assistance in the translation. The project %is approved by IRB at Harvard University (IRB21-0137), and 
% was preregistered in the Ooen Science Foundation registry (osf.io/fy6gd).}}
% %\author{OECD}
% \author{Adrien Fabre, Thomas Douenne, and Linus Mattauch}
% \date{\today}

% \begin{document}

% \maketitle

% \begin{small}
% \begin{abstract}

% \noindent \cite{dechezlepretre_fighting_2022-1} %Using new surveys on more than 40,000 respondents in twenty countries that account for 72\% of global CO$_\text{2}$ emissions, we study the understanding of and attitudes toward climate change and climate policies. We show that, across countries, support for climate policies hinges on three key perceptions centered around the effectiveness of the policies in reducing emissions (effectiveness concerns), their distributional impacts on lower-income households (inequality concerns), and their impact on the respondents' household (self-interest). We show experimentally that information  specifically addressing these key concerns can substantially increase the support for climate policies in many countries. Explaining how policies work and who can benefit from them is critical to foster policy support, whereas simply informing people about the impacts of climate change is not effective. Furthermore, we identify several socioeconomic and lifestyle factors -- most notably education, political leanings, and availability of public transportation -- that are significantly correlated with both policy views and overall reasoning and beliefs about climate policies. However, it is difficult to predict beliefs or policy views based on these characteristics only. 

% \end{abstract}

% \textbf{JEL codes:} P48, Q58, H23, Q54
% % Q54 Climate • Natural Disasters and Their Management • Global Warming
% % Q58 Government Policy (Q is Environmental econ)
% % D78 Positive Analysis of Policy Formulation and Implementation
% % H23 Externalities • Redistributive Effects • Environmental Taxes and Subsidies (H is public econ)
% % P48 Political Economy • Legal Institutions • Property Rights • Natural Resources • Energy • Environment • Regional Studies (P4 is Other economic systems)
% % H41 Public Goods
% % H54 Infrastructures • Other Public Investment and Capital Stock


% \textbf{Keywords:} Climate change, global policies, cap-and-trade, perceptions, survey, inequality, wealth tax.

% \end{small} 

% \clearpage
% % \startcontents
% % \printcontents{ }{1}{\section*{\contentsname}}
% % \clearpage
% \section{Introduction\label{sec:intro}}

% % \clearpage
% \renewcommand{\bibsection}{\section{\refname}}
% \bibliographystyle{naturemag}
% \bibliography{global_tax_attitudes}
% % \stopcontents

% \end{document}

% The following allows keeping figures within the text (otherwise nature.cls would ignore them)
\usepackage{graphicx}
\makeatletter
\let\saved@includegraphics\includegraphics
\AtBeginDocument{\let\includegraphics\saved@includegraphics}
\renewenvironment*{figure}{\@float{figure}}{\end@float}
\makeatother

\title{International Attitudes Toward Global Policies %\\ Addressing Climate Change and Inequality
} 

\author{Adrien Fabre$^{1,2}$, Thomas Douenne$^3$ and Linus Mattauch$^4$}


\begin{document}

% Nature guidelines (not NCC!)
% Sections can only be used in Articles.  Contributions should be organized in the sequence: title, text, methods, references, Supplementary Information line (if any), acknowledgements, interest declaration, corresponding author line, tables, figure legends.

% Spelling must be British English (Oxford English Dictionary)

%Each figure legend should begin with a brief title for the whole figure and continue with a short description of each panel and the symbols used. For contributions with methods sections, legends should not contain any details of methods, or exceed 100 words (fewer than 500 words in total for the whole paper). In contributions without methods sections, legends should be fewer than 300 words (800 words or fewer in total for the whole paper).

% Articles are restricted to 50 references,

% In addition, a cover letter needs to be written with the
% following:
% \begin{enumerate}
%  \item A 100 word or less summary indicating on scientific grounds
% why the paper should be considered for a wide-ranging journal like
% \textsl{Nature} instead of a more narrowly focussed journal.
%  \item A 100 word or less summary aimed at a non-scientific audience,
% written at the level of a national newspaper.  It may be used for
% \textsl{Nature}'s press release or other general publicity.
%  \item The cover letter should state clearly what is included as the
% submission, including number of figures, supporting manuscripts
% and any Supplementary Information (specifying number of items and
% format).
%  \item The cover letter should also state the number of
% words of text in the paper; the number of figures and parts of
% figures (for example, 4 figures, comprising 16 separate panels in
% total); a rough estimate of the desired final size of figures in
% terms of number of pages; and a full current postal address,
% telephone and fax numbers, and current e-mail address.
% \end{enumerate}

% See \textsl{Nature}'s website
% (\texttt{http://www.nature.com/nature/submit/gta/index.html}) for
% complete submission guidelines.

\maketitle

\begin{affiliations}
\item CNRS
\item CIRED
\item University of Amsterdam
\item TU Berlin
\end{affiliations}

\begin{abstract} % 250 words. TODO? mention more the other measures?
  The ``global climate scheme'' (a global carbon price funding a global basic income) would be an effective and progressive way to combat climate change, and poverty. Yet, such policy is mostly absent from political platforms and the policy debate. Using surveys on 40,000 respondents in 20 countries covering 72\% of global CO$_\text{2}$ emissions, we document majority support for this and other global policies. Using a complementary survey on 3,000 U.S. respondents,% and four European countries, 
  we test several hypotheses that could could reconcile strong stated support with a lack of salience of these issues. The complementary analyses show that the stated support is mostly sincere, although we cannot rule out insincerity for 3\% to 9\% of the population from the willingness to sign a real-stake petition and a list experiment, respectively. Global redistributive policies rank high (though not highest) in the prioritization of policies. Conjoint analyses reveal that the Democratic party would not significantly lose votes if it endorsed the global climate scheme, while a candidate at the Democratic primary would actually win votes by doing so. Accurate beliefs about the level of support for the scheme dismisses the hypothesis of pluralistic ignorance of the support. Strong universalistic attitudes are confirmed in more general questions, suggesting that the support cannot be explained away by malleable opinion or experimenter demand. Finally, we conclude that there is no compelling reason why global policies do not enter the public debate or political platforms, as they seem genuinely supported by a majority of the population.
\end{abstract}

% Intro
Transfers from high- to low-income people, hence from high- to low-income countries, are warranted by many ethical paradigms. This is the case of utilitarianism, the benchmark ethical theory used in economics. Utilitarianism assigns the same weight to each person and thus considers that a dollar is better allocated to a low-income person, which has a higher marginal utility than a high-income person.\cite{mill_utilitarianism_1861} 

Addressing global poverty, inequalities and climate change are at the heart of the universally agreed Sustainable Development Goals (SDG). % 12 out of  17
It has been pointed out that low-income countries generally do not have enough domestic resources to eliminate the poverty gap in the short run.\cite{bolch_arithmetics_2022} In other words, it would hardly be possible to achieve the first SDG and end extreme poverty by 2030 without international transfers. 

Climate change is another issue that calls for a global response and international transfers. Postulating %Assuming
that each human has an equal right to emit CO$_\text{2}$, low emitters have a legitimate claim \textit{vis-à-vis} high emitters, that can be settled by monetary transfers. Coupling this burden-sharing principle to the carbon budget (remaining emissions that would be compatible with the Paris agreement) naturally defines a global climate policy. We call it the ``Global climate scheme'' and denote it \textit{G}; it consists of a global cap-and-trade system where emission rights are auctioned each year to polluting firms and the revenues finance a global basic income. Using the price and emissions trajectories from the Stern-Stiglitz report,\cite{stern_report_2017} we estimate that the basic income would amount to \$30 per month for each human above 15 in 2030, enough to lift out of extreme poverty the 700 million people who live with less than PPP \$2 per day. Conversely, high emitters like a typical American (with median U.S. CO$_\text{2}$ emissions) would lose in net \$85 per month, as they would face \$115 per month in price increases (assuming a carbon price of \$90/tCO$_\text{2}$ in 2030).
% G default policy for economists, we focus on it; transfers at heart of COP; global wealth tax proposed by Piketty, Saez (fair and effective); democratisation of int'l institutions recurring topic.
% Few studies on CC burden-sharing, all compatible with G
% Few studies on global policies, but they show support (Ghassim, Carattini 19)
% Here, two sets of results. First, twenty countries. Second, dig deeper using complementary survey.

If high emitters share universalistic ethical values, we expect strong support for G, even in high-income countries. If, on the contrary, people defend their own financial interest, we expect low support for G in high-income countries. 

In this paper, we study attitudes toward global policies that address climate change, global poverty or inequalities, with a focus on G. We measure stated support for different global policies using unpublished results from a recent survey\cite{dechezlepretre_fighting_2022-1} on climate attitudes on 40,000 respondents in 20 countries covering 72\% of global CO$_\text{2}$ emissions. We then conduct a representative survey on 3,000 U.S. respondents to study in detail the sincerity and rationales behind the support for G, the attitudes toward various global policies, global redistribution, and universalistic values.

\section*{Results}
% % 4 most important figures: heatmap OECD, heatmap support, prioritization or conjoint (r), list exp (table)
% H0: Majority support for each global policies except maximum wealth and debt cancellation
% H0: Foreign aid: less than 20\% want a decrease (because nationalist), median wants increase at some conditions (no diversion, human rights) => GCS mostly addresses these points
% H1: List experiment: There seems to be a 8pp social norm (differential of 3pp with NR). No effect of the number of options. TODO: check litterature
% H1: Petition: Small effect against GCS: -4pp
% H1, H2: Conjoint analysis: G|C+R 56%, G|R 59%, G 48% ~ C (|R), G+C|R 56%, C|R 64%, Left+G - Left = -3pp, A+G vs. B 59%
% => G is supported for itself, rather independently from R or C, with similar support to both, and it doesn't significantly penalize the Left, and would help a Democratic candidate
% H1: Prioritization: G has mean only slightly lower than average, makes better than ban of cars and coal exit; global tax on millionaires does as well as wealth tax and almost as good as $15 minimum wage
% H3 belief: No pluralistic ignorance
% H4: A strong majority is universalist/cosmopolitan (TODO: which word?), even a majority for non-Republican

\begin{figure}
  \caption{Share of support (somewhat or strongly) for the main global policies among non-\textit{indifferent}.} % TODO: add G to this heatmap, remove Dependence on what other countries do, change label titles to make it clear that the first one was multiple answers while the others were likert
  \makebox[\textwidth][c]{\includegraphics[width=\textwidth]{../figures/OECD/Heatplot_burden_share_all_positive_countries.pdf}}
\end{figure}


\begin{figure}
  \caption{Support for various global policies in the U.S.}
  \makebox[\textwidth][c]{\includegraphics[width=\textwidth]{../figures/US1/support_likert.pdf}}
\end{figure}


\begin{figure}
  \caption{Points.}
  \makebox[\textwidth][c]{\includegraphics[width=\textwidth]{../figures/US1/points_us.pdf}}
\end{figure}


\begin{figure}
  \caption{Conjoint analysis.}
  \makebox[\textwidth][c]{\includegraphics[width=\textwidth]{../figures/US1/ca_r.png}}
\end{figure}

\begin{table}

\begin{tabular}{@{\extracolsep{5pt}}lc} 
\\[-1.8ex]\hline 
\hline \\[-1.8ex] 
\\[-1.8ex] & Number of supported policies \\ 
\hline \\[-1.8ex] 
Mean & 1.354  \\ \hline \\[-1.8ex]
 List contains: G & 0.496$^{***}$ \\ 
  & (0.069) \\ 
  List contains: R & 0.574$^{***}$ \\ 
  & (0.068) \\ 
  List contains: G \times R & $-$0.033 \\ 
  & (0.119) \\ 
 \hline \\[-1.8ex] 

Observations & 1,045 \\ 
R$^{2}$ & 0.132 \\ 
\hline 
\hline \\[-1.8ex] 
\end{tabular} 
\end{table}

\section*{Discussion} % Summary, conclusion


\begin{methods}
%Put methods in here.  If you are going to subsection it, use \subsection commands.  Methods section should be less than 800 words and if it is less than 200 words, it can be incorporated into the main text.

% \subsection{Method subsection.}

% Here is a description of a specific method used.  Note that the subsection heading ends with a full stop (period) and that the command is \verb|\subsection{}| not \verb|\subsection*{}|.

\end{methods}


\bibliographystyle{naturemag_noURL} % nature class works only with style naturemag or naturemag_noURL, and naturemag bugs if there are certain URLs (like .pdf). Also, nature class only works with \cite, not \citet or \citep.
\bibliography{global_tax_attitudes}


%% Here is the endmatter stuff: Supplementary Info, etc.
%% Use \item's to separate, default label is "Acknowledgements"
\begin{addendum} % 177 words
 \item We are grateful for financial support from the University of Amsterdam and TU Berlin. We are grateful for financial support from the OECD, the French Ministry of Foreign Affairs, the French Conseil d’Analyse Economique and the Spanish Ministry for the Ecological Transition and Demographic Challenge. We also acknowledge support from the Grantham Foundation for the Protection of the Environment and the Economic and Social Research Council through the Centre for Climate Change Economics and Policy. We thank Antoine Dechezleprêtre, Tobias Kruse, Bluebery Planterose, Ana Sanchez Chico, and Stefanie Stantcheva for their invaluable inputs for the project. We thank Auriane Meilland for feedback. We thank Laura Schepp, Martín Fernández-Sánchez, Samuel Gervais, Samuel Haddad, and Guadalupe Manzo for assistance in the translation. 
 \item[Registration] The project %is approved by IRB at Harvard University (IRB21-0137), and 
 was preregistered in the Ooen Science Foundation registry (osf.io/fy6gd).
 \item[Competing Interests] The authors declare that they have no
competing interests.
\item[JEL codes] P48, Q58, H23, Q54.
\item[Keywords] Climate change, global policies, cap-and-trade, perceptions, survey, inequality, wealth tax.
 \item[Correspondence] Correspondence and requests for materials
should be addressed to Adrien Fabre~(email: fabre.adri1@gmail.com).
\end{addendum}

%%
%% TABLES
%%
%% If there are any tables, put them here.
%%

% \begin{table}
% \centering
% \caption{This is a table with scientific results.}
% \medskip
% \begin{tabular}{ccccc}
% \hline
% 1 & 2 & 3 & 4 & 5\\
% \hline
% aaa & bbb & ccc & ddd & eee\\
% aaaa & bbbb & cccc & dddd & eeee\\
% aaaaa & bbbbb & ccccc & ddddd & eeeee\\
% aaaaaa & bbbbbb & cccccc & dddddd & eeeeee\\
% 1.000 & 2.000 & 3.000 & 4.000 & 5.000\\
% \hline
% \end{tabular}
% \end{table}

\end{document}
