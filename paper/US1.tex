% GCS/NR vs. G/R
% test hypotheses 2 (is G realistic) and 3 (is G new to you?) in US2? Bof, on le fait déjà plus ou moins
% taxe carbone progressive
% TODO effect vis-à-vis baseline conjoint analysis r (cf. figure)
% TODO appendices sources, calcul net gain
% TODO not asked this way in FR, DK, US: comment
% TODO check randomization of conjoint analysis

%%%%%%%%%%%%%%%%%%%%%%%%%%%%%%%%%%%%%%%%
%%%%% NATURE CLIMATE CHANGE FORMAT %%%%%
%%%%%%%%%%%%%%%%%%%%%%%%%%%%%%%%%%%%%%%%
%% Comment "% WPcomment" lines, uncomment "% NCCcomment" lines as well as the lines below, replace all citet/citep by cite

% \documentclass{nature}
% \usepackage{amsmath}
% \usepackage{amssymb}
% \usepackage{eurosym}
% % The following allows keeping figures within the text (otherwise nature.cls would ignore them)
% \usepackage{graphicx}
% \makeatletter
% \let\saved@includegraphics\includegraphics
% \AtBeginDocument{\let\includegraphics\saved@includegraphics}
% \renewenvironment*{figure}{\@float{figure}}{\end@float}
% \makeatother

% Nature guidelines (not NCC!)
% Sections can only be used in Articles.  Contributions should be organized in the sequence: title, text, methods, references, Supplementary Information line (if any), acknowledgements, interest declaration, corresponding author line, tables, figure legends.

% No subsubsection nor paragraph

% Spelling must be British English (Oxford English Dictionary)

%Each figure legend should begin with a brief title for the whole figure and continue with a short description of each panel and the symbols used. For contributions with methods sections, legends should not contain any details of methods, or exceed 100 words (fewer than 500 words in total for the whole paper). In contributions without methods sections, legends should be fewer than 300 words (800 words or fewer in total for the whole paper).

% Articles are restricted to 50 references,

% In addition, a cover letter needs to be written with the
% following:
% \begin{enumerate}
%  \item A 100 word or less summary indicating on scientific grounds
% why the paper should be considered for a wide-ranging journal like
% \textsl{Nature} instead of a more narrowly focussed journal.
%  \item A 100 word or less summary aimed at a non-scientific audience,
% written at the level of a national newspaper.  It may be used for
% \textsl{Nature}'s press release or other general publicity.
%  \item The cover letter should state clearly what is included as the
% submission, including number of figures, supporting manuscripts
% and any Supplementary Information (specifying number of items and
% format).
%  \item The cover letter should also state the number of
% words of text in the paper; the number of figures and parts of
% figures (for example, 4 figures, comprising 16 separate panels in
% total); a rough estimate of the desired final size of figures in
% terms of number of pages; and a full current postal address,
% telephone and fax numbers, and current e-mail address.
% \end{enumerate}

% See \textsl{Nature}'s website
% (\texttt{http://www.nature.com/nature/submit/gta/index.html}) for
% complete submission guidelines.

%%%%%%%%%%%%%%%%%%%%%%%%%%%%%%%%
%%%%% WORKING PAPER FORMAT %%%%%
%%%%%%%%%%%%%%%%%%%%%%%%%%%%%%%%
%% Comment "% NCCcomment" lines, uncomment "% WPcomment" lines as well as the lines below
\documentclass[12pt,english]{article}
\usepackage[utf8]{inputenc}
\usepackage{tgpagella} % Palatino text only
\usepackage{mathpazo}  % Palatino math & text
\usepackage[left=1.5in,right=1.5in,top=1.5in,bottom=1.5in]{geometry}
% \linespread{1.5}
\usepackage[super,comma,sort]{natbib}
% \usepackage[round,sort&compress]{natbib}
\usepackage{url} % [hyphens]
\usepackage[hyperpageref]{backref} % back references biblio. Needs latexmk at compilation.
\usepackage[pagebackref]{hyperref}
% \usepackage{multibib} % incompatible with backref
\hypersetup{
  colorlinks=true, % breaklinks=true,
  urlcolor=purple,    % color of external links
  linkcolor=blue,  % color of toc, list of figs etc.
  citecolor=violet,   % color of links to bibliography
}
\usepackage{bm}
\usepackage{indentfirst}
\usepackage{tocbibind}
\setcitestyle{aysep={}} 
\usepackage{amsmath}
\usepackage{amssymb}
\usepackage{eurosym}
\usepackage{amsfonts}
\usepackage{enumerate}
\usepackage{babel}
\usepackage{caption}
\usepackage{supertabular}
\usepackage{tabularx}
\usepackage{float}
\usepackage{dsfont}
\usepackage{fancyvrb}
\usepackage{verbatim}
\usepackage{enumitem}
\usepackage{setspace}
\usepackage{comment}
\usepackage{subcaption}
\usepackage{graphicx}
\usepackage{tikz}
\usepackage{gensymb}
\usepackage{textcomp}

\usepackage{tabulary}
\usepackage{tabularx}
\usepackage{booktabs}
\usepackage{fullpage}
\usepackage{morefloats}
\usepackage{makecell}
\usepackage{lscape}
\usepackage{pdflscape}
\usepackage{longtable}
\usepackage{rotating}
\usepackage{fancyhdr}
\usepackage{tocloft}
\usepackage{titletoc}
\usepackage[export]{adjustbox}
\usepackage[anythingbreaks]{breakurl} % for links
\usepackage{multicol}
\newsavebox\ltmcbox % For net gain table over two columns
%\usepackage[nomarkers,figuresonly]{endfloat} % Figures at the end
%\usepackage[section,below]{placeins} % Floats placed in the section they appear in.
\renewcommand{\floatpagefraction}{.9}

% % Getting landscape page and page number/footer on bottom of page (instead of to the left)
% \fancypagestyle{mylandscape}{
% \fancyhf{} %Clears the header/footer
% \fancyfoot{% Footer
% \makebox[\textwidth][r]{% Right
%   \rlap{\hspace{1.5cm}% Push out of margin by \footskip
%     \smash{% Remove vertical height
%       \raisebox{13.6cm}{% Raise vertically
%         \rotatebox{90}{\thepage}}}}}}% Rotate counter-clockwise
% \renewcommand{\headrulewidth}{0pt}% No header rule
% \renewcommand{\footrulewidth}{0pt}% No footer rule
% }

% \fancypagestyle{page_left}{%
% 	\renewcommand{\headrulewidth}{0pt}
%   \fancyhf{}
%   \fancyfoot[OC]{%
%       \begin{tikzpicture}[remember picture,overlay]
%           \node[xshift=1cm] (number) at (current page.west) {\thepage};
%       \end{tikzpicture}
%   }%
% }
% \renewcommand{\thesubfigure}{\Alph{subfigure}}

% \newcites{App}{Appendix References}

% \captionsetup[table]{skip=-10pt}
% \begin{document}

% \maketitle

% \clearpage
% % \startcontents
% % \printcontents{ }{1}{\section{\contentsname}}
% % \clearpage
% \section{Introduction\label{sec:intro}}

% % \clearpage
% \renewcommand{\bibsection}{\section{\refname}}
% \bibliographystyle{naturemag}
% \bibliography{global_tax_attitudes}
% % \stopcontents

% \end{document}


\title{International Attitudes Toward Global Policies %\\ Addressing Climate Change and Inequality 
} 

\author{Adrien Fabre$^{1,2}$, Thomas Douenne$^3$ and Linus Mattauch$^{4,5,6}$} % WPcomment
\author{Adrien Fabre\footnote{CNRS, CIRED. E-mail: fabre.adri1@gmail.com (corresponding author).}, Thomas Douenne\footnote{University of Amsterdam}\; and Linus Mattauch\footnote{Technical University Berlin, Potsdam Institute for Climate Impact Research and University of Oxford}~~\thanks{The project %is approved by IRB at Harvard University (IRB21-0137), and 
was preregistered in the Open Science Foundation registry (\href{https://osf.io/fy6gd}{osf.io/fy6gd}). \\ We are grateful for financial support from the University of Amsterdam and TU Berlin. %We are grateful for financial support from the OECD, the French Ministry of Foreign Affairs, the French Conseil d’Analyse Economique and the Spanish Ministry for the Ecological Transition and Demographic Challenge. We also acknowledge support from the Grantham Foundation for the Protection of the Environment and the Economic and Social Research Council through the Centre for Climate Change Economics and Policy. 
We thank Antoine Dechezleprêtre, Tobias Kruse, Bluebery Planterose, Ana Sanchez Chico, and Stefanie Stantcheva for their invaluable inputs for the project. We thank Auriane Meilland for feedback. We thank Laura Schepp, Martín Fernández-Sánchez, Samuel Gervais, Samuel Haddad, and Guadalupe Manzo for assistance in the translation. }} % NCCcomment

\date{\today} % NCCcomment

\begin{document}

\maketitle

\begin{center}
{\textbf{\href{https://github.com/bixiou/global_tax_attitudes/raw/main/paper/US1.pdf}{Link to most recent version}}}
\end{center}


% WPcomment
% \begin{affiliations}
% \item CNRS
% \item CIRED
% \item University of Amsterdam
% \item Technical University Berlin
% \item Potsdam Institute for Climate Impact Research 
% \item University of Oxford
% \end{affiliations}

% \begin{small} % NCCcomment
\begin{abstract} % 250 words. TODO? mention more the other measures?
  The ``Global Climate Scheme'' (a global carbon price funding a global basic income) would be an effective and progressive way to combat climate change and poverty. Yet, such policy is mostly absent from political platforms and the policy debate. Using surveys on 40,000 respondents in 20 countries covering 72\% of global CO$_\text{2}$ emissions, we document majority support for this and other global policies. Using a complementary survey on 3,000 U.S. respondents, % and four European countries, 
  we test several hypotheses that could reconcile strong stated support with a lack of salience of these issues. The complementary analyses show that the stated support is mostly sincere, although we cannot rule out insincerity for 3\% to 9\% of the population from the willingness to sign a real-stake petition and a list experiment, respectively. Global redistributive policies rank high (though not highest) in the prioritization of policies. Conjoint analyses reveal that the Democratic party would not significantly lose votes if it endorsed the Global Climate Scheme, while a candidate at the Democratic primary would actually win votes by doing so. Accurate beliefs about the level of support for the scheme dismisses the hypothesis of pluralistic ignorance of the support. Strong universalistic attitudes are confirmed in more general questions, suggesting that the support cannot be explained away by malleable opinion or experimenter demand. In sum, our findings indicate that global policies are genuinely supported by a majority of the population. Public opinion is therefore not the reason that they do not prominently enter political debates. %Finally, we conclude that, at odds with the absence of global policies in the public debate or political platforms, there is no evidence that most people would reject them. %there is no compelling reason why global policies do not enter the public debate or political platforms, as they seem genuinely supported by a majority of the population.
  % The ``Global Climate Scheme'' (a global carbon price funding a global basic income) would be an effective and progressive way to combat climate change and poverty. Yet, such policy is mostly absent from political platforms and the policy debate. Using surveys on 40,000 respondents in 20 countries covering 72% of global CO2 emissions, we document majority support for this and other global policies. Using a complementary survey on 3,000 U.S. respondents, we test several hypotheses that could reconcile strong stated support with a lack of salience of these issues. The complementary analyses show that the stated support is mostly sincere, although we cannot rule out insincerity for 3% to 9% of the population from the willingness to sign a real-stake petition and a list experiment, respectively. Global redistributive policies rank high (though not highest) in the prioritization of policies. Conjoint analyses reveal that the Democratic party would not significantly lose votes if it endorsed the Global Climate Scheme, while a candidate at the Democratic primary would actually win votes by doing so. Accurate beliefs about the level of support for the scheme dismisses the hypothesis of pluralistic ignorance of the support. Strong universalistic attitudes are confirmed in more general questions, suggesting that the support cannot be explained away by malleable opinion or experimenter demand. In sum, our findings indicate that global policies are genuinely supported by a majority of the population. Public opinion is therefore not the reason that they do not prominently enter political debates.
\end{abstract}
% Conf submissions: AFSE, EAERE, JMA, AFEP, Earth System governance, Philo Éco
% \end{small} % NCCcomment
% TODO! update figures, shorten

\textbf{JEL codes:} P48, Q58, H23, Q54 % NCCcomment
% Q54 Climate • Natural Disasters and Their Management • Global Warming
% Q58 Government Policy (Q is Environmental econ)
% D78 Positive Analysis of Policy Formulation and Implementation
% H23 Externalities • Redistributive Effects • Environmental Taxes and Subsidies (H is public econ)
% P48 Political Economy • Legal Institutions • Property Rights • Natural Resources • Energy • Environment • Regional Studies (P4 is Other economic systems)
% H41 Public Goods
% H54 Infrastructures • Other Public Investment and Capital Stock

\textbf{Keywords:} Climate change, global policies, cap-and-trade, perceptions, survey.%, inequality, wealth tax. % NCCcomment

\onehalfspacing % NCCcomment

\section{Introduction}  % NCCcomment
% TD change the intro, e.g. Global poverty and climate change are among the most critical issues faced by the world today", and then explain that the first could be solved by transfer, the second by capping pollution, hence an effective policy to tackle these two problems is the GCS. Yet, this policy is nowhere to be seen in policy debates. Why? In this paper we provide evidence from surveys showing that people all over the world support this policy. To explain this paradox (people stated support vs absence of the policy), we further investigate the sincerity of these claims and rationales behind the support, etc.

Extreme poverty and climate change are among the most critical issues of our time. The first could be solved by redistributive transfers, the second by capping global emissions. %Indeed, transfers from high- to low-income countries are warranted by widespread ethical theories like utilitarianism,\cite{mill_utilitarianism_1861}.  
A fair and effective policy to tackle these two problems is the ``Global Climate Scheme'' (GCS), which combines these two solutions. The GCS consists of a global cap-and-trade system, where emission rights are auctioned each year to polluting firms, and of a global basic income, funded by the auction revenues. %Using the price and emissions trajectories from the Stern-Stiglitz report,\cite{stern_report_2017} we estimate that the basic income would amount to \$30 per month for each human above 15 in 2030, enough to lift out of extreme poverty the 700 million people who live with less than PPP \$2 per day. Conversely, high emitters like a typical American (with median U.S. CO$_\text{2}$ emissions) would lose in net \$85 per month, as they would face \$115 per month in price increases (assuming a carbon price of \$90/tCO$_\text{2}$ in 2030). 

% On top of addressing both global poverty and climate change, we provide evidence from surveys showing that people all over the world support this policy. Yet, the GCS is nowhere to be seen in policy debates. Why? To explain this paradox (absence of the policy despite majority stated support), we further investigate rationales behind the support for the GCS and the sincerity of these claims, as well as attitudes toward other global policies, global redistribution, and universalistic values. % TODO: rework

In this paper, we study attitudes toward global policies that address climate change, global poverty or inequalities, with a focus on the GCS. Using an international survey on climate attitudes, we document majority support for global policies like the GCS in 20 among the largest countriest. Yet, such global policies are nowhere to be seen in policy debates. Why? To explain this paradox (absence of the policy despite majority stated support), we run a complementary survey on 3,000 U.S. respondents and test different hypotheses: insincerity of support for the GCS, pluralistic ignorance (i.e. false belief that most do not support it), defavorable electoral outcomes for a candidate that would support it, or low priority given to global issues. Furthermore, we also study attitudes toward other global policies, global redistribution, and universalistic values.

% Ethical theories often warrant transfers from high- to low-income people, hence from high- to low-income countries. This is the case of utilitarianism, the benchmark ethical theory used in economics. Utilitarianism assigns the same weight to each person and thus considers that a dollar is better allocated to a low-income person, which has a higher marginal utility than a high-income person.\cite{mill_utilitarianism_1861} 

% Addressing global poverty, inequalities and climate change are at the heart of the universally agreed Sustainable Development Goals (SDG). % 12 out of  17
% It has been pointed out that low-income countries generally do not have enough domestic resources to eliminate the poverty gap in the short run.\cite{bolch_arithmetics_2022} In other words, it would hardly be possible to achieve the first SDG and end extreme poverty by 2030 without international transfers. 

% Climate change is another issue that calls for a global response and in particular international transfers. Postulating %Assuming
% that each human has an equal right to emit CO$_\text{2}$, low emitters have a legitimate claim \textit{vis-à-vis} high emitters, that can be settled by monetary transfers. Coupling this burden-sharing principle to the carbon budget (remaining emissions that would be compatible with the Paris agreement) naturally defines a global climate policy. We call it the ``Global climate scheme'' (GCS); it consists of a global cap-and-trade system where emission rights are auctioned each year to polluting firms and the revenues finance a global basic income. Using the price and emissions trajectories from the Stern-Stiglitz report,\cite{stern_report_2017} we estimate that the basic income would amount to \$30 per month for each human above 15 in 2030, enough to lift out of extreme poverty the 700 million people who live with less than PPP \$2 per day. Conversely, high emitters like a typical American (with median U.S. CO$_\text{2}$ emissions) would lose in net \$85 per month, as they would face \$115 per month in price increases (assuming a carbon price of \$90/tCO$_\text{2}$ in 2030). % TD Give the numbers in the Results section
% % G default policy for economists, we focus on it; transfers at heart of COP; global wealth tax proposed by Piketty, Saez (fair and effective); democratisation of int'l institutions recurring topic.
% % Few studies on CC burden-sharing, all compatible with G
% % Few studies on global policies, but they show support (Ghassim, Carattini 19)
% % Here, two sets of results. First, twenty countries. Second, dig deeper using complementary survey.

% If high emitters share universalistic ethical values, we expect strong support for the GCS, even in high-income countries. On the contrary, if people defend their own financial interest, we expect low support for the GCS in high-income countries. 

% In this paper, we study attitudes toward global policies that address climate change, global poverty or inequalities, with a focus on the GCS. We measure stated support for different global policies using unpublished results from a survey\cite{dechezlepretre_fighting_2022} on climate attitudes conducted in 2021 on 40,680 respondents from 20 countries covering 72\% of global CO$_\text{2}$ emissions. We then conduct a representative survey on 3,000 U.S. respondents to study in detail the sincerity and rationales behind the support for the GCS, the attitudes toward various global policies, global redistribution, and universalistic values.

\section{Results}
% % 4 most important figures: heatmap OECD, heatmap support, prioritization or conjoint (r), list exp (table)
\subsection{Data}
\textcolor{red}{\textbf{Beware, data collection is still ongoing (we have 80\% of the final sample) so results are partial and not definitive. Please do not cite at this stage.}} \\
We measure stated support for different global policies using % TD better way to sell these results?
a survey on climate attitudes conducted in 2021 on 40,680 respondents from 20 countries covering 72\% of global CO$_\text{2}$ emissions (the questions of this survey on climate attitudes national policies are analysed in another paper\cite{dechezlepretre_fighting_2022}). We then conduct a representative survey on 3,000 U.S. respondents to study in detail the sincerity and rationales behind the support for the GCS, the attitudes toward various global policies, global redistribution, and universalistic values.

\subsection{International support}
The global survey shows strong support for climate policies at the global level (Figure \ref{fig:oecd}). When asked ``At which level(s) do you think public policies to tackle climate change need to be put in place?'', 70\% (in the U.S.) to 94\% (in Japan) choose the global level. Meanwhile, the European level is chosen by less than half of the European respondents while the federal level is chosen by only 52\% of U.S. respondents. More local levels are generally chosen less than broader ones. This preference for the global level is consistent with (at least) two of the three key motives to support climate policies identified in the literature:\cite{klenert_making_2018,douenne_yellow_2022,dechezlepretre_fighting_2022} effectiveness and fairness (the third being self-interest). 

\begin{figure} % TD have a simpler title (e.g. "Attitudes towards global climate policies"), and a note explaining what the figure exactly does (unit of the numbers, meaning of the colors, exclusion of indifferent people, etc.
  \caption{Support for global climate policies. \\ Share of \textit{Somewhat} or \textit{Strongly support} among non-\textit{indifferent} answers (in percent, $n$ = 40,680). The color blue denotes a relative majority. } 
  \makebox[\textwidth][c]{\includegraphics[width=\textwidth]{../figures/OECD/Heatplot_burden_share_all_share_countries.pdf}}\label{fig:oecd}
\end{figure}

Several global policies obtain an absolute majority %more than 70\% relative %
support in all countries: ``a tax on all millionaires in dollars around the world to finance low-income countries that comply with international standards regarding climate action [which] would finance infrastructure and public services such as access to drinking water, healthcare, and education'', % TODO TD shorten
``a global democratic assembly whose role would be to draft international treaties against climate change [where] each adult across the world would have one vote to elect members of the assembly'' (though this one receives only 48\% of support in the U.S.), and an international emission trading scheme where ``countries that emit more than their national share would pay a fee to countries that emit less than their share''. 
In high-income countries, this global quota obtains 64\% of absolute (i.e. \textit{somewhat} or \textit{strong}) support and 84\% of relative support (i.e. excluding \textit{indifferent} answers). The support is even higher in middle-income countries, though one should interpret the results with caution in middle-income countries as their samples are only representative of the online population (young, graduated and urban people are over-represented). % TODO!: not asked this way in FR, DK, US: comment
After the support for the global quota, we ask how the carbon budget should be divided among countries. 
The preferred burden-sharing rule is to allocate the rights to emit on an equal per capita basis: this fairness principle secures an absolute majority support in all countries, and a relative majority support never below 84\%. 
Taking into account historical responsibilities and vulnerability to climate damages is also popular, though less consensual, while grand-fathering (i.e. allocating emission shares in proportion to current emissions) comes last everywhere. 
The Global Climate Scheme, i.e. a global quota where emission rights are allocated on an equal per capita basis, has the same distributive effects as a global carbon tax that would fund a global basic income. We also test the support for this policy, but here we specify to the respondents the distributive effects: that it would lift the 700 million people who earn less than \$2/day out of extreme poverty, and that the typical person in their country would lose a certain amount (that we specify) due to the price increases.  % The average British person would lose a bit from this policy as they would face £42 per month in price increases, which is higher that the £22 they would receive.
Despite their similarity, the global tax is less supported than the global quota, and it even fails to obtain a majority in Anglo-saxon countries. This lower support is likely due to the fact that distributive effects are made salient in the case of the tax, an interpretation that is consistent with the level of support for the global quota once we make the distributive effects salient, which we do in the complementary surveys. % though we cannot exclude that people find a quota more effective than a tax to reduce emissions. 


\subsection{Stated support for various policies}
% H0: Majority support for each global policies except maximum wealth and debt cancellation

\subsubsection{Global Climate Scheme} % NCCcomment
In the complementary U.S. survey, we describe the Global Climate Scheme, explain its distributive effects (specifying the amounts at stake), test the understanding that typical people would lose in high-income countries and that the poorest humans would win using an incentivized question, and then give the correct answer. We proceed the same way for a National Redistribution Scheme (NR) that would tax the top 5\% to finance cash transfers offseting the monetary loss of the GCS for the median emitter, expecting people to find out at the comprehension question that the richest would lose and the typical people in their country would win. Then, we display summaries of the schemes' description to make sure that the respondents remember them. Right after, we ask again incentivized question of comprehension, and latter give the expected answer that a typical fellow citizen would neither win nor lose with the GCS and NR combined. Finally, we directly ask the support for the GCS and for NR in simple \textit{Yes}/\textit{No}: the stated support for each is at 54\% ($n$ = 3,000).% TD add something if equality remains

\subsubsection{Other global policies} % NCCcomment
We also test support for other %more realistic
global policies (Figure \ref{fig:support}). All receive relative majority support but two: ``a maximum wealth limit of \$10 billion'' and the ``cancellation of low-income countries' public debt''. Climate-related policies are particularly popular: ``high-income countries funding renewable energy in low-income countries'' obtains absolute majority support while loss and damages compensation (which was approved at the COP27) receives a relative support of  57\%. 

\begin{figure}
  \caption{Support for various global policies in the U.S. ($n$ = 3,000).}
  \makebox[\textwidth][c]{\includegraphics[width=\textwidth]{../figures/US1/support_likert.pdf}}\label{fig:support}
\end{figure}

% H0: Foreign aid: less than 20\% want a decrease (because nationalist), median wants increase at some conditions (no diversion, human rights) => GCS mostly addresses these points
\subsubsection{Foreign aid} % NCCcomment
After explaining that ``0.4\% of U.S. government spending (that is, 0.2\% of U.S. GDP) is spent on foreign aid to reduce poverty in low-income countries'', less than 20\% state that U.S. foreign aid should be reduced while 57\% state that it should be increased, including 14\% who support an unconditional increase. To the 43\% who answer that aid should be increased but only if some conditions are respected, we later ask them what condition(s) should be required. The three conditions most chosen are all largely respected by the Global Climate Scheme: ``that we can be sure the aid reaches people in need and money is not diverted'' (chosen by 74\%), ``that recipient countries comply with climate targets and human rights'' (59\%), and ``that other high-income countries also increase their foreign aid'' (44\%). %, we propose different conditions. The most chosen condition (by 74\%) is  and the second most (by 59\%) ``That recipient countries comply with climate targets and human rights''. 
On the other side, not wishing to increase their country's foreign aid is mostly justified by prioritizing one's fellow citizens or viewing each country as responsible for its own fate. 

\subsection{Sincerity of support}

We use several methods to assess the sincerity of the support for the Global Climate Scheme: a list experiment, a real-stake petition, conjoint analyses, and the prioritization of policies. All methods suggest that the support is either completely sincere, or the share of insincere answers is limited. 

\subsubsection{List experiment}  % NCCcomment
% H1: List experiment: There seems to be a 8pp social norm (differential of 3pp with NR). No effect of the number of options. TODO: check literature
The tacit support for the GCS measured through the list experiment is 46\%, i.e. 8 p.p. lower than at the direct question. This may be the sign of a social norm pushing some people to state that they support the GCS although they secretly do not. Still, if there is a social norm in favor of the GCS, there is a similar norm in favor of the National Redistribution Scheme, as the gap between the tacit and direct support for it is comparable (at 7 p.p.). %However, two observations qualify this interpretation. First, the gap between the tacit and direct support for the National Redistribution Scheme is comparable (at 7 p.p.) though we did not expect such a social norm in the case of the national redistribution, as the 95\% who would benefit from it should not feel ashamed to oppose a policy that would benefit them. Second, while we tested the questionnaire on random people in cafés, we noticed that some were confused by the question of the list experiment (asking how many policies from the list they supported), upset with the conservative societal policy (``Marriage only for opposite-sex couples in the U.S.'', ``Death penalty for major crimes'' in Europe), to the point that they did not answer attentively.

\begin{table}[h]\label{tab:list_exp}
  \caption{Number of supported policies in the list experiment in function of the composition of the list. $G$ stands for the Global Climate Scheme and $R$ for the National Redistribution Scheme ($n$ = 3,000).} % Beware, this question is quite unusual. \\ Among the policies below, how many do you support?  \\ Coal exit, Marriage only for opposite-sex couples 
  \makebox[\textwidth][c]{
\begin{tabular}{@{\extracolsep{5pt}}lc} 
\\[-1.8ex]\hline 
\hline \\[-1.8ex] 
\\[-1.8ex] & Number of supported policies \\ 
\hline \\[-1.8ex] 
Mean & 1.354  \\ \hline \\[-1.8ex]
 List contains: G & 0.496$^{***}$ \\ 
  & (0.069) \\ 
  List contains: R & 0.574$^{***}$ \\ 
  & (0.068) \\ 
  List contains: G \times R & $-$0.033 \\ 
  & (0.119) \\ 
 \hline \\[-1.8ex] 

Observations & 1,045 \\ 
R$^{2}$ & 0.132 \\ 
\hline 
\hline \\[-1.8ex] 
\end{tabular} }
\end{table}

% Donation addresses experimenter demand
\subsubsection{Petition} % Addresses hypothetical bias  % NCCcomment
% H1: Petition: Small effect against GCS: -4pp
When told that ``we will send the results to the U.S. President's office, informing him what share of American people are willing to endorse the Global Climate Scheme'', 4 p.p. fewer people are willing to sign a petition for the GCS than to simply state their support. For the National Redistribution Scheme, the proportion of support is not significantly different in the petition and in the simple question. 

\subsubsection{Conjoint analyses} % Addresses acquiescence bias  % NCCcomment
% H1, H2: Conjoint analysis: G|C+R 56%, G|R 59%, G 48% ~ C (|R), G+C|R 56%, C|R 64%, Left+G - Left = -3pp, A+G vs. B 59%
% => G is supported for itself, rather independently from R or C, with similar support to both, and it doesn't significantly penalize the Left, and would help a Democratic candidate
In our \textit{conjoint analyses}, we ask respondents to make five choices between pairs of political platforms. The first conjoint analysis suggests that the GCS is supported for itself, independently of being complemented by a national climate policy (``Coal exit''% in the U.S., ``Thermal insulation plan'' in Europe
, denoted C) or the National Redistribution Scheme. Indeed, 55\% of ($n$ = 3,000) respondents prefer the combination of C, NR and the GCS to the combination of C and NR alone, indicating a similar support for the GCS conditional on NR and C than for the GCS alone.% (as it does not significantly differ from the direct support of 53\%). 
For the second analysis, we split the sample into four random branches. Results from the first branch show that 55\% ($n$ = 750) prefer the combination of C, NR and the GCS to NR alone. The second shows that the support for the GCS conditional on NR, at 59\% ($n$ = 750), is somewhat higher than the direct support for the GCS. The third, that the support for C conditional on NR is even higher, at 63\% ($n$ = 750). This is confirmed by the fourth, showing that 52\% ($n$ = 750) prefer C to the GCS, both conditional on NR. In other words, there is majority support for the GCS and for C, slightly more people prefer C but C does not act as a substitute for the GCS, and some people find the GCS complementary to NR though the number of people requiring NR to support the GCS remains small. % TD rework this paragraph

The third analysis suggests that a Democratic candidate would not significantly lose voting share at the 2024 presidential election if he or she were to endorse the GCS. To estimate this, we present to two random branches of the sample hypothetical Democratic and Republican platforms that differ only by the presence (or not) of the GCS in the Democratic platform. Although the share of respondents choosing ``None of them'' is slightly higher (at 13\% instead of 11\%) when the Democratic platform includes the GCS, the share choosing the Democrat is not significantly lower (52\% in both cases). 

Our last two analyses is run on the subsample of non-Republicans ($n$ = 2,000), i.e. the respondents who choose \textit{Democrat}, \textit{Independent}, \textit{Non-Affiliated} or \textit{Other} for their political affiliation. We frame the choice between two platforms as a hypothetical duel at the 2024 Democratic primary and force the respondents to choose between candidate A or B. In the fourth analysis, a policy (or an absence of policy) is randomly drawn for each platform in each of five categories: \textit{economic issues}, \textit{societal issues}, \textit{climate policy}, \textit{tax system}, \textit{foreign policy} (Figure \ref{fig:ca_r}). Except for the category \textit{foreign policy}, which features the GCS 42\% of the time, the policies are prominent progressive policies and they are drawn uniformly. % TODO!: check
When a platform features the GCS and not the other, the one with the GCS is chosen 53\% (which is significantly more than half) of the time ($n$ = 3,000). % TODO!: effect vis-à-vis baseline (cf. figure)
The fifth analysis draws random Democratic platforms in a similar ways, except that candidate A's platform always contains the GCS while candidate B includes no foreign policy. In this case, 61\% of respondents choose A ($n$ = 3,000). In short, our conjoint analyses indicate that a candidate at the Democratic primary would have more chances to obtain the nomination by endorsing the GCS, and this endorsement would not penalize her or him at the presidential election. This result reminds the finding that 12\% of Germans shift their voting intention from SPD and CDU/CSU to the Greens and the Left when they are told that the latter parties support global democracy.\cite{ghassim_who_2020}

\begin{figure}
  % Imagine that at the 2024 Democratic party presidential primaries, the two main candidates campaign with the following key policies in their platforms. \\ Which of these candidates do you prefer?

  \caption{Conjoint analysis (asked only to non-Republicans). Average Marginal Component Effects (relative to the baseline: an absence of policy of that category) of policies in the choice of candidate for a hypothetical duel in the 2024 Democratic primary, where both platforms are randomly drawn ($n$ = 2,000).}\label{fig:ca_r}
  \makebox[\textwidth][c]{\includegraphics[width=\textwidth]{../figures/US1/ca_r.png}}
\end{figure}

%\subsubsection{Prioritization} % Addresses acquiescence bias and social desirability bias
% H1: Prioritization: G has mean only slightly lower than average, makes better than ban of cars and coal exit; global tax on millionaires does as well as wealth tax and almost as good as $15 minimum wage
At the end of the survey, we pick six policies at random (and uniformly) among the progressive policies used in the last conjoint analyses, and ask respondents to allocate 100 points (using sliders) among them, with the instruction that ``the more you give points to a policy, the more you support it''. For each policy presented, the average support is thus 16.67 points (Figure \ref{fig:points}). The GCS ranks in the middle of all policies (9th. out of 17), with an average number of points of 15.3 which is only slightly lower than average. It is higher than to ``ban the sale of new combustion-engine cars by 2030'' (13.4) and ``coal exit'' (10.0), but lower than the third climate policy: ``trillion dollar investment in clean transportation infrastructure and building insulation'' (20.3). The support for other globally redistributive policies is variable: ``Doubling foreign aid'' is the least supported policy (8.4), while the ``Global tax on millionaires'' is one of the five policies with more than 20 points (20.2), and the ``global democratic assembly on climate change'' is just below the GCS (14.5). The most supported policies are ``Funding affordable housing'' (28.5), ``\$15 minimum wage'' (23.8), and ``Universal childcare/pre-K'' (22.1). % TODO share that allocated at least 1

\begin{figure}
  \caption{Prioritization of policies. Each respondent faces six policies taken at random from the ones below and allocates 100 points among them to signal the strength of their support for each one ($n$ = 3,000).} % Imagine you have 100 points that you can allocate to different policies. The more you give points to a policy, the more you support it.  \\  How do you allocate the points among the following policies?  
  
  \makebox[\textwidth][c]{\includegraphics[width=\textwidth]{../figures/US1/points_us.pdf}}\label{fig:points}
\end{figure}


\subsection{Second-order beliefs}
% H3 belief: No pluralistic ignorance
To explain a strong support for the GCS despite its absence from political platforms and the public debate, we hypothesized pluralistic ignorance, i.e. that most people and policy-makers wrongly perceive the GCS as unpopular. People would then hide their support for such globally redistributive policies, knowing that advocating for them would be vain. We do not find any evidence of pluralistic ignorance in an incentivized question on the perceived support. On the contrary, people have quite accurate beliefs regarding the level of support for the GCS. Indeed, the mean (resp. quartiles) perceived support is 52.0\% (resp. 36, 53, 69\%, $n$ = 1,500) vs. an actual support of 53\%. For the record, the second-order beliefs are equally accurate for the National Redistribution Scheme, with mean (resp. quartiles) perceived support of 54.7\% (resp. 40, 55, 71\%, $n$ = 1,500) vs. 56\%.

\subsection{Universalistic values}
% H4: A strong majority is universalist/cosmopolitan (TODO: which word?), even a majority for non-Republican
% TD It is not obvious how these answers are informative of malleable opinions. So I don't think we should state the hypothesis and sell this as a test.
%Another hypothesis to explain the discrepancy between the lack of interest for global policies in the public debate despite a strong stated support is that opinions on the topic are weak and malleable. A way to test this is to
We ask broad question on people's values, to see whether their core values are consistent with universalism. Asked what group they defend when they vote ($n$ = 3,000), 19\% choose ``sentient beings (humans and animals)'', 25\% ``humans'', 34\% ``Americans'', 15\% ``My family and myself'', and the rest (7\%) choose another group (mostly ``My State or region'' or ``People sharing my culture or religion''). The first two categories can be described as universalist, and they represent close to one out of two people. The share of universalist even constitutes a majority (at 51\%) of non-Republicans. 
When asked what should U.S. diplomats defend in international climate negotiations, only 14\% prefer ``U.S. interests, even if it goes against global justice''; 25\% prefer global justice (mitigated or not by U.S. interests) and the bulk of respondents (37\%) prefer ``U.S. interests, to the extent it respects global justice'' ($n$ = 3,000). 
Furthermore, when asked to judge the extent to which climate change, global poverty and U.S. inequality are an issue, climate change is generally viewed as the biggest problem (with a mean of 0.40 once we recode answers between $-2$ and $2$), followed by global poverty (0.20) and U.S. inequality (0.19, $n$ = 3,000). 
Finally, we elicit unversalistic values through a lottery experiment. We automtically enroll the respondents in a lottery with one \$100 prize. Respondents have to choose which share of the prize to keep for themself vs. give to a person living in poverty. The charity donation is destined either for an Africa or an American, depending on the respondent's random branch. We observe no significant variation in the willingness to donate in function of the recipient's origin (the average donation is around \$33).
Overall, answers to these broad value questions are consistent with half of Americans supporting global policies like the GCS, as people are as much willing to give to poor Africans than to poor Americans, they find that global issues are among the biggest problems, almost half of them are universalist when they vote, and most of them wish that U.S. diplomats take into account global justice.


\section{Discussion} % Summary, conclusion
In 20 among the largest countries, we find strong majority support for global climate policies, even in high-income countries that would financially lose from the globally redistributive policies that we test. The complementary survey in the U.S. confirms these results. For example, there is a strong support for global taxes on the wealthiest, and majority support for our main policy of interest, a Global Climate Scheme that would establish both carbon pricing at the global level through an emission trading system, and a global basic income funded by its revenues. A list experiment and a real-stake petition show that the support for the GCS is mostly sincere. This genuine support is confirmed by the prioritization of this global climate policy above some prominent national climate policies, and consistent with around half of the population holding universalistic (rather than nationalistic or egoistic) values. Moreover, the conjoint analyses reveals that a Democratic candidate should not lose voting shares by endorsing the GCS, and would even win votes at the Democratic primary by doing so. Besides a potential lack of sincerity and weak opinions, we dismiss another hypothesis to explain the scarcity of global policies in the public debate despite a strong support: that people underestimate the support of their fellow citizens. As we ruled out all hypotheses of our registration plan,\footnote{The project was preregistered in the Open Science Foundation registry (\href{https://osf.io/fy6gd}{osf.io/fy6gd}).} we now need to study new explanations. %formulate new hypotheses.

% LM Similarly to Thomas' comment, I had to read this three times before getting it. I had the other idea that it might help to link to the pre-analysis plan here. That would help me understand what is implied. 
We see four potential explanations for the scarce mention of globally redistributive policies in the public debate. Among the new hypotheses, the first two are variations of pluralistic ignorance, and the last two represent complementary (rather than substitute) explanations. First, there may be pluralistic ignorance of univeralistic values, of the support for the GCS, or of the electoral advantage of endorsing it \textit{among policy makers}. We intend to test this hypothesis by running a survey on Congress staffers and Members of the European Parliament. Second, there may be a more subtle form of pluralistic ignorance: although people correctly predict what people would answer to a survey question, they may view globally redistributive policies as unrealistic, perhaps because they have never reflected upon the fact that many people across the world hold univeralistic values and are supportive of global solidarity. Third, most people and perhaps even most policy makers may have simply never heard of the GCS, let alone built their political ideas upon it. The ignorance of the GCS itself seems supported by the feedback fields, where the most common answer is a variation upon ``thank you for this interesting, thought-provoking survey''. Fourth, most institutions are national: the largest scale votes take place at the national level, most media target a national audience, most commentators frame their discourse from a national perspective, and relations to foreign countries as conflictual. The prominence of national institutions may create a nationalistic bias in political thoughts, silencing the univeralistic values of people. % TODO test hypotheses 2 (is G realistic) and 3 (is G new to you?) in US2?
% TODO add swing state analysis
% TODO find references

% LM This are all legtimiate and important points. For a global transfer (rather than say a EU transfer), I am really missing remarks about (lack) of quality of governance, rent creation and capture, corruption etc. To play devil's advocate, as much as I am sympathetic to the idea, I'm not sure I would currently vote in favor of such policies. Why? Not because I dislike the idea, but I am not sure that my carbon tax rent as a German would actually reach the poor population of Indonesia or Nigeria rather than the pockets of cleptocratic elites :) I'd like the draft to reflect that and could make an attempt to pre-empt that objection based on some references later in the project.
%In any case, 
If any (or several) of the remaining hypotheses is confirmed by evidence, we could draw the same conclusion. % TD But what if they are not? I don't think that would invalidate the conclusion stated below.
There is a strong support for global policies that address climate change and global inequality, even in high-income countries, and the frontier of what is considered politically realistic might soon shift on this issue. Uncovering evidence for this might actually contribute itself to garner more attention to global policies in the public debate and political platforms. % TD I don't like this because it sounds a bit like asking editors to publish our work.

% \begin{methods}  % WPcomment
\begin{small} % NCCcomment
%Put methods in here.  If you are going to subsection it, use \subsection commands.  Methods section should be less than 800 words and if it is less than 200 words, it can be incorporated into the main text.
\section*{\normalsize Methods} % NCCcomment
% \subsection{Data collection.} % WPcomment
\paragraph{\small Data collection.} % NCCcomment
The paper relies on two different surveys. The first survey was conducted between March 2021 and March 2022 on 40,680 respondents from 20 countries  (between 1,465 and 2,488 respondents per country). The second, denoted US1, was conducted on 3,000 U.S. respondents in January and February 2023. We used the survey companies \emph{Dynata} and \emph{Respondi}. Stratified quotas ensure that the samples are representative along the dimensions of gender, age (5 brackets), income (4), region (4), education level (3), as well as ethnicity (3) for the U.S. % TODO and urbanity instead of education for OECD, TODO table representativeness
To correct for small remaining imbalances, we apply survey weights throughout the analysis, constructed using the quotas variables as well as the degree of urbanity, and trimmed between 0.25 and 4. %\footnote{We trim weights so that no respondent receives a weight below 0.25 or above 4. Overall, trimming changes the weights for xx\% of the respondents.} 
Appendix \ref{app:representativeness} confirms that our samples are representative

% \subsection*{\small Data quality.} % WPcomment % TODO attrition analysis
\paragraph{\small Data quality.} % NCCcomment
The median duration is 28 minutes for the global survey and 15 minutes in the US1 survey. To ensure the best possible data quality, we exclude respondents who fail an attention test or rush through the survey (i.e. answer in less than 11.5 minutes in the global survey or 4 minutes in US1). 
%\textit{Ex post}, we checked that there were only a few careless response patterns (such as choosing the same answer for all items in a matrix of questions; see Appendix \ref{app:data_quality}). At the end of the survey, we ask whether respondents thought that our survey was politically biased and provide some feedback. 74\% of the respondents found the survey unbiased. 15\% found it left-wing biased, and 11\% found it right-wing biased.

% \subsection*{\small Questionnaires and raw results.} % WPcomment
\paragraph{\small Questionnaires and raw results.} % NCCcomment
% Possible confusion in the questionnaire: people confuse GCS with the four policies together (so support for GCS can suffer from dislike of death penalty), although this confusion is mitigating by the fact that we right after ask about NR; people may answer about revenue-use rather than the whole measure for ETS2 support; people may answer that GCS will or will not have the effects proposed rather than these effects being important or not in their attitude towards GCS; they may answer that it's important that others (not them) get more information; the minimum wage could be reduced at 50% of local median wage.
The questionnaire and raw results of the global survey can be found in the Appendix of the companion paper.\cite{dechezlepretre_fighting_2022} The US1 raw results are reported in Appendix \ref{app:raw_results_US1} while the US1 survey structure and questionnaire are given in Appendix \ref{app:questionnaire_US1}. The questionnaires are the same as the ones given \textit{ex ante} in the registration plan (\href{https://osf.io/fy6gd}{osf.io/fy6gd}).

% \subsection*{\small Incentives.} % WPcomment
\paragraph{\small Incentives.} % NCCcomment
To encourage respondents to answer accurately and truthfully, several questions of the US1 survey use incentives. For each of the three comprehension questions that follow the policies' descriptions, we reward three (randomly drawn) respondents with the correct answer with a \$50 gift certificate. For each of the questions asking respondents to guess the share of support for the GCS and NR, we reward three people who are closest to the true value with a \$50 gift certificate. For the donation lottery question, we randomly draw one respondent and split the \$100 prize between the NGO GiveDirectly and the winner according to the winner's choice. In total, our incentives scheme distributes gift certificates (and donation) for a value of \$850. Finally, respondents have an incentive to answer truthfully to the petition question, given that they know that the results to that question (the share of respondents supporting the policy) will be transmitted to the U.S. President's office.

% Here is a description of a specific method used.  Note that the subsection heading ends with a full stop (period) and that the command is \verb|\subsection{}| not \verb|\subsection**{}|.

% \end{methods} % WPcomment
\end{small}  % NCCcomment

% \bibliographystyle{naturemag_noURL} % nature class works only with style naturemag or naturemag_noURL, and naturemag bugs if there are certain URLs (like .pdf). Also, nature class only works with \cite, not \citet or \citep.  % WPcomment
\renewcommand{\url}[1]{\href{#1}{Link}} % NCCcomment
\bibliographystyle{plainnaturl_clean} % NCCcomment
\bibliography{global_tax_attitudes}

\appendix % NCCcomment

% \clearpage
\section{Literature review}\label{sec:literature}

\subsection{Attitudes and perceptions}\label{subsec:literature_attitudes}

\subsubsection{Population attitudes on global policies}\label{subsubsec:literature_attitudes_policies}
Our surveys fill gaps in the knowledge of attitudes toward global policies. 
We are not aware of any other survey on a global wealth tax. 
\citet{carattini_how_2019} test the support for different variants of a global carbon tax, but their samples are representative only along gender and age, and as respondents face only one variant, the sample size for a given variant is about 167 respondents per country. They find more than 80\% of support for any variant in India, between 50 and 65\% in Australia, the UK and South Africa, and 43 to 59\% of support in the U.S., depending on the variant. The support for a global carbon tax funding an equal dividend for each human is close to 50\% in high-income countries (e.g. at 44\% in the U.S.), consistently with what we find in the OECD survey (see Figure \ref{fig:oecd}). 
Using a conjoint analysis in the U.S. and Germany, \citet{beiser-mcgrath_could_2019} find that the support for a carbon tax increases by up to 50\% % e.g. in their Fig. 4 the DE support for $70/t jumps from 26 to 39% with extension to all industrialized countries
if it applies to all industralized countries rather than just one's own country. % Variant of carbon tax is 8 (US) - 17 (DE) p.p. more likely to be preferred if tax is extended to all industrialized countries

In surveys in Brazil, Germany, Japan, the UK and the U.S., \citet{ghassim_who_2020} finds 55 to 74\% of support for ``a global democracy including both a global government and a global parliament, directly elected by the world population, to recommend and implement policies on global issues''. % (for example, international peace, world poverty, and climate change)''
Using an experiment, he also finds that, in countries where the government stems from a coalition, voting shares would shift by 8 (Brazil) to 12 p.p. (Germany) from parties who are said to oppose global democracy to parties that supposedly support it. For example, the Greens and the Left gained respectively 9 and 3 p.p. in vote intentions while the SPD and the CDU-CSU each lost 6 p.p., when Germans respondents were told that (only) the former parties support global democracy. 
\citet{ghassim_who_2020} also document survey results which show strong majorities support in each of 18 countries for the direct election of one's country's UN representative. % GlobeScan 2005; also: half/half (majorities or not depend on the country) for “Global Parliament, where votes are based on country population sizes, and the global parliament is able to make binding policies” (Synovate 2007); also: (GlobeScan 22, not from Ghassim) in 31 countries: 77% agree that “Rich countries must pay for poorer countries do deal with the effects of CC”
Similarly, in each of 10 countries, there are clear majorities in favor of ``a new supranational entity [taking] enforceable global decisions in order to solve global risks'' \citep{global_challenges_foundation_attitudes_2018}. Actually, already in 1946, 54\% of Americans agreed (and 24\% disagreed) that ``the UN should be strengthened to make it a world government with the power to control the armed forces of all nations'' \citep{gallup_seventy_1946}. 
In surveys in Argentina, China, India, Russia, Spain, and the U.S., \citet{ghassim_public_2022} find support for UN reform that would make United Nations' decisions binding, give veto powers at the Security Council to a few other major countries, and complement the highest body of the UN with a chamber of directly elected representatives. 
% TODO Schleich 16: international agreements are important but current ones are unsuccessful, people find themselves poorly represented in climate negotiations

These specific questions are in line with the answers to more general questions. In each of 36 countries, \citet{issp_research_group_international_2010} find near consensus that ``for environmental problems, there should be international agreements that [their country] and other countries should be made to follow'' (overall, 82\% agree and 4\% disagree). % No question like this in the next Envi wave in 2022
In each of 29 countries, \citet{issp_international_2019} find near consensus that ``resent economic differences between rich and poor countries are too large'' (overall, 78\% agree and 5\% disagree). 
%* Also in ISSP (19): slight minorities (in rich countries) that “People in wealthy countries should make an additional tax contribution to help people in poor countries.” p. 104, but strong majorities everywhere that “People from poor countries should be allowed to work in wealthy countries.” p. 106
Relatedly, \citet{meilland_international_2023} find that Americans and French people prefer an international settlement of climate justice even if it empedes on sovereignty. In a 2013 survey in China, Germany and the U.S., \citet{schleich_citizens_2016} show that more than 73\% of people find important future international climate agreements, while less than 26\% think that international reached so far are successful. 

%* ISSP (19): Near consensus that “Present economic differences between rich and poor countries are too large.” p. 102, slight minorities (in rich countries) that “People in wealthy countries should make an additional tax contribution to help people in poor countries.” p. 104, but strong majorities everywhere that “People from poor countries should be allowed to work in wealthy countries.” p. 106
%* Ghassim et al. (22): support for stronger UN with more direct elections.
%* Ghassim (20):  in Germany those two parties that supposedly endorse global democracy – the Greens and the Left – benefitted, gaining nine and three percentage points respectively in terms of voting intentions. Meanwhile, the traditional centrist parties – SPD and CDU – each lost six percentage points due to their supposed opposition to global democracy.
%* Beiser-McGrath & Bernauer (19): Conjoint analysis in US, DE. Variant of carbon tax is 8 (US) - 17 (DE) p.p. more likely to be preferred and 50% more likely to be supported if tax is extended to all industrialized countries (Fig 1, 4). (Unfortunately, don't test extension to global level).
%- Çarkoğlu.. (15) International Social Survey Program 2010 data reveal that people in LDCs are less supportive of international agreements forcing their country to take necessary environmental measures than are citizens in the developed world [80% instead of 85%]. (‘for environmental problems, there should be international agreements that [their country] and other countries should be made to follow.’)
%* Carattini et al. (Nature, 19): 1k in US, IA, ZA, AU, UK. Each respondent receives one variant at random of global carbon price of 40/60/80 $/t redistributed as international dividend / national dividend / mitigation in all countries / mitigation in developing countries / domestic mitigation / reduced labour tax. Immense majorities for any scheme in India, small majorities for each elsewhere except US international dividend (44%) or mitigation in developing (43%), and AU mitigation in developing (49,6%). PB: very low sample size (~167) for a given redistribution, even lower (~55) for a given variant (that also specifies the price). Appendix also contains estimation of distributive impacts. Representative only along the two quotas: gender and age. Don't give the representativeness in terms of income (the third socio-demos that they ask) so it's probably bad.

\subsubsection{Population attitudes on climate burden sharing}\label{subsubsec:literature_attitudes_burden_sharing}

Despite their differences in the description of the fairness principles, the surveys burden-sharing rules show consistent attitudes. Or at least, their various results can be made compatible with the following interpretation. 
Concerning emissions reductions, most people want that every country engage in strong decarbonization effort together, with a global quota converging to climate neutrality in the medium run. Concerning the financial effort, most people support high-emitting countries paying and low-income countries receive funding. The most supported rules are those that appear equitable, in particular an equal right to emit per person. 
% When the rankings between rules differ, it can be due to the difference in countries surveyed, but it is most often due to differences in definitions and wording. 

This interpretation helps understanding the apparent differences between articles, which approach burden sharing from different angles: cost sharing (i.e. in money terms), effort sharing (in terms of emissions reductions), or resource sharing (in terms of rights to emit). Most papers adopt the cost sharing or effort sharing approaches and preclude any country being a net receiver of money. Also, by focusing on either the financial or the decarbonization effort, these surveys miss the other half of the picture, which can explain why some papers find strong support for the ability-to-pay principle while others find strong support for grandfathering (defined as emissions reductions being the same in every country). The literature follow these approaches to stick to the terms used by the UNFCCC. Yet, we argue that the resource sharing approach is preferable to uncover attitudes, as it unambiguously describes the distributive implications of each rule while achieving an efficient location of emissions reductions and explicitly allowing for monetary gains for some countries. % TODO? say more simply that the location of emissions reductions is flexible in resource sharing
% TODO? appendix with the definitions for each author, incl. us

Now, let us summarize the different papers' results in the light of this clarification. 
\citet{schleich_citizens_2016} find an identical ranking in the support for the burden-sharing principles in China, Germany, and the U.S.: polluter-pays followed by ability-to-pay, equal emissions per capita, and grandfathering. 
% \footnote{The survey of \citet{schleich_citizens_2016} defines these rules as follows: \\
% \textit{Polluter-pays}: ``Every country has to bear costs according to the emissions it causes (hence countries causing higher emissions have a higher share of the costs).'' \\
% \textit{Ability-to-pay}: ``Every country has to bear costs according to its economic strength (hence richer countries have a higher share of the costs).''
% \textit{Egalitarianism}: ``Every country is allowed to produce the same amount of emissions per capita (hence countries with currently high emissions per capita have higher costs).''
% \textit{Sovereignty} (i.e. grandfathering): ``Every country is allowed to produce the same share of global emissions as in the past (hence the proportional reduction of emissions is the same for every country).''} 
Note that the authors do not allow for emissions trading in their description of equal \textit{emissions per capita}, which may explain its relatively low support. 
Yet, the relative support for egalitarianism also depends on how \textit{the other} rules are described. Indeed, \citet{carlsson_is_2011} find that Swedes prefer that ``all countries are allowed to emit an equal amount per capita'' rather than options where emissions are reduced in relation to current or historical emissions for which it is explicitly written that high-emitting countries ``will continue to emit more than others''. 
\citet{bechtel_mass_2013} find agreement that rich countries should pay more and historical emissions matter, but that rich countries should not be the only one to make the efforts. More precisely, their conjoint analysis in France, Germany, the UK and the U.S. shows that a climate agreement is 15 p.p. more likely to be preferred  (to a random alternative) if it includes 160 countries rather than 20, and 5 p.p. less likely to be preferred if ``only rich countries pay'' comapred other burden-sharing rules: ``rich countries pay more than poor'', ``countries pay proportional to current emissions'' or ``countries pay proportional to historical emissions''. %=> confirms preference for global policies (rather than only partial coverage). Finds that costs is what matters most: preference decreases by 30pp if it’s 2.5\% of GDP compared to 0.5\%.
Using a choice experiment, \citet{carlsson_fair_2013} find that the least preferred option in China and the U.S. is when low-emitting countries are exempted from any effort. Ability-to-pay is appreciated in both countries, though the preferred option in China is another one, which accounts for historical responsibility. %that Americans prefer capacity to pay > current responsibility > historical responsibility > equal emissions per capita while Chinese prefer historical > capacity > current > equal emissions.
%   Capacity to pay: Countries with high income levels must pay a larger share of the costs than countries with low income levels. This option says that countries with greater ability to pay should pay more
%   Current responsibility: Countries with currently high emissions levels must pay a larger share of the costs than countries with currently low emissions levels. This option says that those countries that are currently a larger part of the problem should pay more.
%   Historical responsibility: Countries with a history of high emissions levels must pay a larger share of the costs than countries with a history of lower emissions. This option recognizes that CO2 builds up in the atmosphere over many years. Thus, countries with a history of high emissions should pay more because they caused more of the problem.
%   Equal emissions pc: Countries with emissions per person greater than an agreed amount must pay, and they must pay more the higher their emissions per person are.
% > "equal emissions" is a misnomer as this is about costs (not emissions) and it's just a more progressive version of current responsibility / polluter-pay, where high-emitting pay more and low-emitting don't pay. The result for US is compatible with the other papers as Americans agree that rich countries (or high-emitting, the diff is small) should pay more. The Chinese position could also be reconciliable once we define responsibility from footprint rather than territorial and that there will be transfers from rich to poor countries.
In the U.S. and France, \citet{meilland_international_2023} find that the most favored fairness principle is that ``all countries commit to converge to the same average of total emissions per inhabitant, compatible with a controlled climate change''. Furthermore, in each country, 73\% disagree with grandfathering defined as ``countries which emitted a lot of carbon in the past have a right to continue emitting more than others in the future''. \citet{meilland_international_2023} contain many other results, for example majorities prefers to hold countries accountable for their consumption-based rather than territorial emissions, and the median choice regarding historical responsibility is to hold a country accountable for their post-1990 emissions (rather than post-1850 or just their current emissions). 
% - Meilland et al. (23) find that in US and France, most favored fairness principle is Equality in per capita emissions: "all countries commit to converge to the same average of total emissions per inhabitant, compatible with a controlled climate change" and second-most (which closely follows) is grandfathering: "all countries commit to reduce their emissions by a same proportion". 73% in each disagree with grandfathering when defined as "countries which emitted a lot of carbon in the past have a right to continue emitting more than others in the future". To rationalize these contrasted views with grandfathering, we can interpret them as: equal rights, equal emission reductions, and transfers. 
%   convergence per capita (70%): all countries commit to converge to the same average of total emissions per inhabitant, compatible with a controlled climate change
%   grandfathering (60%): all countries commit to reduce their emissions by a same proportion
%   past emissions (20% choose it among their two favorite): countries which emitted less in the past commit to reduce their emissions less than other countries
%   poor countries (20%): poorer countries commit to reduce their emissions less than richer countries
%   cost-efficiency (20%): countries where reducing emissions is more costly commit to reduce their emissions less than other countries
% - Meilland et al. (23) Other findings: people prefer international settlement on CC even if it empedes on sovereignty, a majority prefers to target footprint rather than territorial emissions, median is that countries should be held accountable for post-1990 emissions, self-serving bias when judging e.g. India vs. EU, no shared understanding of fairness when asked to coordinate between French and Americans
Finally, in each of 28 (among the largest) countries, \citet{dabla-norris_public_2023} find strong majority for ``all countries'' to the question ``Which countries do you think should be paying to reduce carbon emissions?''. Asked to choose between a cost sharing based on \textit{current} vs. \textit{accumulated historic emissions}, a majority prefers \textit{current emissions} in all countries but China and Saudi Arabia (where the two options are close to equally preferred). 


\subsubsection{Population attitudes on foreign aid}\label{subsubsec:literature_foreign_aid}

\subsubsection{Population attitudes on wealth tax}\label{subsubsec:literature_wealth_tax}

\subsubsection{Population attitudes on ethical norms}\label{subsubsec:literature_wealth_tax}
\paragraph{Universalism}
% TODO WVS on world citizenship (e.g. Bayram 15), Reysen and Katzarska-Miller 2018
%- Buntaine & Prather (18), Diedrich & Goeschl (18) Willingness to act for domestic vs. international climate action (lab experiment) READ
\paragraph{Free-riding}

\subsubsection{Second-order beliefs}\label{subsubsec:literature_beliefs}

\subsubsection{Elite attitudes}\label{subsubsec:literature_beliefs}


\subsection{Proposals and analyses of global policy-making}\label{subsec:literature_policies}

\subsubsection{Global carbon pricing}\label{subsubsec:literature_pricing}

\subsubsection{Climate burden sharing}\label{subsubsec:literature_burden_sharing}

\subsubsection{Global redistribution}\label{subsubsec:literature_redistribution}

\subsubsection{Global democracy}\label{subsubsec:literature_democracy}


% Burden-sharing
%- Agarwal & Narain (91) first to defend an equal right to emit per capita (equal to the absorbing capacity of the Earth)
%- Gampfer (14): lab experiment (ultimatum game) to test whether preferences respect fairness principles
%- Chancel & Piketty (15): global progressive carbon tax
% cf. bottom

% Global policies attitudes 
%* ISSP (19): Near consensus that “Present economic differences between rich and poor countries are too large.” p. 102, slight minorities (in rich countries) that “People in wealthy countries should make an additional tax contribution to help people in poor countries.” p. 104, but strong majorities everywhere that “People from poor countries should be allowed to work in wealthy countries.” p. 106
%* Ghassim et al. (22): support for stronger UN with more direct elections.
%* Ghassim (20):  in Germany those two parties that supposedly endorse global democracy – the Greens and the Left – benefitted, gaining nine and three percentage points respectively in terms of voting intentions. Meanwhile, the traditional centrist parties – SPD and CDU – each lost six percentage points due to their supposed opposition to global democracy.
%* Beiser-McGrath & Bernauer (19): Conjoint analysis in US, DE. Variant of carbon tax is 8 (US) - 17 (DE) p.p. more likely to be preferred and 50% more likely to be supported if tax is extended to all industrialized countries (Fig 1, 4). (Unfortunately, don't test extension to global level).
%- Çarkoğlu.. (15) International Social Survey Program 2010 data reveal that people in LDCs are less supportive of international agreements forcing their country to take necessary environmental measures than are citizens in the developed world [80% instead of 85%]. (‘for environmental problems, there should be international agreements that [their country] and other countries should be made to follow.’)
%* Carattini et al. (Nature, 19): 1k in US, IA, ZA, AU, UK. Each respondent receives one variant at random of global carbon price of 40/60/80 $/t redistributed as international dividend / national dividend / mitigation in all countries / mitigation in developing countries / domestic mitigation / reduced labour tax. Immense majorities for any scheme in India, small majorities for each elsewhere except US international dividend (44%) or mitigation in developing (43%), and AU mitigation in developing (49,6%). PB: very low sample size (~167) for a given redistribution, even lower (~55) for a given variant (that also specifies the price). Appendix also contains estimation of distributive impacts. Representative only along the two quotas: gender and age. Don't give the representativeness in terms of income (the third socio-demos that they ask) so it's probably bad.


% Global policies
% Pottier et al (17): A survey of global climate justice 
%* Hickel (17): The Divide: A Brief Guide to Global Inequality and its Solutions
%* Kopczuk et al (EER, 17) Compute optimal linear tax rate for all countries in two ways: decentralized or globally. Shows that the tax rate increases with inequality of skills (calibrated with the gini). The average decentralized rate is 0.41 The global one 0.62, with a global demogrant of 250$/month (higher than 73 countries' GDP). Show that within decentralized/country optimal taxation would not decrease global inequality by much (gini from 0.695 to 0.69, but down to 0.25 with global income tax). Show that USA don't give a damn of poor countries' people. citizens in the US (one of the richest) attach only 1/(2,000\*a) of the weight to the welfare of citizens in poorest countries, where a is the share  of transfer (supposedly) effectively arriving to the recipients. e.g. if half of aid is wasted by corrupt politicians, the weight is 1/1000.
% Carthy & Walsh (Oxfam, 22) propose various sources of funding for damages.
% Piketty (2014) "At what rate would [a global wealth tax] be levied? One might imagine a rate of 0 percent for net assets below 1 million euros, 1 percent between 1 and 5 million, and 2 percent above 5 million. Or one might prefer a much more steeply progressive tax on the largest fortunes (for example, a rate of 5 or 10 percent on assets above 1 billion euros). There might also be advantages to having a minimal rate on modest-to-average wealth (for example, 0.1 percent below 200,000 euros and 0.5 percent between 200,000 and 1 million)" He doesn't explicitly talk about revenue use, but implicitly they would be retained by each collecting country: "Le rôle principal de l'impôt sur le capital n'est pas de financer l'État social, mais de réguler le capitalisme.", "En principe, chaque pays de l'Union européenne pourrait obtenir des recettes du même ordre en appliquant seul un tel système."

% Global carbon pricing TODO! find current advocates of GCS
%* Grubb (90), Betram (92) advocate for global market with equal pc right
%* Bergh et al. (20) call for a "dual-track transition to global carbon pricing": an expanding climate club, and "a reorientation of UNFCCC negotiations creates room for talking seriously about a global carbon price schedule, including redistribution-of-revenues rules." They don't specify which equity rules to use.
%* Jamieson (01) advocates of equal pc burden-sharing (after the precursors Agarwal & Narain (91))
%* Bear et al (Science, 00), Bear (02), Athanasiou & Baer (02) advocate for equal pc burden-sharing (although weirdly, Bear & Athanasiou then change mind and advocate for the Greenhouse Development Rights, accounting for capacity and responsibility)
%* Cramton et al (17): Livre de pontes. Tout le monde est d'accord : un prix mondial du carbone est requis, il ne peut être obtenu que par la réciprocité des engagements (style climate club), et il faut quelques transferts des riches vers les pauvres ainsi que des sanctions commerciales pour aligner les incitations. Ch 4 (also Cramton et al 15) propose la formule suivante de transfert (positif ou négatif) à un fonds climat : générosité*émissions en excès (par rapport à la cible)*prix du carbone. On demanderait aux États autour de la moyenne d'émission de fixer ce paramètre de générosité, pour qu'il soit fixé de sorte à maximiser le prix, puis on fixerait le prix comme le prix minimum proposé (après avoir éjecté qqs pays récalcitrants des négos). Puis, sanctions commerciales pour ceux qui ne respectent pas le prix. Ch Gollier & Tirole proposent une formule aussi simple que l'autre : quota global*((1-g)*part des émissions à t=0 + g*part de la population), où g joue le même rôle de paramètre de générosité/éthique (que je voudrais mettre à 1, mais qu'ils disent tous de mettre < 1 pour que les pays riches acceptent. Le livre argumente bcp sur prix vs. quantité (TLM préfère prix sauf Gollier & Tirole), l'argument le plus convaincant en faveur du prix c'est qu'avec la procédure proposée le prix négocié serait le plus élevé possible, alors qu'avec la quantité c'est le budget carbone qui serait le point focal et ça aboutirait à une impasse (objectif trop ambitieux).
%* MacKay et al (Nature, 15) summarizes the above
%* Weitzman (17) advocates for a World Climate Assembly, choosing the price level with the median voter, and each country retaining the revenues.
% Fleurbaey & Zuber (13): The discount rate converges to the worst-off (affected by the measure) to the worst-off (beneficiary of the measure) discount rate, which depends on the growth between both agents. Applied to real data, we can consider that the worst-off affected by a global tax on CO_2 is the average-earner on earth (around 75% centile i.e. ~1000€/month, cf. Chancel & Piketty, Lakner & Milanovic, Chakravorty) while the worst-off beneficiary is the worst-off person in the future (among those less affected by CC thanks to the measure), probably below 1000€/month => negative discount rate.
% Stanton (11): Negishi weights obviate the IAMs’ equalization of income. 4 ways to solve this problem: 1. be more transparent, 2. stop weighting, 3. take linear utility (i.e. maximize global GDP), 4. stop optimizing. 
% Hoel (91): Shows that an international tax can be designed so that it is both efficient and satisfies whatever distributional objectives one might have.
% IMF (2019): global pricing (with either differentiated prices or international transfers) or, as a first step, a carbon price floor. 25% of revenues should be rebated to the bottom 40%, the rest used to reduce distortionary taxes or for green investments. Estimate that $75/t is needed in 2030 for 2°C.
% Parry et al (21): Proposal for an International Carbon Price Floor Among Large Emitters. Acknowledges that transfers could be necessary to induce climate action in low/middle-income countries, talks about transferring 1% of carbon revenues.
%- Sager: distributive effects of global pricing without int'l transfers.
%- Budolfson et al. (incl. Fleurbaey, Méjean, Zuber, Dennig) (21): global carbon price with within-country per capita dividend. Acknowledge that "The overall benefits to society are even greater if total carbon tax revenues are returned on an equal per capita basis globally, which directs more of the revenues towards the poorest populations in the world (rather than the poorest within each country or region)." Very short (3p, no appendix, no suppl. info)

% Foreign aid TODO: find more recent, check .lyx for already written paragraph
%* Kaufmann et al (12) Shows the level of perceived and desired aid in 26 countries between 2005 and 2008 (cf. Table 1). In most countries (incl. UK, DE, FR, ES but not U.S.) desired aid is larger than perceived. Argue that this is due to political influence efforts/possibilities of the rich, as they prefer less aid due to vested interests (support this by a theoretical model + correlations between level of lobbying and actual aid level, controling for desired aid). In most countries the gap between the two is small, except in the U.S. where perceived is 7.5% of GDP and preferred is 3%.Use WVS and Gallup (like Chong & Gradstein, Paxton & Knack) but have more waves and the others don't use the question on perceived aid. Shows that richer want less aid ("those in the top income quintile favour ODA (as a share of GNI) that is 0.13 percentage points lower than the preferred share for individuals in the bottom 40\% of the income distribution" after controling for perceived aid - our regression results are sensibly the same.). from 0 to higher than 25%: threshold at 0.05; 0.15; 0.35; 0.75; 1.5; 2.5; 4; 7.5; 17.5; 25, i.e. same number of thresholds but small than ours below 2.5 and higher above. 
%*? Milner & Tingley (13): (highly cited but no original data, don't think we need to cite it) In 2008, 44% of American wanted foreign aid cut (american elections study, 08). fraction of federal budget going to foreign aid (mean: 27%, median: 25%) / should go (mean: 13%, median: 10%) (WorldPublicOpinion, 10)
% PIPA (01): Overwhelming majorities support a multilateral effort to cut hunger in half by the year 2015 and say that they would be willing to pay for the costs of such a program. However, most do not think that the average American would be as willing to pay the necessary costs. when PIPA asked respondents to estimate how much of the federal budget was devoted to foreign aid, the median estimate was 15% -- 15 times the actual amount, which was just under 1%. More dramatically, when asked what an appropriate percentage would be, the median response was 5% -- 5 times the actual amount. And when asked to imagine that they heard the real amount was only 1%, only 18% of respondents said they thought that would be too much--as compared to the 75% who had initially said that the US was spending too much. what percentage of their "tax dollars that go to help poor people at home and abroad...should go to help poor people in other countries." The mean response was 16% (down a bit from 22% in response to this question in a 1996 PIPA poll). Strikingly, this turns out to be a far higher percentage than is currently given. In 1999, a bit less than 4% of the total spent on the poor went to the poor abroad. Sixty percent of respondents proposed a percentage that was higher than 4%.
%- DFID (10): Priorities: 1 NHS, 2 education, 3 support to poor countries, 4 police, 5 defence (p. 19). Show majority support for increased aid until 07, then median is to support stable aid (due to crisis?). It seems they don't give the info on actual amount though.
%* PIPA (08): Across 20 countries, 81% support that "developed countries have a moral responsibility to help reduce hunger ansevere poverty in poor countries (majority in every country). “the World Bank (Shantayanan et al, 2002) has estimated that it will require an extra US$39-54 billion per year to meet Millennium Development Goal 1 (MDG1). (…) The per person cost of meeting MDG1 came to £25 for the UK, $56 for the US, €27 for Germany, and so on. On average 77 per cent of respondents are in favour of contributing towards meeting the goal (provided that all others do too). To take the US example, 75 per cent of people supported paying an extra $56 per year to meet MDG1. What is significant about this figure is that it is only slightly below the support for the ‘cost free’ question as to whether the US should be willing to share a  small portion of its wealth with those who are in great need (79%).” Hudson & van Heerde (12)
%* Hudson & van Heerde (12):Reviews literature on foreign aid and criticizes it on a number of points (e.g. not uncovering the determinants, and not asking well the questions). Shows strong support for poverty alleviation, (at least partly) out of intrinsic altruism. Use 4 main sources: PIPA (01, 08) UK DIDP, Eurobarometer; cf. Table 1 for all surveys on foreign aid / Public support for development has been famously described as a mile wide and and inch deep (Smillie, 1996: ref impossible to find). Hard times at home have meant that public support appears to have turned against international development efforts (Henson and Lindstrom, 2010). / Monitor public support: (Fransman and Solignac Lacomte, 2004; McDonnell et al, 2003), Paxton and Knack, 2008; Chong & Gradstein 2006. Review surveys on aid. / ~75% support aid in developed countries (stable) but ‘84 per cent agreed with the assertion that ‘taking care of problems at home is more important than giving aid to foreign countries’ (PIPA, 2001:9).” / References on covariates of aid support / PIPA 2001, "On average, Americans thought just under 25 per cent of the US budget was allocated to foreign aid, and government should allocate less than 14 per cent of the national budget. However, when told that US spends approximately 1 per cent of the federal budget on foreign aid, 37 per cent of respondents thought this was too little, 44 per cent thought it was about right, and 13 per cent thought it too much."  Think that only 23% of aid really goes to the poor / “The 2009 UK survey, Public Attitudes towards Development, reports ‘public support for overseas aid’ at 72 per cent (DFID, 2009); while in the US support was a comparable 79 per cent (PIPA, 2001); and average support across the EU trends slightly higher than in the US and UK with 91 per cent saying it was either very (53%) or fairly (38%) important to provide aid to poor countries (Eurobarometer, 2005).” / “DFID has now begun asking questions that provide relative measures of the salience of development aid vis-à-vis other competing policy issues (DFID, 2009; IDC, 2009). / "high proportion (61%) of US citizens who felt that the US spends too much on foreign aid. [from another source]” / “The distinction between foreign aid, which includes military spending, and development aid/assistance is an important one” / “81 per cent of respondents believed that developed countries do have a moral responsibility to work towards reducing hunger and severe poverty (WorldPublicOpinion.org, 2008). (…) there are a good number of people who support aid despite the fact they do not think it works. What this suggests – but cannot show in any detail – is that people have nonutilitarian motives for supporting aid.” / “support for development assistance is highly contingent on respondents’ perceptions of the effectiveness of aid, especially with regard to corruption (Henson et al, 2010). For example, in the UK, 47 per cent of respondents thought that aid was wasted, with sizable majorities citing corruption and poor management and/or delivery as primary factors (DFID, 2008). More disconcertingly, US respondents thought that only 23 per cent of US aid money that goes to poor countries ends up helping the people who really need it and 54 per cent of US aid money that goes to poor countries ends up in the pockets of corrupt government officials (PIPA, 2001). (…) international charities and NGOs are deemed best suited/most effective compared to donor countries” / UK ‘MyAid’ plan – where the public gets to vote on how a pot of money should be distributed – / "public engagement should be about ‘opening up the political and wider societal space to the possibility of deeper change’ (Darnton and Kirk, 2011:14).”
%* Gilens (01) 17% fewer American with high political knowledge want to cut foreign aid when we provide them specific information about aid amount.
%- Chong & Gradstein (16): from WVS 95-99, 58% want that their country give more foreign aid (but misperceptions are not taken into account)
%* Bauhr et al (13): Support for aid is reduced by perception of corruption in recipient countries. However, this effect is reduced by the aid-corruption paradox (and other things): most corrupt countries need more help.
%- Nair (18): (lack of) Aid support in US driven by information on global distribution, because people underestimate their rank by 27 centiles and overestimate global median income by a factor 10.
%- Williamson (19): Public Ignorance or Elitist Jargon? Reconsidering Americans’ Overestimates of Government Waste and Foreign Aid. "Foreign aid" encompasses military spending, in the mind of American.
%- McDonnell et al (03) Public Opinion and the Fight against Poverty
%- Nair (16): preferences driven by worldviews rather than self-interest
%- Bodenstein & Faust (17): Determinants of support for aid conditionality. They are: perceived corruption in donor country, right-wing.
%- Scotto et al (17): We Spend How Much? Misperceptions, Innumeracy, and Support for the Foreign Aid in the United States and Great Britain. Less American and British want aid cut when information on current aid is given in % of GDP rather than in $.
%* Paxton & Knack (12): Majorities want more aid, and main determinants are trust, ideology, interest in politics, and female (all positive). Gallup 02: in US 45% want more aid (rather than stable) vs. 68-91 in DE-UK-ES. Like Chong & Gradstein, find that desired aid increases with income, contrary to Kaufmann et al. but the latter contains more datasets.
%- Wood (15): Determinants for aid support in Australia. Wood (18) Examine Australian support for aid: although there is support to help foreign poor, people back recent aid cuts.
%- Bayram (17): Aid support associated with trust, i.e. seeing integrity and trustworthiness in others.
%- Cheng & Smyth (16): Why Give it Away When You Need it Yourself? Understanding Public Support for Foreign Aid in China. Political ideology and patriotism main explaining variables for aid support. People in poorer provinces less supportive.
%- Milner & Tingley (10) theory + empirics: who supports aid and why. owners of capital in donor countries tend to gain from aid and thus are more likely to support giving aid
%- Easterly (JEP, 03) Can Foreign Aid Buy Growth? No (disproves Hansen & Tarp).
%- Hansen & Tarp (01) Aid increases growth (empirical evidence)
%- Tresch et al. (22): 66% of Swiss people want to increase their foreign aid
%- Harris (17): majority of French want to decrease foreign aid


% Universalism
%- Enke et al. (Manag. Science, 23): measures universalism by asking to split donation to domestic and foreigner of same absolute income (US).
%- Enke et al. (ReStud, 23): unviersalism more correlated to policy attitudes than income, education, religiosity or beliefs about government efficiency (West).
%- Cappelen et al. (NBER, 22): how unviversalism (as measured above) varies across countries. Comparable in Europe and US (lower in China, higher in Africa)
%- Cherry et al (17) show in the lab that some people prefer policies detrimental to them due to their worldview.


% Free-riding
%- Mildenberg (2019): people are not free riders
%- McGrath & Bernauer (17): review paper. people are not free riders. Preferences concerning climate policy tend to be driven primarily by a range of personal predispositions and cost considerations, which existing research has already explored quite extensively, rather than by considerations of what other countries do
%- Bernauer & Gampfer (15): US and IA people are not free riders. They each overestimate their country's emissions at one third of global total.


% Social norms
%- Bursztyn et al. (AER, 20): social norms can change following new public information such as unexpected election outcome. After Trump election, people express more xenophobic views and judge less severely those who do.
%- Farrow et al. (17): review of effect of social norm intervention on environmental attitudes

% Incentive compatibility
%- Danz et al


% Second-order beliefs
%* Mildenberg & Tingley (19): survey elites (Congress staffers, scholars) and public in U.S. and China and show pluralistic ignorance of pro-climate attitudes, egocentric bias, and increasing support after beliefs are updated.
%- Bursztyn & Yang (21): Review of the field. Misperceptions about others are widespread, asymmetric, much larger when about out-group members, and positively associated with one’s own attitudes.
%- Drews et al. (22): in Spain, supporters (resp. opponents) of carbon tax overestimate (resp. underestimate) support. Providing information doesn't change the overall support.
%* Falk et al. (21): Respondents vastly underestimate the prevalence of climate- friendly behaviors and norms among their fellow citizens. Providing respondents with correct information causally raises individual willingness to fight climate change as well as individual support for climate policies. The effects are strongest for individuals who are skeptical about the existence and threat of global warming.
%- Di Tella et al. (AER, 15): The results of the lab experiment favor the hypothesis that people avoid altruistic actions by distorting beliefs about others' altruism
%- Allport (1924): first book on pluralistic ignorance
%- Allport (40): function of poll is to correct pluralistic ignorance
%- Studies on pluralistic ignorance: business (Buckley et al. 00), against affirmative action (Van Boven 00), political correctness (Braghieri, AER 21), alcohol (Suls & Green, 03), white support for racial segregation (O'Gorman 75), CC (Geiger & Swim 16), hooking up (Lambert et al 03, cf. note for paragraph of pluralistic ignorance), women working outside home in Saudi Arabia (Bursztyn et al. 20)
%- Geiger & Swim (16) Shows that pluralistic ignorance of others' concern about CC leads people to talk less about CC and self-silence themselves.
%- Miller & MacFarland (87) Shows that pluralistic ignorance emerges because individuals believe that fear of embarrassment is a sufficient cause for their own behavior but not for the behavior of others.


% Elite surveys TODO find more
%* Mildenberg & Tingley (19): Congress staffers, cf. second-order beliefs
%- Hertel-Fernandez et al. (2019): Survey on US Congress staffers (not on climate)
%- Milner & Tingley (10) (not sure it's a survey) owners of capital in donor countries tend to gain from aid and thus are more likely to support giving aid
%- Lange et al. (Energy Econ, 2007): climate negotiators
%- Lange et al. (EER, 2010): same data as Lange et al. (10)
%- Dannenberg et al. (ERE, 2010): elicit climate negotiators’ equity preferences using Fehr & Schmidt (99) method => regional differences in addressing climate change are driven more by national interests than by different equity concerns
%- Kesternich et al. (EEPS, 2020): survey on climate negotiators about their preferred burden-sharing rules: we observe tendencies for a more harmonized view among key groups towards the ability-to-pay rule in a setting of weighted burden sharing rules
%- Lange & Schwirplies (ERE, 2017): combines Lange et al. (10) and Schleich et al.
%* Hjerpe et al. (2011): Delegates at COP2009. The results indicate that voluntary contribution, indicated as willingness to contribute, was the least preferred principle among both negotiators and observers. Three of the four principles for allocating mitigation commitments were recognized widely across the major geographical regions: historic 1990, capacity to pay, and equal per capita emissions. The difference was never below 25 percentage units, and the opponent share never exceeded 16%.
%- Scholte et al. (2020)
%- Bayram (17): cosmopolitanism of German politicians and their respect of international law


% Global poverty gap
%* Bolch et al. (22)
%- Zhang (16) estimates the poverty gap in each country. Global one is at $80G/year.


% Basic income TODO find more
%* Egger et al. (19): positive gen eq effects. We provided one-time cash transfers of about USD 1000 to over 10,500 poor households across 653 randomized villages in rural Kenya. The implied fiscal shock was over 15 percent of local GDP. We find large impacts on consumption and assets for recipients. Importantly, we document large positive spillovers on non-recipient households and firms, and minimal price inflation.
%* Haushofer & Shapiro (16): The Short-term Impact of Unconditional Cash Transfers to the Poor: Experimental Evidence from Kenya. Monthly transfers are more likely than lump-sum transfers to improve food security


% Unequal exchange / embodided labour
%- Reyes et al (17)
%- Sakai et al (17)
%- Alsamawi et al. 2014


% NDCs assessments or burden-sharing computations. TODO check the contraction & convergence scheme proposed by France
%- Bourban (18): Soutient un marché du carbone avec droits en proportion des émissions cumulées depuis 1990. Et des “mesures volontaires de contrôle de la population mondiale”.
%- Raupach et al (NCC, 14) 
%- Grasso (2012)
%- van den Berg et al (20)
%- Meyer (04) Contraction and Convergence (i.e. grandfathering converging to equal pc, within an ETS)
%- >Baer et al (08)< (cite this one, others don't give more info), Baer (13), Athanasiou et al (22), Holz et al (19) https://calculator.climateequityreference.org/ Athanasiou, Greenhouse Development Rights, EcoEquity calculator, US fair share. Effort-sharing approach based on splitting emissions reductions in function of capacity to pay (~ share of global income in top 30%) and responsibility (share of emissions since 1950), weighted equally. Corresponds to UNFCCC wording. Pb of this method (applying to any choice of parameters): A country with relatively low incomes (e.g. equal distribution slightly above the p70) and that has few historical responsibility would have a relatively low effort. Even more problematic, the **poorest countries would have virtually 0% of the effort, hence they would be allowed to emit following the baseline trajectory… but this baseline is not fair; it amounts to grandfathering**. It is computed as the “product of the projected GDP and CO2 emission intensity”. ([https://climateequityreference.org/calculator-information/gdp-and-emissions-baselines/](https://climateequityreference.org/calculator-information/gdp-and-emissions-baselines/)), and give for example 0.8tCO2e/cap for RDC in 2030 (16% more than in 2020, but lot lower than the objective of ~4t). => Compared to an equal right to emit pc, this method favors countries like China (allowed to remain stable over 2020-30 vs. reduced by 35-40%) and penalizes countries like the U.S. and Africa. 
%  in Athanasiou et al (22) Justification of Greenhouse Development Rights instead of Equal per capita right is on p. 36. It is weak, and basically that historical responsibility should be taken into account. Conversely, justification against historical resp. is that the latter doesn’t take into account capacity to pay (it is not said like this, but we can think of ex-USSR).
%- Pachauri et al. (Science, 2022) 
%- Robiou du Pont et al. (NCC, 2016)
%- Robiou du Pont et al. (ERL, 2016)
%- Höhne et al. (Climate Policy, 2014): review of 40 papers
%- Gao et al. (FEM, 2019)
%- Gignac & Matthews (ERL, 15)
%- Matthews (16) Quantifying carbon debts among nations
%- https://climateequitymonitor.in/ computes carbon debt based on equal per capita cumulative emissions. contact@climateequitymonitor.in https://twitter.com/equity4climate


% Mismatch between preferences and climate action
%- McCright & Dunlap (03) show that it's an organized conservative movement that succeeded in the U.S. not ratifying Kyoto, through lobbying and disinformation.


% Wealth tax attitudes
% look for surveys on global tax => I've found no result with survey or attitudes + "global tax" or "global wealth tax" in google scholar
% Fisman et al (17): Americans want a 3% tax on inherited wealth
%- Christensen et al. (Oxfam, 23) p. 32 gives references on rich tax attitudes, with always strong majority support:
%* OECD (19): 52-80% of absolute support for "government tax the rich more than they currently do in order to support the poor" in 21 OECD countries
%* Isbell (22): 34 African countries
%- Patriotic Millionaires (22), UK
%- Americans for Tax Fairness (21), US
%- Gallup (22), US
%- Fight Inequality Alliance India (22), IA

% Different framing of burden-sharing, depending on what should be split:
% - mitigation costs: this is the most used as it is easiest to explain. The issue is that it is not specified how agents pay (or if some agents receive payments) and implicitly, there is no negative costs (transfers exceeding the costs) and the carbon price is not uniform. Used in .
% - emission: this one is vague as it doesn't state at which date emissions pc converge (if they do) and whether there are side payments.
% - emission rights: this one is the most accurate as there is no need of a BAU scenario to compute the mitigation needed and its cost.

% Different fairness principles:
% - equal emission right per capita: using this as a baseline, we can call 'grandfathering' any principle that is more regressive and 'historical responsibility' any principle that is more progressive
% - equal emission reduction (in share of current emission) per capita: grandfathering
% - emission rights proportional to current emissions: grandfathering
% - costs proportional to current emissions: polluter-pay principle
% - costs proportional to cumulative emissions: so-called historical responsibility but may actually have a grandfathering component

% Surveys of population:
% - Schleich et al. (Climate Policy, 16) ask for ranking (TODO check) and find an identical ranking of fairness principles in China, Germany, and the US: accountability (costs according to emissions) followed by capability (according to economic strength), egalitarianism (equal emission per capita), and sovereignty (constant share of global emission) (see Lange & Schiwplies (17) for the computations). 
%   Polluter-pays: Every country has to bear costs according to the emissions it causes (hence countries causing higher emissions have a higher share of the costs).
%   Ability-to-pay: Every country has to bear costs according to its economic strength (hence richer countries have a higher share of the costs).
%   Egalitarian: Every country is allowed to produce the same amount of emissions per capita (hence countries with currently high emissions per capita have higher costs).
%   Sovereignty: Every country is allowed to produce the same share of global emissions as in the past (hence the proportional reduction of emissions is the same for every country).
% other findings: international agreements are important but current ones are unsuccessful, people find themselves poorly represented in climate negotiations
% - Bechtel & Scheve (PNAS, 13) find with a conjoint analysis on FR, DE, UK, US that a climate agreement is 5 p.p. less likely to be preferred (to a random alternative) if only rich countries pay (other burden-sharing are: pay prop. to current emissions / historical emissions / rich countries pay more than poor countries) [TODO: check SI that these are the verbatim] and 15 p.p. more likely to be preferred if it includes 160 (out of 192) countries rather than 20 => confirms preference for global policies (rather than only partial coverage). Finds that costs is what matters most: preference decreases by 30pp if it’s 2.5% of GDP compared to 0.5%.
% - Carlsson et al. (REE, 13) find using a 09 choice experiment that Americans prefer capacity to pay > current responsibility > historical responsibility > equal emissions per capita while Chinese prefer historical > capacity > current > equal emissions.
%   Capacity to pay: Countries with high income levels must pay a larger share of the costs than countries with low income levels. This option says that countries with greater ability to pay should pay more
%   Current responsibility: Countries with currently high emissions levels must pay a larger share of the costs than countries with currently low emissions levels. This option says that those countries that are currently a larger part of the problem should pay more.
%   Historical responsibility: Countries with a history of high emissions levels must pay a larger share of the costs than countries with a history of lower emissions. This option recognizes that CO2 builds up in the atmosphere over many years. Thus, countries with a history of high emissions should pay more because they caused more of the problem.
%   Equal emissions pc: Countries with emissions per person greater than an agreed amount must pay, and they must pay more the higher their emissions per person area.
% > "equal emissions" is a misnomer as this is about costs (not emissions) and it's just a more progressive version of current responsibility / polluter-pay, where high-emitting pay more and low-emitting don't pay. The result for US is compatible with the other papers as Americans agree that rich countries (or high-emitting, the diff is small) should pay more. The Chinese position could also be reconciliable once we define responsibility from footprint rather than territorial and that there will be transfers from rich to poor countries.
% - Carlsson et al. (Ecol Eco, 11) find that Swedes prefer that "all countries are allowed to emit an equal amount per capita" rather than options where emissions reduce in relation to current or historical emissions and continue to be higher in high-emitting countries. 
% - Meilland et al. (23) find that in US and France, most favored fairness principle is Equality in per capita emissions: "all countries commit to converge to the same average of total emissions per inhabitant, compatible with a controlled climate change" and second-most (which closely follows) is grandfathering: "all countries commit to reduce their emissions by a same proportion". 73% in each disagree with grandfathering when defined as "countries which emitted a lot of carbon in the past have a right to continue emitting more than others in the future". To rationalize these contrasted views with grandfathering, we can interpret them as: equal rights, equal emission reductions, and transfers. 
%   convergence per capita (70%): all countries commit to converge to the same average of total emissions per inhabitant, compatible with a controlled climate change
%   grandfathering (60%): all countries commit to reduce their emissions by a same proportion
%   past emissions (20% choose it among their two favorite): countries which emitted less in the past commit to reduce their emissions less than other countries
%   poor countries (20%): poorer countries commit to reduce their emissions less than richer countries
%   cost-efficiency (20%): countries where reducing emissions is more costly commit to reduce their emissions less than other countries
% Other findings: people prefer international settlement on CC even if it empedes on sovereignty, a majority prefers to target footprint rather than territorial emissions, median is that countries should be held accountable for post-1990 emissions, self-serving bias when judging e.g. India vs. EU, no shared understanding of fairness when asked to coordinate between French and Americans
% - Dechezleprêtre et al. (WP, 22) find that equal per capita right > historical responsability, capabilities > grandfathering; that global CC policies are needed; 50% support for global T&D; strong support for global tax on millionaires; no free-riding. TODO: check FR, US wording
% - Dabla-Norris et al. (WP, 23) find strong majority for “all countries” everywhere in “Which countries do you think should be paying to reduce carbon emissions?”, and majority for current rather than historical in all countries but China and Saudi Arabia in “Should countries be paying to reduce carbon emissions based on their current or accumulated historic levels of emissions?”

% > Position making all this compatible: people want that every country engage in strong decarbonization effort together, with a global quota, converging to climate neutrality in the medium run, based on an equal right to emit per person, implying that rich countries pay and low-emitting countries receive funding. Where the rankings differ, it is likely because the definitions or wordings are different, and also because it involves different countries (Sweden != US != China).
% - Schleich find support for costs according to emissions and against immediate equalization of emissions (but nothing against convergence to equal emissions per capita).
% - This is just in contradiction with Carlsson (11) which finds that Swedes prefer the equalization (with a similar wording) to other reduction options. TODO: check wording of the latter.
% - Bechtel find agreement that rich countries should pay more and historical emissions matter, but just that they should not be the only one to make the efforts. 
% - Carlsson (13) find that the least preferred option in China and US is when low-emitting countries don't participate to the effort. Ability to pay is liked in both countries.
% - Meilland find that convergence is the most preferred, followed by emission reductions of same proportion, disagreement with grandfathering expressed in terms of emission rights.
% - Dechezleprêtre find support for equal right is strongest, although historical responsibility and capabilities are also supported. The quota system is strongly supported.

% Surveys of negotiators:
% - Hjerpe et al. (WP, 11)
% - Dannenberg et al. (ERE, 10): measuring negotiators' equity preferences, regional differences in addressing climate change are driven more by national interests than by different equity concerns.
% - Lange et al. (Energy Econ, 07): Mix of self-serving bias and support for egalitarian principle.
% - Kesternich et al. (EEPS, 21): kind of convergence on ability-to-pay.

% Other papers:
% - Lange & Schwirplies (ERE, 17) develop a theoretical model (building on Buchholz et al. (05)), supported by data, justifying that climate negotiators (chosen by the citizens) have lower environmental preferences than their citizens and equity views more aligned with the other negotiators. 
\clearpage
\section{Raw results from the first U.S. complementary survey}
% TODO! socio-demos, vote, interest politics, left_right, govt_involvement, donation_charities, bias, feedback

% \begin{figure}[h!]
%     \caption{label}\label{fig:vars}
%     \makebox[\textwidth][c]{\includegraphics[width=\textwidth]{../figures/US1/vars.pdf}} 
% \end{figure}

\begin{figure}[h!]
    \caption{Correct answers to comprehension questions.}\label{fig:understood_each}
    \makebox[\textwidth][c]{\includegraphics[width=\textwidth]{../figures/US1/understood_each.pdf}} 
\end{figure}

\begin{figure}[h!]
    \caption{Number of correct answers to comprehension questions.}\label{fig:understood_score}
    \makebox[\textwidth][c]{\includegraphics[width=\textwidth]{../figures/US1/understood_score.pdf}} 
\end{figure}

\begin{figure}[h!]
    \caption{Support for the GCS, NC and the combination of GCS, NR and C.}\label{fig:support_binary}
    \makebox[\textwidth][c]{\includegraphics[width=\textwidth]{../figures/US1/support_binary.pdf}} 
\end{figure}

\begin{figure}[h!]
    \caption{Beliefs regarding the support for the GCS and NR.}\label{fig:belief}
    \makebox[\textwidth][c]{\includegraphics[width=\textwidth]{../figures/US1/belief.pdf}} 
\end{figure}

\begin{figure}[h!]
    \caption{List experiment.}\label{fig:list_exp}
    \makebox[\textwidth][c]{\includegraphics[width=\textwidth]{../figures/US1/list_exp.pdf}} 
\end{figure}

\begin{figure}[h!]
    \caption{Conjoint analyses.}\label{fig:conjoint}
    \makebox[\textwidth][c]{\includegraphics[width=\textwidth]{../figures/US1/conjoint.pdf}} 
\end{figure}

\begin{figure}[h!] % already in text
    \caption{[Asked only to non-Republicans] Conjoint analysis n°4: random programs at the Democratic primary.}\label{fig:ca_r}
    \makebox[\textwidth][c]{\includegraphics[width=\textwidth]{../figures/US1/ca_r.png}} 
\end{figure}

\begin{figure}[h!]
    \caption{Donation in case of lottery win.}\label{fig:variables_donation}
    \makebox[\textwidth][c]{\includegraphics[width=\textwidth]{../figures/US1/variables_donation.pdf}} 
\end{figure}

\begin{figure}[h!]
    \caption{Willingness to sign real-stake petition for the GCS or NR.}\label{fig:variables_petition}
    \makebox[\textwidth][c]{\includegraphics[width=.7\textwidth]{../figures/US1/variables_petition.pdf}} 
\end{figure}

\begin{figure}[h!] % already in text
    \caption{Support for various global policies.}\label{fig:support_likert}
    \makebox[\textwidth][c]{\includegraphics[width=\textwidth]{../figures/US1/support_likert.pdf}} 
\end{figure}

% \begin{figure}[h!]
%     \caption{label}\label{fig:climate_policies}
%     \makebox[\textwidth][c]{\includegraphics[width=\textwidth]{../figures/US1/climate_policies.pdf}} 
% \end{figure}

% \begin{figure}[h!]
%     \caption{label}\label{fig:global_policies}
%     \makebox[\textwidth][c]{\includegraphics[width=\textwidth]{../figures/US1/global_policies.pdf}} 
% \end{figure}

\begin{figure}[h!]
    \caption{Attitudes regarding the evolution of U.S. foreign aid.}\label{fig:foreign_aid_raise_support}
    \makebox[\textwidth][c]{\includegraphics[width=\textwidth]{../figures/US1/foreign_aid_raise_support.pdf}} 
\end{figure}

\begin{figure}[h!]
    \caption{[Asked to those who wish an increase of foreign aid at some conditions.] Conditions at which foreign aid should be increased.}\label{fig:foreign_aid_condition}
    \makebox[\textwidth][c]{\includegraphics[width=\textwidth]{../figures/US1/foreign_aid_condition.pdf}} 
\end{figure}

\begin{figure}[h!]
    \caption{[Asked to those who wish a decrease or stability of foreign aid.] Reasons why foreign aid should not be increased.}\label{fig:foreign_aid_no}
    \makebox[\textwidth][c]{\includegraphics[width=\textwidth]{../figures/US1/foreign_aid_no.pdf}} 
\end{figure}

\begin{figure}[h!]
    \caption{Preferred approach of U.S. diplomats at international climate negotiations.}\label{fig:negotiation}
    \makebox[\textwidth][c]{\includegraphics[width=\textwidth]{../figures/US1/negotiation.pdf}} 
\end{figure}

% \begin{figure}[h!]
%     \caption{label}\label{fig:vote}
%     \makebox[\textwidth][c]{\includegraphics[width=\textwidth]{../figures/US1/vote.pdf}} 
% \end{figure}

\begin{figure}[h!]
    \caption{Extent to which selected issues are viewed as important problems.}\label{fig:problem}
    \makebox[\textwidth][c]{\includegraphics[width=\textwidth]{../figures/US1/problem.pdf}} 
\end{figure}

\begin{figure}[h!]
    \caption{Group defended when voting.}\label{fig:group_defended_agg}
    \makebox[\textwidth][c]{\includegraphics[width=\textwidth]{../figures/US1/group_defended_agg.pdf}} 
\end{figure}

% \begin{figure}[h!]
%     \caption{label}\label{fig:group_defended}
%     \makebox[\textwidth][c]{\includegraphics[width=\textwidth]{../figures/US1/group_defended.pdf}} 
% \end{figure}

\begin{figure}[h!] % already in text
    \caption{Prioritization of policies.}\label{fig:points_us}
    \makebox[\textwidth][c]{\includegraphics[width=\textwidth]{../figures/US1/points_us.pdf}} 
\end{figure}

% \begin{figure}[h!]
%     \caption{label}\label{fig:share_policies_supported}
%     \makebox[\textwidth][c]{\includegraphics[width=\textwidth]{../figures/US1/share_policies_supported.pdf}} 
% \end{figure} % TODO? uncomment?

% \begin{figure}[h!]
%     \caption{label}\label{fig:vars}
%     \makebox[\textwidth][c]{\includegraphics[width=\textwidth]{../figures/US1/vars.pdf}} 
% \end{figure}

% WPcomment
%% Here is the endmatter stuff: Supplementary Info, etc.
%% Use \item's to separate, default label is "Acknowledgements"
% \begin{addendum} % 177 words
%  \item We are grateful for financial support from the University of Amsterdam and TU Berlin. We are grateful for financial support from the OECD, the French Ministry of Foreign Affairs, the French Conseil d’Analyse Economique and the Spanish Ministry for the Ecological Transition and Demographic Challenge. We also acknowledge support from the Grantham Foundation for the Protection of the Environment and the Economic and Social Research Council through the Centre for Climate Change Economics and Policy. We thank Antoine Dechezleprêtre, Tobias Kruse, Bluebery Planterose, Ana Sanchez Chico, and Stefanie Stantcheva for their invaluable inputs for the project. We thank Auriane Meilland for feedback. We thank Laura Schepp, Martín Fernández-Sánchez, Samuel Gervais, Samuel Haddad, and Guadalupe Manzo for assistance in the translation. 
%  \item[Registration] The project %is approved by IRB at Harvard University (IRB21-0137), and 
%  was preregistered in the Open Science Foundation registry (osf.io/fy6gd).
%  \item[Competing Interests] The authors declare that they have no
% competing interests.
% \item[JEL codes] P48, Q58, H23, Q54.
% \item[Keywords] Climate change, global policies, cap-and-trade, perceptions, survey, inequality, wealth tax.
%  \item[Correspondence] Correspondence and requests for materials
% should be addressed to Adrien Fabre~(email: fabre.adri1@gmail.com).
% \end{addendum}

%%
%% TABLES
%%
%% If there are any tables, put them here.
%%

% \begin{table}
% \centering
% \caption{This is a table with scientific results.}
% \medskip
% \begin{tabular}{ccccc}
% \hline
% 1 & 2 & 3 & 4 & 5\\
% \hline
% aaa & bbb & ccc & ddd & eee\\
% aaaa & bbbb & cccc & dddd & eeee\\
% aaaaa & bbbbb & ccccc & ddddd & eeeee\\
% aaaaaa & bbbbbb & cccccc & dddddd & eeeeee\\
% 1.000 & 2.000 & 3.000 & 4.000 & 5.000\\
% \hline
% \end{tabular}
% \end{table}

\end{document}
