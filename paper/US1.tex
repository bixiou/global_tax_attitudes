\documentclass{nature}
% \documentclass[12pt,english]{article}
% \usepackage[utf8]{inputenc}
% \usepackage[left=1.5in,right=1.5in,top=1.5in,bottom=1.5in]{geometry}
% \usepackage{bm}
\usepackage{amsmath}
\usepackage{amssymb}
% \usepackage{indentfirst}
% \usepackage[hyperpageref]{backref} % back references biblio
% \usepackage{tocbibind}
% \usepackage[round,sort&compress]{natbib}
% \setcitestyle{aysep={}} 
% \usepackage{amsfonts}
% \usepackage{enumerate}
% \usepackage{babel}
% \usepackage{caption}
% \usepackage{supertabular}
% \usepackage{tabularx}
% \usepackage{float}
% \usepackage{dsfont}
% \usepackage{fancyvrb}
% \usepackage{verbatim}
% \usepackage[hyphens]{url}
% \usepackage{hyperref}
% \usepackage{enumitem}
% \usepackage{setspace}
% \usepackage{comment}
% \usepackage{subcaption}
% \usepackage{graphicx}
% \usepackage{tikz}
% \usepackage{gensymb}
\usepackage{eurosym}
% \usepackage{textcomp}

% \usepackage{tabulary}
% \usepackage{tabularx}
% \usepackage{booktabs}
% \usepackage{fullpage}
% \usepackage{morefloats}
% % \usepackage[utf8]{inputenc}
% % \usepackage{bm}
% % \usepackage{indentfirst}
% % \usepackage{tocbibind}
% % \usepackage{enumerate}
% \usepackage{makecell}
% \usepackage{lscape}
% \usepackage{pdflscape}
% \usepackage{longtable}
% \usepackage{rotating}
% \usepackage{fancyhdr}
% \usepackage{tocloft}
% \usepackage{multibib}
% \usepackage{titletoc}
% \usepackage[export]{adjustbox}
% \usepackage[anythingbreaks]{breakurl} % for links
% %\usepackage[nomarkers,figuresonly]{endfloat} % Figures at the end
% \hypersetup{
%   colorlinks=true, % breaklinks=true,
%   urlcolor=purple,    % color of external links
%   linkcolor=blue,  % color of toc, list of figs etc.
%   citecolor=violet,   % color of links to bibliography
% }
% %\usepackage[section,below]{placeins} % Floats placed in the section they appear in.

% % Getting landscape page and page number/footer on bottom of page (instead of to the left)
% \fancypagestyle{mylandscape}{
% \fancyhf{} %Clears the header/footer
% \fancyfoot{% Footer
% \makebox[\textwidth][r]{% Right
%   \rlap{\hspace{1.5cm}% Push out of margin by \footskip
%     \smash{% Remove vertical height
%       \raisebox{13.6cm}{% Raise vertically
%         \rotatebox{90}{\thepage}}}}}}% Rotate counter-clockwise
% \renewcommand{\headrulewidth}{0pt}% No header rule
% \renewcommand{\footrulewidth}{0pt}% No footer rule
% }

% \fancypagestyle{page_left}{%
% 	\renewcommand{\headrulewidth}{0pt}
%   \fancyhf{}
%   \fancyfoot[OC]{%
%       \begin{tikzpicture}[remember picture,overlay]
%           \node[xshift=1cm] (number) at (current page.west) {\thepage};
%       \end{tikzpicture}
%   }%
% }
% \renewcommand{\thesubfigure}{\Alph{subfigure}}

% \newcites{App}{Appendix References}

% \captionsetup[table]{skip=-10pt}

% \title{International Attitudes Toward Global Policies \\ Addressing Climate Change and Inequality\thanks{Corresponding author: Fabre: CNRS, CIRED (fabre.adri1@gmail.com).
% Thomas Douenne: University of Amsterdam (t.r.g.r.douenne@uva.nl). Linus Mattauch: TU Berlin (linus.mattauch@tu-berlin.de). We are grateful for financial support from the University of Amsterdam and TU Berlin. We are grateful for financial support from the OECD, the French Ministry of Foreign Affairs, the French Conseil d’Analyse Economique and the Spanish Ministry for the Ecological Transition and Demographic Challenge. We also acknowledge support from the Grantham Foundation for the Protection of the Environment and the Economic and Social Research Council through the Centre for Climate Change Economics and Policy. We thank Antoine Dechezleprêtre, Tobias Kruse, Bluebery Planterose, Ana Sanchez Chico, and Stefanie Stantcheva for their invaluable inputs for the project. We thank Auriane Meilland for feedback. We thank Laura Schepp, Martín Fernández-Sánchez, Samuel Gervais, Samuel Haddad, and Guadalupe Manzo for assistance in the translation. The project %is approved by IRB at Harvard University (IRB21-0137), and 
% was preregistered in the Ooen Science Foundation registry (osf.io/fy6gd).}}
% %\author{OECD}
% \author{Adrien Fabre, Thomas Douenne, and Linus Mattauch}
% \date{\today}

% \begin{document}

% \maketitle

% \begin{small}
% \begin{abstract}

% \noindent \cite{dechezlepretre_fighting_2022-1} %Using new surveys on more than 40,000 respondents in twenty countries that account for 72\% of global CO$_\text{2}$ emissions, we study the understanding of and attitudes toward climate change and climate policies. We show that, across countries, support for climate policies hinges on three key perceptions centered around the effectiveness of the policies in reducing emissions (effectiveness concerns), their distributional impacts on lower-income households (inequality concerns), and their impact on the respondents' household (self-interest). We show experimentally that information  specifically addressing these key concerns can substantially increase the support for climate policies in many countries. Explaining how policies work and who can benefit from them is critical to foster policy support, whereas simply informing people about the impacts of climate change is not effective. Furthermore, we identify several socioeconomic and lifestyle factors -- most notably education, political leanings, and availability of public transportation -- that are significantly correlated with both policy views and overall reasoning and beliefs about climate policies. However, it is difficult to predict beliefs or policy views based on these characteristics only. 

% \end{abstract}

% \textbf{JEL codes:} P48, Q58, H23, Q54
% % Q54 Climate • Natural Disasters and Their Management • Global Warming
% % Q58 Government Policy (Q is Environmental econ)
% % D78 Positive Analysis of Policy Formulation and Implementation
% % H23 Externalities • Redistributive Effects • Environmental Taxes and Subsidies (H is public econ)
% % P48 Political Economy • Legal Institutions • Property Rights • Natural Resources • Energy • Environment • Regional Studies (P4 is Other economic systems)
% % H41 Public Goods
% % H54 Infrastructures • Other Public Investment and Capital Stock


% \textbf{Keywords:} Climate change, global policies, cap-and-trade, perceptions, survey, inequality, wealth tax.

% \end{small} 

% \clearpage
% % \startcontents
% % \printcontents{ }{1}{\section{\contentsname}}
% % \clearpage
% \section{Introduction\label{sec:intro}}

% % \clearpage
% \renewcommand{\bibsection}{\section{\refname}}
% \bibliographystyle{naturemag}
% \bibliography{global_tax_attitudes}
% % \stopcontents

% \end{document}

% The following allows keeping figures within the text (otherwise nature.cls would ignore them)
\usepackage{graphicx}
\makeatletter
\let\saved@includegraphics\includegraphics
\AtBeginDocument{\let\includegraphics\saved@includegraphics}
\renewenvironment*{figure}{\@float{figure}}{\end@float}
\makeatother

\title{International Attitudes Toward Global Policies %\\ Addressing Climate Change and Inequality
} 

\author{Adrien Fabre$^{1,2}$, Thomas Douenne$^3$ and Linus Mattauch$^4$}


\begin{document}

% Nature guidelines (not NCC!)
% Sections can only be used in Articles.  Contributions should be organized in the sequence: title, text, methods, references, Supplementary Information line (if any), acknowledgements, interest declaration, corresponding author line, tables, figure legends.

% No subsubsection nor paragraph

% Spelling must be British English (Oxford English Dictionary)

%Each figure legend should begin with a brief title for the whole figure and continue with a short description of each panel and the symbols used. For contributions with methods sections, legends should not contain any details of methods, or exceed 100 words (fewer than 500 words in total for the whole paper). In contributions without methods sections, legends should be fewer than 300 words (800 words or fewer in total for the whole paper).

% Articles are restricted to 50 references,

% In addition, a cover letter needs to be written with the
% following:
% \begin{enumerate}
%  \item A 100 word or less summary indicating on scientific grounds
% why the paper should be considered for a wide-ranging journal like
% \textsl{Nature} instead of a more narrowly focussed journal.
%  \item A 100 word or less summary aimed at a non-scientific audience,
% written at the level of a national newspaper.  It may be used for
% \textsl{Nature}'s press release or other general publicity.
%  \item The cover letter should state clearly what is included as the
% submission, including number of figures, supporting manuscripts
% and any Supplementary Information (specifying number of items and
% format).
%  \item The cover letter should also state the number of
% words of text in the paper; the number of figures and parts of
% figures (for example, 4 figures, comprising 16 separate panels in
% total); a rough estimate of the desired final size of figures in
% terms of number of pages; and a full current postal address,
% telephone and fax numbers, and current e-mail address.
% \end{enumerate}

% See \textsl{Nature}'s website
% (\texttt{http://www.nature.com/nature/submit/gta/index.html}) for
% complete submission guidelines.

\maketitle

\begin{affiliations}
\item CNRS
\item CIRED
\item University of Amsterdam
\item TU Berlin
\end{affiliations}

\begin{abstract} % 250 words. TODO? mention more the other measures?
  The ``global climate scheme'' (a global carbon price funding a global basic income) would be an effective and progressive way to combat climate change, and poverty. Yet, such policy is mostly absent from political platforms and the policy debate. Using surveys on 40,000 respondents in 20 countries covering 72\% of global CO$_\text{2}$ emissions, we document majority support for this and other global policies. Using a complementary survey on 3,000 U.S. respondents,% and four European countries, 
  we test several hypotheses that could could reconcile strong stated support with a lack of salience of these issues. The complementary analyses show that the stated support is mostly sincere, although we cannot rule out insincerity for 3\% to 9\% of the population from the willingness to sign a real-stake petition and a list experiment, respectively. Global redistributive policies rank high (though not highest) in the prioritization of policies. Conjoint analyses reveal that the Democratic party would not significantly lose votes if it endorsed the global climate scheme, while a candidate at the Democratic primary would actually win votes by doing so. Accurate beliefs about the level of support for the scheme dismisses the hypothesis of pluralistic ignorance of the support. Strong universalistic attitudes are confirmed in more general questions, suggesting that the support cannot be explained away by malleable opinion or experimenter demand. Finally, we conclude that there is no compelling reason why global policies do not enter the public debate or political platforms, as they seem genuinely supported by a majority of the population.
\end{abstract}

% Intro
Ethical theories often warrent transfers from high- to low-income people, hence from high- to low-income countries. This is the case of utilitarianism, the benchmark ethical theory used in economics. Utilitarianism assigns the same weight to each person and thus considers that a dollar is better allocated to a low-income person, which has a higher marginal utility than a high-income person.\cite{mill_utilitarianism_1861} 

Addressing global poverty, inequalities and climate change are at the heart of the universally agreed Sustainable Development Goals (SDG). % 12 out of  17
It has been pointed out that low-income countries generally do not have enough domestic resources to eliminate the poverty gap in the short run.\cite{bolch_arithmetics_2022} In other words, it would hardly be possible to achieve the first SDG and end extreme poverty by 2030 without international transfers. 

Climate change is another issue that calls for a global response and international transfers. Postulating %Assuming
that each human has an equal right to emit CO$_\text{2}$, low emitters have a legitimate claim \textit{vis-à-vis} high emitters, that can be settled by monetary transfers. Coupling this burden-sharing principle to the carbon budget (remaining emissions that would be compatible with the Paris agreement) naturally defines a global climate policy. We call it the ``Global climate scheme'' and denote it \textit{G}; it consists of a global cap-and-trade system where emission rights are auctioned each year to polluting firms and the revenues finance a global basic income. Using the price and emissions trajectories from the Stern-Stiglitz report,\cite{stern_report_2017} we estimate that the basic income would amount to \$30 per month for each human above 15 in 2030, enough to lift out of extreme poverty the 700 million people who live with less than PPP \$2 per day. Conversely, high emitters like a typical American (with median U.S. CO$_\text{2}$ emissions) would lose in net \$85 per month, as they would face \$115 per month in price increases (assuming a carbon price of \$90/tCO$_\text{2}$ in 2030).
% G default policy for economists, we focus on it; transfers at heart of COP; global wealth tax proposed by Piketty, Saez (fair and effective); democratisation of int'l institutions recurring topic.
% Few studies on CC burden-sharing, all compatible with G
% Few studies on global policies, but they show support (Ghassim, Carattini 19)
% Here, two sets of results. First, twenty countries. Second, dig deeper using complementary survey.

If high emitters share universalistic ethical values, we expect strong support for G, even in high-income countries. On the contrary, if people defend their own financial interest, we expect low support for G in high-income countries. 

In this paper, we study attitudes toward global policies that address climate change, global poverty or inequalities, with a focus on G. We measure stated support for different global policies using unpublished results from a survey\cite{dechezlepretre_fighting_2022-1} on climate attitudes conducted in 2021 on 40,680 respondents from 20 countries covering 72\% of global CO$_\text{2}$ emissions. We then conduct a representative survey on 3,000 U.S. respondents to study in detail the sincerity and rationales behind the support for G, the attitudes toward various global policies, global redistribution, and universalistic values.

\section{Results}
% % 4 most important figures: heatmap OECD, heatmap support, prioritization or conjoint (r), list exp (table)
\subsection{Global support}
The global survey shows strong support for climate policies at the global level (Figure \ref{fig:oecd}). When asked ``At which level(s) do you think public policies to tackle climate change need to be put in place?'', 70\% (in the U.S.) to 94\% (in Japan) choose the global level. Meanwhile, the European level is chosen by less than half of the European respondents while the federal level is chosen by only 52\% of U.S. respondents. More local levels are generally chosen less than broader ones. This preference for the global level is consistent with the two of the three key motives identified to support climate policies:\cite{klenert_making_2018,douenne_yellow_2022,dechezlepretre_fighting_2022} effectiveness and fairness (the third being self-interest). 

\begin{figure}
  \caption{Share of support (somewhat or strongly) for the main global policies among non-\textit{indifferent} ($n$ = 40,680).} % TODO: add G to this heatmap, remove Dependence on what other countries do, change label titles to make it clear that the first one was multiple answers while the others were likert
  \makebox[\textwidth][c]{\includegraphics[width=\textwidth]{../figures/OECD/Heatplot_burden_share_all_share_countries.pdf}}\label{fig:oecd}
\end{figure}

Several global policies obtain an absolute majority %more than 70\% relative %
support in all countries: ``a tax on all millionaires in dollars around the world to finance low-income countries that comply with international standards regarding climate action [which] would finance infrastructure and public services such as access to drinking water, healthcare, and education'', ``a global democratic assembly whose role would be to draft international treaties against climate change [where] each adult across the world would have one vote to elect members of the assembly'' (though this one receives only 48\% of support in the U.S.), and an international emission trading scheme where ``countries that emit more than their national share would pay a fee to countries that emit less than their share''. 
In high-income countries, this global quota obtains 64\% of absolute (i.e. \textit{somewhat} or \textit{strong}) support and 84\% of relative support (i.e. excluding \textit{indifferent} answers). The support is even higher in middle-income countries, though one should interpret the results with caution in middle-income countries as their samples are only representative of the online population (young, graduated and urban people are over-represented). % TODO: not asked this way in FR, DK, US: comment
After the support for the global quota, we ask how the carbon budget should be divided among countries. 
The preferred burden-sharing rule is to allocate the rights to emit on an equal per capita basis: this fairness principle secures an absolute majority support in all countries, and a relative majority support never below 84\%. 
Taking into account historical responsibilities and vulnerability to climate damages is also popular, though less consensual, while grand-fathering (i.e. allocating emission shares in proportion to current emissions) comes last everywhere. 

\subsection{Stated support for various policies}
% H0: Majority support for each global policies except maximum wealth and debt cancellation

In the complementary U.S. survey, we test support for more realistic %other
global policies (Figure \ref{fig:support}). All receive relative majority support but two: ``a maximum wealth limit of \$10 billion'' and the ``cancellation of low-income countries' public debt''. Climate-related policies are particularly popular: ``high-income countries funding renewable energy in low-income countries'' obtains absolute majority support while loss and damages compensation (which was approved at the COP27) receives a relative support of  57\%. 

\begin{figure}
  \caption{Support for various global policies in the U.S. ($n$ = 3,000).}
  \makebox[\textwidth][c]{\includegraphics[width=\textwidth]{../figures/US1/support_likert.pdf}}\label{fig:support}
\end{figure}

% H0: Foreign aid: less than 20\% want a decrease (because nationalist), median wants increase at some conditions (no diversion, human rights) => GCS mostly addresses these points
After explaining that ``0.4\% of U.S. government spending (that is, 0.2\% of U.S. GDP) is spent on foreign aid to reduce poverty in low-income countries'', less than 20\% state that U.S. foreign aid should be reduced while 57\% state that it should be increased, including 14\% who support an inconditional increase. To the 43\% who answer that aid should be increased but only if some conditions are respected. The three conditions most chosen are all largely respected by the Global climate scheme: ``that we can be sure the aid reaches people in need and money is not diverted'' (chosen by 74\%),`` that recipient countries comply with climate targets and human rights'' (59\%), and ``that other high-income countries also increase their foreign aid'' (44\%). %, we propose different conditions. The most chosen condition (by 74\%) is  and the second most (by 59\%) ``That recipient countries comply with climate targets and human rights''. 
On the other side, not wishing to increase their country's foreign aid is mostly justified by prioritizing one's fellow citizens or viewing each country as responsible for its own fate. 

\subsection{Sincerity of support}

We use several methods to assess the sincerity of the support for the Global climate scheme (G): a list experiment, a real-stake petition, conjoint analyses, and the prioritization of policies. All methods suggest that the support is either completely sincere, or the share of insincere answers is limited. 

%\subsubsection{List experiment}
% H1: List experiment: There seems to be a 8pp social norm (differential of 3pp with NR). No effect of the number of options. TODO: check litterature
The tacit support for G measured through the list experiment is 46\%, i.e. 8 p.p. lower than at the direct question. This may be the sign of a social norm pushing some people to state that they support G although they secretly do not. Still, if there is a social norm in favor of G, there is a similar norm in favor of the National redistribution scheme, as the gap between the tacit and direct support for it is comparable (at 7 p.p.). %However, two observations qualify this interpretation. First, the gap between the tacit and direct support for the National redistribution scheme is comparable (at 7 p.p.) though we did not expect such a social norm in the case of the national redistribution, as the 95\% who would benefit from it should not feel ashamed to oppose a policy that would benefit them. Second, while we tested the questionnaire on random people in cafés, we noticed that some were confused by the question of the list experiment (asking how many policies from the list they supported), upset with the conservative societal policy (``Marriage only for opposite-sex couples in the U.S.'', ``Death penalty for major crimes'' in Europe), to the point that they did not answer attentively.

\begin{table}\label{tab:list_exp}
  \caption{Number of supported policies in the list experiment in function of the composition of the list. $G$ stands for the Global climate scheme and $R$ for the National redistribution scheme ($n$ = 3,000).} % Beware, this question is quite unusual. \\ Among the policies below, how many do you support?  \\ Coal exit, Marriage only for opposite-sex couples 

\begin{tabular}{@{\extracolsep{5pt}}lc} 
\\[-1.8ex]\hline 
\hline \\[-1.8ex] 
\\[-1.8ex] & Number of supported policies \\ 
\hline \\[-1.8ex] 
Mean & 1.354  \\ \hline \\[-1.8ex]
 List contains: G & 0.496$^{***}$ \\ 
  & (0.069) \\ 
  List contains: R & 0.574$^{***}$ \\ 
  & (0.068) \\ 
  List contains: G \times R & $-$0.033 \\ 
  & (0.119) \\ 
 \hline \\[-1.8ex] 

Observations & 1,045 \\ 
R$^{2}$ & 0.132 \\ 
\hline 
\hline \\[-1.8ex] 
\end{tabular} 
\end{table}

%\subsubsection{Petition}
% H1: Petition: Small effect against GCS: -4pp
When told that ``we will send the results to the U.S. President's office, informing him what share of American people are willing to endorse the Global climate scheme'', 4 p.p. fewer people are willing to sign a petition for G than to simply state their support. For the National redistribution scheme, the proportion of support is not significantly different in the petition and in the simple question. 

%\subsubsection{Conjoint analyses}
% H1, H2: Conjoint analysis: G|C+R 56%, G|R 59%, G 48% ~ C (|R), G+C|R 56%, C|R 64%, Left+G - Left = -3pp, A+G vs. B 59%
% => G is supported for itself, rather independently from R or C, with similar support to both, and it doesn't significantly penalize the Left, and would help a Democratic candidate
%\subsubsection{Prioritization}
% H1: Prioritization: G has mean only slightly lower than average, makes better than ban of cars and coal exit; global tax on millionaires does as well as wealth tax and almost as good as $15 minimum wage

\begin{figure}
  \caption{Prioritization of policies. Each respondent faces six policies taken at random from the ones below and allocates 100 points among them to signal the strength of their support for each one ($n$ = 3,000).} % Imagine you have 100 points that you can allocate to different policies. The more you give points to a policy, the more you support it.  \\  How do you allocate the points among the following policies?  
  
  \makebox[\textwidth][c]{\includegraphics[width=\textwidth]{../figures/US1/points_us.pdf}}\label{fig:points}
\end{figure}

\begin{figure}
  % Imagine that at the 2024 Democratic party presidential primaries, the two main candidates campaign with the following key policies in their platforms. \\ Which of these candidates do you prefer?

  \caption{Conjoint analysis (asked only to non-Republicans). Average Marginal Component Effects (relative to the baseline: an absence of policy of that category) of policies in the choice of candidate for a hypothetical duel in the 2024 Democratic primary, where both platforms are randomly drawn ($n$ = 2,000).}\label{fig:ca_r}
  \makebox[\textwidth][c]{\includegraphics[width=\textwidth]{../figures/US1/ca_r.png}}
\end{figure}

\subsection{Second-order beliefs}
% H3 belief: No pluralistic ignorance


\subsection{Universalistic values}
% H4: A strong majority is universalist/cosmopolitan (TODO: which word?), even a majority for non-Republican



\section{Discussion} % Summary, conclusion


\begin{methods}
%Put methods in here.  If you are going to subsection it, use \subsection commands.  Methods section should be less than 800 words and if it is less than 200 words, it can be incorporated into the main text.

% \subsection{Method subsection.}

% Here is a description of a specific method used.  Note that the subsection heading ends with a full stop (period) and that the command is \verb|\subsection{}| not \verb|\subsection{}|.

\end{methods}


\bibliographystyle{naturemag_noURL} % nature class works only with style naturemag or naturemag_noURL, and naturemag bugs if there are certain URLs (like .pdf). Also, nature class only works with \cite, not \citet or \citep.
\bibliography{global_tax_attitudes}


%% Here is the endmatter stuff: Supplementary Info, etc.
%% Use \item's to separate, default label is "Acknowledgements"
\begin{addendum} % 177 words
 \item We are grateful for financial support from the University of Amsterdam and TU Berlin. We are grateful for financial support from the OECD, the French Ministry of Foreign Affairs, the French Conseil d’Analyse Economique and the Spanish Ministry for the Ecological Transition and Demographic Challenge. We also acknowledge support from the Grantham Foundation for the Protection of the Environment and the Economic and Social Research Council through the Centre for Climate Change Economics and Policy. We thank Antoine Dechezleprêtre, Tobias Kruse, Bluebery Planterose, Ana Sanchez Chico, and Stefanie Stantcheva for their invaluable inputs for the project. We thank Auriane Meilland for feedback. We thank Laura Schepp, Martín Fernández-Sánchez, Samuel Gervais, Samuel Haddad, and Guadalupe Manzo for assistance in the translation. 
 \item[Registration] The project %is approved by IRB at Harvard University (IRB21-0137), and 
 was preregistered in the Ooen Science Foundation registry (osf.io/fy6gd).
 \item[Competing Interests] The authors declare that they have no
competing interests.
\item[JEL codes] P48, Q58, H23, Q54.
\item[Keywords] Climate change, global policies, cap-and-trade, perceptions, survey, inequality, wealth tax.
 \item[Correspondence] Correspondence and requests for materials
should be addressed to Adrien Fabre~(email: fabre.adri1@gmail.com).
\end{addendum}

%%
%% TABLES
%%
%% If there are any tables, put them here.
%%

% \begin{table}
% \centering
% \caption{This is a table with scientific results.}
% \medskip
% \begin{tabular}{ccccc}
% \hline
% 1 & 2 & 3 & 4 & 5\\
% \hline
% aaa & bbb & ccc & ddd & eee\\
% aaaa & bbbb & cccc & dddd & eeee\\
% aaaaa & bbbbb & ccccc & ddddd & eeeee\\
% aaaaaa & bbbbbb & cccccc & dddddd & eeeeee\\
% 1.000 & 2.000 & 3.000 & 4.000 & 5.000\\
% \hline
% \end{tabular}
% \end{table}

\end{document}
