\clearpage
\section{Literature review}\label{sec:literature}

\subsection{Attitudes and perceptions}\label{subsec:literature_attitudes}

\subsubsection{Population attitudes on global policies}\label{subsubsec:literature_attitudes_policies}
% Our surveys fill gaps in the knowledge of attitudes toward global policies. 
% We are not aware of any other survey on a global wealth tax. 
\citet{carattini_how_2019} test the support for different variants of a global carbon tax, but their samples are representative only along gender and age, and as respondents face only one variant, the sample size for a given variant is about 167 respondents per country. They find more than 80\% of support for any variant in India, between 50 and 65\% in Australia, the UK and South Africa, and 43 to 59\% of support in the U.S., depending on the variant. The support for a global carbon tax funding an equal dividend for each human is close to 50\% in high-income countries (e.g. at 44\% in the U.S.), consistently with what we find in the OECD survey (see Figure \ref{fig:oecd}). 
Using a conjoint analysis in the U.S. and Germany, \citet{beiser-mcgrath_could_2019} find that the support for a carbon tax increases by up to 50\% % e.g. in their Fig. 4 the DE support for $70/t jumps from 26 to 39% with extension to all industrialized countries
if it applies to all industralized countries rather than just one's own country. % Variant of carbon tax is 8 (US) - 17 (DE) p.p. more likely to be preferred if tax is extended to all industrialized countries

In surveys in Brazil, Germany, Japan, the UK and the U.S., \citet{ghassim_who_2020} finds 55 to 74\% of support for ``a global democracy including both a global government and a global parliament, directly elected by the world population, to recommend and implement policies on global issues''. % (for example, international peace, world poverty, and climate change)''
Using an experiment, he also finds that, in countries where the government stems from a coalition, voting shares would shift by 8 (Brazil) to 12 p.p. (Germany) from parties who are said to oppose global democracy to parties that supposedly support it. For example, the Greens and the Left gained respectively 9 and 3 p.p. in vote intentions while the SPD and the CDU-CSU each lost 6 p.p., when Germans respondents were told that (only) the former parties support global democracy. 
\citet{ghassim_who_2020} also document survey results which show strong majorities support in each of 18 countries for the direct election of one's country's UN representative. % GlobeScan 2005; also: half/half (majorities or not depend on the country) for “Global Parliament, where votes are based on country population sizes, and the global parliament is able to make binding policies” (Synovate 2007); also: (GlobeScan 22, not from Ghassim) in 31 countries: 77% agree that “Rich countries must pay for poorer countries do deal with the effects of CC”
Similarly, in each of 10 countries, there are clear majorities in favor of ``a new supranational entity [taking] enforceable global decisions in order to solve global risks'' \citep{global_challenges_foundation_attitudes_2018}. Actually, already in 1946, 54\% of Americans agreed (and 24\% disagreed) that ``the UN should be strengthened to make it a world government with the power to control the armed forces of all nations'' \citep{gallup_seventy_1946}. 
In surveys in Argentina, China, India, Russia, Spain, and the U.S., \citet{ghassim_public_2022} find support for UN reform that would make United Nations' decisions binding, give veto powers at the Security Council to a few other major countries, and complement the highest body of the UN with a chamber of directly elected representatives. 

Relatedly, \citet{meilland_international_2023} find that Americans and French people prefer an international settlement of climate justice even if it empedes on sovereignty. In a 2013 survey in China, Germany and the U.S., \citet{schleich_citizens_2016} show that more than three quarter of people think that international reached so far are not successful and that future agreements are important. % 73\% of people find important future international climate agreements, while less than 26\% think that international reached so far are successful. 

These specific questions are in line with the answers to more general questions. In each of 36 countries, \citet{issp_research_group_international_2010} find near consensus that ``for environmental problems, there should be international agreements that [their country] and other countries should be made to follow'' (overall, 82\% agree and 4\% disagree). % No question like this in the next Envi wave in 2022
In each of 29 countries, \citet{issp_international_2019} find near consensus that ``resent economic differences between rich and poor countries are too large'' (overall, 78\% agree and 5\% disagree). 
%* Also in ISSP (19): slight minorities (in rich countries) that “People in wealthy countries should make an additional tax contribution to help people in poor countries.” p. 104, but strong majorities everywhere that “People from poor countries should be allowed to work in wealthy countries.” p. 106

%* ISSP (19): Near consensus that “Present economic differences between rich and poor countries are too large.” p. 102, slight minorities (in rich countries) that “People in wealthy countries should make an additional tax contribution to help people in poor countries.” p. 104, but strong majorities everywhere that “People from poor countries should be allowed to work in wealthy countries.” p. 106
%* Ghassim et al. (22): support for stronger UN with more direct elections.
%* Ghassim (20):  in Germany those two parties that supposedly endorse global democracy – the Greens and the Left – benefitted, gaining nine and three percentage points respectively in terms of voting intentions. Meanwhile, the traditional centrist parties – SPD and CDU – each lost six percentage points due to their supposed opposition to global democracy.
%* Beiser-McGrath & Bernauer (19): Conjoint analysis in US, DE. Variant of carbon tax is 8 (US) - 17 (DE) p.p. more likely to be preferred and 50% more likely to be supported if tax is extended to all industrialized countries (Fig 1, 4). (Unfortunately, don't test extension to global level).
%- Çarkoğlu.. (15) International Social Survey Program 2010 data reveal that people in LDCs are less supportive of international agreements forcing their country to take necessary environmental measures than are citizens in the developed world [80% instead of 85%]. (‘for environmental problems, there should be international agreements that [their country] and other countries should be made to follow.’)
%* Carattini et al. (Nature, 19): 1k in US, IA, ZA, AU, UK. Each respondent receives one variant at random of global carbon price of 40/60/80 $/t redistributed as international dividend / national dividend / mitigation in all countries / mitigation in developing countries / domestic mitigation / reduced labour tax. Immense majorities for any scheme in India, small majorities for each elsewhere except US international dividend (44%) or mitigation in developing (43%), and AU mitigation in developing (49,6%). PB: very low sample size (~167) for a given redistribution, even lower (~55) for a given variant (that also specifies the price). Appendix also contains estimation of distributive impacts. Representative only along the two quotas: gender and age. Don't give the representativeness in terms of income (the third socio-demos that they ask) so it's probably bad.

\subsubsection{Population attitudes on climate burden sharing}\label{subsubsec:literature_attitudes_burden_sharing}

Despite their differences in the description of the fairness principles, the surveys on burden-sharing rules show consistent attitudes. Or at least, their various results can be made compatible with the following interpretation. 
Concerning emissions reductions, most people want that every country engage in strong decarbonization effort together, with a global quota converging to climate neutrality in the medium run. Concerning the financial effort, most people support high-emitting countries paying and low-income countries receive funding. The most supported rules are those that appear equitable, in particular an equal right to emit per person. 
% When the rankings between rules differ, it can be due to the difference in countries surveyed, but it is most often due to differences in definitions and wording. 

This interpretation helps understanding the apparent differences between articles, which approach burden sharing from different angles: cost sharing (i.e. in money terms), effort sharing (in terms of emissions reductions), or resource sharing (in terms of rights to emit). Extant papers adopt the cost sharing or effort sharing approaches and preclude any country being a net receiver of money. Also, by focusing on either the financial or the decarbonization effort, these surveys miss the other half of the picture, which can explain why some papers find strong support for the ability-to-pay principle while others find strong support for grandfathering (defined as emissions reductions being the same in every country). The literature follow these approaches to stick to the terms used by the UNFCCC. Yet, we argue that the resource sharing approach is preferable to uncover attitudes, as it unambiguously describes the distributive implications of each rule while achieving an efficient location of emissions reductions and explicitly allowing for monetary gains for some countries. % TODO? say more simply that the location of emissions reductions is flexible in resource sharing
% TODO? appendix with the definitions for each author, incl. us

Now, let us summarize the different papers' results in the light of this clarification. 
\citet{schleich_citizens_2016} find an identical ranking in the support for the burden-sharing principles in China, Germany, and the U.S.: polluter-pays followed by ability-to-pay, equal emissions per capita, and grandfathering. 
% \footnote{The survey of \citet{schleich_citizens_2016} defines these rules as follows: \\
% \textit{Polluter-pays}: ``Every country has to bear costs according to the emissions it causes (hence countries causing higher emissions have a higher share of the costs).'' \\
% \textit{Ability-to-pay}: ``Every country has to bear costs according to its economic strength (hence richer countries have a higher share of the costs).''
% \textit{Egalitarianism}: ``Every country is allowed to produce the same amount of emissions per capita (hence countries with currently high emissions per capita have higher costs).''
% \textit{Sovereignty} (i.e. grandfathering): ``Every country is allowed to produce the same share of global emissions as in the past (hence the proportional reduction of emissions is the same for every country).''} 
Note that the authors do not allow for emissions trading in their description of equal \textit{emissions per capita}, which may explain its relatively low support. 
Yet, the relative support for egalitarianism also depends on how \textit{the other} rules are described. Indeed, \citet{carlsson_is_2011} find that Swedes prefer that ``all countries are allowed to emit an equal amount per capita'' rather than options where emissions are reduced in relation to current or historical emissions for which it is explicitly written that high-emitting countries ``will continue to emit more than others''. 
\citet{bechtel_mass_2013} find agreement that rich countries should pay more and historical emissions matter, but that rich countries should not be the only one to make the efforts. More precisely, their conjoint analysis in France, Germany, the UK and the U.S. shows that a climate agreement is 15 p.p. more likely to be preferred  (to a random alternative) if it includes 160 countries rather than 20, and 5 p.p. less likely to be preferred if ``only rich countries pay'' comapred other burden-sharing rules: ``rich countries pay more than poor'', ``countries pay proportional to current emissions'' or ``countries pay proportional to historical emissions''. %=> confirms preference for global policies (rather than only partial coverage). Finds that costs is what matters most: preference decreases by 30pp if it’s 2.5\% of GDP compared to 0.5\%.
Using a choice experiment, \citet{carlsson_fair_2013} find that the least preferred option in China and the U.S. is when low-emitting countries are exempted from any effort. Ability-to-pay is appreciated in both countries, though the preferred option in China is another one, which accounts for historical responsibility. %that Americans prefer capacity to pay > current responsibility > historical responsibility > equal emissions per capita while Chinese prefer historical > capacity > current > equal emissions.
%   Capacity to pay: Countries with high income levels must pay a larger share of the costs than countries with low income levels. This option says that countries with greater ability to pay should pay more
%   Current responsibility: Countries with currently high emissions levels must pay a larger share of the costs than countries with currently low emissions levels. This option says that those countries that are currently a larger part of the problem should pay more.
%   Historical responsibility: Countries with a history of high emissions levels must pay a larger share of the costs than countries with a history of lower emissions. This option recognizes that CO2 builds up in the atmosphere over many years. Thus, countries with a history of high emissions should pay more because they caused more of the problem.
%   Equal emissions pc: Countries with emissions per person greater than an agreed amount must pay, and they must pay more the higher their emissions per person are.
% > "equal emissions" is a misnomer as this is about costs (not emissions) and it's just a more progressive version of current responsibility / polluter-pay, where high-emitting pay more and low-emitting don't pay. The result for US is compatible with the other papers as Americans agree that rich countries (or high-emitting, the diff is small) should pay more. The Chinese position could also be reconciliable once we define responsibility from footprint rather than territorial and that there will be transfers from rich to poor countries.
In the U.S. and France, \citet{meilland_international_2023} find that the most favored fairness principle is that ``all countries commit to converge to the same average of total emissions per inhabitant, compatible with a controlled climate change''. Furthermore, in each country, 73\% disagree with grandfathering defined as ``countries which emitted a lot of carbon in the past have a right to continue emitting more than others in the future''. \citet{meilland_international_2023} contain many other results, for example majorities prefers to hold countries accountable for their consumption-based rather than territorial emissions, and the median choice regarding historical responsibility is to hold a country accountable for their post-1990 emissions (rather than post-1850 or just their current emissions). 
% - Meilland et al. (23) find that in US and France, most favored fairness principle is Equality in per capita emissions: "all countries commit to converge to the same average of total emissions per inhabitant, compatible with a controlled climate change" and second-most (which closely follows) is grandfathering: "all countries commit to reduce their emissions by a same proportion". 73% in each disagree with grandfathering when defined as "countries which emitted a lot of carbon in the past have a right to continue emitting more than others in the future". To rationalize these contrasted views with grandfathering, we can interpret them as: equal rights, equal emission reductions, and transfers. 
%   convergence per capita (70%): all countries commit to converge to the same average of total emissions per inhabitant, compatible with a controlled climate change
%   grandfathering (60%): all countries commit to reduce their emissions by a same proportion
%   past emissions (20% choose it among their two favorite): countries which emitted less in the past commit to reduce their emissions less than other countries
%   poor countries (20%): poorer countries commit to reduce their emissions less than richer countries
%   cost-efficiency (20%): countries where reducing emissions is more costly commit to reduce their emissions less than other countries
% - Meilland et al. (23) Other findings: people prefer international settlement on CC even if it empedes on sovereignty, a majority prefers to target footprint rather than territorial emissions, median is that countries should be held accountable for post-1990 emissions, self-serving bias when judging e.g. India vs. EU, no shared understanding of fairness when asked to coordinate between French and Americans
Finally, in each of 28 (among the largest) countries, \citet{dabla-norris_public_2023} find strong majority for ``all countries'' to the question ``Which countries do you think should be paying to reduce carbon emissions?''. Asked to choose between a cost sharing based on \textit{current} vs. \textit{accumulated historic emissions}, a majority prefers \textit{current emissions} in all countries but China and Saudi Arabia (where the two options are close to equally preferred). 

%- Gampfer (14): lab experiment (ultimatum game) to test whether preferences respect fairness principles

\subsubsection{Population attitudes on foreign aid}\label{subsubsec:literature_foreign_aid}

There is an extensive literature on attitudes toward foreign aid in donor countries. Its main insights are that most people overestimate the amount of foreign aid and that only a minority wants a cut in foreign aid compared to actual amounts, especially once they know them. 

\citet{pipa_americans_2001} shows that 83\% of Americans support a multilateral effort to cut world hunger in half. 
\citet{pipa_publics_2008} shows that in each of 20 countries, a majority thinks that developed countries ``have a moral responsibility to work to reduce hunger and severe poverty in poor countries'', with an average agreement of 81\%. In 7 OECD countries, they find that at least 75\% are willing to pay for a program to cut hunger in half (at an estimated cost of e.g. \$50 a year for each American). 

\citet{kaufmann_foreign_2012} find that in each of 24 countries, perceived aid is overestimated, on average by a facto 7. In most countries, desired aid is larger than perceived aid.\footnote{\citet{kaufmann_foreign_2012} offer the best results on desired aid because (as \citet{hudson_mile_2012} criticize), other studies did not take into acount misperceptions of actual aid.} They show that those in the top income quintile desire aid 0.13 p.p. lower than those in the bottom 40\% -- which is very close to what we find. Then, using a theoretical model as well as correlations between the level of lobbying and the actual aid (controling for desired aid), they argue that the gap between actual and desired and is due to political influence of the rich who defend their vested interests. 
In \citet{kaufmann_foreign_2012}, the U.S. is an outlier: desired aid is at the other countries' average (3\% of GNI), but as misperceptions are enormous, perceived aid is twice as large as desired aid. Indeed, \citet{gilens_political_2001} shows that even American with high political knowledge misperceive actual aid, and finds that 17\% fewer of them want to cut aid when we provide them specific information about aid amount. % same for Hurst et al
Similarly, \citet{nair_misperceptions_2018} finds that the relatively low support for aid in the U.S. is driven by information on global distribution, as people underestimate their rank by 27 centiles on average and overestimate the global median income by a factor 10. 

\citet{hudson_mile_2012} offer a critical review of the literature and show that the strong support for poverty alleviation largely stems from intrinsic altruism. Indeed, citing \citet{dfid_aid_2009} and \citet{pipa_americans_2001}, they note that 47\% of British people find that the aid is wasted due mainly to corruption, while Americans estimate that less than a quarter of the aid reaches people who really need it and more than half ends up in the hands of corrupt government officials. And yet, most people still support aid, suggesting that they have nonutilitarian motives for doing so. Consistent with \citet{henson_public_2010}, \citet{bauhr_does_2013} find that support for aid is reduced by perception of corruption in recipient countries. However, this effect is reduced by the aid-corruption paradox: most corrupt countries need more help. \citet{bodenstein_who_2017} further show that right-wing Europeans or those who perceive strong corruption in their country are more likely to agree that recipient countries should ``follow certain rules regarding democracy, human rights and governance as a condition for receiving EU development aid.'' 
Using a 2002 Gallup survey as well as the 2006 World Values Survey, and consistently with \citet{bayram_aiding_2017}, \citet{paxton_individual_2012} show that the main determinants for wanting more aid are trust, ideology, interest in politics, and being a woman (all positively associated). %Likewise, \citet{nair_preferences_2016} shows that preferences for international redistribution in the U.S. are netter explained by worldviews rather than socio-demographic variables. 
% heinrich_voters_2018 also show that the country's interest also play a role in aid support (as support increases when the donation can be in the country's interest) 

% Reviews, determinants
%*? Milner & Tingley (13): (highly cited but no original data, don't think we need to cite it) In 2008, 44% of American wanted foreign aid cut (american elections study, 08). fraction of federal budget going to foreign aid (mean: 27%, median: 25%) / should go (mean: 13%, median: 10%) (WorldPublicOpinion, 10)
% PIPA (01): when PIPA asked respondents to estimate how much of the federal budget was devoted to foreign aid, the median estimate was 15% -- 15 times the actual amount, which was just under 1%. More dramatically, when asked what an appropriate percentage would be, the median response was 5% -- 5 times the actual amount. And when asked to imagine that they heard the real amount was only 1%, only 18% of respondents said they thought that would be too much--as compared to the 75% who had initially said that the US was spending too much. 
%- DFID (10): Priorities: 1 NHS, 2 education, 3 support to poor countries, 4 police, 5 defence (p. 19). Show majority support for increased aid until 07, then median is to support stable aid (due to crisis?). It seems they don't give the info on actual amount though.
%* Hudson & van Heerde (12):Reviews literature on foreign aid and criticizes it on a number of points (e.g. not uncovering the determinants, and not asking well the questions). Shows strong support for poverty alleviation, (at least partly) out of intrinsic altruism. Use 4 main sources: PIPA (01, 08) UK DIDP, Eurobarometer; cf. Table 1 for all surveys on foreign aid / Public support for development has been famously described as a mile wide and and inch deep (Smillie, 1996: ref impossible to find). Hard times at home have meant that public support appears to have turned against international development efforts (Henson and Lindstrom, 2010). / Monitor public support: (Fransman and Solignac Lacomte, 2004; McDonnell et al, 2003), Paxton and Knack, 2008; Chong & Gradstein 2006. Review surveys on aid. / ~75% support aid in developed countries (stable) but ‘84 per cent agreed with the assertion that ‘taking care of problems at home is more important than giving aid to foreign countries’ (PIPA, 2001:9).” / References on covariates of aid support / PIPA 2001, "On average, Americans thought just under 25 per cent of the US budget was allocated to foreign aid, and government should allocate less than 14 per cent of the national budget. However, when told that US spends approximately 1 per cent of the federal budget on foreign aid, 37 per cent of respondents thought this was too little, 44 per cent thought it was about right, and 13 per cent thought it too much."  Think that only 23% of aid really goes to the poor / “The 2009 UK survey, Public Attitudes towards Development, reports ‘public support for overseas aid’ at 72 per cent (DFID, 2009); while in the US support was a comparable 79 per cent (PIPA, 2001); and average support across the EU trends slightly higher than in the US and UK with 91 per cent saying it was either very (53%) or fairly (38%) important to provide aid to poor countries (Eurobarometer, 2005).” / “DFID has now begun asking questions that provide relative measures of the salience of development aid vis-à-vis other competing policy issues (DFID, 2009; IDC, 2009). / "high proportion (61%) of US citizens who felt that the US spends too much on foreign aid. [from another source]” / “The distinction between foreign aid, which includes military spending, and development aid/assistance is an important one” / “81 per cent of respondents believed that developed countries do have a moral responsibility to work towards reducing hunger and severe poverty (WorldPublicOpinion.org, 2008). (…)  %/ “support for development assistance is highly contingent on respondents’ perceptions of the effectiveness of aid, especially with regard to corruption (Henson et al, 2010). For example, 
%(…) international charities and NGOs are deemed best suited/most effective compared to donor countries” / UK ‘MyAid’ plan – where the public gets to vote on how a pot of money should be distributed – / "public engagement should be about ‘opening up the political and wider societal space to the possibility of deeper change’ (Darnton and Kirk, 2011:14).”
%- Chong & Gradstein (16): from WVS 95-99, 58% want that their country give more foreign aid (but misperceptions are not taken into account)
%- Williamson (19): Public Ignorance or Elitist Jargon? Reconsidering Americans’ Overestimates of Government Waste and Foreign Aid. "Foreign aid" encompasses military spending, in the mind of American.
%- McDonnell et al (03) Public Opinion and the Fight against Poverty
%- Nair (16): preferences driven by worldviews rather than self-interest
%- Scotto et al (17): We Spend How Much? Misperceptions, Innumeracy, and Support for the Foreign Aid in the United States and Great Britain. Less American and British want aid cut when information on current aid is given in % of GDP rather than in $.
%* Paxton & Knack (12): Majorities want more aid, and main determinants are trust, ideology, interest in politics, and female (all positive). Gallup 02: in US 45% want more aid (rather than stable) vs. 68-91 in DE-UK-ES. Like Chong & Gradstein, find that desired aid increases with income, contrary to Kaufmann et al. but the latter contains more datasets.
%- Wood (15): Determinants for aid support in Australia. Wood (18) Examine Australian support for aid: although there is support to help foreign poor, people back recent aid cuts.
%- Cheng & Smyth (16): Why Give it Away When You Need it Yourself? Understanding Public Support for Foreign Aid in China. Political ideology and patriotism main explaining variables for aid support. People in poorer provinces less supportive.
%- Milner & Tingley (10) theory + empirics: who supports aid and why. owners of capital in donor countries tend to gain from aid and thus are more likely to support giving aid
%- Easterly (JEP, 03) Can Foreign Aid Buy Growth? No (disproves Hansen & Tarp).
%- Hansen & Tarp (01) Aid increases growth (empirical evidence)
%- Tresch et al. (22): 66% of Swiss people want to increase their foreign aid; also Borofsky
%- Harris (17): majority of French want to decrease foreign aid when face with a trade-off with other public spending

\subsubsection{Population attitudes on rich tax}\label{subsubsec:literature_wealth_tax}

We are not aware of any previous survey on a global wealth tax,\footnote{We did not find any using the combination of ``survey'' or ``attitudes'' with ``wealth tax'' or ``global wealth tax'' in Google Scholar.} though surveys consistently show strong level of support from national wealth taxes. 
By asking how much taxes per year should a person with a certain income and wealth level pay, \citet{fisman_americans_2017} finds that the average Americans favors a 0.8\% linear tax rate on unspecified wealth until \$2 million, and a 3\% linear rate on inherited wealth. 
In 21 OECD countries, \citet{oecd_main_2019} find strong majority support for higher taxes on the rich to support the poor (with nearly 70\% overall agreement and less than 20\% disagreement), while \citet{isbell_footing_2022} finds similarly high level of support in 34 African countries. 
In the UK, \citet{patriotic_millionaires_patriotic_2022} find 69\% support (and 7\% opposition) for a 1.1\% tax on wealth in excess of £10 million. 
In the U.S., \citet{americans_for_tax_fairness_support_2021} find 67 to 71\% support to to ``raise taxes for those earning more than \$400,000 a year'', ``raise the income tax rate for those earning over \$1 million a year by 10 percentage points'', or ``apply a 2\% tax on an individual's wealth above \$50 million each year, and 3\% on wealth above \$1 billion''.
% TODO! ISSP
%- Gallup (22), US
%- Fight Inequality Alliance India (22), IA

% PIPA (01): what percentage of their "tax dollars that go to help poor people at home and abroad...should go to help poor people in other countries." The mean response was 16% (down a bit from 22% in response to this question in a 1996 PIPA poll). Strikingly, this turns out to be a far higher percentage than is currently given. In 1999, a bit less than 4% of the total spent on the poor went to the poor abroad. Sixty percent of respondents proposed a percentage that was higher than 4%.

\subsubsection{Population attitudes on ethical norms}\label{subsubsec:literature_wealth_tax}
\paragraph{Universalism}
% TODO! WVS on world citizenship (e.g. Bayram 15), Reysen and Katzarska-Miller 2018
%- Buntaine & Prather (18), Diedrich & Goeschl (18) Willingness to act for domestic vs. international climate action (lab experiment) READ

\citet{enke_moral_2023-1} measure universalism, by asking American respondents to split \$100 between a random stranger and a random person closer to them with the same income. They distinguish different facets of universalism, and define \textit{foreign universalism} as giving to a foreigner rather than a fellow citizen. They find a home bias for most people, which may partly be due to concerns for inequality, as the split involves two persons with the same income, with the foreigner most certainly living in a poorer country than the American and thus enjoying a higher social status. 
That being said, a home bias probably remains once removing the concern for inequality, as 84\% of Americans agree that ``taking care of problems at home is more important than giving aid to foreign countries'' \citep{pipa_americans_2001}. 
\citet{enke_moral_2023} measure universalism and analyze its correlates in 7 countries, and \citet{cappelen_universalism_2022} deploy this method in 60 countries. 
In a lab experiment with students in the U.S., \citet{cherry_accepting_2017} show that a substantial share of people prefer policies detrimental to them due to their egalitarian worldview. 
% Evidence that people are altruistic: they experience higher temperature when they learn they are doing good (Taufik et al. 15), they are sensitive to self-transcending more than self-interested reasons (Evans et al. 13), 

\paragraph{Free-riding}

Although researchers have long explained the lack of climate action by free-riding, surveys consistently show that people support climate mitigation in their country even if other countries defect. \citet{bernauer_how_2015} show this for Americans and Indians, who both overestimate their country's emissions at one third of global total. \citet{beiser-mcgrath_commitment_2019} show this in the U.S. and China using an experimental design. \citet{mcevoy_prospects_2016} show that Americans mostly invoke leadership and morality to justify unliteral climate action. Using a range of methods, \citet{aklin_prisoners_2020} show that the empirical evidence for free-riding is not compelling, and that climate inaction can be equally well explained by distributive conflicts. Finally, through a review of the literature, \citet{mcgrath_how_2017} show that climate attitudes are largely nonreciprocal, and primarily driven by values and perceptions of the policies, rather than by considerations of what other countries do.

\subsubsection{Second-order beliefs}\label{subsubsec:literature_beliefs}

\citet{allport_social_1924} introduced the concept of pluralistic ignorance: a shared misperception concerning others' beliefs. The concept became notorious when \citet{ogorman_pluralistic_1975} showed that, towards the end of the civil rights movement, 47\% of Americans believed that most white people favored segregation while only 18\% did so. %\citet{miller_pluralistic_1987} shows that pluralistic ignorance emerges because individuals believe that fear of embarrassment is a sufficient cause for their own behavior but not for the behavior of others. 
\citet{pipa_americans_2001} has shown that 75\% of Americans are willing to pay \$50 a year to cut world hunger in half (the cost of the program), but only 32\% think that the majority would be willing to pay.
\citet{andre_fighting_2021} have documented pluralistic ignorance of climate-friendly norms in the U.S. Similarly, \citet{sparkman_americans_2022} show that Americans underestimate the support for climate policies by nearly half, while \citet{drews_biased_2022} document pluralistic ignorance of carbon tax support in Spain. 
\citet{geiger_climate_2016} show that pluralistic ignorance concern about climate change leads people to talk less about it as they self-silence themselves. 
% TODO READ Mildenberg & Tingley (19) and improve summary
%- Bursztyn & Yang (21): Review of the field. Misperceptions about others are widespread, asymmetric, much larger when about out-group members, and positively associated with one’s own attitudes.
%* Andre et al. (21): Respondents vastly underestimate the prevalence of climate- friendly behaviors and norms among their fellow citizens. Providing respondents with correct information causally raises individual willingness to fight climate change as well as individual support for climate policies. The effects are strongest for individuals who are skeptical about the existence and threat of global warming.
%- Di Tella et al. (AER, 15): The results of the lab experiment favor the hypothesis that people avoid altruistic actions by distorting beliefs about others' altruism
%- Allport (1924): first book on pluralistic ignorance
%- Allport (40): function of poll is to correct pluralistic ignorance
%- Studies on pluralistic ignorance: business (Buckley et al. 00), against affirmative action (Van Boven 00), political correctness (Braghieri, AER 21), alcohol (Suls & Green, 03), white support for racial segregation (O'Gorman 75), CC (Geiger & Swim 16), hooking up (Lambert et al 03, cf. note for paragraph of pluralistic ignorance), women working outside home in Saudi Arabia (Bursztyn et al. 20)

% \subsubsection{Elite attitudes}\label{subsubsec:literature_beliefs} % TODO!

% \citet{mildenberger_beliefs_2019} survey elites (Congress staffers and international relations scholars) as well as the population in U.S. and China. They document pluralistic ignorance of pro-climate attitudes, egocentric bias, and increasing support after beliefs are updated. 
% \citet{lange_importance_2007} \citet{lange_self-interested_2010}
% \citet{dannenberg_equity_2010}
% \citet{kesternich_negotiating_2021} 
% \citet{hjerpe_common_2011}

% Elite surveys 
%* Mildenberg & Tingley (19): Congress staffers, cf. second-order beliefs
%- Hertel-Fernandez et al. (2019): Survey on US Congress staffers (not on climate)
%- Milner & Tingley (10) (not sure it's a survey) owners of capital in donor countries tend to gain from aid and thus are more likely to support giving aid
%- Lange et al. (Energy Econ, 2007): climate negotiators.  Mix of self-serving bias and support for egalitarian principle.
%- Lange et al. (EER, 2010): same data as Lange et al. (10)
%- Dannenberg et al. (ERE, 2010): elicit climate negotiators’ equity preferences using Fehr & Schmidt (99) method => regional differences in addressing climate change are driven more by national interests than by different equity concerns
%- Kesternich et al. (EEPS, 2020): survey on climate negotiators about their preferred burden-sharing rules: we observe tendencies for a more harmonized view among key groups towards the ability-to-pay rule in a setting of weighted burden sharing rules
%- Lange & Schwirplies (ERE, 2017): combines Lange et al. (10) and Schleich et al.
%* Hjerpe et al. (2011): Delegates at COP2009. The results indicate that voluntary contribution, indicated as willingness to contribute, was the least preferred principle among both negotiators and observers. Three of the four principles for allocating mitigation commitments were recognized widely across the major geographical regions: historic 1990, capacity to pay, and equal per capita emissions. The difference was never below 25 percentage units, and the opponent share never exceeded 16%.
%- Scholte et al. (2020)
%- Bayram (17): cosmopolitanism of German politicians and their respect of international law

\subsection{Proposals and analyses of global policy-making}\label{subsec:literature_policies}

\subsubsection{Global carbon pricing}\label{subsubsec:literature_pricing}

Economists generally consider global carbon pricing as the benchmark climate policy, as it would efficiently correct the carbon emissions externality. For example, \citet{hoel_carbon_1991} shows that an international carbon tax can be designed so that it is both efficient and satisfies whatever distributional objectives one might have. 
Concerning the distributional objective, \citet{grubb_greenhouse_1990}, \citet{agarwal_global_1991} and \citet{bertram_tradeable_1992} were the first advocates of an equal right to emit for each human. As \citet{grubb_greenhouse_1990} states it: ``by far the best combination of long term effectiveness, feasibility, equity, and simplicity, is obtained from a system based upon tradable permits for carbon emission which are allocated on an adult per capita basis''. The support for such solution has been renewed ever since \citep{baer_equity_2000,jamieson_climate_2001,blanchard_major_2021,rajan_global_2021}. 

While many endorse the egalitarian allocation of emissions permits, economists also considered this outcome as politically irrealistic. Thus, they tweaked their (integrated assessment) models by assigning more weight to rich countries' interests to preserve the current level of inequalities between countries, precluding any transfer between them \citep{stanton_negishi_2011}. 
\citet{gollier_negotiating_2015} synthesize the distributional decision with a \textit{generosity} parameter which would allocate emissions permit to countries in proportion to their population if set to one, in proportion to their emissions (on the start date of the policy) if set to zero, and as a mixture of the egalitarian rule and grandfathering if set in between. Using a similar formula in the context of a tax, \citet{cramton_international_2015} (summarized in \citealp{mackay_price_2015}) propose that countries around the average emission per capita fix the generosity parameter, so that it is strategically chosen to maximize the tax rate, and to fix the tax rate at the minimum price proposed by participating countries. Negotiations would exclude countries with low ambition beforehand; and the treaty would impose trade sanctions on non-participating countries. %  In \citet{cramton_global_2017}, prominent economists discuss how such negotiations can succeed, and whether a tax or tradable quotas has better chances. The tax is recommended in most chapters (except in the one of Gollier \& Tirole)
\citet{bergh_dual-track_2020} propose a ``dual-track transition to global carbon pricing'': an expanding climate club that would integrate existing and new emissions trading systems, and a reorientation of UNFCC negotiations towards a global carbon price, including burden-sharing rules. 
The \citet{imf_how_2019} also supports global carbon pricing or, as a first step, a carbon price floor. They propose either differentiated prices among countries, or international transfers, and estimate that a price of \$75/tCO$_\text{2}$ in 2030 would be compatible with a 2\textdegree{}C trajectory. %Similarly, \citet{parry_proposal_2021} acknowledges that transfers could be necessary to induce climate action in low- and middle-income countries, though they mention transferring only 1\% of carbon revenues. 

Other authors have advanced more radical ideas. \citet{weitzman_world_2017} envisions a World Climate Assembly with proportional representation at the global scale, so that the median (human) voter would choose the carbon price level. % and each country retaining the revenues TODO?
To finance an adaptation fund, \citet{chancel_carbon_2015} propose a global \textit{progressive} carbon tax (or a progressive tax on air tickets as a first step), so that rich people (who are high emitters) contribute more to the public good. 
\citet{fleurbaey_climate_2013} highlight that, given that current emitters are probably richer than future victims of climate change damages, climate policies deserve a \textit{negative} discount rate. Said differently, we cannot abstract the climate issue from global inequalities, and an ethical response requires global redistribution. 
%* Cramton et al (17): Livre de pontes. Tout le monde est d'accord : un prix mondial du carbone est requis, il ne peut être obtenu que par la réciprocité des engagements (style climate club), et il faut quelques transferts des riches vers les pauvres ainsi que des sanctions commerciales pour aligner les incitations. Ch 4 (also Cramton et al 15) propose la formule suivante de transfert (positif ou négatif) à un fonds climat : générosité*émissions en excès (par rapport à la cible)*prix du carbone. Le livre argumente bcp sur prix vs. quantité (TLM préfère prix sauf Gollier & Tirole), l'argument le plus convaincant en faveur du prix c'est qu'avec la procédure proposée le prix négocié serait le plus élevé possible, alors qu'avec la quantité c'est le budget carbone qui serait le point focal et ça aboutirait à une impasse (objectif trop ambitieux). % TODO: read Cramton notes + chapter Stiglitz
%* Weitzman (17) advocates for a World Climate Assembly, choosing the price level with the median voter, and each country retaining the revenues.
% Fleurbaey & Zuber (13): The discount rate converges to the worst-off (affected by the measure) to the worst-off (beneficiary of the measure) discount rate, which depends on the growth between both agents. Applied to real data, we can consider that the worst-off affected by a global tax on CO_2 is the average-earner on earth (around 75% centile i.e. ~1000€/month, cf. Chancel & Piketty, Lakner & Milanovic, Chakravorty) while the worst-off beneficiary is the worst-off person in the future (among those less affected by CC thanks to the measure), probably below 1000€/month => negative discount rate.
% Parry et al (21): Proposal for an International Carbon Price Floor Among Large Emitters. Acknowledges that transfers could be necessary to induce climate action in low/middle-income countries, talks about transferring 1% of carbon revenues.
%- Sager: distributive effects of global pricing without int'l transfers.
%- Budolfson et al. (incl. Fleurbaey, Méjean, Zuber, Dennig) (21): global carbon price with within-country per capita dividend. Acknowledge that "The overall benefits to society are even greater if total carbon tax revenues are returned on an equal per capita basis globally, which directs more of the revenues towards the poorest populations in the world (rather than the poorest within each country or region)." Very short (3p, no appendix, no suppl. info)
%- Chancel & Piketty (15): global progressive carbon tax
% Pottier et al (17): A survey of global climate justice 

\subsubsection{Climate burden sharing}\label{subsubsec:literature_burden_sharing}

% NDCs assessments or burden-sharing computations. TODO! check the contraction & convergence scheme proposed by France
%- Bourban (18): Soutient un marché du carbone avec droits en proportion des émissions cumulées depuis 1990. Et des “mesures volontaires de contrôle de la population mondiale”.
%- Raupach et al (NCC, 14) 
%- Grasso (2012)
%- van den Berg et al (20)
%- Meyer (04) Contraction and Convergence (i.e. grandfathering converging to equal pc, within an ETS)
%- >Baer et al (08)< (cite this one, others don't give more info), Baer (13), Athanasiou et al (22), Holz et al (19) https://calculator.climateequityreference.org/ Athanasiou, Greenhouse Development Rights, EcoEquity calculator, US fair share. Effort-sharing approach based on splitting emissions reductions in function of capacity to pay (~ share of global income in top 30%) and responsibility (share of emissions since 1950), weighted equally. Corresponds to UNFCCC wording. Pb of this method (applying to any choice of parameters): A country with relatively low incomes (e.g. equal distribution slightly above the p70) and that has few historical responsibility would have a relatively low effort. Even more problematic, the **poorest countries would have virtually 0% of the effort, hence they would be allowed to emit following the baseline trajectory… but this baseline is not fair; it amounts to grandfathering**. It is computed as the “product of the projected GDP and CO2 emission intensity”. ([https://climateequityreference.org/calculator-information/gdp-and-emissions-baselines/](https://climateequityreference.org/calculator-information/gdp-and-emissions-baselines/)), and give for example 0.8tCO2e/cap for RDC in 2030 (16% more than in 2020, but lot lower than the objective of ~4t). => Compared to an equal right to emit pc, this method favors countries like China (allowed to remain stable over 2020-30 vs. reduced by 35-40%) and penalizes countries like the U.S. and Africa. 
%  in Athanasiou et al (22) Justification of Greenhouse Development Rights instead of Equal per capita right is on p. 36. It is weak, and basically that historical responsibility should be taken into account. Conversely, justification against historical resp. is that the latter doesn’t take into account capacity to pay (it is not said like this, but we can think of ex-USSR).
%- Pachauri et al. (Science, 2022) 
%- Robiou du Pont et al. (NCC, 2016)
%- Robiou du Pont et al. (ERL, 2016)
%- Höhne et al. (Climate Policy, 2014): review of 40 papers
%- Gao et al. (FEM, 2019)
%- Gignac & Matthews (ERL, 15)
%- Matthews (16) Quantifying carbon debts among nations
%- https://climateequitymonitor.in/ computes carbon debt based on equal per capita cumulative emissions. contact@climateequitymonitor.in https://twitter.com/equity4climate

\subsubsection{Global redistribution}\label{subsubsec:literature_redistribution}

% Global poverty gap
%* Bolch et al. (22)
%- Zhang (16) estimates the poverty gap in each country. Global one is at $80G/year.

% Unequal exchange / embodided labour
%- Reyes et al (17)
%- Sakai et al (17)
%- Alsamawi et al. 2014

%* Beyond the welfare state: Myrdal 58
%* Hickel (17): The Divide: A Brief Guide to Global Inequality and its Solutions
%* Kopczuk et al (EER, 17) Compute optimal linear tax rate for all countries in two ways: decentralized or globally. Shows that the tax rate increases with inequality of skills (calibrated with the gini). The average decentralized rate is 0.41 The global one 0.62, with a global demogrant of 250$/month (higher than 73 countries' GDP). Show that within decentralized/country optimal taxation would not decrease global inequality by much (gini from 0.695 to 0.69, but down to 0.25 with global income tax). Show that USA don't give a damn of poor countries' people. citizens in the US (one of the richest) attach only 1/(2,000\*a) of the weight to the welfare of citizens in poorest countries, where a is the share  of transfer (supposedly) effectively arriving to the recipients. e.g. if half of aid is wasted by corrupt politicians, the weight is 1/1000.
% Carthy & Walsh (Oxfam, 22) propose various sources of funding for damages.
% Piketty (2014) "At what rate would [a global wealth tax] be levied? One might imagine a rate of 0 percent for net assets below 1 million euros, 1 percent between 1 and 5 million, and 2 percent above 5 million. Or one might prefer a much more steeply progressive tax on the largest fortunes (for example, a rate of 5 or 10 percent on assets above 1 billion euros). There might also be advantages to having a minimal rate on modest-to-average wealth (for example, 0.1 percent below 200,000 euros and 0.5 percent between 200,000 and 1 million)" He doesn't explicitly talk about revenue use, but implicitly they would be retained by each collecting country: "Le rôle principal de l'impôt sur le capital n'est pas de financer l'État social, mais de réguler le capitalisme.", "En principe, chaque pays de l'Union européenne pourrait obtenir des recettes du même ordre en appliquant seul un tel système."

% Basic income TODO find more
%* Egger et al. (19): positive gen eq effects. We provided one-time cash transfers of about USD 1000 to over 10,500 poor households across 653 randomized villages in rural Kenya. The implied fiscal shock was over 15 percent of local GDP. We find large impacts on consumption and assets for recipients. Importantly, we document large positive spillovers on non-recipient households and firms, and minimal price inflation.
%* Haushofer & Shapiro (16): The Short-term Impact of Unconditional Cash Transfers to the Poor: Experimental Evidence from Kenya. Monthly transfers are more likely than lump-sum transfers to improve food security

\subsubsection{Global democracy}\label{subsubsec:literature_democracy}
% TODO!

% Burden-sharing
%- Agarwal & Narain (91) first to defend an equal right to emit per capita (equal to the absorbing capacity of the Earth)
%- Gampfer (14): lab experiment (ultimatum game) to test whether preferences respect fairness principles
%- Chancel & Piketty (15): global progressive carbon tax
% cf. bottom

% Global policies attitudes 
%* ISSP (19): Near consensus that “Present economic differences between rich and poor countries are too large.” p. 102, slight minorities (in rich countries) that “People in wealthy countries should make an additional tax contribution to help people in poor countries.” p. 104, but strong majorities everywhere that “People from poor countries should be allowed to work in wealthy countries.” p. 106
%* Ghassim et al. (22): support for stronger UN with more direct elections.
%* Ghassim (20):  in Germany those two parties that supposedly endorse global democracy – the Greens and the Left – benefitted, gaining nine and three percentage points respectively in terms of voting intentions. Meanwhile, the traditional centrist parties – SPD and CDU – each lost six percentage points due to their supposed opposition to global democracy.
%* Beiser-McGrath & Bernauer (19): Conjoint analysis in US, DE. Variant of carbon tax is 8 (US) - 17 (DE) p.p. more likely to be preferred and 50% more likely to be supported if tax is extended to all industrialized countries (Fig 1, 4). (Unfortunately, don't test extension to global level).
%- Çarkoğlu.. (15) International Social Survey Program 2010 data reveal that people in LDCs are less supportive of international agreements forcing their country to take necessary environmental measures than are citizens in the developed world [80% instead of 85%]. (‘for environmental problems, there should be international agreements that [their country] and other countries should be made to follow.’)
%* Carattini et al. (Nature, 19): 1k in US, IA, ZA, AU, UK. Each respondent receives one variant at random of global carbon price of 40/60/80 $/t redistributed as international dividend / national dividend / mitigation in all countries / mitigation in developing countries / domestic mitigation / reduced labour tax. Immense majorities for any scheme in India, small majorities for each elsewhere except US international dividend (44%) or mitigation in developing (43%), and AU mitigation in developing (49,6%). PB: very low sample size (~167) for a given redistribution, even lower (~55) for a given variant (that also specifies the price). Appendix also contains estimation of distributive impacts. Representative only along the two quotas: gender and age. Don't give the representativeness in terms of income (the third socio-demos that they ask) so it's probably bad.


% Global policies
%* Beyond the welfare state: Myrdal 58
% Pottier et al (17): A survey of global climate justice 
%* Hickel (17): The Divide: A Brief Guide to Global Inequality and its Solutions
%* Kopczuk et al (EER, 17) Compute optimal linear tax rate for all countries in two ways: decentralized or globally. Shows that the tax rate increases with inequality of skills (calibrated with the gini). The average decentralized rate is 0.41 The global one 0.62, with a global demogrant of 250$/month (higher than 73 countries' GDP). Show that within decentralized/country optimal taxation would not decrease global inequality by much (gini from 0.695 to 0.69, but down to 0.25 with global income tax). Show that USA don't give a damn of poor countries' people. citizens in the US (one of the richest) attach only 1/(2,000\*a) of the weight to the welfare of citizens in poorest countries, where a is the share  of transfer (supposedly) effectively arriving to the recipients. e.g. if half of aid is wasted by corrupt politicians, the weight is 1/1000.
% Carthy & Walsh (Oxfam, 22) propose various sources of funding for damages.
% Piketty (2014) "At what rate would [a global wealth tax] be levied? One might imagine a rate of 0 percent for net assets below 1 million euros, 1 percent between 1 and 5 million, and 2 percent above 5 million. Or one might prefer a much more steeply progressive tax on the largest fortunes (for example, a rate of 5 or 10 percent on assets above 1 billion euros). There might also be advantages to having a minimal rate on modest-to-average wealth (for example, 0.1 percent below 200,000 euros and 0.5 percent between 200,000 and 1 million)" He doesn't explicitly talk about revenue use, but implicitly they would be retained by each collecting country: "Le rôle principal de l'impôt sur le capital n'est pas de financer l'État social, mais de réguler le capitalisme.", "En principe, chaque pays de l'Union européenne pourrait obtenir des recettes du même ordre en appliquant seul un tel système."

% Global carbon pricing TODO find current advocates of GCS
%* Grubb (90), Betram (92) advocate for global market with equal pc right
%* Bergh et al. (20) call for a "dual-track transition to global carbon pricing": an expanding climate club, and "a reorientation of UNFCCC negotiations creates room for talking seriously about a global carbon price schedule, including redistribution-of-revenues rules." They don't specify which equity rules to use.
%* Jamieson (01) advocates of equal pc burden-sharing (after the precursors Agarwal & Narain (91))
%* Bear et al (Science, 00), Bear (02), Athanasiou & Baer (02) advocate for equal pc burden-sharing (although weirdly, Bear & Athanasiou then change mind and advocate for the Greenhouse Development Rights, accounting for capacity and responsibility)
%* Cramton et al (17): Livre de pontes. Tout le monde est d'accord : un prix mondial du carbone est requis, il ne peut être obtenu que par la réciprocité des engagements (style climate club), et il faut quelques transferts des riches vers les pauvres ainsi que des sanctions commerciales pour aligner les incitations. Ch 4 (also Cramton et al 15) propose la formule suivante de transfert (positif ou négatif) à un fonds climat : générosité*émissions en excès (par rapport à la cible)*prix du carbone. On demanderait aux États autour de la moyenne d'émission de fixer ce paramètre de générosité, pour qu'il soit fixé de sorte à maximiser le prix, puis on fixerait le prix comme le prix minimum proposé (après avoir éjecté qqs pays récalcitrants des négos). Puis, sanctions commerciales pour ceux qui ne respectent pas le prix. Ch Gollier & Tirole proposent une formule aussi simple que l'autre : quota global*((1-g)*part des émissions à t=0 + g*part de la population), où g joue le même rôle de paramètre de générosité/éthique (que je voudrais mettre à 1, mais qu'ils disent tous de mettre < 1 pour que les pays riches acceptent. Le livre argumente bcp sur prix vs. quantité (TLM préfère prix sauf Gollier & Tirole), l'argument le plus convaincant en faveur du prix c'est qu'avec la procédure proposée le prix négocié serait le plus élevé possible, alors qu'avec la quantité c'est le budget carbone qui serait le point focal et ça aboutirait à une impasse (objectif trop ambitieux).
% Blanchard & Tirole
%* MacKay et al (Nature, 15) summarizes the above
%* Weitzman (17) advocates for a World Climate Assembly, choosing the price level with the median voter, and each country retaining the revenues.
% Fleurbaey & Zuber (13): The discount rate converges to the worst-off (affected by the measure) to the worst-off (beneficiary of the measure) discount rate, which depends on the growth between both agents. Applied to real data, we can consider that the worst-off affected by a global tax on CO_2 is the average-earner on earth (around 75% centile i.e. ~1000€/month, cf. Chancel & Piketty, Lakner & Milanovic, Chakravorty) while the worst-off beneficiary is the worst-off person in the future (among those less affected by CC thanks to the measure), probably below 1000€/month => negative discount rate.
% Stanton (11): Negishi weights obviate the IAMs’ equalization of income. 4 ways to solve this problem: 1. be more transparent, 2. stop weighting, 3. take linear utility (i.e. maximize global GDP), 4. stop optimizing. 
% Hoel (91): Shows that an international tax can be designed so that it is both efficient and satisfies whatever distributional objectives one might have.
% IMF (2019): global pricing (with either differentiated prices or international transfers) or, as a first step, a carbon price floor. 25% of revenues should be rebated to the bottom 40%, the rest used to reduce distortionary taxes or for green investments. Estimate that $75/t is needed in 2030 for 2°C.
% Parry et al (21): Proposal for an International Carbon Price Floor Among Large Emitters. Acknowledges that transfers could be necessary to induce climate action in low/middle-income countries, talks about transferring 1% of carbon revenues.
%- Sager: distributive effects of global pricing without int'l transfers.
%- Budolfson et al. (incl. Fleurbaey, Méjean, Zuber, Dennig) (21): global carbon price with within-country per capita dividend. Acknowledge that "The overall benefits to society are even greater if total carbon tax revenues are returned on an equal per capita basis globally, which directs more of the revenues towards the poorest populations in the world (rather than the poorest within each country or region)." Very short (3p, no appendix, no suppl. info)

% Foreign aid 
%* Kaufmann et al (12) Shows the level of perceived and desired aid in 26 countries between 2005 and 2008 (cf. Table 1). In most countries (incl. UK, DE, FR, ES but not U.S.) desired aid is larger than perceived. Argue that this is due to political influence efforts/possibilities of the rich, as they prefer less aid due to vested interests (support this by a theoretical model + correlations between level of lobbying and actual aid level, controling for desired aid). In most countries the gap between the two is small, except in the U.S. where perceived is 7.5% of GDP and preferred is 3%.Use WVS and Gallup (like Chong & Gradstein, Paxton & Knack) but have more waves and the others don't use the question on perceived aid. Shows that richer want less aid ("those in the top income quintile favour ODA (as a share of GNI) that is 0.13 percentage points lower than the preferred share for individuals in the bottom 40\% of the income distribution" after controling for perceived aid - our regression results are sensibly the same.). from 0 to higher than 25%: threshold at 0.05; 0.15; 0.35; 0.75; 1.5; 2.5; 4; 7.5; 17.5; 25, i.e. same number of thresholds but small than ours below 2.5 and higher above. 
%*? Milner & Tingley (13): (highly cited but no original data, don't think we need to cite it) In 2008, 44% of American wanted foreign aid cut (american elections study, 08). fraction of federal budget going to foreign aid (mean: 27%, median: 25%) / should go (mean: 13%, median: 10%) (WorldPublicOpinion, 10)
% PIPA (01): Overwhelming majorities support a multilateral effort to cut hunger in half by the year 2015 and say that they would be willing to pay for the costs of such a program. However, most do not think that the average American would be as willing to pay the necessary costs. when PIPA asked respondents to estimate how much of the federal budget was devoted to foreign aid, the median estimate was 15% -- 15 times the actual amount, which was just under 1%. More dramatically, when asked what an appropriate percentage would be, the median response was 5% -- 5 times the actual amount. And when asked to imagine that they heard the real amount was only 1%, only 18% of respondents said they thought that would be too much--as compared to the 75% who had initially said that the US was spending too much. what percentage of their "tax dollars that go to help poor people at home and abroad...should go to help poor people in other countries." The mean response was 16% (down a bit from 22% in response to this question in a 1996 PIPA poll). Strikingly, this turns out to be a far higher percentage than is currently given. In 1999, a bit less than 4% of the total spent on the poor went to the poor abroad. Sixty percent of respondents proposed a percentage that was higher than 4%.
%- DFID (10): Priorities: 1 NHS, 2 education, 3 support to poor countries, 4 police, 5 defence (p. 19). Show majority support for increased aid until 07, then median is to support stable aid (due to crisis?). It seems they don't give the info on actual amount though.
%* PIPA (08): Across 20 countries, 81% support that "developed countries have a moral responsibility to help reduce hunger ansevere poverty in poor countries (majority in every country). “the World Bank (Shantayanan et al, 2002) has estimated that it will require an extra US$39-54 billion per year to meet Millennium Development Goal 1 (MDG1). (…) The per person cost of meeting MDG1 came to £25 for the UK, $56 for the US, €27 for Germany, and so on. On average 77 per cent of respondents are in favour of contributing towards meeting the goal (provided that all others do too). To take the US example, 75 per cent of people supported paying an extra $56 per year to meet MDG1. What is significant about this figure is that it is only slightly below the support for the ‘cost free’ question as to whether the US should be willing to share a  small portion of its wealth with those who are in great need (79%).” Hudson & van Heerde (12)
%* Hudson & van Heerde (12):Reviews literature on foreign aid and criticizes it on a number of points (e.g. not uncovering the determinants, and not asking well the questions). Shows strong support for poverty alleviation, (at least partly) out of intrinsic altruism. Use 4 main sources: PIPA (01, 08) UK DIDP, Eurobarometer; cf. Table 1 for all surveys on foreign aid / Public support for development has been famously described as a mile wide and and inch deep (Smillie, 1996: ref impossible to find). Hard times at home have meant that public support appears to have turned against international development efforts (Henson and Lindstrom, 2010). / Monitor public support: (Fransman and Solignac Lacomte, 2004; McDonnell et al, 2003), Paxton and Knack, 2008; Chong & Gradstein 2006. Review surveys on aid. / ~75% support aid in developed countries (stable) but ‘84 per cent agreed with the assertion that ‘taking care of problems at home is more important than giving aid to foreign countries’ (PIPA, 2001:9).” / References on covariates of aid support / PIPA 2001, "On average, Americans thought just under 25 per cent of the US budget was allocated to foreign aid, and government should allocate less than 14 per cent of the national budget. However, when told that US spends approximately 1 per cent of the federal budget on foreign aid, 37 per cent of respondents thought this was too little, 44 per cent thought it was about right, and 13 per cent thought it too much."  Think that only 23% of aid really goes to the poor / “The 2009 UK survey, Public Attitudes towards Development, reports ‘public support for overseas aid’ at 72 per cent (DFID, 2009); while in the US support was a comparable 79 per cent (PIPA, 2001); and average support across the EU trends slightly higher than in the US and UK with 91 per cent saying it was either very (53%) or fairly (38%) important to provide aid to poor countries (Eurobarometer, 2005).” / “DFID has now begun asking questions that provide relative measures of the salience of development aid vis-à-vis other competing policy issues (DFID, 2009; IDC, 2009). / "high proportion (61%) of US citizens who felt that the US spends too much on foreign aid. [from another source]” / “The distinction between foreign aid, which includes military spending, and development aid/assistance is an important one” / “81 per cent of respondents believed that developed countries do have a moral responsibility to work towards reducing hunger and severe poverty (WorldPublicOpinion.org, 2008). (…) there are a good number of people who support aid despite the fact they do not think it works. What this suggests – but cannot show in any detail – is that people have nonutilitarian motives for supporting aid.” / “support for development assistance is highly contingent on respondents’ perceptions of the effectiveness of aid, especially with regard to corruption (Henson et al, 2010). For example, in the UK, 47 per cent of respondents thought that aid was wasted, with sizable majorities citing corruption and poor management and/or delivery as primary factors (DFID, 2008). More disconcertingly, US respondents thought that only 23 per cent of US aid money that goes to poor countries ends up helping the people who really need it and 54 per cent of US aid money that goes to poor countries ends up in the pockets of corrupt government officials (PIPA, 2001). (…) international charities and NGOs are deemed best suited/most effective compared to donor countries” / UK ‘MyAid’ plan – where the public gets to vote on how a pot of money should be distributed – / "public engagement should be about ‘opening up the political and wider societal space to the possibility of deeper change’ (Darnton and Kirk, 2011:14).”
%* Gilens (01) 17% fewer American with high political knowledge want to cut foreign aid when we provide them specific information about aid amount.
%- Chong & Gradstein (16): from WVS 95-99, 58% want that their country give more foreign aid (but misperceptions are not taken into account)
%* Bauhr et al (13): Support for aid is reduced by perception of corruption in recipient countries. However, this effect is reduced by the aid-corruption paradox (and other things): most corrupt countries need more help.
%- Nair (18): (lack of) Aid support in US driven by information on global distribution, because people underestimate their rank by 27 centiles and overestimate global median income by a factor 10.
%- Williamson (19): Public Ignorance or Elitist Jargon? Reconsidering Americans’ Overestimates of Government Waste and Foreign Aid. "Foreign aid" encompasses military spending, in the mind of American.
%- McDonnell et al (03) Public Opinion and the Fight against Poverty
%- Nair (16): preferences driven by worldviews rather than self-interest
%- Bodenstein & Faust (17): Determinants of support for aid conditionality. They are: perceived corruption in donor country, right-wing.
%- Scotto et al (17): We Spend How Much? Misperceptions, Innumeracy, and Support for the Foreign Aid in the United States and Great Britain. Less American and British want aid cut when information on current aid is given in % of GDP rather than in $.
%* Paxton & Knack (12): Majorities want more aid, and main determinants are trust, ideology, interest in politics, and female (all positive). Gallup 02: in US 45% want more aid (rather than stable) vs. 68-91 in DE-UK-ES. Like Chong & Gradstein, find that desired aid increases with income, contrary to Kaufmann et al. but the latter contains more datasets.
%- Wood (15): Determinants for aid support in Australia. Wood (18) Examine Australian support for aid: although there is support to help foreign poor, people back recent aid cuts.
%- Bayram (17): Aid support associated with trust, i.e. seeing integrity and trustworthiness in others.
%- Cheng & Smyth (16): Why Give it Away When You Need it Yourself? Understanding Public Support for Foreign Aid in China. Political ideology and patriotism main explaining variables for aid support. People in poorer provinces less supportive.
%- Milner & Tingley (10) theory + empirics: who supports aid and why. owners of capital in donor countries tend to gain from aid and thus are more likely to support giving aid
%- Easterly (JEP, 03) Can Foreign Aid Buy Growth? No (disproves Hansen & Tarp).
%- Hansen & Tarp (01) Aid increases growth (empirical evidence)
%- Tresch et al. (22): 66% of Swiss people want to increase their foreign aid; also Borofsky
%- Harris (17): majority of French want to decrease foreign aid


% Universalism
%- Enke et al. (Manag. Science, 23): measures universalism by asking to split donation to domestic and foreigner of same absolute income (US).
%- Enke et al. (ReStud, 23): unviersalism more correlated to policy attitudes than income, education, religiosity or beliefs about government efficiency (West).
%- Cappelen et al. (NBER, 22): how unviversalism (as measured above) varies across countries. Comparable in Europe and US (lower in China, higher in Africa)
%- Cherry et al (17) show in the lab that some people prefer policies detrimental to them due to their worldview.


% Free-riding
%- Mildenberg (2019): people are not free riders
%- McGrath & Bernauer (17): review paper. people are not free riders. Preferences concerning climate policy tend to be driven primarily by a range of personal predispositions and cost considerations, which existing research has already explored quite extensively, rather than by considerations of what other countries do
%- Bernauer & Gampfer (15): US and IA people are not free riders. They each overestimate their country's emissions at one third of global total.


% Social norms
%- Bursztyn et al. (AER, 20): social norms can change following new public information such as unexpected election outcome. After Trump election, people express more xenophobic views and judge less severely those who do.
%- Farrow et al. (17): review of effect of social norm intervention on environmental attitudes

% Incentive compatibility
%- Danz et al


% Second-order beliefs
%* Mildenberg & Tingley (19): survey elites (Congress staffers, scholars) and public in U.S. and China and show pluralistic ignorance of pro-climate attitudes, egocentric bias, and increasing support after beliefs are updated.
%- Bursztyn & Yang (21): Review of the field. Misperceptions about others are widespread, asymmetric, much larger when about out-group members, and positively associated with one’s own attitudes.
%- Drews et al. (22): in Spain, supporters (resp. opponents) of carbon tax overestimate (resp. underestimate) support. Providing information doesn't change the overall support.
%* Falk et al. (21): Respondents vastly underestimate the prevalence of climate- friendly behaviors and norms among their fellow citizens. Providing respondents with correct information causally raises individual willingness to fight climate change as well as individual support for climate policies. The effects are strongest for individuals who are skeptical about the existence and threat of global warming.
%- Di Tella et al. (AER, 15): The results of the lab experiment favor the hypothesis that people avoid altruistic actions by distorting beliefs about others' altruism
%- Allport (1924): first book on pluralistic ignorance
%- Allport (40): function of poll is to correct pluralistic ignorance
%- Studies on pluralistic ignorance: business (Buckley et al. 00), against affirmative action (Van Boven 00), political correctness (Braghieri, AER 21), alcohol (Suls & Green, 03), white support for racial segregation (O'Gorman 75), CC (Geiger & Swim 16), hooking up (Lambert et al 03, cf. note for paragraph of pluralistic ignorance), women working outside home in Saudi Arabia (Bursztyn et al. 20)
%- Geiger & Swim (16) Shows that pluralistic ignorance of others' concern about CC leads people to talk less about CC and self-silence themselves.
%- Miller & MacFarland (87) Shows that pluralistic ignorance emerges because individuals believe that fear of embarrassment is a sufficient cause for their own behavior but not for the behavior of others.


% Elite surveys TODO find more
%* Mildenberg & Tingley (19): Congress staffers, cf. second-order beliefs
%- Hertel-Fernandez et al. (2019): Survey on US Congress staffers (not on climate)
%- Milner & Tingley (10) (not sure it's a survey) owners of capital in donor countries tend to gain from aid and thus are more likely to support giving aid
%- Lange et al. (Energy Econ, 2007): climate negotiators
%- Lange et al. (EER, 2010): same data as Lange et al. (10)
%- Dannenberg et al. (ERE, 2010): elicit climate negotiators’ equity preferences using Fehr & Schmidt (99) method => regional differences in addressing climate change are driven more by national interests than by different equity concerns
%- Kesternich et al. (EEPS, 2020): survey on climate negotiators about their preferred burden-sharing rules: we observe tendencies for a more harmonized view among key groups towards the ability-to-pay rule in a setting of weighted burden sharing rules
%- Lange & Schwirplies (ERE, 2017): combines Lange et al. (10) and Schleich et al.
%* Hjerpe et al. (2011): Delegates at COP2009. The results indicate that voluntary contribution, indicated as willingness to contribute, was the least preferred principle among both negotiators and observers. Three of the four principles for allocating mitigation commitments were recognized widely across the major geographical regions: historic 1990, capacity to pay, and equal per capita emissions. The difference was never below 25 percentage units, and the opponent share never exceeded 16%.
%- Scholte et al. (2020)
%- Bayram (17): cosmopolitanism of German politicians and their respect of international law


% Global poverty gap
%* Bolch et al. (22)
%- Zhang (16) estimates the poverty gap in each country. Global one is at $80G/year.


% Basic income TODO find more
%* Egger et al. (19): positive gen eq effects. We provided one-time cash transfers of about USD 1000 to over 10,500 poor households across 653 randomized villages in rural Kenya. The implied fiscal shock was over 15 percent of local GDP. We find large impacts on consumption and assets for recipients. Importantly, we document large positive spillovers on non-recipient households and firms, and minimal price inflation.
%* Haushofer & Shapiro (16): The Short-term Impact of Unconditional Cash Transfers to the Poor: Experimental Evidence from Kenya. Monthly transfers are more likely than lump-sum transfers to improve food security


% Unequal exchange / embodided labour
%- Reyes et al (17)
%- Sakai et al (17)
%- Alsamawi et al. 2014


% NDCs assessments or burden-sharing computations. TODO! check the contraction & convergence scheme proposed by France
%- Bourban (18): Soutient un marché du carbone avec droits en proportion des émissions cumulées depuis 1990. Et des “mesures volontaires de contrôle de la population mondiale”.
%- Raupach et al (NCC, 14) 
%- Grasso (2012)
%- van den Berg et al (20)
%- Meyer (04) Contraction and Convergence (i.e. grandfathering converging to equal pc, within an ETS)
%- >Baer et al (08)< (cite this one, others don't give more info), Baer (13), Athanasiou et al (22), Holz et al (19) https://calculator.climateequityreference.org/ Athanasiou, Greenhouse Development Rights, EcoEquity calculator, US fair share. Effort-sharing approach based on splitting emissions reductions in function of capacity to pay (~ share of global income in top 30%) and responsibility (share of emissions since 1950), weighted equally. Corresponds to UNFCCC wording. Pb of this method (applying to any choice of parameters): A country with relatively low incomes (e.g. equal distribution slightly above the p70) and that has few historical responsibility would have a relatively low effort. Even more problematic, the **poorest countries would have virtually 0% of the effort, hence they would be allowed to emit following the baseline trajectory… but this baseline is not fair; it amounts to grandfathering**. It is computed as the “product of the projected GDP and CO2 emission intensity”. ([https://climateequityreference.org/calculator-information/gdp-and-emissions-baselines/](https://climateequityreference.org/calculator-information/gdp-and-emissions-baselines/)), and give for example 0.8tCO2e/cap for RDC in 2030 (16% more than in 2020, but lot lower than the objective of ~4t). => Compared to an equal right to emit pc, this method favors countries like China (allowed to remain stable over 2020-30 vs. reduced by 35-40%) and penalizes countries like the U.S. and Africa. 
%  in Athanasiou et al (22) Justification of Greenhouse Development Rights instead of Equal per capita right is on p. 36. It is weak, and basically that historical responsibility should be taken into account. Conversely, justification against historical resp. is that the latter doesn’t take into account capacity to pay (it is not said like this, but we can think of ex-USSR).
%- Pachauri et al. (Science, 2022) 
%- Robiou du Pont et al. (NCC, 2016)
%- Robiou du Pont et al. (ERL, 2016)
%- Höhne et al. (Climate Policy, 2014): review of 40 papers
%- Gao et al. (FEM, 2019)
%- Gignac & Matthews (ERL, 15)
%- Matthews (16) Quantifying carbon debts among nations
%- https://climateequitymonitor.in/ computes carbon debt based on equal per capita cumulative emissions. contact@climateequitymonitor.in https://twitter.com/equity4climate


% Mismatch between preferences and climate action TODO: cite
%- McCright & Dunlap (03) show that it's an organized conservative movement that succeeded in the U.S. not ratifying Kyoto, through lobbying and disinformation.


% Wealth tax attitudes
% look for surveys on global tax => I've found no result with survey or attitudes + "global tax" or "global wealth tax" in google scholar
% Fisman et al (17): Americans want a 3% tax on inherited wealth
%- Christensen et al. (Oxfam, 23) p. 32 gives references on rich tax attitudes, with always strong majority support:
%* OECD (19): 52-80% of absolute support for "government tax the rich more than they currently do in order to support the poor" in 21 OECD countries
%* Isbell (22): 34 African countries
%- Patriotic Millionaires (22), UK
%- Americans for Tax Fairness (21), US
%- Gallup (22), US
%- Fight Inequality Alliance India (22), IA

% Different framing of burden-sharing, depending on what should be split:
% - mitigation costs: this is the most used as it is easiest to explain. The issue is that it is not specified how agents pay (or if some agents receive payments) and implicitly, there is no negative costs (transfers exceeding the costs) and the carbon price is not uniform. Used in .
% - emission: this one is vague as it doesn't state at which date emissions pc converge (if they do) and whether there are side payments.
% - emission rights: this one is the most accurate as there is no need of a BAU scenario to compute the mitigation needed and its cost.

% Different fairness principles:
% - equal emission right per capita: using this as a baseline, we can call 'grandfathering' any principle that is more regressive and 'historical responsibility' any principle that is more progressive
% - equal emission reduction (in share of current emission) per capita: grandfathering
% - emission rights proportional to current emissions: grandfathering
% - costs proportional to current emissions: polluter-pay principle
% - costs proportional to cumulative emissions: so-called historical responsibility but may actually have a grandfathering component

% Surveys of population:
% - Schleich et al. (Climate Policy, 16) ask for ranking and find an identical ranking of fairness principles in China, Germany, and the US: accountability (costs according to emissions) followed by capability (according to economic strength), egalitarianism (equal emission per capita), and sovereignty (constant share of global emission) (see Lange & Schiwplies (17) for the computations). 
%   Polluter-pays: Every country has to bear costs according to the emissions it causes (hence countries causing higher emissions have a higher share of the costs).
%   Ability-to-pay: Every country has to bear costs according to its economic strength (hence richer countries have a higher share of the costs).
%   Egalitarian: Every country is allowed to produce the same amount of emissions per capita (hence countries with currently high emissions per capita have higher costs).
%   Sovereignty: Every country is allowed to produce the same share of global emissions as in the past (hence the proportional reduction of emissions is the same for every country).
% other findings: international agreements are important but current ones are unsuccessful, people find themselves poorly represented in climate negotiations
% - Bechtel & Scheve (PNAS, 13) find with a conjoint analysis on FR, DE, UK, US that a climate agreement is 5 p.p. less likely to be preferred (to a random alternative) if only rich countries pay (other burden-sharing are: pay prop. to current emissions / historical emissions / rich countries pay more than poor countries) and 15 p.p. more likely to be preferred if it includes 160 (out of 192) countries rather than 20 => confirms preference for global policies (rather than only partial coverage). Finds that costs is what matters most: preference decreases by 30pp if it’s 2.5% of GDP compared to 0.5%.
% - Carlsson et al. (REE, 13) find using a 09 choice experiment that Americans prefer capacity to pay > current responsibility > historical responsibility > equal emissions per capita while Chinese prefer historical > capacity > current > equal emissions.
%   Capacity to pay: Countries with high income levels must pay a larger share of the costs than countries with low income levels. This option says that countries with greater ability to pay should pay more
%   Current responsibility: Countries with currently high emissions levels must pay a larger share of the costs than countries with currently low emissions levels. This option says that those countries that are currently a larger part of the problem should pay more.
%   Historical responsibility: Countries with a history of high emissions levels must pay a larger share of the costs than countries with a history of lower emissions. This option recognizes that CO2 builds up in the atmosphere over many years. Thus, countries with a history of high emissions should pay more because they caused more of the problem.
%   Equal emissions pc: Countries with emissions per person greater than an agreed amount must pay, and they must pay more the higher their emissions per person area.
% > "equal emissions" is a misnomer as this is about costs (not emissions) and it's just a more progressive version of current responsibility / polluter-pay, where high-emitting pay more and low-emitting don't pay. The result for US is compatible with the other papers as Americans agree that rich countries (or high-emitting, the diff is small) should pay more. The Chinese position could also be reconciliable once we define responsibility from footprint rather than territorial and that there will be transfers from rich to poor countries.
% - Carlsson et al. (Ecol Eco, 11) find that Swedes prefer that "all countries are allowed to emit an equal amount per capita" rather than options where emissions reduce in relation to current or historical emissions and continue to be higher in high-emitting countries. 
% - Meilland et al. (23) find that in US and France, most favored fairness principle is Equality in per capita emissions: "all countries commit to converge to the same average of total emissions per inhabitant, compatible with a controlled climate change" and second-most (which closely follows) is grandfathering: "all countries commit to reduce their emissions by a same proportion". 73% in each disagree with grandfathering when defined as "countries which emitted a lot of carbon in the past have a right to continue emitting more than others in the future". To rationalize these contrasted views with grandfathering, we can interpret them as: equal rights, equal emission reductions, and transfers. 
%   convergence per capita (70%): all countries commit to converge to the same average of total emissions per inhabitant, compatible with a controlled climate change
%   grandfathering (60%): all countries commit to reduce their emissions by a same proportion
%   past emissions (20% choose it among their two favorite): countries which emitted less in the past commit to reduce their emissions less than other countries
%   poor countries (20%): poorer countries commit to reduce their emissions less than richer countries
%   cost-efficiency (20%): countries where reducing emissions is more costly commit to reduce their emissions less than other countries
% Other findings: people prefer international settlement on CC even if it empedes on sovereignty, a majority prefers to target footprint rather than territorial emissions, median is that countries should be held accountable for post-1990 emissions, self-serving bias when judging e.g. India vs. EU, no shared understanding of fairness when asked to coordinate between French and Americans
% - Dechezleprêtre et al. (WP, 22) find that equal per capita right > historical responsability, capabilities > grandfathering; that global CC policies are needed; 50% support for global T&D; strong support for global tax on millionaires; no free-riding. 
% - Dabla-Norris et al. (WP, 23) find strong majority for “all countries” everywhere in “Which countries do you think should be paying to reduce carbon emissions?”, and majority for current rather than historical in all countries but China and Saudi Arabia in “Should countries be paying to reduce carbon emissions based on their current or accumulated historic levels of emissions?”

% > Position making all this compatible: people want that every country engage in strong decarbonization effort together, with a global quota, converging to climate neutrality in the medium run, based on an equal right to emit per person, implying that rich countries pay and low-emitting countries receive funding. Where the rankings differ, it is likely because the definitions or wordings are different, and also because it involves different countries (Sweden != US != China).
% - Schleich find support for costs according to emissions and against immediate equalization of emissions (but nothing against convergence to equal emissions per capita).
% - This is just in contradiction with Carlsson (11) which finds that Swedes prefer the equalization (with a similar wording) to other reduction options. 
% - Bechtel find agreement that rich countries should pay more and historical emissions matter, but just that they should not be the only one to make the efforts. 
% - Carlsson (13) find that the least preferred option in China and US is when low-emitting countries don't participate to the effort. Ability to pay is liked in both countries.
% - Meilland find that convergence is the most preferred, followed by emission reductions of same proportion, disagreement with grandfathering expressed in terms of emission rights.
% - Dechezleprêtre find support for equal right is strongest, although historical responsibility and capabilities are also supported. The quota system is strongly supported.

% Surveys of negotiators:
% - Hjerpe et al. (WP, 11)
% - Dannenberg et al. (ERE, 10): measuring negotiators' equity preferences, regional differences in addressing climate change are driven more by national interests than by different equity concerns.
% - Lange et al. (Energy Econ, 07): Mix of self-serving bias and support for egalitarian principle.
% - Kesternich et al. (EEPS, 21): kind of convergence on ability-to-pay.

% Other papers:
% - Lange & Schwirplies (ERE, 17) develop a theoretical model (building on Buchholz et al. (05)), supported by data, justifying that climate negotiators (chosen by the citizens) have lower environmental preferences than their citizens and equity views more aligned with the other negotiators. 
% - List experiment: Kuklinski et al. 97 or https://blogs.lse.ac.uk/europpblog/2022/04/06/do-russians-tell-the-truth-when-they-say-they-support-the-war-in-ukraine-evidence-from-a-list-experiment/