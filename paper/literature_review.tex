\clearpage
\section{Literature review}\label{app:literature}

% Burden-sharing
%- Agarwal & Narain (91) first to defend an equal right to emit per capita (equal to the absorbing capacity of the Earth)
%- Gampfer (14): lab experiment (ultimatum game) to test whether preferences respect fairness principles
% cf. notes

% Global policies attitudes TODO! find more, e.g. citations of Carattini et al, global (wealth) tax, check literature in Ghassim for global democracy
%* Ghassim et al. (22): support for stronger UN with more direct elections.
%* Ghassim (20):  in Germany those two parties that supposedly endorse global democracy – the Greens and the Left – benefitted, gaining nine and three percentage points respectively in terms of voting intentions. Meanwhile, the traditional centrist parties – SPD and CDU – each lost six percentage points due to their supposed opposition to global democracy.
%- Bechtel et al. (19): READ
%* Beiser-McGrath & Bernauer (19): Conjoint analysis in US, DE. Variant of carbon tax is 8 (US) - 17 (DE) p.p. more likely to be preferred and 50% more likely to be supported if tax is extended to all industrialized countries (Fig 1, 4). (Unfortunately, don't test extension to global level).
%* Carattini et al. (Nature, 19): READ (also check for global policies in Carattini et al 18 & Klenert et al 18)
%* Çarkoğlu.. (15) International Social Survey Program 2010 data reveal that people in LDCs are less supportive of international agreements forcing their country to take necessary environmental measures than are citizens in the developed world. READ


% Global policies
% Pottier et al (17): A survey of global climate justice 
%* Hickel (17): The Divide: A Brief Guide to Global Inequality and its Solutions
%* Kopczuk et al (EER, 17) Compute optimal linear tax rate for all countries in two ways: decentralized or globally. Shows that the tax rate increases with inequality of skills (calibrated with the gini). The average decentralized rate is 0.41 The global one 0.62, with a global demogrant of 250$/month (higher than 73 countries' GDP). Show that within decentralized/country optimal taxation would not decrease global inequality by much (gini from 0.695 to 0.69, but down to 0.25 with global income tax). Show that USA don't give a damn of poor countries' people. citizens in the US (one of the richest) attach only 1/(2,000\*a) of the weight to the welfare of citizens in poorest countries, where a is the share  of transfer (supposedly) effectively arriving to the recipients. e.g. if half of aid is wasted by corrupt politicians, the weight is 1/1000.


% Global carbon pricing TODO! find advocates of GCS
%* Grubb (90), Betram (92) READ advocate for global market with equal pc right
%* Jamieson (01) advocates of equal pc burden-sharing (after the precursors Agarwal & Narain (91))
%* Bear et al (Science, 00), Bear (02), Athanasiou & Baer (02) advocate for equal pc burden-sharing (although weirdly, Bear & Athanasiou then change mind and advocate for the Greenhouse Development Rights, accounting for capacity and responsibility)
%* Cramton et al (17): Livre de pontes. Tout le monde est d'accord : un prix mondial du carbone est requis, il ne peut être obtenu que par la réciprocité des engagements (style climate club), et il faut quelques transferts des riches vers les pauvres ainsi que des sanctions commerciales pour aligner les incitations. Ch 4 (also Cramton et al 15) propose la formule suivante de transfert (positif ou négatif) à un fonds climat : générosité*émissions en excès (par rapport à la cible)*prix du carbone. On demanderait aux États autour de la moyenne d'émission de fixer ce paramètre de générosité, pour qu'il soit fixé de sorte à maximiser le prix, puis on fixerait le prix comme le prix minimum proposé (après avoir éjecté qqs pays récalcitrants des négos). Puis, sanctions commerciales pour ceux qui ne respectent pas le prix. Ch Gollier & Tirole proposent une formule aussi simple que l'autre : quota global*((1-g)*part des émissions à t=0 + g*part de la population), où g joue le même rôle de paramètre de générosité/éthique (que je voudrais mettre à 1, mais qu'ils disent tous de mettre < 1 pour que les pays riches acceptent. Le livre argumente bcp sur prix vs. quantité (TLM préfère prix sauf Gollier & Tirole), l'argument le plus convaincant en faveur du prix c'est qu'avec la procédure proposée le prix négocié serait le plus élevé possible, alors qu'avec la quantité c'est le budget carbone qui serait le point focal et ça aboutirait à une impasse (objectif trop ambitieux).
%* MacKay et al (Nature, 15) summarizes the above
%* Weitzman (17) advocates for a World Climate Assembly, choosing the price level with the median voter, and each country retaining the revenues.
% Fleurbaey & Zuber (13): The discount rate converges to the worst-off (affected by the measure) to the worst-off (beneficiary of the measure) discount rate, which depends on the growth between both agents. Applied to real data, we can consider that the worst-off affected by a global tax on CO_2 is the average-earner on earth (around 75% centile i.e. ~1000€/month, cf. Chancel & Piketty, Lakner & Milanovic, Chakravorty) while the worst-off beneficiary is the worst-off person in the future (among those less affected by CC thanks to the measure), probably below 1000€/month => negative discount rate.
% Stanton (11): Negishi weights obviate the IAMs’ equalization of income. 4 ways to solve this problem: 1. be more transparent, 2. stop weighting, 3. take linear utility (i.e. maximize global GDP), 4. stop optimizing. 
% Hoel (91): Shows that an international tax can be designed so that it is both efficient and satisfies whatever distributional objectives one might have.
% IMF (2019): global pricing (with either differentiated prices or international transfers) or, as a first step, a carbon price floor. 25% of revenues should be rebated to the bottom 40%, the rest used to reduce distortionary taxes or for green investments. Estimate that $75/t is needed in 2030 for 2°C.
% Parry et al (21): Proposal for an International Carbon Price Floor Among Large Emitters. Acknowledges that transfers could be necessary to induce climate action in low/middle-income countries, talks about transferring 1% of carbon revenues.
%- Sager: distributive effects of global pricing without int'l transfers.

% Foreign aid TODO: find more recent, check .lyx for already written paragraph
%* Kaufmann et al (12) Shows the level of perceived and desired aid in 26 countries between 2005 and 2008. Shows that richer want less aid. from 0 to higher than 25%: threshold at 0.05; 0.15; 0.35; 0.75; 1.5; 2.5; 4; 7.5; 17.5; 25. READ
%- Milner & Tingley (13): Stern (1998) (fig 1) shows that country's taste for foreign aid is correlated to its amount of aid. Within a country, richer support more aid. In 2008, 44% of American wanted foreign aid cut. fraction of federal budget going to foreign aid (mean: 27%, median: 25%) / should go (mean: 13%, median: 10%). U.S. 2010 READ
% PIPA (01): Overwhelming majorities support a multilateral effort to cut hunger in half by the year 2015 and say that they would be willing to pay for the costs of such a program. However, most do not think that the average American would be as willing to pay the necessary costs. when PIPA asked respondents to estimate how much of the federal budget was devoted to foreign aid, the median estimate was 15% -- 15 times the actual amount, which was just under 1%. More dramatically, when asked what an appropriate percentage would be, the median response was 5% -- 5 times the actual amount. And when asked to imagine that they heard the real amount was only 1%, only 18% of respondents said they thought that would be too much--as compared to the 75% who had initially said that the US was spending too much. what percentage of their "tax dollars that go to help poor people at home and abroad...should go to help poor people in other countries." The mean response was 16% (down a bit from 22% in response to this question in a 1996 PIPA poll). Strikingly, this turns out to be a far higher percentage than is currently given. In 1999, a bit less than 4% of the total spent on the poor went to the poor abroad. Sixty percent of respondents proposed a percentage that was higher than 4%.
%- Hudson & van Heerde (12): ~75% support aid in developed countries (stable) but 84% prefer to address domestic pb in the US. PIPA 2001, "On average, Americans thought just under 25 per cent of the US budget was allocated to foreign aid, and government should allocate less than 14 per cent of the national budget. However, when told that US spends approximately 1 per cent of the federal budget on foreign aid, 37 per cent of respondents thought this was too little, 44 per cent thought it was about right, and 13 per cent thought it too much." => think that only 23% of aid really goes to the poor. UK ‘MyAid’ plan – where the public gets to vote on how a pot of money should be distributed – READ
%* Gilens (01) 17% fewer American with high political knowledge want to cut foreign aid when we provide them specific information about aid amount.
%- Chong & Gradstein (16): from WVS 95-99, 58% want that their country give more foreign aid (but misperceptions are not taken into account)
%* Bauhr et al (13): Support for aid is reduced by perception of corruption in recipient countries. However, this effect is reduced by the aid-corruption paradox (and other things): most corrupt countries need more help.
%- Nair (18): (lack of) Aid support in US driven by information on global distribution, because people underestimate their rank by 27 centiles and overestimate global median income by a factor 10.
%- Williamson (19): Public Ignorance or Elitist Jargon? Reconsidering Americans’ Overestimates of Government Waste and Foreign Aid. "Foreign aid" encompasses military spending, in the mind of American.
%- McDonnell et al (03) Public Opinion and the Fight against Poverty
%- Nair (16): preferences driven by worldviews rather than self-interest
%- Bodenstein & Faust (17): Determinants of support for aid conditionality. They are: perceived corruption in donor country, right-wing.
%- Scotto et al (17): We Spend How Much? Misperceptions, Innumeracy, and Support for the Foreign Aid in the United States and Great Britain. Less American and British want aid cut when information on current aid is given in % of GDP rather than in $.
%- Paxton & Knack (12): Determinants for aid support (US outlier). READ
%- Wood (15): Determinants for aid support in Australia. Wood (18) Examine Australian support for aid: although there is support to help foreign poor, people back recent aid cuts.
%- Bayram (17): Aid support associated with trust, i.e. seeing integrity and trustworthiness in others.
%- Cheng & Smyth (16): Why Give it Away When You Need it Yourself? Understanding Public Support for Foreign Aid in China. Political ideology and patriotism main explaining variables for aid support. People in poorer provinces less supportive.
%- Milner & Tingley (10) theory + empirics: who supports aid and why. owners of capital in donor countries tend to gain from aid and thus are more likely to support giving aid
%- Easterly (JEP, 03) Can Foreign Aid Buy Growth? No (disproves Hansen & Tarp).
%- Hansen & Tarp (01) Aid increases growth (empirical evidence)
%- Tresch et al. (22): 66% of Swiss people want to increase their foreign aid
%- Harris (17): majority of French want to decrease foreign aid


% Universalism
%- Enke et al. (Manag. Science, 23): measures universalism by asking to split donation to domestic and foreigner of same absolute income (US).
%- Enke et al. (ReStud, 23): unviersalism more correlated to policy attitudes than income, education, religiosity or beliefs about government efficiency (West).
%- Cappelen et al. (NBER, 22): how unviversalism (as measured above) varies across countries. % TODO higher in Europe than US?
%- Cherry et al (17) show in the lab that some people prefer policies detrimental to them due to their worldview.


% Free-riding
%- Mildenberg (2019): people are not free riders
%- McGrath & Bernauer (17): review paper. people are not free riders. Preferences concerning climate policy tend to be driven primarily by a range of personal predispositions and cost considerations, which existing research has already explored quite extensively, rather than by considerations of what other countries do
%- Bernauer & Gampfer (15): US and IA people are not free riders. They each overestimate their country's emissions at one third of global total.


% Social norms
%- Bursztyn et al. (AER, 20): social norms can change following new public information such as unexpected election outcome. After Trump election, people express more xenophobic views and judge less severely those who do.
%- Farrow et al. (17): review of effect of social norm intervention on environmental attitudes

% Incentive compatibility
%- Danz et al


% Second-order beliefs
%* Mildenberg & Tingley (19): survey elites (Congress staffers, scholars) and public in U.S. and China and show pluralistic ignorance of pro-climate attitudes, egocentric bias, and increasing support after beliefs are updated.
%- Bursztyn & Yang (21): Review of the field. Misperceptions about others are widespread, asymmetric, much larger when about out-group members, and positively associated with one’s own attitudes.
%- Drews et al. (22): in Spain, supporters (resp. opponents) of carbon tax overestimate (resp. underestimate) support. Providing information doesn't change the overall support.
%* Falk et al. (21): Respondents vastly underestimate the prevalence of climate- friendly behaviors and norms among their fellow citizens. Providing respondents with correct information causally raises individual willingness to fight climate change as well as individual support for climate policies. The effects are strongest for individuals who are skeptical about the existence and threat of global warming.
%- Di Tella et al. (AER, 15): The results of the lab experiment favor the hypothesis that people avoid altruistic actions by distorting beliefs about others' altruism
%- Allport (1924): first book on pluralistic ignorance
%- Allport (40): function of poll is to correct pluralistic ignorance
%- Studies on pluralistic ignorance: business (Buckley et al. 00), against affirmative action (Van Boven 00), political correctness (Braghieri, AER 21), alcohol (Suls & Green, 03), white support for racial segregation (O'Gorman 75), CC (Geiger & Swim 16), hooking up (Lambert et al 03, cf. note for paragraph of pluralistic ignorance), women working outside home in Saudi Arabia (Bursztyn et al. 20)
%- Geiger & Swim (16) Shows that pluralistic ignorance of others' concern about CC leads people to talk less about CC and self-silence themselves.
%- Miller & MacFarland (87) Shows that pluralistic ignorance emerges because individuals believe that fear of embarrassment is a sufficient cause for their own behavior but not for the behavior of others.


% Elite surveys TODO find more
%* Mildenberg & Tingley (19): Congress staffers, cf. second-order beliefs
%- Hertel-Fernandez et al. (2019): Survey on US Congress staffers (not on climate)
%- Milner & Tingley (10) (not sure it's a survey) owners of capital in donor countries tend to gain from aid and thus are more likely to support giving aid
%- Lange et al. (Energy Econ, 2007): climate negotiators
%- Lange et al. (EER, 2010): same data as Lange et al. (10)
%- Dannenberg et al. (ERE, 2010): elicit climate negotiators’ equity preferences using Fehr & Schmidt (99) method => regional differences in addressing climate change are driven more by national interests than by different equity concerns
%- Kesternich et al. (EEPS, 2020): survey on climate negotiators about their preferred burden-sharing rules: we observe tendencies for a more harmonized view among key groups towards the ability-to-pay rule in a setting of weighted burden sharing rules
%- Lange & Schwirplies (ERE, 2017): combines Lange et al. (10) and Schleich et al.
%* Hjerpe et al. (2011): Delegates at COP2009. The results indicate that voluntary contribution, indicated as willingness to contribute, was the least preferred principle among both negotiators and observers. Three of the four principles for allocating mitigation commitments were recognized widely across the major geographical regions: historic 1990, capacity to pay, and equal per capita emissions. The difference was never below 25 percentage units, and the opponent share never exceeded 16%.
%- Scholte et al. (2020)


% Global poverty gap
%* Bolch et al. (22)
%- Zhang (16) estimates the poverty gap in each country. Global one is at $80G/year.


% Basic income TODO find more
%* Egger et al. (19): positive gen eq effects. We provided one-time cash transfers of about USD 1000 to over 10,500 poor households across 653 randomized villages in rural Kenya. The implied fiscal shock was over 15 percent of local GDP. We find large impacts on consumption and assets for recipients. Importantly, we document large positive spillovers on non-recipient households and firms, and minimal price inflation.
%* Haushofer & Shapiro (16): The Short-term Impact of Unconditional Cash Transfers to the Poor: Experimental Evidence from Kenya. Monthly transfers are more likely than lump-sum transfers to improve food security


% Unequal exchange / embodided labour
%- Reyes et al (17)
%- Sakai et al (17)
%- Alsamawi et al. 2014


% NDCs assessments or burden-sharing computations. TODO check the contraction & convergence scheme proposed by France
%- Raupach et al (NCC, 14) 
%- Grasso (2012)
%- van den Berg et al (20)
%- Meyer (04) Contraction and Convergence (i.e. grandfathering converging to equal pc, within an ETS)
%- Baer et al (08), Baer (13), Athanasiou et al (22) READ, Holz et al (19) https://calculator.climateequityreference.org/ Athanasiou, Greenhouse Development Rights, EcoEquity calculator, US fair share. Effort-sharing approach based on splitting emissions reductions in function of capacity to pay (~ share of global income in top 30%) and responsibility (share of emissions since 1950), weighted equally. Corresponds to UNFCCC wording. Pb of this method (applying to any choice of parameters): A country with relatively low incomes (e.g. equal distribution slightly above the p70) and that has few historical responsibility would have a relatively low effort. Even more problematic, the **poorest countries would have virtually 0% of the effort, hence they would be allowed to emit following the baseline trajectory… but this baseline is not fair; it amounts to grandfathering**. It is computed as the “product of the projected GDP and CO2 emission intensity”. ([https://climateequityreference.org/calculator-information/gdp-and-emissions-baselines/](https://climateequityreference.org/calculator-information/gdp-and-emissions-baselines/)), and give for example 0.8tCO2e/cap for RDC in 2030 (16% more than in 2020, but lot lower than the objective of ~4t). => Compared to an equal right to emit pc, this method favors countries like China (allowed to remain stable over 2020-30 vs. reduced by 35-40%) and penalizes countries like the U.S. and Africa.
%- Pachauri et al. (Science, 2022) 
%- Robiou du Pont et al. (NCC, 2016)
%- Robiou du Pont et al. (ERL, 2016)
%- Höhne et al. (Climate Policy, 2014): review of 40 papers
%- Gao et al. (FEM, 2019)
%- Gignac & Matthews (ERL, 15)
%- Matthews (16) Quantifying carbon debts among nations


% Mismatch between preferences and climate action
%- McCright & Dunlap (03) show that it's an organized conservative movement that succeeded in the U.S. not ratifying Kyoto, through lobbying and disinformation.
