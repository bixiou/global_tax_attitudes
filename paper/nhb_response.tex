\documentclass[12pt,english]{article}
\usepackage[utf8]{inputenc}
\usepackage[left=1.5in,right=1.5in,top=1.5in,bottom=1.5in]{geometry}
\usepackage{bm}
\usepackage{amsmath}
\usepackage{amssymb}
\usepackage{indentfirst}
\usepackage[hyperpageref]{backref} % back references biblio
\usepackage{tocbibind}
\usepackage[round,sort&compress]{natbib}
\setcitestyle{aysep={}} 
\usepackage{amsfonts}
\usepackage{enumerate}
\usepackage{babel}
\usepackage{caption}
\usepackage{supertabular}
\usepackage{tabularx}
\usepackage{float}
\usepackage{dsfont}
\usepackage{fancyvrb}
\usepackage{verbatim}
\usepackage[hyphens]{url}
\usepackage{hyperref}
\usepackage[shortlabels]{enumitem}
\usepackage{setspace}
\usepackage{comment}
\usepackage{subcaption}
\usepackage{graphicx}
\usepackage{tikz}
\usetikzlibrary{shapes,backgrounds,positioning}
\usepackage{gensymb}
\usepackage{eurosym}
\usepackage{textcomp}
\usepackage{color,soul}

\usepackage{multicol}
\usepackage{changepage}
\usepackage{enumitem}


\usepackage{tabulary}
\usepackage{tabularx}
\usepackage{booktabs}
\usepackage{fullpage}
\usepackage{morefloats}
% \usepackage[utf8]{inputenc}
% \usepackage{bm}
% \usepackage{indentfirst}
% \usepackage{tocbibind}
% \usepackage{enumerate}
\usepackage{makecell}
\usepackage{multirow}
\usepackage{lscape}
\usepackage{pdflscape}
\usepackage{longtable}
\usepackage{rotating}
\usepackage{fancyhdr}
\usepackage{tocloft}
\usepackage{multibib}
\usepackage{titletoc}
\usepackage[export]{adjustbox}
\usepackage[anythingbreaks]{breakurl} % for links
%\usepackage[nomarkers,figuresonly]{endfloat} % Figures at the end
\hypersetup{
  colorlinks=true, % breaklinks=true,
  urlcolor=blue, % color of external links
  linkcolor=blue,  % color of toc, list of figs etc.
  citecolor=violet,   % color of links to bibliography
}   

\title{International Attitudes Toward Global Policies ~\\ \textbf{Responses to the Editor and Reviewers}}

\date{\today}
\begin{document}
	
\maketitle

\paragraph*{Editor's comments}

\textit{Dear Dr Fabre,}

\textit{Thank you once again for your manuscript, entitled "International Attitudes Toward Global Policies", and for your patience during the peer review process.}

\textit{Your Article has now been evaluated by 3 referees. You will see from their comments copied below that, although they find your work of potential interest, they have raised quite substantial concerns. In light of these comments, we cannot accept the manuscript for publication in the current form, but we would be interested in considering a revised version if you are willing and able to fully address reviewer and editorial concerns.}

\textit{We hope you will find the referees' comments useful as you decide how to proceed. If you wish to submit a substantially revised manuscript, please bear in mind that we will be reluctant to approach the referees again in the absence of major revisions.}

\textit{In particular, as Referee \#2 says, it will be important to revise the text to transparently acknowledge and clarify the relationship of this manuscript to the Dechezleprêtre paper which has been accepted at AER, and to avoid presenting any results here which are also presented in Dechezleprêtre et al. (e.g. Fig 1 here appears to be the same as Fig A20 in the AER paper).}

There might have been a confusion concerning Fig 1 (now Figure 2). Although that figure appeared in an online appendix of an author's version of Dechezleprêtre et al., it does not appear in the AER paper. Besides, no version of Dechezleprêtre et al. has ever commented on this figure nor analyzed results on global policies. We strived to make clear that that paper only deals with questions on national policies, e.g. in the sentence: ``Detailed information about the data collection process, sample representativeness, and analysis of questions on national policies can be found in a companion paper.''
~\\

% Finally, your revised manuscript must comply fully with our editorial policies and formatting requirements. Failure to do so will result in your manuscript being returned to you, which will delay its consideration. To assist you in this process, I have attached a checklist that lists all of our requirements. If you have any questions about any of our policies or formatting, please don't hesitate to contact me.

% If you wish to submit a suitably revised manuscript, we would hope to receive it within 2 months. I would be grateful if you could contact us as soon as possible if you foresee difficulties with meeting this target resubmission date.

% With your revision, please:

% • Include a “Response to the editors and reviewers” document detailing, point-by-point, how you addressed each editor and referee comment. If no action was taken to address a point, you must provide a compelling argument. When formatting this document, please respond to each reviewer comment individually, including the full text of the reviewer comment verbatim followed by your response to the individual point. This response will be used by the editors to evaluate your revision and sent back to the reviewers along with the revised manuscript.

% TODO:
% • Highlight all changes made to your manuscript or provide us with a version that tracks changes.

% Please use the link below to submit your revised manuscript and related files:

% https://mts-nathumbehav.nature.com/cgi-bin/main.plex?el=A5Co1DDI6A5uXz3J2A9ftdrq33rwC5ehBabcLWtReHhQZ

% Note: This URL links to your confidential home page and associated information about manuscripts you may have submitted, or that you are reviewing for us. If you wish to forward this email to co-authors, please delete the link to your homepage.

% Thank you for the opportunity to review your work. Please do not hesitate to contact me if you have any questions or would like to discuss the required revisions further.

% Sincerely,

% Jamie

% Dr Jamie Horder
% Senior Editor
% Nature Human Behaviour

% ----

% REVIEWER COMMENTS:

\paragraph*{Reviewer \#1:}
\textit{This is an interesting study on an important topic, with some novel features. These major upsides are, however, to a large extent overshadowed by a confusing study design and exposition of the results. The study needs a major (structural) cleanup and is often sorely missing in justification for design choices made.}

To address the latter concern, we added a paragraph on \textit{Design choices} in the Methods section.

\textit{The study analyses the results from one 20-country 40k respondent survey, and three “complementary” surveys – one for four European countries, and two (with different sample sizes and questions) for the US.
1) What is the rationale for conducting one large multi-country study, and then follow it up with a more “detailed” study in some countries?}

As stated in the \textit{Data} section, the reason for the Main survey is to ``delve deeper into the sincerity and rationales behind support for the GCS and attitudes towards global policies, global redistribution, and universalistic values.'' Chronologically, we were not expecting such a high support for global redistributive policies before obtaining results from the Global survey. We then wanted to dig deeper this topic in countries that would have to pay the burden of global redistribution.
~\\

\textit{2) When the larger survey is global in scope, why focus only on Western countries (for which public support has been much more extensively studied) in the “complementary” surveys? And given that you chose Western countries, why exactly these five?}

In the complementary surveys, we focused on high-income countries as they are the ones losing from global redistributive policies. Therefore, we expect high-income countries to be less supportive of such policies, an hypothesis confirmed in the Global survey (see Figure 2). Surveying less supportive countries is thus a conservative way to assess the support of citizens around the world. 

The main criterion governing our choice of countries was country size. We preselected the largest high-income countries: the U.S., Japan, Germany, France, the UK, Italy, South Korea, and Spain. Our second criterion was the level of support for global redistributive policies. We conservatively selected countries with a relatively lower average support over the four policies tested in the Global survey. While this average support ranges from 66 to 77\% in the selected countries, it is at least 80\% in the remaining preselected countries.
~\\

%  CA  DE  FR  IT  JP  PL  SK  SP  UK  US
%  73  71  76  81  80  73  80  77  73  66

\textit{3) In the “complementary surveys”, why do you focus in greater detail on the fee-and-dividend scheme and not one of the other policies?}

We focus on the fee-and-dividend to conservatively assess the support for global redistribution. Indeed, for policies such as a global wealth tax, the cost is concentrated on the very rich. If citizens support a globally redistributive policy (such as the Global Climate Scheme) for which they would have to bear a direct cost, we can reasonably deduce that they would support a globally redistributive policy that would leave their purchasing power unaffected. Relatedly, there were more doubt that the stated support for the GCS could be insincere compared to the stated support for a global wealth tax.
~\\

\textit{4) What is the rationale for having 3000 respondents across four European countries plus 3000 in the US in one survey and then an additional 2000 in the US in a second survey? While I do of course understand budgetary constraints (referred to in the pre-registration), why make a different choice for the US than for the European countries? Why not focus for example on two countries and run the exact same study design in those two countries?}

Given that each survey contained blocks of questions with four branches (in particular the list experiment and the wealth tax), we needed at least 2,000 respondents per survey to get precise estimates. We also wanted to test the effect on the support of providing the information on the actual support, hence the need to split the \textit{US} survey into two waves. We could have run two waves in Europe (or in one European country) too, with 2,000 respondents in each wave and each continent. However, we preferred to get a bigger sample (of 3,000 respondents) at least in the U.S., to get more power in our analyses.
~\\

\textit{A number of other design choices are also simply presented with no justification or discussion. Examples include:
• How did you select the four policies you focus on in the global survey? (I.e., why these policies and not other policies?).}

We acknowledge that the selection of policies necessarily involved some level of arbitrariness. We wanted to cover what seemed to be three key areas for global redistribution: climate change, inequality, and global governance. As the Global survey was dedicated to climate change, we emphasized this area. As for the policies themselves, we selected policies theorized as optimal (namely carbon pricing) or backed by prominent actors (the Brazilian government in the case of the wealth tax, Mobilizing an Earth Governance Alliance (MEGA) in the case of democratic climate governance). 
We left aside policies deemed too technical to explain (recapitalization of Multilateral Development Banks; rechannelling of Special Drawing Rights; public guarantees on foreign exchange risk) or similar to policies that we already survey (taxes on financial transactions; maritime fuel; aviation fuel). 
~\\

\textit{• For the “complementary surveys” it is unclear how the “other global policies” were selected.}

We selected policies that are either on the agenda of international negotiations (international transfers for mitigation; adaptation; or loss and damages; cancellation of public debt; reform of voting rights at the UN or IMF; global wealth tax) or advocated by prominent NGOs or scholars (global asset registry by \textit{ICRICT}; limits on wealth by \textit{Ingrid Robeyns} and \textit{Thomas Piketty}; democratic climate governance by \textit{MEGA}; global minimum wage by \textit{ITUC}; % Thomas Palley
fair trade by \textit{Jason Hickel}; carbon pricing by e.g. Cramton, MacKay \& Ockenfels (2017); increased foreign by \textit{CONCORD}). 
~\\

\textit{• For the “complementary surveys”, why do you ask about a national redistribution scheme targeting the top 5\% in the US, but the top 1\% in Europe?}

The wider base in the U.S. was chosen because emissions are larger in the U.S. than in Europe, and it would hardly be feasible to offset the median American's loss from the GCS by taxing only the top 1\%. 

This detail about the National Redistribution scheme, which is not central to our analysis, distracted the reader more than it provided a useful information. Therefore, we replaced ``the top 5\% (in the U.S.) or top 1\% (in Europe)'' by ``top incomes'' in the sentence: ``The same approach is applied to a National Redistribution scheme (NR) targeting top incomes with the aim of financing cash transfers to all adults, calibrated to offset the monetary loss of the GCS for the median emitter in their country.''
~\\

\textit{• Why is the “maximum wealth limit” also (hugely) different, at USD10 billion in the US and USD 100 million in Europe.} 

It is well known that Americans favor less redistribution (let alone confiscatory redistribution) than people in other countries. For example, the U.S. is the only country among 29 without an absolute majority agreeing with the statement: ``It is the responsibility of the government to reduce the differences in income between people with high incomes and those with low incomes'' \citep{issp_international_2019}. Similarly, while the median answer to ``How much do you think a chairman of a large national company should earn?'' is \euro{}10,000 in France and Germany, it is \$250,000 in the U.S. Suspecting that support for a wealth cap would be lower in the U.S., we tested a higher wealth limit in that country. However, we reckon that for comparability purposes, it would have been more appropriate to test the same wealth limit in every country.
~\\

\textit{• Why did you elicit underlying values focusing on “universalistic values” instead of for instance Schwartz Value Survey or cultural worldviews?}% 

Admittedly, seminal are the questionnaires developed by \citet{schwartz_are_1994} to measure one's values and \citet{douglas_risk_1982} to classify one's values on the Individualism--Communitarianism and Hierarchy--Egalitarianism scales. However, none of the questions involved in these questionnaire addresses international issues. In the Schwartz values survey, the 6 items (out of 56) pertaining to ``universalism'' relate to the appreciation of social justice without reference to the society considered. In the cultural worldviews survey, none of the 24 items refer to the international context, while many of them relate to the ``government''. In fact, none of these questionnaires departs from the prevalent norm that associates one's society with one's nation. Therefore, they are inappropriate to assess the degree of \textit{universalism} understood as the equal consideration for foreigners compared to fellow citizens. 1
~\\

\textit{• For the section on “robustness of support”, which is novel, why did you pick certain strategies and not others?}

To test the robustness of support, we picked all strategies we could think of. We would be very grateful if you could suggest other strategies that we could have used, as we intend to run similar surveys in the future.
~\\

\textit{Other things are quite simply unclear: “We survey respondents to gather their perspectives… utilizing either an open-ended or a closed question”. What does this mean? Do half the respondents get the closed and the other half the open-ended question, or is this done in some other way?}

Thank you for pointing out this lack of clarity. Yes, we meant that the sample is split into two random branches. We replace the unclear sentence by ``We survey respondents to gather their perspectives on the pros and cons of the GCS, \textbf{randomly} utilizing an open-ended or a closed question.'' We hope that we managed to be clear with our new wording while keeping its brevity.
~\\

\textit{When it comes to analyzing the results, I’m not sure it is appropriate to group answers across countries/surveys when the details of the questions differ (e.g. for the national redistribution scheme), as done in Figure 2. A similar concern applies to the comparisons in “2.3.4 Prioritization” where direct comparisons are made between outcomes from the different countries, yet only “non-republicans” were surveyed in the US (as far as I can gather from Figure S6).}

The National Redistribution scheme does not appear in Figure 2 (which is now Figure 3) and only appears in figures of the Supplementary Material. Besides, this policy is peripheral to our analysis. The other case where the details of the question differ is on the wealth cap (in Figure 3). We believe that the legend is self-explanatory as its specifies that the wealth cap tested was \euro{}100 million in Europe vs. \$10 billion in the U.S. As it seems to us that the confusion is avoided, we prefer to stick to the current, condensed figure.

Meanwhile, there seems to be a confusion between the prioritization (Figures S37-38) % TODO: check number
and the conjoint analysis (Figure S6). Indeed, Figure S6 presents results country-by-country, and clearly specifies that the fourth conjoint analysis was restricted to non-Republicans in the U.S. Yet, the prioritization is conducted on all respondents.

Although we always specify in Figures' legend that, in the U.S., the fourth and fifth conjoint analyses were only asked to non-Republicans, you are right to point out that we did not always recall it in the text. To avoid the confusion, we replaced ``A is chosen by 60\% in Europe and 58\% in the U.S.'' by ``A is chosen by 60\% of Europeans and 58\% of non-Republican Americans.'' We also added ``(among non-Republicans)'' after the two occurrences of ``in the U.S.'' in the paragraph on the fourth conjoint analysis.
~\\

\textit{I also find the overall structure a bit confusing. For example, after discussing the findings on support for the “other global policies”, it’s back to the global climate scheme, but this time to focus on how robust the support is. (I believe what is now section 2.2.2 should be followed immediately by section 2.3, and not “interrupted” by 2.2.3-2.2.5).}

Thank you for this feedback. We now moved (old) sections 2.2.3-2.2.5 after 2.2.2, so that everything related to the GCS is in a common section, while results pertaining to other policies now in a separate, subsequent section.
~\\

\textit{Before these overall (structural) issues are addressed, I find it difficult to go into details on the individual analyses conducted, in particular with regard to such questions as what data can and should be pooled, and what should not (e.g. questions asked only to non-Republicans in the US).}

With all the above changes, we hope to have clarified the exposition and permitted a smooth reading of our analyses.
~\\

\textit{In my opinion a major revision is clearly required, but also desirable as there are important and novel ideas contained in the manuscript – they just need to be brought out in a much more structured and understandable manner. One potential idea is to simply drop the 20-country survey, and focus only on what is labeled the “complementary surveys (a term that somewhat diminishes its importance). I also believe some kind of map/visual illustration of the structure of the surveys would be helpful for the reader (especially to understand the “branches” of the analyses). (Appendix D contains something like this, but it should be shortened and made much more visually appealing).}

We are very grateful for these very good suggestions. Although you suggest to take out the Global survey, we note that Reviewer \#2 thinks that ``Figure 1 is the main insight of the paper.'' We decided to maintain the Global survey, given that it provides unique results on attitudes towards climate burden-sharing and confirms the support for global redistribution beyond the five countries of our complementary surveys. That being said, we followed your insights that called these surveys ``complementary'' diminshed their importance, and now designate them as our ``Main'' surveys. 

Furthermore, we followed your excellent advice to illustrate the survey flow in the main text. We replaced Table 1 (which did not bring extra information compared to the \textit{Data} section) by Figure 1, which depicts the outline of our Main surveys in a simplified way. While Figure 1 does not report the different branches of the analyses, we replaced the Figures in Appendix D (one for each survey) by a combined Figure S48 % TODO check number
that explains the randomized branches in a shortened and more visually appealing way. 

With these and other changes aimed at improving readability, we hope that our results are now presented in a sufficiently structured and understandable manner.
~\\


\paragraph*{Reviewer \#2:}

\textit{This paper presents evidence from various surveys on attitudes towards global policies, in particular climate change and poverty. The main finding is that a global carbon price funding a global basic income (CGS) is strongly supported by respondents from 20 countries. Additional complementary findings from four country samples report similar support, and discuss methodological concerns, how electoral candidates would win votes endorsing these policies and further support in favour of wealth tax and foreign aid.}

\textit{I think the main message of the paper – widespread support for climate and redistributive policies – is both interesting and important; but has been reported in a companion paper already. I list my main comments in the following.}

\textit{1. I was surprised to see that the main finding from Figure 1 “Relative support for global climate policies” has been shown in the authors’ companion paper (Dechezlepreˆtre et al. 2022). In the latter paper, Figure A20 presents the exact same figure. (The only difference is that in that figure the authors report “absolute” not “relative” support, which was confusing to me because for the first four questions in the first block, numbers are actually exactly the same; in blocks 2 and 3 numbers are higher in the “relative” presentation in comparison to “absolute”.) In any case, I think Figure 1 is the main insight of the paper and has been reported before. Maybe I have overlooked it but I think that the authors are not fully transparent about this fact, and I had to double check. In fact, they write in the introduction that “other (!) questions from these surveys are analyzed in a companion paper”, which I read as saying that the questions analyzed in Figure 1 have not been reported previously.}

The companion paper is now accepted for publication at the \textit{American Economic Review} and the figure mentioned will not appear in the publication. The figure does not appear in the working paper versions (\href{https://www.oecd.org/climate-change/international-attitudes-toward-climate-policies/}{OECD}, \href{https://data.nber.org/data-appendix/w30265/International_Attitudes_Toward_Climate_Change_OA.pdf}{NBER}, \href{https://repec.cepr.org/repec/cpr/ceprdp/DP17602.pdf}{CEPR}, \href{https://www.lse.ac.uk/granthaminstitute/wp-content/uploads/2022/12/working-paper-384-Dechezlepretre-et-al.pdf}{LSE}), it only appears in the appendix of an author's version. Besides, no version of Dechezleprêtre et al. has ever commented on this figure nor analyzed results on global policies, even in the appendix. Therefore, we confirm that Figure 1 (now Figure 2) has not been previously reported in any publication and that the companion paper analyzes other questions of the Global survey, pertaining to attitudes towards climate change and \textit{national} climate policies. We tried to make clear that that paper only deals with questions on national policies, e.g. in the sentence: ``Detailed information about the data collection process, sample representativeness, and analysis of questions on national policies can be found in a companion paper.''

On a side note, the results of the first block are the same whether the support is expressed in absolute or relative terms, as there was no \textit{Indifferent} option for this (multiple choice) question.
~\\

\textit{2. The paper is a bit difficult to read and lacks focus. It would be good to structure the paper better and to summarize the findings in a more concise way. In its current form, too many details and findings are presented in a somewhat disconnected way. What are the key findings and what are supporting/supplementary findings? Perhaps it would make sense to focus on GCS only (including robustness), and move all other findings to a separate section? Likewise, the introduction is a bit “all over the place”, perhaps you could relegate some parts to the discussion section.}

Thank you for this insightful comment. We followed your suggestion and moved all findings related to other policies (than the GCS) to a separate section. % TODO intro
~\\

\textit{3. Data come from a “Global Survey” and “Complementary Surveys”. It is not obvious to me how the two data sets relate to each other. Are you replicating the findings from the global study in the four countries from the complementary survey? Are results consistent between both data sets?}

We tried to clarify the relation between the two surveys, in particular by adding a paragraph on \textit{Design choices} in the Methods section which states that: ``As Global survey results indicated strong support for global redistributive policies worldwide, we conducted our Main [formerly \textit{complementary}] surveys to test the robustness of these results.'' 

While we do not strictly replicate the findings from the Global survey (as we do not ask the same questions), we check that the results of both surveys are consistent. In particular, the GCS can be understood as a combination of two questions from the Global survey: a global carbon tax with equal rebate per capita, where net costs are specified; and a global quota with equal right to emit per capita, where distributive effects are not explained in the question. The GCS is a quota with equal rebate per capita and the question specifies the net costs. Two reasons may explain why, in the Global survey, support is lower for the tax compared to the quota: (i) distributive effects are made salient in the case of the tax; (ii) people may prefer a quota, perhaps because they find it more effective than a tax to reduce emissions. Support for the GCS is at an intermediate level between support for the two policies tested in the Global survey, an outcome that was expected if the two previous reasons held true. Therefore, we write that ``This interpretation [the two possible reasons] is consistent with the level of support for the global quota once we make the distributive effects salient, as we do in the Main surveys.'' Besides, we also compare the responses to similar questions on a global tax on millionaires that would finance low-income countries: support ranges from 73\% (in the U.S.) to 86\% (in Spain) in the Global survey vs. 69\% (in the U.S.) to 87\% (in Spain) in the Main surveys. Therefore, we write ``Consistent with the results of the Global survey, a `tax on millionaires of all countries to finance low-income countries' garners relative support of over 69\% in each country.''

~\\

\textit{4. I found the term “relative support”, which seems to include “indifferent” responses, misleading. For a better understanding, I would prefer Figure 1 to show actual “support” for the respective items. Also, the authors report both “absolute” and “relative” support in different parts of the paper. To me that was confusing. I would stick to one definition of support throughout the paper.}

To clarify the two notions, we added a paragraph on \textit{Absolute vs. relative support} in the Methods section. It reads: ``In most questions, support or opposition for a policy is asked using a 5-Likert scale, with \textit{Indifferent} as the middle option and compulsory response. We call \textit{absolute support} the share of \textit{Somewhat} or \textit{Strong support}. We generally favor the notion of \textit{relative support}, which reports the share of support after excluding \textit{Indifferent} answers. Indeed, the \textit{relative support} is better suited to assess whether there is more people in favor vs. against a policy.''

We hope that it is now clear that ``relative support'' does not include \textit{Indifferent} responses in the numerator, but excludes them from the denominator. Even though we emphasize on the relative notion (which is better suited to assess whether there is more people in favor vs. against a policy), we always make the other results easily accessible. For example, in the Note of Figure 1 (now Figure 2), we refer to Figure S11 for the absolute support.
~\\

\textit{5. The first finding, that local policies receive least support is interesting. I wonder if this has to do with an aversion against being the “only one who contributes”. The notion of conditional cooperation, i.e., the willingness to contribute if others contribute as well, is one of the most important motives to cooperate. Federal or even global policies reduce the likelihood/fear of being the “sucker”, i.e., the guys who have to pay. Perhaps you may want to mention that.}

You are right. Although we cannot test (using our data) whether the preference for climate policy at the global scale stems from conditional cooperation or the perception that it would be more fair and effective, it is worth mentioning the first hypothesis alongside the second one. Therefore, we added: ``It could also stem from conditional cooperation, although previous studies indicate that the support for climate policies does not depend on climate action abroad \citep{aklin_prisoners_2020,tingley_conditional_2014}.''
~\\

\textit{6. How did you choose the global policies, in particular in sections 2.2.3 and 2.2.4? The collection of items reads a bit arbitrary. For example, why is the question about minimum wage in all countries included? Or the democratisation of institutions, etc.? Where are specific numbers in these items coming from? Is there a conceptual framework or is the selection of items based on some higher order principles? More guidance would be good.}

We selected policies that are either on the agenda of international negotiations (international transfers for mitigation; adaptation; or loss and damages; cancellation of public debt; reform of voting rights at the UN or IMF; global wealth tax) or advocated by prominent NGOs or scholars (global asset registry by \textit{ICRICT}; limits on wealth by \textit{Ingrid Robeyns} and \textit{Thomas Piketty}; democratic climate governance by \textit{MEGA}; global minimum wage by \textit{ITUC} and \textit{Thomas Palley}; 
fair trade by \textit{Jason Hickel}; carbon pricing by e.g. Cramton, MacKay \& Ockenfels (2017); increased foreign by \textit{CONCORD}). 
We left aside policies deemed too technical to explain (recapitalization of Multilateral Development Banks; rechannelling of Special Drawing Rights; public guarantees on foreign exchange risk) or similar to policies that we already survey (taxes on financial transactions; maritime fuel; aviation fuel). 

Increase in the minimum wage is a long-standing demand of unions and social movements. \citet{palley_financial_2013} advocates for a \href{https://www.ft.com/content/fa0af8ca-345a-318b-8850-d8d93e61feaa}{global minimum wage system} where each country would set a minimum wage at or above 50\% of the median wage. 

The reform of voting rights in international organizations such as the UN or the IMF is another long-standing issue. Several proposals are on the table \citep{woodward_imf_2007}. We chose voting rights in proportion to a country's population as this is both the most radical proposal and the easiest to explain. 

In the new Methods section on \textit{Design choices}, we now provide guidance on how the selection of policies was made. % TODO!! add references?
~\\

\textit{7. I like the results from section 2.3 and I believe they strengthen the paper. However, I would tone this down a bit. Many potential objections remain of course, and framing an answer in terms of a real-stake petition is not the same as asking participants to actually pay, for example. Also, the description of the various experiments (list, petition) was lacking details and it was therefore difficult to understand what exactly was done.}

We replaced ``Even when framed as a real-stake petition'' by ``Even when framed as a petition that might have real stakes'' and replaced the two occurrences of ``real-stake petition'' in the paragraph by the mere ``petition'', to tone down the expression. % TODO!! add details on list and petition in Methods
~\\

\textit{8. What exactly is the purpose of the conjoint analysis in section 2.3.3 and the prioritization in section 2.3.4? How convincing are the results from section 2.3.5? % on pros and cons 
Is it possible to simulate a public debate? What exactly can we learn from the exercise?}

We added a paragraph on \textit{Prioritization} in the Methods section to explain its purpose compared to the conjoint analysis. It reads: ``The prioritization allows inferring individual-level preferences for one policy over another. This slightly differs from a conjoint analysis, which only allows inferring individual-level preferences for one platform over another or collective-level preferences for one policy over another. Also, by comparing platforms, conjoint analyses may be subject to interaction effects between policies of a platform (which can be seen as complementary, subsitute, or antagonistic) while the prioritization frames the policies as independent.''

Although it is not possible to satisfactorily simulate a public debate in a survey experiment, providing arguments to the respondents helps assessing the extent to which support for the GCS is firm or influenceable. That the support decreases by 11 p.p. after 6 cons and 3 pros were presented indicates that the support for the GCS is somewhat context-dependent, in that a sizable fraction of the population could change their attitude depending on the public discourse about the policy. To avoid conflating our experiment with a real-world media campaign, we replaced the word ``simulate'' by ``mimic'' in the sentence: ``The objective of the `pros and cons treatment' was to mimic a `campaign effect'.''
~\\

\textit{9. I don’t think section 2.4 adds much, if anything.}

% TODO!? take out?
~\\

\textit{10. I read your results on beliefs in section 2.5. a bit different. To me there is strong support for a systematic underestimation in all countries (both for GCS and NR) with the exception of the US.}

This is a fair point. We removed the questionable interpretation. We removed the sentence ``However, the evidence for pluralistic ignorance is limited based on an incentivized question about perceived support'' and we cut ``and underestimation of fellow citizens' support'' in ``Having ruled out insincerity %and underestimation of fellow citizens' support 
as potential explanation for the scarcity of global policies in the public debate, we propose alternative explanations.''
~\\

\textit{11. Figure S6 should be translated.}

We interverted Figure S6 with Figure S16 so that the translated version of the figures appears first and more prominently.
~\\

\paragraph*{Reviewer \#3:}
\textit{This is a bold paper on the important topic of international public opinion on global governance reforms to address some of the most pressing global challenges, in particular, climate change and world poverty. The schemes it considers seem well-considered and a particular strength of their paper is that they go into a lot of the practical details how such reforms could be implemented. I would recommend its publication in Nature Human Behaviour, but subject to some substantial revisions.}

~\\

\textit{Major points:}

\textit{1. One of the main issues is the anchoring of the specific proposals that the authors test in academic or policy debates. To what extent do they reflect such debates, or how do they deviate from them? In their current form, the proposals seem somewhat detached from existing debates (e.g. the idea of population-proportional voting, line 2030), which may limit their appeal in the eyes of some readers and relevant audiences. Better anchoring would be key here.}

% TODO
~\\

\textit{2. While the authors cannot change their country sample anymore, it would be important that they elaborate on their choice of countries and provide better justifications for their concentration on Western countries and bigger countries from the global South. They should also reflect on and analyze to what extent these design choices might be impacting their results, e.g. the greater concern for climate rather than poverty (page 5, line 143-144).}

We added a paragraph on \textit{Design choices} in the Methods section where we explain our selection of countries. In the Global survey, we selected 20 among the largest countries. Some large countries were excluded for diplomatic reasons (Russia) or because the survey company was unable to provide good-quality samples (Iran, Saudi Arabia) while some relatively small countries (Denmark, Ukraine) were included due to a condition of some funders.

In the complementary surveys, we focused on high-income countries as they are the ones losing from global redistributive policies. Therefore, we expect high-income countries to be less supportive of such policies, an hypothesis confirmed in the Global survey (see Figure 2). Surveying less supportive countries is thus a conservative way to assess the support of citizens around the world. 

The main criterion governing our choice of countries was country size. We preselected the largest high-income countries: the U.S., Japan, Germany, France, the UK, Italy, South Korea, and Spain. Our second criterion was the level of support for global redistributive policies. We conservatively selected countries with a relatively lower average support over the four policies tested in the Global survey. While this average support ranges from 66 to 77\% in the selected countries, it is at least 80\% in the remaining preselected countries.

While the choice of countries can certainly impact some results, in our exposition, we emphasize on the similarities rather than the differences across countries. On the greater concern for climate rather than poverty, we emphasize on the fact that both global issues obtain greater concern than a national issue (national inequality). 
~\\

\textit{3. Since Political Scientists are presumably an important target audience for this article, I was a bit struck by the relative lack of conceptualization and theory in the article. While it is not an essential part for articles in Nature Human Behaviour, I would still expect a Political Science piece to offer more in terms of concepts and theorization – at least in the appendix.}

We acknowledge that, coming from environmental economics, we are not properly trained to relate our attitudinal surveys to theories from political science. % TODO! 
~\\

\textit{4. One of the main issues of the article lies in its presentation. The authors show us a lot of survey results, but readers are not adequately led through them or presented with accompanying information such that they can easily grasp what is going on. It would be good to start off with a few punchy headline findings, then structure the paper along those headlines, and always offer all necessary information for readers to grasp what is going on without having to go to the appendix. For example, in Figures S1 and S2, the authors don’t provide the questions, but just reference them. In Table S2, it is not clear what the “open-ended field” is referring to. It would be useful to always cite the questions in the figures (like they do in Figures S27, S28, and S46, for instance). Sometimes such omissions are in the way of understanding the figures, and force readers to go to the questionnaire (e.g. Figure S37). For instance, what is the unit in Figure S25?}

We note that all the Figures and Tables that you cite are intended for publication in the appendix. Furthermore, we made sure that the survey question is accessible in one click from the figures' legends, and that the figures are accessible in one click from the associated questions in the questionnaire. As both will appear in the online appendix, and as we do not want to overcharge the figures' legend with long survey questions, we prefer to keep the current structure when it is not practical. 

Wherever practical (in particular in Figures S37 and S38), we added the original question to avoid the need to go to the questionnaire. % TODO!!
In Figure S25, we specified that the unit was ``in percent of public spending''.

To clarify what the ``open-ended field'' refers to in Table S2, % TODO check number
we added the parenthesis (here in bold) in the Table's legend: ``Effects on the support for the GCS of a question on its pros and cons \textbf{(either in open-ended of closed format)} and on information about the actual support''.
~\\

\textit{5. The quality of online survey samples should not be overstated. While the samples may reflect population proportions along certain criteria, they don’t do so along others – as evident in your own data (e.g. see the voting preference splits in Tables S9 and S10). You should tone down claims to the representativeness of your sample (line 598), acknowledge sample imbalances more clearly, and provide some discussion of opt-in online sampling vs. traditional random sampling methods (including research that specifically explores biases in online panels).}

We toned down claims of representativeness. Namely, we changed ``Appendix G confirms that our samples are representative of the population'' into ``Appendix G shows how our samples compare to actual population frequencies.'' and we replaced ``Tables S9-S10 confirm that our samples closely match population frequencies'' by ``Tables S9-S10 % TODO check number
detail how our samples match population frequencies.'' 

We clearly acknowledged sample imbalances by adding: ``Our samples match well actual frequencies, except for some imbalance on vote in the U.S. (which does not affect our results, as show the results reweighted by vote in the below section \textit{Support for the GCS}).'' We also mentioned biases in online panels in the new sentence: ``Stratified quotas followed by reweighting is the usual method to reduce selection bias from opt-in online panels, when better sampling methods (such as compulsory participation of random dwellings) are unavailable \citep{scherpenzeel_how_2010}.''
~\\

\textit{6. Excluding inattentive respondents is not best practice, as that can lead to additional biases in your data. At the very least, you should keep that data for robustness checks of your main results.}

Before we report new robustness checks based on the extended sample (including the 11\% of inattentive respondents), note that our quotas apply to the final sample. In other words, the extended sample is by construction less representative of targeted socio-demographics than the final sample.
% TODO!!
~\\

\textit{7. Some questions appear somewhat leading (e.g. the phrase “democratise” on line 2030), and indeed many respondents (albeit no majorities) seem to agree (Figure S45). It would be good to discuss such potential shortcomings, at least in the appendix.}

We thrived to use neutral wording in every question. Although we reckon that ``democracy'' is generally associated with a positive view in Western countries, we believe that ``democratise international institutions'' is an accurate description in ``Democratise international institutions (UN, IMF) by making a country's voting right proportional to its population''. Indeed, \citet{woodward_imf_2007} states that this allocation of voting rights would be the most democratic.

As redistribution and internationalism are values associated with the political left, we are not surprised that three times more respondents view our survey left-wing rather than right-wing biased. We believe that this impression is due to the topic rather than the wording of questions. 

To ackowledge this potential shortcoming, however, we added the following sentence in the paragraph on \textit{Data quality} in the Methods section: ``At the end of the survey, we ask whether respondents thought that our survey was politically biased and provide some feedback. 67\% of the respondents found the survey unbiased. 25\% found it left-wing biased, and 8\% found it right-wing biased.''
~\\

\textit{8. From Figure S48 it seems like the order of survey blocks was not randomized, right? If so, you should try to justify why that does not affect your substantial results, as there may well be unexpected effects due to block sequence, respondent fatigue, etc.}

% TODO!! same questions US1 vs. US2
~\\

\textit{Minor points:}

\textit{9. Consider a punchier title than the current one. Something like: “International majorities of citizens support strong global governance to combat climate change and world poverty”.}

Thank you for this great suggestion. We changed the title to: ``International Majorities Genuinely Support Global Redistributive and Climate Policies.''
~\\

\textit{10. The exclusion of indifferent/neutral answers (e.g. page 9, line 263) is understandable, but should be justified better. Also, results should always be presented in full as well, including such neutral answers, at least in the appendix.} % In high-income countries, the global quota policy obtains 64\% absolute support and 84\% relative support (i.e., excluding ``indifferent'' answers).

We added a paragraph on \textit{Absolute vs. relative support} in the Methods section where we justify the preferential use of the relative notion: ``the \textit{relative support} is better suited to assess whether there is more people in favor vs. against a policy.''

We always present results in terms of both relative and absolute support (relegating the absolute results in the appendix). From these results, one can recover the proportion of \textit{Indifferent} and \textit{Oppose} answers. 
~\\


\textit{11. Clarify line 272-273.}

By adding some qualifiers, we hope to have clarified the sentence. The original sentence was: ``A global quota with equal per capita emission rights should produce the same distributional outcomes as a global carbon tax that funds a global basic income'' and the new sentence reads: ``A global carbon tax that funds a global basic income should produce the same distributional outcomes as a global tradable quota with equal per capita emission rights, provided that each country returns equally to its citizens the revenues from emissions trading and to the extent that the carbon price is the same.''
~\\

\textit{12. Clarify what you mean by “absolute” and “relative” support (e.g. page 14), as you don’t seem to be referring to absolute and relative majorities.} % relative support for loss and damages compensation, as 330 approved in principle at the international climate negotiations in 2022 (“COP27”), ranges 331 from 55% (U.S.) to 81% (Spain), with absolute support ranging from 41% to 62%

To clarify the two notions, we added a paragraph on \textit{Absolute vs. relative support} in the Methods section. It reads: ``In most questions, support or opposition for a policy is asked using a 5-Likert scale, with \textit{Indifferent} as the middle option and compulsory response. We call \textit{absolute support} the share of \textit{Somewhat} or \textit{Strong support}. We generally favor the notion of \textit{relative support}, which reports the share of support after excluding \textit{Indifferent} answers. Indeed, the \textit{relative support} is better suited to assess whether there is more people in favor vs. against a policy.''

Indeed, we do not refer to absolute vs. relative majorities, but to absolute vs. relative support (which can each be majority support or minority support).
~\\


\textit{13. Don’t overstate the stakes of your real-world petition (p. 17).}

We replaced ``Even when framed as a real-stake petition'' by ``Even when framed as a petition that might have real stakes'' and replaced the two occurrences of ``real-stake petition'' in the paragraph by the mere ``petition'', to tone down the expression.
~\\

\textit{14. Use Figure S24 for a plausibility check on Figure S23.}

We added ``, consistently with the other variant of the question'' at the end of the sentence: ``Approximately half of the respondents opt to allocate half of the tax revenues to low-income countries.''
~\\

\textit{15. In Tables S14 and S15, you should state what the bracket below the coefficient shows.}

We added ``Standard errors are reported in parentheses'' in these tables' notes.

\renewcommand{\url}[1]{\href{#1}{Link}} 
\bibliographystyle{plainnaturl_clean} % NCCcomment
\bibliography{global_tax_attitudes}

\end{document}