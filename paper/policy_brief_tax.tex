\documentclass[12pt,english]{article}
\usepackage[utf8]{inputenc}
\usepackage{tgpagella} % Palatino text only
\usepackage{mathpazo}  % Palatino math & text
\usepackage[left=1.5in,right=1.5in,top=1.5in,bottom=1.5in]{geometry}
% \linespread{1.5}
% \usepackage[super,comma,sort]{natbib}
\usepackage[round,sort&compress]{natbib}
\usepackage{url} % [hyphens]
\usepackage[hyperpageref]{backref} % back references biblio. Needs latexmk at compilation.
\usepackage[pagebackref]{hyperref}
% \usepackage{multibib} % incompatible with backref
\hypersetup{
  colorlinks=true, % breaklinks=true,
  urlcolor=purple,    % color of external links
  linkcolor=blue,  % color of toc, list of figs etc.
  citecolor=violet,   % color of links to bibliography
}
\usepackage{bm}
\usepackage{indentfirst}
\usepackage{tocbibind}
\setcitestyle{aysep={}} 
\usepackage{amsmath}
\usepackage{amssymb}
\usepackage{eurosym}
\usepackage{amsfonts}
\usepackage{enumerate}
\usepackage{babel}
\usepackage{caption}
\usepackage{supertabular}
\usepackage{tabularx}
\usepackage{float}
\usepackage{dsfont}
\usepackage{fancyvrb}
\usepackage{verbatim}
\usepackage{enumitem}
\usepackage{setspace}
\usepackage{comment}
\usepackage{subcaption}
\usepackage{graphicx}
\usepackage{tikz}
\usepackage{gensymb}
\usepackage{textcomp}

\usepackage{tabulary}
\usepackage{tabularx}
\usepackage{booktabs}
\usepackage{fullpage}
\usepackage{morefloats}
\usepackage{makecell}
\usepackage{lscape}
\usepackage{pdflscape}
\usepackage{longtable}
\usepackage{rotating}
\usepackage{fancyhdr}
\usepackage{tocloft}
\usepackage{titletoc}
\usepackage[export]{adjustbox}
\usepackage[anythingbreaks]{breakurl} % for links
\usepackage{multicol}
\newsavebox\ltmcbox % For net gain table over two columns
%\usepackage[nomarkers,figuresonly]{endfloat} % Figures at the end
%\usepackage[section,below]{placeins} % Floats placed in the section they appear in.
\renewcommand{\floatpagefraction}{.9}

\title{A Global Wealth Tax -- Policy Brief
} 

\author{Adrien Fabre\footnote{CNRS, CIRED. E-mail: fabre.adri1@gmail.com (corresponding author).}} 

\date{\today} 

\begin{document}

\maketitle

\begin{center}
{\textbf{\href{https://github.com/bixiou/global_tax_attitudes/raw/main/paper/policy_brief_tax.pdf}{Link to most recent version}}}
\end{center}

\begin{figure}[h!]
  \caption{Support for a Global Wealth Tax (in percent).}\label{fig:support}
  \makebox[\textwidth][c]{\includegraphics[width=1.2\textwidth]{../figures/OECD/Heatplot_global_tax_attitudes_tax_share.pdf}} 
\end{figure}

\section{Introduction}\label{sec:intro}

\citet{fabre_international_2023} survey attitudes toward global policies in 20 among the largest countries and find near consensus for a global tax on millionaires that would finance low-income countries. The world's richest 1\% (those with a wealth above \euro{}900,000) own 38\% of global wealth \citep{chancel_world_2022}, and the wealth in excess of \euro{}1 million represents 24\% of global wealth. It is logical that the other 99\% massively support taxing the wealthiest. What is more interesting, 90\% of Americans and 92\% of Europeans want to pool at least 10\% of the revenues of a global wealth tax to finance low-income countries. When asked the preferred amount that should finance low-income countries, the average answer is one third.

In this policy brief, we propose a global wealth tax and specify how its revenues should be allocated between countries (Section \ref{sec:design}), we estimate the distributive effects of such a tax (Section \ref{sec:distribution}), and show that it would be strongly supported all over the world (Section \ref{sec:support}).

\section{Design}\label{sec:design}

While action at a global level reduces tax avoidance, taxing wealth at the national level already significantly makes a dent on inequality and generates important revenues. 
The implementation of national wealth taxes should therefore not be delayed by the wait of a global wealth tax. 
We call like-minded political parties of all countries to include a wealth tax proposal in their platform, and implement it when they arrive in power. We propose below a design of wealth tax that can be replicated in any country so that national wealth taxes would be compatible and part of a global wealth tax system. 

A 2\% tax on wealth in excess of \$5 million would raise \$816 billion each year, that is 0.85\% of the Gross World Product (GWP), half of it coming from the U.S. and less than \$1 billion from all low-income countries (28 countries home to 700 million people) combined. This tax schedule is just a basic example of what a small wealth tax can raise: let us call it the ``basic'' tax schedule. \citet{chancel_world_2022} offers a \href{https://wid.world/world-wealth-tax-simulator/}{world wealth tax simulator} that allows estimating the revenues of a custom global wealth tax by world region.\footnote{Similar simulators exist for \href{https://taxjusticenow.org/}{the U.S.} \citep{saez_triumph_2019} and \href{http://taxsimulator.ukwealth.tax/\#/appendix}{the UK} while \citet{kapeller_european_2021} and \citet{oxfam_taxing_2022} offer similar estimates for the EU and many countries, respectively. Despite differences in some assumptions and the data used, all these simulators and calculations yield comparable estimates.} 
For example, a progressive wealth tax with the following schedule could raise 6\% of the GWP: 0.5\% between \$500k and \$1 million, 1\% between \$1 and \$2 million, 3\% between \$2 and \$5 million, 5\% between \$5 and \$20 million, 10\% between \$20 and \$100 million, and 20\% above \$100 million. % TODO: put in table format
% also 6% GWP: 0.5% > 100k; 2% > 500k; 4% > 1M; 5% > 5M; 7% > 20M; 10% > 100M 
Around 6\% of the GWP is therefore the long-term revenues that can be reasonably expected from a strongly progressive global wealth tax, though more can be collected with an even more progressive tax (see e.g. \citealp{chancel_world_2022} who propose a top marginal rate of 90\%). 
Ideally, the tax schedule should be defined by aggregating democratically people's preferences \citep{fabre_french_2022}. 
% We propose a minimal rate at 2% and a custom schedule by country, for which the second schedule given seems a reasonable proposal.
% Half of the minimal tax, that is 1\% for wealth in excess of 5M, should be pooled to finance LIC.
% TODO: look in Chancel and Piketty proposals for global wealth tax, and if they find maximum perennial revenues

We propose that all countries introduce the \textit{basic tax schedule} as a minimum wealth tax, and then top it up with the progressive wealth tax of their choice, such as the one just described. 
Half of the basic tax (that is, a 1\% tax on wealth in excess of \$5 million) should be pooled and transferred to low-income countries. More could be pooled, at the discretion of each country. The World Bank would be the manage the pooled revenues before allocating them to low-income countries' governments.

The revenues need to be allocated in priority to the poorest countries. A good indicator of poverty is the poverty gap: it expresses the minimum amount that would be required to lift everyone above the poverty line. However, allocating revenues in function of the poverty gap would disincentivize countries' governments to effectivly address poverty. To avoid poor incentives, it is preferrable to allocate the revenues in function of a well-measured indicator correlated to the poverty gap. We propose an allocation key based on GDP per capita, according to the prediction of the poverty gap using the GDP per capita, from a linear regression. 

% 1% above 5M for low-income countries + countries are free to top up that with a more progressive wealth tax. Examples of rates and revenues per country.
% revenues allocation

% Proposal of Piketty in last chapter of Capital and Ideology: 0.1% above 0.5 of average wealth i.e. 37k; 1% > 2 = 150k; 2% > 5 = 370k; 5% > 10 = 740k; 10% > 100 = 7.4M; 60% > 1k = 74M; 90% > 10k = 740M
% Mean global wealth: €74k
% Most progressive scenario in WIR (22): 1% > 1M; 1.5% > 10M; 7% > 100M; 15% > 1G; 50% > 10G; 90% > 100G


% \~ 50M millionaires \~ top 1\% \~ 1/3 wealth \~ 3M/millionaire \~ 150T\$ (Table 4.1 (p. 90) de World Inequality Report 2022)
% => Progressive millionaire tax can yield max 5-7T\$/year i.e. 5-7\% of world GDP.
% Tax at 1\% above 1M, 2\% > 2M, 3\% > 5M, 5\% > 20M, 10\% > 100M yield 3.2\% of world GDP.
% Tax at 2\% above 5M, 5\% > 20M, 7\% > 100M yield 1.9\% of world GDP. https://wid.world/world-wealth-tax-simulator/

% China has probably around 1/6 of millionaire wealth, i.e. 1\% of world GDP from max tax. 
% China's exported emissions is probably around 6\% of world total. Counting 1.7\% of world GDP in ETS revenues, that's 0.1\% of world GDP for China's exported emissions.
\section{Distributive effects}\label{sec:distribution}

\section{Support}\label{sec:support}

\citet{fabre_international_2023} asks to representative samples of about 2,000 respondents in 20 countries the support for
``a tax on all millionaires in dollars around the world to finance low-income countries that comply with international standards regarding climate action [which] would finance infrastructure and public services such as access to drinking water, healthcare, and education'', in a 5-Likert scale from \textit{strongly oppose} to \textit{strongly support}. There is absolute majority support in each country, from 53\% in the U.S. to 86\% in China. Figure \ref{fig:support} shows that the relative support (excluding \textit{Indifferent} answers) ranges from 72\% in Denmark to 98\% in China. 

\citet{fabre_international_2023} also run complementary surveys on 2,000 Americans and on 3,000 Europeans (representative of France, Germany, Spain and the UK). Asking almost the same question (the only difference being that the revenues are allocated to low-income countries unconditional on their climate action), they find comparable levels of support. The global wealth tax obtains absolute majority support in each of the five Western countries, with a relative support ranging from 70\% in the U.S. to 90\% in Spain (Figure \ref{fig:global_tax}). %Respondents are also asked to imagine that a global wealth tax is inplace, and w
\citet{fabre_international_2023} also ask respondents what percentage of the global tax revenues should be pooled to finance low-income countries, if a global tax on wealth (in excess of \$5 million) were in place. In each country, at least 88\% of respondents answer a positive amount, with an overall average of 30\% (Germany) to 36\% (U.S., France) (Figure \ref{fig:global_share_mean}).

\begin{figure}[h!]
    \caption[Support for a global wealth tax]{Support for a global wealth tax. \\
    ``Do you support or oppose a tax on millionaires of all countries to finance low-income countries? \\
    Such tax would finance infrastructure and public services such as access to drinking water, healthcare, and education.''}\label{fig:global_tax}
    \makebox[\textwidth][c]{\includegraphics[width=\textwidth]{../figures/country_comparison/global_tax_support.pdf}} 
\end{figure}

\begin{figure}
    \centering 
    \caption[Preferred share of wealth tax for low-income countries]{Percent of global wealth tax that should finance low-income countries (\textit{mean}).} % TODO! n
    \includegraphics[width=1\textwidth]{../figures/country_comparison/global_tax_global_share_mean.pdf} \label{fig:global_share_mean}
\end{figure}

% \begin{figure}[h!]
%     \caption[Support for sharing half of global tax revenues with low-income countries]{Support for sharing half of global tax revenues with low-income countries, rather that each country retaining all the revenues it collects (in percent). (Question \ref{q:global_tax_sharing})}\label{fig:global_tax_sharing}
%     \makebox[\textwidth][c]{\includegraphics[width=\textwidth]{../figures/country_comparison/global_tax_sharing_positive.pdf}} 
% \end{figure}

% TODO: influence on electoral prospects


\renewcommand{\url}[1]{\href{#1}{Link}} % NCCcomment
\bibliographystyle{plainnaturl_clean} % NCCcomment
\bibliography{global_tax_attitudes}

\end{document}