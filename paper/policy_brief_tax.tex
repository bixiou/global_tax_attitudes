\documentclass[12pt,english]{article}
\usepackage[utf8]{inputenc}
\usepackage{tgpagella} % Palatino text only
\usepackage{mathpazo}  % Palatino math & text
\usepackage[left=1.5in,right=1.5in,top=1.5in,bottom=1.5in]{geometry}
% \linespread{1.5}
\usepackage[super,comma,sort]{natbib}
% \usepackage[round,sort&compress]{natbib}
\usepackage{url} % [hyphens]
\usepackage[hyperpageref]{backref} % back references biblio. Needs latexmk at compilation.
\usepackage[pagebackref]{hyperref}
% \usepackage{multibib} % incompatible with backref
\hypersetup{
  colorlinks=true, % breaklinks=true,
  urlcolor=purple,    % color of external links
  linkcolor=blue,  % color of toc, list of figs etc.
  citecolor=violet,   % color of links to bibliography
}
\usepackage{bm}
\usepackage{indentfirst}
\usepackage{tocbibind}
\setcitestyle{aysep={}} 
\usepackage{amsmath}
\usepackage{amssymb}
\usepackage{eurosym}
\usepackage{amsfonts}
\usepackage{enumerate}
\usepackage{babel}
\usepackage{caption}
\usepackage{supertabular}
\usepackage{tabularx}
\usepackage{float}
\usepackage{dsfont}
\usepackage{fancyvrb}
\usepackage{verbatim}
\usepackage{enumitem}
\usepackage{setspace}
\usepackage{comment}
\usepackage{subcaption}
\usepackage{graphicx}
\usepackage{tikz}
\usepackage{gensymb}
\usepackage{textcomp}

\usepackage{tabulary}
\usepackage{tabularx}
\usepackage{booktabs}
\usepackage{fullpage}
\usepackage{morefloats}
\usepackage{makecell}
\usepackage{lscape}
\usepackage{pdflscape}
\usepackage{longtable}
\usepackage{rotating}
\usepackage{fancyhdr}
\usepackage{tocloft}
\usepackage{titletoc}
\usepackage[export]{adjustbox}
\usepackage[anythingbreaks]{breakurl} % for links
\usepackage{multicol}
\newsavebox\ltmcbox % For net gain table over two columns
%\usepackage[nomarkers,figuresonly]{endfloat} % Figures at the end
%\usepackage[section,below]{placeins} % Floats placed in the section they appear in.
\renewcommand{\floatpagefraction}{.9}

\title{Global Wealth Tax -- Policy Brief
} 

\author{Adrien Fabre\footnote{CNRS, CIRED. E-mail: fabre.adri1@gmail.com (corresponding author).}} 

\date{\today} 

\begin{document}

\maketitle

\begin{center}
{\textbf{\href{https://github.com/bixiou/global_tax_attitudes/raw/main/paper/policy_brief_tax.pdf}{Link to most recent version}}}
\end{center}


\~ 50M millionaires \~ top 1\% \~ 1/3 wealth \~ 3M/millionaire \~ 150T\$ (Table 4.1 (p. 90) de World Inequality Report 2022)
=> Progressive millionaire tax can yield max 5-7T\$/year i.e. 5-7\% of world GDP.
Tax at 1\% above 1M, 2\% > 2M, 3\% > 5M, 5\% > 20M, 10\% > 100M yield 3.2\% of world GDP.
Tax at 2\% above 5M, 5\% > 20M, 7\% > 100M yield 1.9\% of world GDP. https://wid.world/world-wealth-tax-simulator/

China has probably around 1/6 of millionaire wealth, i.e. 1\% of world GDP from max tax. 
China's exported emissions is probably around 6\% of world total. Counting 1.7\% of world GDP in ETS revenues, that's 0.1\% of world GDP for China's exported emissions.

\renewcommand{\url}[1]{\href{#1}{Link}} % NCCcomment
\bibliographystyle{plainnaturl_clean} % NCCcomment
\bibliography{global_tax_attitudes}
