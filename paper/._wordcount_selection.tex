Nous appelons les dirigeants mondiaux à examiner des mesures de redistribution mondiale telles que celles décrites ci-dessus lors des réunions de l'ONU, du G20 et des COP. Nous exhortons les décideurs à mettre en œuvre des politiques mondiales redistribuant au moins 1~000 milliards de dollars par an (ou 1~\% du revenu mondial) des pays à hauts revenus vers les pays à bas revenus. Ce ne serait qu'un premier pas vers un monde moins inégalitaire.

Nous sommes un groupe divers d'organisations de la société civile, d'universitaires, de responsables politiques, de syndicats, de groupes religieux, de célébrités et de citoyens du monde. Chacun et chacune est invitée à rejoindre notre mouvement en signant cette lettre ouverte
\footnote{\href{https://global-redistribution-advocates.org/fr/signer-les-petitions/?238=true}{global-redistribution-advocates.org/fr/signer-les-petitions}.}%
, en diffusant son message, en faisant campagne pour la redistribution mondiale ou en faisant un don à la cause. Nous manifesterons notre force et notre détermination dans un an, le jeudi 17 octobre 2024, à l'occasion de la Journée internationale pour l'élimination de la pauvreté. Inscrivez cette date sur vos calendriers, car elle constituera un moment décisif dans la quête mondiale pour la justice et l'équité.