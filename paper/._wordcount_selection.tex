Third, the poorest countries would be granted emissions rights close to the Business as Usual trajectory, as they would bear virtually none of the effort. But this trajectory carries the current (unfair) income distribution and amounts to grandfathering. Indeed, the baseline trajectory\footnote{The baseline trajectory is computed as the ``product of the projected GDP and CO$_\text{2}$ emission intensity''.} yields for example a mere 0.8 tCO$_\text{2}$e p.c. for the DRC in 2030, which is five times less than the world average emissions right p.c. 
Also, note that if the DRC grows faster than projected, it will actually have to pay to the rest of the world for mitigation efforts. This is what is likely to happen to countries like Mexico or Senegal, from our simulation of the net gains of CERF compared to a situation without international transfers (see Figure \ref{fig:gain_gdr_over_gdp_2030}). 