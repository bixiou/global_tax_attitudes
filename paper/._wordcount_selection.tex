Le prix ici est plus faible que dans le modèle et les émissions plus élevées, ce qui se compense au niveau du revenu de base. Si on avait recalé le prix. Si on avait recalé le prix et les émissions, on aurait obtenu des recettes (3.2% PIB) et transferts riches->pauvres (1.3%) un peu plus élevés, plus éloignés du modèle (recettes à 2.5% PIB). Si on avait aussi recalé les revenus au modèle pour y coller parfaitement, on aurait obtenu un poverty gap ex ante de 1.4%, contre 2.1% d'après PIP.