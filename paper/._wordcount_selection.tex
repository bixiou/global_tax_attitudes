At the Paris agreement in 2015, all countries have agreed to contain global warming ``well below +2 $\mathrm{{}^\circ}$C''. To limit global warming to this level,~\textbf{there is a maximum amount of greenhouse gases we can emit globally}.\\
To meet the climate target, a limited number of permits to emit greenhouse gases can be created globally. Polluting firms would be required to buy permits to cover their emissions. Such a policy would~\textbf{make fossil fuel companies pay}~for their emissions and progressively raise the price of fossil fuels.~\textbf{Higher prices would encourage people and companies to use less fossil fuels, reducing greenhouse gas emissions.}\\
In accordance with the principle that each human has an equal right to pollute, the revenues generated by the sale of permits could finance a global basic income.~\textbf{Each adult in the world would receive } [US1, US2: \textbf{\$30/month}; UK: \textbf{\$30 (that is £25) per month}; FR, DE, ES:  \textbf{\euro{}30/month}], thereby lifting out of extreme poverty the 700 million people who earn less than \$2/day.\\
\textbf{The typical }[\textbf{American}]\textbf{ would lose out financially }[US1, US2: \textbf{\$85}, FR: \textbf{\euro{}10}, DE: \textbf{\euro{}25}, ES: \textbf{\euro{}5}, UK: \textbf{£20}]\textbf{ per month}~(as he or she would face [\$115] per month in price increases, which is higher than the [\$30] they would receive). 
\\The policy could be put in place as soon as countries totaling more than 60\% of global emissions agree on it. Countries that would refuse to take part in the policy could face sanctions (like tariffs) from the rest of the World and would be excluded from the basic income. 