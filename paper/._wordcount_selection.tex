Our proposal is twofold. First, voluntary countries would introduce a basic tax schedule on individual wealth: 2% marginal tax rate above $5 million, 6% above $100 million, and 10% above $1 billion. Second, half of the wealth tax revenues would be allocated to lower-income countries: the poorer the country, the more it would receive. This would involve a transfer of 1% of the world GDP from high- to lower-income countries, doubling national income of countries like the DRC. Recent academic surveys show massive support for such policy worldwide (cf. https://bit.ly/Fabre2023).

% The details of our proposal are sorted out in our policy brief, including estimates of transfers for each country (cf. https://bit.ly/GRA_tax). Here are some useful elements:
% - Each participating country could of course set a more progressive tax schedule, on top of the basic tax schedule. Only half of the basic tax revenues would be pooled globally.
% - To prevent middle-income countries from being net contributor to the rest of the world, the pooling of revenues would be phased in linearly for GDP per capita between 75% and 125% of the world average. This rule ensures that most Latin American countries would be net recipients from the policy (with Colombia receiving 0.38% of its GDP).
% - The allocation key (for pooled revenues) shall respect two conditions: that poorer countries receive more in per capita terms, and that the policy does not affect the ranking of countries in terms of national income. We propose an allocation key that respect these conditions, and allocate revenues based on a decreasing function of the GDP per capita, proportional to the predicted poverty gap.
% - A new Fund would be created to collect the pooled revenues and allocate them. It would make sure that recipient countries use the money to fund public services and social protection. If severe violations of human rights or important embezzlement is observed in a recipient country, the Fund could decide to channel the revenues belonging to that country through projects (in that country) overseen by Multilateral Development Banks rather than through its government.
% - In the last decades of the century, as poverty would have hopefully been eradicated, the wealth tax revenues would be used to fund carbon removal.