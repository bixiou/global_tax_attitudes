Major sustainability objectives could be achieved by global approaches to mitigating climate change and inequality. For instance, a global carbon price funding a global basic income, called the “Global Climate Scheme” (GCS), would be an effective and progressive way to combat climate change and poverty. A key condition for the success of global cooperation is the support of citizens in affluent countries for such globally redistributive policies. Yet, prior attitudinal surveys have examined support for global policies. To explore relevant public attitudes, we survey over 48,000 respondents from 20 high- and middle-income countries. The responses reveal strong support for global policies, including the GCS and a global wealth tax aimed at financing low-income countries. A list experiment shows no evidence of social desirability bias, majorities are willing to sign a real-stake petition, and global redistribution ranks high in the prioritization of policies. Conjoint analyses reveal that a political platform is more likely to be preferred if it contains the GCS or a global tax on millionaires. In sum, our findings indicate that global policies are genuinely supported by a majority of the population. Public opinion is therefore not the reason that they do not prominently enter political debates. These results could help draw attention to global policies in the public debate and contribute to their increased prominence.