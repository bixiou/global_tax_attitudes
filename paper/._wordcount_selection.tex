
% FAQ
% - Les émissions vont augmenter si on distribue un revenu de base et les revenus augmentent
% - Qui paient : entreprises ou consommateurs ?
% - Les quotas profitent aux plus riches qui peuvent s'acheter plus de quotas, les plus pauvres bradent leurs quotas pour des rétributions immédiates en obérant des potentialités pour le futur 
% - pas sûr que ça ait fait ses preuves 
% - il y a certaines pratiques qui vont devoir s'arrêter, il faut interdire
% => Le Plan fournit un cadre qui assure des réductions d'émissions adéquates; qui incite les États à prendre des mesures complémentaires type interdiction; on peut remplacer le système d'échanges de permis par une taxe, c'est équivalent. 
% Pourquoi pas taxe progressive ?
% Problème de la fraude / du monitoring.
% Problème de la transparence sur les implications (les gens réalisent-ils les changements de mode de vie nécessaires ?)
% Qu'est-ce qu'on fait de l'EU ETS? On le laisse en place. 
% Les taxes affectées ne sont-elles pas interdites ? => no, e.g. Chirac tax on flying
