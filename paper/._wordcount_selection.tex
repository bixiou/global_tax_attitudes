A global carbon price funding a global basic income, called the ``Global Climate Scheme'' (GCS), would be an effective and progressive way to combat climate change and poverty. Yet, such policy is mostly absent from political platforms and the policy debate. 
  Using surveys on 40,680 respondents in 20 countries covering 72\% of global CO$_\text{2}$ emissions, we document majority support for this and other global policies. Using a complementary survey on 8,000 respondents in the U.S., France, Germany, Spain and the UK, we test several hypotheses that could reconcile strong stated support with a lack of salience of these issues. The complementary analyses show that the stated support is mostly sincere, as a list experiment shows no evidence of social desirability bias, majorities are also willing to sign a real-stake petition, and global redistributive policies rank high %(though not highest) 
  in the prioritization of policies. Conjoint analyses reveal that a progressive candidate would not significantly lose voting share by endorsing the GCS in any country, and may even gain 11 p.p. %in voting intention 
  in France. Likewise, a platform is more likely to be preferred if it contains the GCS or a global tax on millionaires. 
  Accurate beliefs about the level of support for the GCS dismisses the hypothesis of pluralistic ignorance of the support. 
  Universalistic attitudes are confirmed in more general questions, suggesting that the support cannot be explained away by malleable opinion or experimenter demand. In sum, our findings indicate that global policies are genuinely supported by a majority of the population. Public opinion is therefore not the reason that they do not prominently enter political debates.