\documentclass[12pt,english]{article}
\usepackage[utf8]{inputenc}
\usepackage{tgpagella} % Palatino text only
\usepackage{mathpazo}  % Palatino math & text
\usepackage[left=1.5in,right=1.5in,top=1.5in,bottom=1.5in]{geometry}
% \linespread{2}
\usepackage[super,comma,sort]{natbib} % WPcomment
% \usepackage[round,sort&compress]{natbib} % NCCcomment
% \usepackage[natbibapa]{apacite}
% \bibliographystyle{apacite} 
\usepackage{url} % [hyphens]
\usepackage[hyperpageref]{backref} % back references biblio. Needs latexmk at compilation. Incompatible with apacite.
% \usepackage{hyperref}
\usepackage[pagebackref]{hyperref} % Incompatible with apacite.
% \usepackage{hyperref}
% \usepackage{multibib} % incompatible with backref
\hypersetup{
  colorlinks=true, % breaklinks=true,
  urlcolor=purple,    % color of external links
  linkcolor=blue,  % color of toc, list of figs etc.
  citecolor=violet,   % color of links to bibliography
}
\usepackage{bm}
\usepackage{indentfirst}
\setcitestyle{aysep={}} 
\usepackage{amsmath}
\usepackage{tcolorbox}
\usepackage{amssymb}
\usepackage{eurosym}
\usepackage{amsfonts}
\usepackage{enumerate}
\usepackage{babel}
\usepackage{graphicx}
\usepackage{caption}
\usepackage{supertabular}
\usepackage{tabularx}
\usepackage{float}
\usepackage{dsfont}
\usepackage{fancyvrb}
\usepackage{verbatim}
\usepackage{enumitem}
\usepackage{setspace}
\usepackage{comment}
\usepackage{subcaption}
\usepackage{tikz}
\usepackage{gensymb}
\usepackage{textcomp}
\usepackage{placeins} % Floats appear in their section (to use with \FloatBarrier or [section])
\usepackage{tabulary}
\usepackage{tabularx}
\usepackage{booktabs}
\usepackage{fullpage}
\usepackage{morefloats}
\usepackage{makecell}
\usepackage{lscape}
\usepackage{pdflscape}
\usepackage{longtable}
\usepackage{rotating}
\usepackage{fancyhdr}
\usepackage{tocbibind} % Adds biblio to ToC
% \usepackage{tocloft}
\usepackage{titletoc} % Adds title to ToC (use with tocloft?)
\usepackage[export]{adjustbox}
\usepackage[anythingbreaks]{breakurl} % for links
\usepackage{multicol}
\usepackage{lineno}
\usepackage{endnotes}
% \let\footnote=\endnote
% \linenumbers
\makeatletter
\renewcommand\tableofcontents{% removes "Contents" from ToC
    \@starttoc{toc}%
}
\makeatother
\newsavebox\ltmcbox % For net gain table over two columns
%\usepackage[nomarkers,figuresonly]{endfloat} % Figures at the end
%\usepackage[section,below]{placeins} % Floats placed in the section they appear in.
\renewcommand{\floatpagefraction}{.99}
\newenvironment{stretchpars}{\par\setlength{\parfillskip}{0pt}}{\par} % to justify a line


% \newcites{App}{Appendix References}

% \captionsetup[table]{skip=-10pt}
% \begin{document}

% \maketitle

% % \clearpage
% \renewcommand{\bibsection}{\section{\refname}}
% \bibliographystyle{naturemag}
% \bibliography{global_tax_attitudes}
% % \stopcontents

% \end{document}


\title{Global Policies to Phase Out Fossil Fuels } 

% \author{Adrien Fabre$^{1,2}$} % WPcomment
\author{Adrien Fabre\footnote{CNRS, CIRED. E-mail: adrien.fabre@cnrs.fr.}}

\date{\today} % NCCcomment

\begin{document}

\sloppy
\maketitle

% \begin{center}
% {\textbf{\href{https://github.com/bixiou/domestic_poverty_eradication/raw/main/paper/poverty.pdf}{Link to most recent version}}}
% \end{center}

% WPcomment
% \begin{affiliations}
% \item CNRS
% \item CIRED
% \end{affiliations}

% \begin{small} % NCCcomment
\begin{abstract}
% /200 words

\end{abstract}
% \textbf{JEL codes:}
% \textbf{Keywords:} 

\clearpage
\tableofcontents

% \onehalfspacing % NCCcomment

%\clearpage

\section{Where do we stand? What do we need?\label{sec:now}}% NCCcomment


\subsection{A critical assessment of the current regime\label{subsec:criticism}}

The international climate policy regime is laid down in the United Nations Framework Convention on Climate Change (UNFCCC), and its offshoot, the Paris Agreement. The consensus of the international community in favor this regime and its common  temperature target is an immense success: the UNFCCC has been universally adopted, and the Paris Agreement has been ratified by all countries but three (Iran, Libya, and Yemen), before the U.S. withdrawal. As the UNFCCC takes its decisions by consensus, this also results in major limitations: agreements rest on the lowst common denominator and fall short of achieving any substantial progress on international climate action. In this section, we review the current regime and its most likely developments.

\subsubsection{Developed nations taking the lead\label{subsubsec:developed}}
The UNFCCC introduces the distinction between developed and developing nations: the former shall provide financial resources to the latter to promote their sustainable development and climate action. While aimed at sharing fairly the costs of climate action, this classification dates from 1992 and is now outdated. For example, while Singapore, South Korea, Saudi Arabia and Slovenia are all richer than Greece, only the latter is considered by the UNFCCC to be a developed country with financial obligations. This outdated classification is stalling progress in critical negotiations, as newly high-income countries resist being considered developed, and historically developed countries are reluctant to increase their contributions unless all high-income countries do so.

While high-income countries should indeed provide resources for foster climate action in lower-income countries, the determination of required transfers should not rest on an outdated, binary classification; it should be defined using up-to-date, continuous indicators such as the GNI per capita. A simple yet fair rule would be that a country's contributions are to be made in proportion to GNI and entitlements in proportion to population. 

\subsubsection{CBDR\label{subsubsec:cbdr}} 
In its Article 1, the UNFCCC states what is now known as the \textit{CBDR} principle: ``Parties should protect the climate system (...) on the basis of equity and in accordance with their common but differentiated responsibilities and respective capabilities.'' This Article is commendable in its objective to guide the allocation of the burden of climate action between countries and reconcile different burden-sharing principles: common action, equity, historical responsibility, ability to pay, etc. Unfortunately, the CBDR principle only offers vague and inconsistent guidance. For example, does equity refer to equal per capita emissions rights or to something else (equal cost of emissions reductions, equal access to development)? How should we balance rules that result in different allocations of emissions rights, such as common action, equal per capita, historical responsibilities and ability to pay? As the key question of the burden-sharing rule was left unresolved by the CBDR principle and its multiple possible interpretations, countries are not able to agree on binding targets of emissions reductions and financial transfers by country. 

\subsubsection{NDCs\label{subsubsec:ndc}} 

This absence of consensus on burden-sharing led to the system of Nationally Determined Contributions (NDCs), where each country sets its own targets. Countries are not sanctioned if they fail their targets. Countries do not even have to define their target using a common indicator (such as their future cumulative emissions). As NDCs rarely specify a cumulative emissions target, researchers need to formulate hypotheses to assess whether NDCs are jointly consistent with the universally agreed temperature target.\footnote{Note that the temperature is itself vague. Article 2 of the Paris Agreement aims at ``holding the increase in the global average temperature to well below 2~\textdegree{}C above pre-industrial levels and pursuing efforts to limit the temperature increase to 1.5~\textdegree{}C above pre-industrial levels.'' Yet, given the uncertainty around the climate system, this (double) target is not precisely defined: does it mean a 83\% chance to limit global warming to 2~\textdegree{}C? A 67\% chance? A 50\% chance? Each probability is associated with a different carbon budget -- respectively 900, 1,150, and 1,350 GtCO$_\text{2}$ starting in 2020, according to the IPCC (AR6, WGI, p. 39).} 
Even in the most optimistic hypotheses, NDCs are insufficient to meet the temperature target. If all countries respect their NDCs, global GHG emissions should be 51~GtCO$_\text{2}$e in 2030, while 41 Gt would be needed to meet the 2~\textdegree{}C target with a 66\% chance.\cite{den_elzen_updated_2022} According to the \href{https://climateactiontracker.org/}{Climate Action Tracker}, 
current policies and actions correspond to a global warming of +2.7~\textdegree{}C by 2100, a warming may continue to rise beyond that date.

\subsubsection{ITMOs\label{subsubsec:itmo}} 

The article 6.2 of the Paris Agreement allows Parties to exchange Internationally Transferred Mitigation Outcomes (ITMOs). This enables a country to nominally reduce its emissions (the emissions as counted to assess its NDC) by purchasing an ITMO to another country. The latter country will then be credited with the buyer's ITMO emissions. As any bilateral agreement on ITMO is permitted, the use of ITMOs risks reducing the buyers' domestic decarbonization efforts. % Take Switzerland's purchase of ITMOs from Thailand: if they meet their target, these two countries will jointly reach 5.5~tCO$_\text{2}$e per capita in 2030, more than the global level of 4.5~tCO$_\text{2}$e that would be in line with the 2\textdegree{}C target. This observation alone is insufficient to state that these two countries lack climate ambition: for example, if Switzerland and Thailand had the same NDCs but other countries had more ambitious ones, such that the world total is in line with the temperature target, 
Indeed, to the extent that the NDCs do not add up to the global emissions reductions objective, ITMOs would not reflect the required mitigation constraint, and their price of ITMOs will be too low. As a result, ITMOs may propagate a global lack of ambition to countries with otherwise ambitious NDCs, offering a cheap (and less effective) alternative to domestic decarbonization. 
% To the extent that the sum of NDCs' mitigation efforts falls short of the global target, the price of ITMOs will be too low. 
% In other words, the lack of ambition of some countries' NDCs can propagate to other countries through the sale of ITMOs. 
% mention "hot air"?

To prevent ITMOs from weakening domestic action, countries that use them should commit to extra rules, beyond verifying the environmental integrity of the ITMO they buy. In case of linkages between domestic carbon markets, the same rules would be required to the cross-border (or rather, cross-market) purchase of emissions allowances. Let us call \textit{sellers} the countries that are willing to sell ITMOs, and \textit{buyers} the countries they agree to sell them to. The extra rules could be as follows: 
\begin{itemize}
  \item Sellers and buyers should include a cumulative emissions target (i.e. a national carbon budget) in their NDC, decomposed in yearly targets.
  \item The joint carbon budget of sellers and buyers should be compatible with the Paris temperature target. If the group sellers and buyers does not include all countries, their joint carbon budget should correspond to their population share of the world's budget.
  \item The joint carbon budget (of sellers and buyers) in a given year should be lower than their preceding year's joint emissions, by at least (say) 2\%.
\end{itemize}

If a group of sellers and buyers agrees to these rules, they would effectively impose the principle of an equal per capita allocation of emissions rights, at least to govern the allocation between their group and the rest of the world. While alternative allocation principles are possible, the operationalization of the cross-border trading of emissions allowances (or ITMOs) needs to rely on an allocation principle. The inadequacy of NDCs (taken jointly) proves that the global climate regime cannot rely on diverse and self-serving allocation rules to divide the global carbon budget into consistent national targets. 


\subsubsection{Climate finance\label{subsubsec:finance}}

An equal per capita allocation of emissions rights corresponding to the remaining carbon budget would entail transfers approaching 1\% of the world's GDP (or \$1 trillion per year) from high to low emitters, that is from the Global North to the Global South. Taking into account historical responsibilities for emissions, an equal per capita allocation of cumulative (past and future) emissions rights would entail even more transfers (the carbon debt that the North owes to the South is estimated at \$26 trillion\cite{fabre_global_2024}). 

At COP29, the international community reached a compromise concerning the New Collective Quantified Goal (NCQG): Developed countries committed to mobilize \$300 billion per year by 2035 for developing countries for climate action (and countries ``call on all actors'' to mobilize \$1.3 trillion, which would be in line with experts' recommendations\cite{unfccc_new_2024,songwe_raising_2024}). Although the quantum of \$300 billion represents a tripling of the previous climate finance goal, it can be reached through loans (including from the private sector), and does not specify what share should be provided as grants (or grant-equivalent concessional loans). In fact, the current goal of \$100 billion is met with only \$26 billion provided in the form of grants. In theory, the NCQG could be met with the same amount of grants (i.e. North-to-South transfers), or even less. 

In contrast, at COP29, ``India specified that the NCQG should mobilize \$1.3 trillion, of which at least \$600 billion should come in the form of grants and equivalent resources.''\cite{earth_negotiations_bulletin_daily_2024} India, voicing Global South concerns, stated it was ``disappointed in the outcome which clearly brings out the unwillingness of the developed country parties to fulfill their responsibilities. We cannot accept it.'' Transfers aligned with Global South's demands would allow enormous progress towards the Sustainable Development Goals, including climate action but also the deployment of public services and poverty reduction programs. Conversely, an insufficient provision of climate finance does not only infringe on climate justice, it also jeopardizes decarbonization in the Global South, as many countries make their NDC conditional on the adequate provision of climate finance. 

Together with more North-to-South transfers, reforms to the international financial systems are needed to reorient financial resources towards climate action. These reforms are multifaceted and are more likely to be accepted by governments in the Global North than direct transfers, since that they rely on mostly painless, growth-enhancing accounting operations. The government of Barbados (supported by the UN Secretary-General) leads the movement in favor of these reforms. Their ``Bridgetown Initiative'' calls for debt relief for low-income countries, for a new issuance of at least \$650 billion in Special Drawing Rights by the IMF to expand the loans of Multilateral Development Banks (MDBs) to at least \$500 billion per year, and for public guarantees to lower interest rates on sustainable projects in the Global South.\cite{bridgetown_bridgetown_2025} Note that although the Bridgetown Initiative is most famous for its climate finance proposals, it also calls for other reforms, such as a universal carbon price and international taxes on the super-rich funding global public goods. 

While a scaling up of climate finance is crucial, it is not sufficient to decarbonize the world as it does not cap (or directly reduce) emissions. In the worst case scenario, the expansion of low-emissions projects would mostly add up low-carbon infrastructures on top of fossil ones, failing to meaningfully reduce emissions.

\subsubsection{JETPs\label{subsubsec:jetp}}

The last piece of the climate regime worth mentioning is the Just Energy Transition Partnerships (JETPs). JETPs are mechanisms where one developing country essentially commits to emissions reductions through the deployment of renewable energy in exchange for concessional terms on the required loans by a group of developed countries. Four JETPs have been signed so far, involving Indonesia, Vietnam, South Africa, and Senegal.\cite{ha-duong_just_2023} In existing JETPs, the groups of developed countries pledged to offer loans ranging from \$2.5 billion (for Senegal) to \$20 billion (for Indonesia). 

While JETPs offer a promising way to deliver climate finance in a way that guarantees emissions reductions, they currently suffer from several shortcomings. First, their coverage is limited (in terms of sectors and countries). To improve the sectoral coverage and efficiency of JETPs, researchers have proposed to design them as a financial transfer in exchange for a national carbon price. Second, as they focus on emissions reductions rather than sustainable development, JETPs do not contribute to poverty reduction. This concern could be mitigated by JETPs with a higher reliance on grants.\cite{bolton_why_2025} However, a higher provision of grants is difficult to achieve absent a dedicated source of revenue (such as an international tax).
% cite those who say they should be grant based and those who say they should involve pricing/regulations

Lastly, even if JETPs were improved along the previous lines, they would still fail to guarantee that the decarbonization of big emitters like China or the European Union is consistent with required global efforts. 

% I. current model: NDCs (pb: don't sum up to carbon budget, no common burden-sharing norm), ITMO (pb: hot hair), JETPs (pb: limited coverage, even if expanded wouldn't guarantee big emitters like China or EU decarbonize, absence of funding mechanism, don't address poverty); Bridgetown (necessary but insufficient as doesn't cap emissions)

\subsection{Objectives for a truly sustainable regime\label{subsec:objectives}}

Now that we have a critical understanding of the current international climate regime, let us sketch the properties we desire for a new (or improved) regime. We will then be able to assess different proposals in light of these objectives. Here they are:
\begin{itemize}
  \item \textbf{Temperature}. An effective climate regime should achieve the Paris Agreement's temperature target. It should do so by a stabilization of the concentration of each GHG in the atmosphere, and abstain from climate engineering risky bets such as Solar Radiation Managements. This objective would translate into a global carbon budget. For example, the carbon budget could be set at 1,000~GtCO$_\text{2}$ starting in 2025, which corresponds to most likely warming of +1.8\textdegree{}C a 67\% chance to keep global warming below +2\textdegree{}C. 
  \item \textbf{SDGs}. A holistic approach requires solving all humanity's greatest challenges, not just climate change. As explained above, climate justice requires sufficient North-to-South transfers to fund sustainable development (not just climate action). Even though the Sustainable Development Goals (SDGs) and the planetary boundaries would require additional policies and transfers, one important feature of the climate regime is how much climate finance it delivers in the form of transfers to the poorest and improved market conditions. This can be measured through SDGs indicators or GDP per capita in low- and lower-middle-income countries.
  \item \textbf{Efficiency}. As stated by the UNFCCC since 1992,\cite{unfccc_united_1992-1} ``measures to deal with climate change should be cost-effective so as to ensure global benefits at the lowest possible cost.'' Economists have shown that ensuring cost-effectiveness require an economy-wide carbon price, uniform across sectors and countries. This fundamentally results from the fact the social cost of emissions are independent from their source or location, therefore they should be priced uniformly. %Efficiency can then be measured by the dispersion of (implicit) carbon prices across countries and sectors. 
  Note that this argument in favor of carbon pricing does not preclude other, complementary policies: these have also been shown to be optimal by economic analysis.\cite{stiglitz_addressing_2019}
  \item \textbf{Acceptability}. An interesting international agreement is one that has good chances to be accepted by most countries. To measure the success of a proposal, we can use the share of global emissions that are covered by participating jurisdictions. Different elements contribute to acceptability: 
  \begin{itemize}
    \item \textit{Progressivity at the top}. If costs are concentrated on the richest households, the regime can benefit the majority in each country while addressing the excessive level of inequality.
    \item \textit{No loss in middle-income countries}. Countries whose population is not rich should not lose from an international climate policy. To assess whether a country loses or not, we should compare its situation in the new regime compared to the status quo. If we synthesize a country's situation by the carbon budget it is granted, a country would lose if its carbon budget is lower than their unconditional NDC completed by the ambitious emissions trajectory that the country currently envisions. % TODO? add? This challenges the blunt application of the equal per capita allocation of emissions rights, as it would entail transfers from countries with a carbon footprint greater than the world average to countries with a footprint lower than the average. Yet, some middle-income countries such as China, Iran, or South Africa, have a carbon footprint greater than the world average. 
    \item \textit{Win-win}. While (per the SDG objective) transfers would be required from high-income countries, this does not necessarily mean that these countries' population would lose out. First, because (as stated above), redistributive policies can concentrate the costs on the richest households in their country. Second, everyone would benefit from a stabilized climate and from a world where SDGs are met. For example, sustainable development would spur global demand, including for advanced technology and low-carbon exports from industrialized economies. Third, while transfers imply a loss compared to the situation with the same worldwide decarbonization efforts and without international transfers, the latter situation is unlikely (as transfers are necessary to promote decarbonization in the Global South). As above, the situation that should be used as a point of comparison is the status quo where the country's carbon budget corresponds to its unilaterally planned emissions trajectory and where there is no international trade in emissions allowances. To the extent that transfers are the counterpart of the purchase of emissions allowances, a new climate regime could be a \textit{win-win} for all participating countries, as they would all reap the efficiency gains of an optimal location of emissions reductions.
  \end{itemize}
\end{itemize}
% objectives: SDGs (=> transfers / climate finance), 2°C, efficiency (say it's in UNFCCC 92, => uniform price), acceptability (=> concentrate costs on richest, coalition of willing, win win thx to efficiency, LMIC not losing (compared to carbon budget corresponding to their unconditional ambitious scenario)), incentives to expand (remove?)

\paragraph{Coalition of the willing}. International negotiations have shown that it is illusory to seek universal agreement for an ambitious agreement. Therefore, political realism requires pushing for proposal that do not get accepted by all countries, and thus, that may also fail to deliver on the climate target, as countries outside the coalition would not fulfill their part of the temperature objective. If oil exporting countries, representing 25\% of current emissions, do not join the coalition, temperature in 2100 would be about 0.3\textdegree{}C higher than with a universal participation to decarbonization efforts. While this outcome would be a partial renouncement to some objectives (full acceptability, temperature target), it is probably the only type of outcomes that is accessible given the political balance of power.


\section{A Fossil-Free Union\label{sec:ffu}}

Having in mind the shortcomings of the current regime as well as the objectives of a new regime, we are now equipped to propose an international agreement to phase out fossil fuels in way that is cost-effective, promotes sustainable development, and acceptable to most countries. % TODO! elite support: Rajan, vdL, Dasgupta, Blanchard-Tirole...

\subsection{The principles for a Fossil-Free Union\label{subsec:principles}}

The Union would be open to any country, as well as subnational entities (such as U.S. states). 

\paragraph{Emissions Trading System}
The Union would put in place an international Emissions Trading System (ETS), that would add up to existing ones. All sectors except agriculture and land-use (LULUCF) would be covered. In particular, the ETS would cover (domestic and international) aviation. International shipping could also be covered, replacing the system established by the International Maritime Organisation. The ETS would cover all gases emitted in industrial or energy processes, as in the Korean ETS. Namely, the ETS would cover CO$_\text{2}$, N$_\text{2}$O, PFCs, SF$_\text{6}$, HFCs, as well as methane emissions from industrial processes, fossil fuel extraction, and waste management (but not methane emissions from agriculture). % TODO which equivalence between GHG

Complementary policies such as the Tropical Forest Forever Fund would be needed to cover LULUCF sectors. This is important to avoid carbon leakage that would substitute fossil fuels by biomass obtained through deforestation.

Emissions allowances would be fully auctioned by an ad hoc international authority to polluting companies upstream of the supply chain. Carbon pricing revenues would be returned to countries based on their yearly carbon budget.

\paragraph{National carbon budgets}
Each country would be granted a carbon budget between the starting year (say 2030) and net-zero (say in 2080). 

Each country would then describe how they would divide their carbon budget intertemporally. As such, the yearly trajectory of the Union's emissions over the next fifty years would be known at the starting year. Each country would be relatively free on the intertemporal breakdown of their carbon budget, though this choice would have to respect some constraints, developed in Section \ref{subsec:implementation}, and related to the rules to avoid hot hair proposed in Section \ref{subsubsec:itmo}.

\paragraph{Adjusted per capita allocation}
By default, each country would be granted a carbon budget corresponding to an equal per capita share of the remaining global carbon budget. This allocation can be understood as an equal right to pollute for each human, irrespective of their country. Such an allocation would induce international transfers from agents (people or countries) with a carbon footprint higher than the world average, to agents with a lower carbon footprint.

As future population is unknown (and can be affected by policy choices), the benchmark per capita carbon budget would be based on the population share taken at the starting year. Then, some departures from the benchmark would allow adjusting to special circumstances.

To prevent transfers flowing from lower-income countries to high-income countries, high-income countries would be granted a carbon budget corresponding to their ambitious decarbonization pathway. In particular, the European Union would be granted emissions allowances in line with its NDC, with 90\% emissions reductions in 2040 (compared to 1990), and net-zero in 2050. This represents less than half of EU's benchmark equal per capita share.

To prevent middle-income countries from being net contributors of international transfers, countries would be allowed to propose further departures from the benchmark allocation, to the extent that the Union's carbon budget is respected. These departures from the benchmark need to be agreed by a qualified majority of participating countries, i.e. countries representing a majority of emissions within the Union. % TODO! refine

In particular, middle-income countries with emissions per capita above the world average, such as China, Iran, or South Africa, could be granted a carbon budget equal to the cumulative carbon footprint corresponding to their own ambitious decarbonization pathway.

\paragraph{Universal cash transfer} 
The Union would encourage countries to return the ETS revenue to the population through an equal cash transfer. In particular, the Union would develop standard and provide technological resources to distribute the cash transfer. 

The equal cash transfer would compensate people for the rise in fossil fuel prices. The transfer would reflect each person's equal right to pollute, as it would work as if the person would have sold this right to polluting companies at the carbon price.

Countries that choose not to distribute all revenue through a cash transfer would have to prove that they spend it %on social protection, public services, and infrastructure, 
in a way that leaves no one behind.

\subsection{Likely participating countries\label{subsec:scenarios}}

Countries that would not lose from the policy are expected to join: these include all low- and middle-income countries, as well as high-income countries with an ambitious decarbonization plan. The map in Figure \ref{fig:participation} shows which countries are likely to join the Union. These countries represent 74\% of current emissions. % (club_emissions_share <- sum(df$emissions_2025[setdiff(df$code %in% c(union_high, "TUR", "CHL", "URY", "TUN", "UGA", "SDN", "MYS", "UKR", "TKM"), "CAN")])/sum(df$emissions_2025)) # 74%

\begin{figure}
  \centering \caption{Countries likely to participate in the Fossil-Free Union.}\label{fig:participation}
  \includegraphics[width=\textwidth]{../figures/maps/participation_FFU_wo_CAN.png} 
\end{figure}
% /!\ Turkey, Iraq, Iran, and some other MEA countries are not counted in the Union in the code

\subsection{Allocation of emissions rights\label{subsec:allocation}}

Policies currently implemented in the prospective Union would imply emissions of 792~GtCO$_\text{2}$ over 2030--2080, while current NDCs (without accounting for long-term targets) would imply 708~GtCO$_\text{2}$. In both cases, emissions are expected to continue after that date.\footnote{The data on emissions by region from current policies and NDCs (with and without long-term targets) is given by van de Ven et al. (2023)\cite{van_de_ven_multimodel_2023}. Post-2030 action is modelled by extending the average rate of change in emissions intensity of GDP from 2020 to 2030. The global carbon budget (and associated equal per capita rights) follows the scenario SSP226MESGB of Gütschow et al. (2021).\cite{gutschow_country-resolved_2021}} % Post-2030 action is then modelled by measuring the average rate of change in emissions intensity of GDP from 2020 to 2030 in each region and assuming emissions-intensity reduction rates will remain the same after 2030. 
In contrast, enforcing an equal per capita share of the remaining carbon budget would limit Union's emissions to 653~GtCO$_\text{2}$ over the period and would achieve net-zero emissions by 2080.

To determine the ``non-losing'' carbon budget, below which a country could be considered losing, we proceed as follows. For countries with emissions per capita lower than the world average, we use a Contraction \& Convergence benchmark, whre emissions rights per capita start at their value in 2030 for current policies and linearly converge to the equal per capita share in 2050. This benchmark implicitly assumes that countries with relatively low emissions would consider as beneficial to their development the pathway that starts with current policies, gradually grants them extra resources for sustainable development (by rights converging to an equal per capita share of the global sum), and then make them to follow the world decarbonization trend. 
For high-income countries and for China, we use the cumulative emissions implied by their NDCs and long-term targets.\footnote{For China, the value is in line with the domestic 2\textdegree{}C target scenario developed at Tsinghua University.\cite{he_towards_2022}} Doing so implicitly assume that these countries have the domestic capacity to deliver their long-term targets on their own. These carbon budgets imply slightly more rights than the equal per capita share for China, and less for high-income countries. 

Table \ref{tab:budgets} present the cumulative emissions implied by current policies, \textit{non-losing} budgets, equal per capita ones, and the proposed allocation. The proposed allocation departs from the equal per capita one for China and Western Europe only, which are both allocated a carbon budget corresponding to their NDCs and long-term targets. It is worth noting that the proposed allocation grants Eastern Europe, Japan, and South Korea with their equal per capita share. Indeed, these countries are less rich or have significantly higher emissions per capita than the world average (contrary to Western Europe). Therefore, there is few concern that these countries would turn net recipient from international transfers and no need to apply to them the same exception as for Western Europe. % In addition, granting these countries with their equal per capita share improves acceptability. 

\begin{table}[h]
  \caption{Carbon budgets over 2030--2080 for a 1.8\textdegree{}C trajectory (in GtCO$_\text{2}$).\label{tab:budgets}} 
  \makebox[\textwidth][c]{
\begin{tabular}[t]{cccccccccc}
  \toprule
   & Africa & China & \makecell{Latin\\America} & India & Europe & \makecell{Japan \& \\South Korea} & \makecell{Other\\Asia} & \makecell{\textit{Fossil-Free}\\\textit{Union}} & \textit{World} \\
  \midrule % TODO? Add BAU?
  \textbf{Current policies} & 88 & 226 & 80 & 143 & 31 & 46 & 179 & \textit{792} & \textit{1,214} \\
  \textbf{Non-losing} & 124 & 147 & 57 & 115 & 22 & 11 & 104 & \textit{589} & \textit{786} \\ % emissions_cc for all but WEU, JPN, SKO, CHI: emissions_ndc, and world as a whole: emissions_cc
  \textbf{Equal p.c.} & 144 & \textbf{134} & 62 & 135 & \textbf{49} & 16 & 113 & \textit{653} & \textit{754} \\
  \textbf{Proposal} & 144 & \textbf{147} & 62 & 135 & \textbf{23} & 16 & 113 & \textit{640} & \textit{754} \\        
  \bottomrule\\[-0.81em]
\end{tabular}}
\end{table}% TODO! adjust China's budget to account for carbon footprint

Table \ref{tab:budgets} shows that the sum of the Union's proposed carbon budget is 13~GtCO$_\text{2}$ (or 2\%) lower than its equal per capita share of the world's carbon budget. The unallocated emissions allowances could be used to grant additional countries some extra carbon budget. For the moment, we have only modelled such a departure for China, but a similar one should be granted to other fossil-dependent middle-income countries with relatively high emissions: Algeria, Kazakhstan, Iraq, Iran, Libya, Mongolia, South Africa, and/or Turkmenistan. These countries currently represent 5.6\% of global emissions and 3.4\% of global population, translating into an equal per capita carbon budget of 22~GtCO$_\text{2}$; so the extra allowances would cover their needs. 

% \begin{figure} 
%   \centering \caption{CO$_\text{2}$ emissions allowances for selected regions (in tCO$_\text{2}$ p.c.).}
%   \includegraphics[height=.6\textheight]{../figures/policies/fossil_free_union_emission_trajectories.png} % TODO! remove South America from legend, have Africa converge to 0, avoid allowances > current policies
% \end{figure}

% TODO! proposed trajectories

\subsection{A win-win deal\label{subsec:winwin}}

Each country colored in Figure \ref{fig:participation} would have an interest to join the Union: 
\begin{itemize}
  \item Every country would benefit from a stabilized climate, and from the guarantee that all countries in the Union decarbonize.
  \item Most countries (including all countries likely to join) would be granted a carbon budget sufficient to avoid a loss from the status quo. Exceptions include Russia, Saudi Arabia and other high-income countries from the Gulf with low climate ambition. Even the U.S., Australia and Canada would enjoy a non-losing carbon budget.
  \item Lower-income countries would receive transfers from the rest of the world, spurring their sustainable development.
  \item Countries with an important low-carbon industry, such as East Asian countries, would gain from the stronger demand for these goods.
  \item High-income countries would benefit from the efficiency gains allowed by international carbon trading.
  \item Large representative surveys show strong public support in favor of the Fossil-Free Union, even in high-income countries when transfers are presented as a loss and the amount of transfers is specified. For example, there is 54\% support in the U.S. and 76\% in Western Europe.\cite{fabre_international_2023} % TODO update citation
  Moreover, academic research shows that political programs containing the Fossil-Free Union are preferred to similar programs without it by 58\% to 60\% of citizens in Western countries, suggesting that candidates at an election may win vote intentions by campaigning on the proposal.\cite{fabre_international_2023}
\end{itemize}

\subsection{A ratcheted-up ambition\label{subsec:ambition}}

\paragraph{Global temperature reduced by more than half a degree}
Current policies correspond to a temperature trajectory reaching +2.7\textdegree{}C in 2100 (see Section \ref{subsubsec:ndc}). 

While the carbon budgets proposed in Section \ref{subsec:allocation} are based on a +1.8\textdegree{}C trajectory, to the extent that the Union is not universal, they would imply a higher temperature trajectory. The higher temperature achieved is not only due to countries outside the Union exceeding their equal per capita share of the +1.8\textdegree{}C carbon budget. It is also due to higher emissions within the Union that would be efficient in case of universal participation. Indeed, as non-participating countries are those with the largest emissions per capita, their absence from the Union decreases the Union's carbon price below its cost-effective level to achieve +1.8\textdegree{}C. In other words, the non-participation of the largest emitters (in per capita terms) prevents the efficiency gains that would occur should they participate: in this case, they would buy emissions allowances to the rest of the world, raising the demand for allowances and hence the carbon price, and the rest of the world would decarbonize faster (in exchange for transfers). 

If the whole world decarbonized at the same rate as the Union, the temperature would reach +1.95\textdegree{}C in 2100. Assuming that emissions in non-participating countries would follow the trend from current policies, the temperature increase expected in 2100 is +2.15\textdegree{}C. 

Therefore, the Fossil-Free Union studied here would bring a reduction of global temperature in 2100 of half a degree. Of course, a lower temperature target could be reached by choosing a smaller carbon budget: the Union's decarbonization trajectory is a policy choice. % a Fossil-Free Union based on a +1.8\textdegree with the participation scenario of Figure \ref{fig:participation} and the would 

\paragraph{A sufficiently high carbon price}
It is important that the Union's carbon price be sufficiently high, for different reasons. First, as transfers are proportional to the carbon price, a substantial carbon price is required to deliver meaningful transfers, finance sustainable development, and convince lower-income countries to join. Second, a low carbon price would entail few decarbonization incentives and indicate that the carbon budget is too high, i.e. the ambition is low. Third, a low price could result in a price hike if a large emitter (like the U.S.) decides to join the Union. This, in turn, would hinder the interest that high-income countries would have in favor of an expansion of the Union to new countries, as their contributions would increase along with the price.

To make sure that the price is sufficiently high, the Union could implement a (steadily increasing) carbon price floor. However, this is not our favorite option. Indeed, adding a price floor would redefine and obscure the distributive effects implied by the carbon budgets. By inducing a price higher than the equilibrium market price, the price floor would entail emissions lower than the yearly allowances and be equivalent to a reduction of each country's emissions allowances. While countries recipient of transfers would be cushioned against these lower allowances through larger transfers, contributing countries would lose out compared to the situation without a binding price floor. This could jeopardize an agreement on the proposed allocation, that has been designed so that industrialized countries neither gain nor lose from the policy. Furthermore, given that we can hardly predict whether the price floor would be binding or whether the equilibrium price would be higher, we can hardly redefine the proposed allocation to mitigate the effects of the price floor.

Instead of a carbon price floor, we propose rules to ensure that there is no excess allowances and that the carbon price increases sufficiently overtime. These rules correspond to the rules sketched in Section \ref{subsubsec:itmo} and apply to the intertemporal allocation of national budgets. These rules are that the Union's allowances should not exceed its joint emissions at the starting year, and that they have to decrease every year at a minimum rate of, say, 2\%. 

If countries cannot agree on an intertemporal allocation of their emissions allowances that respect these rules, the Union's scientific council would propose how to allocate allowances intertemporally in a way that maximizes welfare, thereby preserving the interests of each country. In case the Union rejects the proposal of the scientific council, a price floor would be implemented (say, starting at \$10/tCO$_\text{2}$ and increasing by \$10 each year). The threat of a strong price floor should help countries find an agreement.

\subsection{Implementation and governance\label{subsec:implementation}}
% TODO: phase in country joins

\subsection{Limitations of the proposal\label{subsec:limitation}}

\subsection{Variants of the proposal\label{subsec:variants}}

% implementation, governance: experts/scientific council from each country to assess allocation and price; treaty negotiated by main players (CH, EU, IA, UA, BR) with different price trajectories / rights allocation depending on who's in
% Pb de FFU: 1. la répartition temporelle des droits doit tenir compte du prix plancher, qui n'est pas connu (participation partielle induit différence distributive par rapport à simple marché). => fixer le prix (en fonction de qui participe); 2. inadapté circonstances changeantes e.g. Chine riche; 3. EU doesn't have interest that US joins 
% Alternatives: differentiated floors (show equivalence, FFU better because efficient); GCP (isn't flexible enough e.g. regarding allocation in first years); C&C: close, main difference is no grandfathering for HIC

\section{A Sustainable Union}

% Conclusion: What next? Scientific council to conduct in-depth analyses and answer to countries' questions

\appendix

\section{The draft treaty}

% III. sustainable union: stronger incentive for LMIC to join; can simplify allocation (transfers purely based on GNI); count anti-climate legislation as negative contributions; explain treaty; distributive effects.
% Annex: Treaty/ies

% Difficult to reconcile:
% 1. countries pay marginal footprint to RoW at universal eq price
% 2. submedial countries are not net contributor in partial union (they lose when price=floor if their planned emissions > rights * club/world average ~ rights * 3/4, when club's rights sum to equal pc). Can be presented as ~1.5°C trajectory with extra rights for submedial or ~1.8°C trajectory with less rights for poor and rich.
% 3. no discontinuity in price/transfers when union expands (so that members have interest in expansion)

% Options:
% a. difference between universal and union eq price is collected/recycled nationally for C-submedial countries and the share over the global average is transferred to submedial countries for supermedial ones. Fails 1,2,3.
%>b. transfers are unrelated to emissions, based on GNI. Fails 1.
% c. transfers based on GNI + departure from expected emissions reductions. May fail 2.
% d. union price is universal eq, extra transfers for submedial countries (to the expense of high and low income ones). Fails 3.
% e. countries with GNI pc < 1.5 average can opt out from revenue sharing. Fails 1 (and rights left to others unpredictable).
% f. keep carbon price floor low. Fails 3, also 2 but mitigated.
%>g. revenue share of surmedial LMIC augmented by their footprint/club average at t=0. Fails 2 to the extend emissions decrease slower than in the rest of the club. 
%>h. Differentiated rights + high price floor. May fail 2 (if differentiated prices don't accommodate enough departure, i.e. if AFR doesn't renounce to enough of its rights).
%>>i No price floor (except for countries who repeal climate laws), MIC rights get as much rights as their needs. Fails 1 (equivalent to higher temperature target), 3 (same issue with Sustainable Union). 
%>j A price (rather than quantity) target with equal pc sharing except for countries < 1.2 average GDP which can be exempted from revenue sharing (with phase out up to 1.5 average) and rule to avoid HIC being net receiver => clear, rule-based, only issue is that price may be too low; and if automatically increased then too big an advantage for exempted countries as they increase emissions without increasing transfers (i.e. fails 1 for exempted countries => could be avoided by adding payments if departure from expected emissions reductions) TODO give this as alternative

% Pq je suis passé de GCP à FFU? 1. pour ne plus dire que EU perd (mais j'peux parler de droits équivalents dans GCP), 2. pour donner de la flexibilité sur les departures (par ex éviter de donner un avantage trop grand à la Chine, qui va se décarboner plus vite que le club) et sur la répartition temporelle (donner plus tard à Inde pour éviter qu'elle ne sorte), 3. parce que je renonçais à un prix conséquent les premières années (hot air)
% Pb de FFU: la répartition temporelle des droits doit tenir compte du prix plancher, qui n'est pas connu (participation partielle induit différence distributive par rapport à simple marché). => fixer le prix (en fonction de qui participe)
% Avantages du GCP? 1. permet un prix plus élevé au début car les surmédiaux sont ajustés relativement à moyenne du club et pas moyenne mondiale comme ds FFU (il le faut pour éviter une augmentation trop rapide lorsque le marché devient binding ou si US rejoignent), 2. rule-based plutôt que discretionary, 3. permet de s'ajuster aux circonstances changeantes (par ex si Chine a rattrapé EU en PIB/hab), 

% Ask to each country: Imagine int'l ETS (membership, price unknown, though probably just Global South, no price floor): 
% How much allowances they would want? (minimum acceptable, fair, ideal number; give justification for each)
% Total/world allowances they would want until net zero (minimum, ideal, maximum).
% Allocation rules they'd accept: equal pc, cumulative equal pc, CC, BAU for MIC, grandfathering
% => work out a solution based on the answers

% - Pricing:
% -- equivalence differentiated prices - rights but uniform price better, countries can always add policies to e.g. subsidize fuels if they wish
% -- for acceptability + fairness, we need departures from equal pc
% -- simulations using NICE
% -- feasibility of global redistributive policy: surveys, informal support of diplomats (even EU saying govts is pb, they personally like)

% Start paper on Fossil-Free Union, then expand.

% Argue in favor of FFU: guarantee the union respects its targets; determines burden sharing.
% Carbon price floors very important as it's binding in first years. Must be complemented by the obligation to pay higher price if country reduces its domestic carbon price / climate regulations. In effect, acts as a generalized JTEP (more favorable to LDCs). 

% PB of FFU: in first phase with carbon price floor, China loses as it has less rights among the union than its emissions share in the union. => No, I think it's been computed so that China doesn't lose. The above effect is mitigated by the carbon trade + larger rights.
% PB of SU: same issue if partial participation => No because China is allowed to skip the redistribution part and just apply the FFU.
% => Think of treaty rules
% Le prix plancher doit correspondre au prix qui serait atteint en cas de participation universelle, pour qu'il y ait suffisamment de réductions d'émission, de transferts, et que les états membres n'aient pas un désintérêt à ce que les US joignent

% Pb du FFU: EU has no financial interest that US joins (as long as EU is a net contributor => determine loss of EU when US joins; determine date of US joining that makes EU indifferent). Indeed, EU pays price*(emissions - rights). If US joins, price will increase and (e - r) be constant. If instead of fixing the rights we fixed the price, EU would have interest that US joins, as its rights (equal to union average) would increase. EU would also have interest of US joining in the case 1% GNI return in fct of pop.
% => What rules leave no one worse off when big emitters join? What rules leave big emitters indifferent when bigger emitters join?

% Global policies to phase out fossil fuels 
% - Principles: non-universal, open, redistributive, non-dichotomic
% - Climate justice -> justice (burden-sharing options, criticism of CERF, grandfathering would be plausible in absence of broader inequality/justice hence can make sense within country)
% - Finance options (multilateral guarantees, climate bonds, money creation - Dafermos...)
% - Common standards (ships, cars, banning coal or oil exploration)
% - Supply-side policies (critique of fossil non proliferation)
% - Pricing:
% -- equivalence differentiated prices p_i = p + a_i where mean(p_i)=p  and differentiated rights per capita r_i : r_i = p*e - a_i*e_i, where e_i is country's emission pc and e global average.
% -- uniform price better, countries can always add policies to e.g. subsidize fuels if they wish
% -- for acceptability + fairness, we need departures from equal pc
% -- simulations using NICE
% -- critiques of other proposals: Pezsko, Edenhofer, Nordhaus, Piketty, rationing, proposals lacking sectoral coverage (CBAM sectors) or redistribution (restricting ITMOs to high-integrity countries)...
% -- feasibility of global redistributive policy: surveys, informal support of diplomats (even EU saying govts is pb, they personally like)
% -- call for expert contributions and common proposal

% \begin{figure}[b!]
%   \caption[Title.]{Caption
%   }\label{fig:antipoverty_tax_7}
%   \makebox[\textwidth][c]{\includegraphics[width=\textwidth]
%   {../figures/antipoverty_2_tax_7_average.pdf}}
% \end{figure}

% \begin{figure}[h!]
%     \caption[Title.]{Caption.}\label{fig:kenya}
%   \begin{subfigure}{.5\textwidth}
%     \caption[]{Caption_a.}\label{fig:kenya_policies}
%     \includegraphics[width=\textwidth]{../figures/Kenya_policies.pdf}
%   \end{subfigure} \;
%   \begin{subfigure}{.5\textwidth}
%     \caption[]{Caption_b.}\label{fig:kenya_cap}
%     \includegraphics[width=\textwidth]{../figures/Kenya_cap.pdf}
%   \end{subfigure}
%   \\ \quad \\
%   \begin{subfigure}{.5\textwidth}
%     \caption[]{Caption_c.}\label{fig:kenya_tax}
%     \includegraphics[width=\textwidth]{../figures/Kenya_tax.pdf}
%   \end{subfigure} \;
%   \begin{subfigure}{.5\textwidth}
%     \caption[]{Caption_d.}\label{fig:kenya_floor}
%     \includegraphics[width=\textwidth]{../figures/Kenya_floor.pdf}
%   \end{subfigure}
% \end{figure}  


\section{Discussion\label{sec:conclusion}} 

% \begin{methods}  % WPcomment
  \begin{small} % NCCcomment
% \section*{\normalsize Data and code availability}

% \end{methods} % WPcomment
\end{small}  % NCCcomment

% \theendnotes

\renewcommand{\url}[1]{\href{#1}{Link}} % NCCcomment
\bibliographystyle{plainnaturl_clean} % NCCcomment
\bibliography{global_tax_attitudes}
% \clearpage
\section{Raw results% from the complementary surveys
}\label{app:raw_results}
% /!\ Do not replace by app_desc_stats_US1 as the latter also contains figures that are already in the main text
% TODO? add country-specific prioritization? No, it's in (separate) country appendices.
% TODO! add share who click on info or reminder
% TODO! Appendix Sources or at least clean up specificities.xlsx

Country-specific raw results are also available as supplementary material files:  \href{https://github.com/bixiou/global_tax_attitudes/raw/main/paper/app_desc_stats_US.pdf}{US}, \href{https://github.com/bixiou/global_tax_attitudes/raw/main/paper/app_desc_stats_EU.pdf}{EU}, \href{https://github.com/bixiou/global_tax_attitudes/raw/main/paper/app_desc_stats_FR.pdf}{FR}, \href{https://github.com/bixiou/global_tax_attitudes/raw/main/paper/app_desc_stats_DE.pdf}{DE}, \href{https://github.com/bixiou/global_tax_attitudes/raw/main/paper/app_desc_stats_ES.pdf}{ES}, \href{https://github.com/bixiou/global_tax_attitudes/raw/main/paper/app_desc_stats_UK.pdf}{UK}.

\begin{figure}[h!]
    \caption[Absolute support for global climate policies]{Absolute support for global climate policies. \\ Share of \textit{Somewhat} or \textit{Strongly support} (in percent, $n$ = 40,680). The color blue denotes an absolute majority. See Figure \ref{fig:oecd} for the relative support. (Questions \ref{q:scale}-\ref{q:millionaire_tax} of the global survey. Reproduced from \citealp{dechezlepretre_fighting_2022}, Figure A20.)} 
    \makebox[\textwidth][c]{\includegraphics[width=1.2\textwidth]{../figures/OECD/Heatplot_global_tax_attitudes_positive.pdf}}\label{fig:oecd_absolute}% with dependence on others (absent from OECD): Heatplot_burden_share_all_positive_countries
    {\footnotesize *In Denmark, France and the U.S., the questions with an asterisk were asked differently, cf. Question \ref{q:burden_sharing_asterisk}. } 
\end{figure}

\begin{figure}[h!]
    \caption[Comprehension]{Correct answers to comprehension questions (in percent). (Questions \ref{q:understood_gcs}-\ref{q:understood_both})}\label{fig:understood_each}
    \makebox[\textwidth][c]{\includegraphics[width=\textwidth]{../figures/country_comparison/understood_each_positive.pdf}} 
\end{figure}

\begin{figure}[h!]
    \caption[Comprehension score]{Number of correct answers to comprehension questions (mean). (Questions \ref{q:understood_gcs}-\ref{q:understood_both})}\label{fig:understood_score}
    \makebox[\textwidth][c]{\includegraphics[width=\textwidth]{../figures/country_comparison/understood_score_mean.pdf}} 
\end{figure}

% \begin{figure}[h!]
%     \caption[Support for the Global Climate Scheme]{Support for the GCS, NR and the combination of GCS, NR and C. (Questions \ref{q:gcs_support}, \ref{q:nr_support} and \ref{q:crg_support})}\label{fig:support_binary}
%     \makebox[\textwidth][c]{\includegraphics[width=.9\textwidth]{../figures/country_comparison/support_binary.pdf}} 
% \end{figure}

% \begin{figure}[h!]
%     \caption[Beliefs about support for the GCS and NR]{Beliefs regarding the support for the GCS and NR. (Questions \ref{q:gcs_belief} and \ref{q:nr_belief})}\label{fig:belief}
%     \makebox[\textwidth][c]{\includegraphics[width=.8\textwidth]{../figures/country_comparison/belief.pdf}} 
% \end{figure}

\begin{figure}[h!]
    \caption[List experiment]{List experiment: mean number of supported policies. (Section \ref{subsubsec:list_exp}, Question \ref{q:list_exp})}\label{fig:list_exp}
    \makebox[\textwidth][c]{\includegraphics[width=.7\textwidth]{../figures/country_comparison/list_exp_mean.pdf}} 
\end{figure}

\begin{figure}[h!]
    \caption[Conjoint analyses 1 and 2]{Conjoint analyses 1 and 2. (Questions \ref{q:conjoint_a}-\ref{q:conjoint_b}, Back to Section \ref{subsubsec:conjoint})}\label{fig:conjoint}
    \makebox[\textwidth][c]{\includegraphics[width=.8\textwidth]{../figures/country_comparison/conjoint_ab_all_positive.pdf}} 
\end{figure}

% \begin{figure}[h!] % already in text
%     \caption{[Asked only to non-Republicans] Conjoint analysis n°4: random programs at the Democratic primary. (Question \ref{q:conjoint_r})}\label{fig:ca_r}
%     \makebox[\textwidth][c]{\includegraphics[width=\textwidth]{../figures/country_comparison/ca_r.png}} 
% \end{figure}

% \begin{figure}[h!]
%     \caption[Influence of the GCS on preferred platform]{Influence of the GCS on preferred platform:\\ Preference for a random platform A that contains the Global Climate Scheme rather than a platform B that does not (in percent). (Question \ref{q:conjoint_d}; in the U.S., asked only to non-Republicans.)}\label{fig:conjoint_left_ag_b}
%     \makebox[\textwidth][c]{\includegraphics[width=\textwidth]{../figures/country_comparison/conjoint_left_ag_b_binary_positive.pdf}} 
% \end{figure}

\begin{figure}[h!]
    \caption[Perceptions of the GCS]{Perceptions of the GCS. Elements seen as important for supporting the GCS in a 4-Likert scale (in percent). (Question \ref{q:gcs_important})  \hfill (Back~to~Section~\ref{subsubsec:pros_cons})}\label{fig:gcs_important}
    \makebox[\textwidth][c]{\includegraphics[width=\textwidth]{../figures/country_comparison/gcs_important_positive.pdf}} 
\end{figure}

\begin{figure}[h!]
    \caption[Classification of open-ended field on the GCS]{Perceptions of the GCS. Elements found in the open-ended field on the GCS (manually recoded, in percent). (Question \ref{q:gcs_field}) \hfill (Back~to~Section~\ref{subsubsec:pros_cons})}\label{fig:gcs_field}
    \makebox[\textwidth][c]{\includegraphics[width=.75\textwidth]{../figures/country_comparison/gcs_field_positive.pdf}} 
\end{figure}

\begin{figure}[h!]
    \caption[Topics of open-ended field on the GCS]{Perceptions of the GCS. Keywords found in the open-ended field on the GCS (automatic search ignoring case, in percent). (Question \ref{q:gcs_field}) \hfill (Back~to~Section~\ref{subsubsec:pros_cons})}\label{fig:gcs_field_contains}
    \makebox[\textwidth][c]{\includegraphics[width=\textwidth]{../figures/country_comparison/gcs_field_contains_positive.pdf}} 
\end{figure}

\begin{table}[h]
    \caption[Campaign and bandwagon effects on the support for the GCS.]{Effects on the support for the GCS of a question on its pros and cons and on information about the actual support, in the U.S. (See Section \ref{subsec:questionnaire_perceptions} in the US2 Questionnaire)  \hfill (Back~to~Section~\ref{subsubsec:pros_cons})} \label{tab:branch_gcs}
    \makebox[\textwidth][c]{
        
\begin{tabular}{@{\extracolsep{5pt}}lcccc} 
\\[-1.8ex]\hline 
\hline \\[-1.8ex] 
 & \multicolumn{4}{c}{Support} \\ 
\cline{2-5} 
\\[-1.8ex] & \multicolumn{2}{c}{Global Climate Scheme} & \multicolumn{2}{c}{National Redistribution} \\ 
\\[-1.8ex] & (1) & (2) & (3) & (4)\\ 
\hline \\[-1.8ex] 
Control group mean & 0.557 & 0.557 & 0.569 & 0.569  \\ \hline \\[-1.8ex]
 Treatment: Open\mbox{-}ended field on GCS pros \& cons & $-$0.073$^{**}$ & $-$0.073$^{**}$ & $-$0.035 & $-$0.031 \\ 
  & (0.035) & (0.031) & (0.035) & (0.032) \\ 
  Treatment: Closed questions on GCS pros \& cons & $-$0.109$^{***}$ & $-$0.096$^{***}$ & $-$0.065$^{*}$ & $-$0.062$^{**}$ \\ 
  & (0.034) & (0.031) & (0.034) & (0.031) \\ 
  Treatment: Info on actual support for GCS and NR & $-$0.021 & $-$0.017 & 0.048 & 0.054$^{*}$ \\ 
  & (0.034) & (0.031) & (0.033) & (0.031) \\ 
 \hline \\[-1.8ex] 
Includes controls &  & \checkmark &  & \checkmark \\

Observations & 2,000 & 1,995 & 2,000 & 1,995 \\ 
R$^{2}$ & 0.007 & 0.169 & 0.007 & 0.153 \\ 
\hline 
\hline \\[-1.8ex] 
\end{tabular} 
    }
    {\footnotesize %\textit{Note}: 
    }
\end{table}

\begin{figure}[h!]
    \caption[Donation to Africa vs. own country]{Donation in case of lottery win, depending on the recipient's (randomly drawn) nationality (mean). (Question \ref{q:donation})\hfill (Back~to~Section~\ref{subsec:universalistic})}\label{fig:donation}
    \makebox[\textwidth][c]{\includegraphics[width=.8\textwidth]{../figures/country_comparison/donation_mean.pdf}} 
\end{figure}

\begin{table}[h]
    \caption[Donation to Africa vs. own country]{Donation in case of lottery win, depending on the recipient's (randomly drawn) nationality. (Question \ref{q:donation})\hfill (Back~to~Section~\ref{subsec:universalistic})} \label{tab:donation}
    \makebox[\textwidth][c]{\input{../tables/continents/donation_interaction.tex}}
\end{table}

\begin{figure}[h!]
    \caption[Support for a global wealth tax]{Support for a global wealth tax. \\
    ``Do you support or oppose a tax on millionaires of all countries to finance low-
    income countries? \\
    Such tax would finance infrastructure and public services such as access to drinking water, healthcare, and education.'' (Question \ref{q:global_tax})}\label{fig:global_tax}
    \makebox[\textwidth][c]{\includegraphics[width=\textwidth]{../figures/country_comparison/global_tax_support.pdf}} 
\end{figure}

\begin{figure}[h!]
    \caption[Support for a national wealth tax]{Support for a national wealth tax financing public services like healthcare, education, and social housing. (Question \ref{q:national_tax})}\label{fig:national_tax}
    \makebox[\textwidth][c]{\includegraphics[width=\textwidth]{../figures/country_comparison/national_tax_support.pdf}} 
\end{figure}

\begin{figure}[h!]
    \caption[Preferred share of global tax for low-income countries]{Preferred share of global wealth tax revenues that should be pooled to finance low-income countries. (Question \ref{q:global_tax_global_share})}\label{fig:global_tax_global_share}
    \makebox[\textwidth][c]{\includegraphics[width=\textwidth]{../figures/country_comparison/global_tax_global_share.pdf}} 
\end{figure}

\begin{figure}[h!]
    \caption[Support for sharing half of global tax revenues with low-income countries]{Support for sharing half of global tax revenues with low-income countries, rather that each country retaining all the revenues it collects (in percent). (Question \ref{q:global_tax_sharing})}\label{fig:global_tax_sharing}
    \makebox[\textwidth][c]{\includegraphics[width=\textwidth]{../figures/country_comparison/global_tax_sharing_positive.pdf}} 
\end{figure}

\begin{figure} 
    \caption[Actual, perceived and preferred amount of foreign aid (mean)]{Actual, perceived and preferred amount of foreign aid, with random info (or not) on actual amount. (\textit{Mean}, Questions \ref{q:foreign_aid_belief}, \ref{q:foreign_aid_preferred})  \hfill (Back~to~Section~\ref{subsubsec:support_foreign_aid})}\label{fig:foreign_aid_amount}
    \makebox[\textwidth][c]{\includegraphics[width=.9\textwidth]{../figures/country_comparison/foreign_aid_amount_mean.pdf} } 
\end{figure}

% \begin{figure} 
%     \caption{Actual, perceived and preferred amount of foreign aid, with random info (or not) on actual amount. (\textit{Median}, Questions \ref{q:foreign_aid_belief}, \ref{q:foreign_aid_preferred})}\label{fig:foreign_aid_amount}
%     \makebox[\textwidth][c]{\includegraphics[width=.9\textwidth]{../figures/country_comparison/foreign_aid_amount_median.pdf} } % TODO? add? not necessary as the info on median can be deduced from below figures
% \end{figure}

\begin{figure} 
    % \caption{Support for increased foreign aid (vs. reduced or stable): from previous question, and directly asked (with info).}\vspace{-.2cm}
    % \includegraphics[height=.32\textheight]{../figures/country_comparison/foreign_aid_more_positive.pdf} 
    \caption[Preferred foreign aid (summary)]{Preferred foreign aid (after info or after perception). (Questions \ref{q:foreign_aid_belief} and \ref{q:foreign_aid_preferred})}\label{fig:foreign_aid_no_less_all}
    \makebox[\textwidth][c]{\includegraphics[width=\textwidth]{../figures/country_comparison/foreign_aid_no_less_all_positive.pdf} }
\end{figure} 

% \begin{figure}
%     \centering 
%     \caption{Your previous answer shows that you would like to increase [UK] foreign aid.\\How would you like to finance such increase in foreign aid? (Multiple answers possible)}
%     \includegraphics[width=\columnwidth]{../figures/all/foreign_aid_raise.pdf} 
% \end{figure}		
% \begin{figure}
%     \centering 
%     \caption{Your previous answer shows that you would like to reduce [UK] foreign aid.\\How would you like to use the freed budget? (Multiple answers possible)}
%     \includegraphics[width=\columnwidth]{../figures/all/foreign_aid_reduce.pdf} 
% \end{figure}

\begin{figure}[h!]
    \caption[Perceived foreign aid]{Perceived foreign aid. ``From your best guess, what percentage of [own country] government spending is allocated to foreign aid (that is, to reduce poverty in low-income countries)?'' (Question \ref{q:foreign_aid_belief})  \hfill (Back~to~Section~\ref{subsubsec:support_foreign_aid}) \\ Actual values: France: 0.8\%; Germany: 1.3\%; Spain: 0.5\%; UK: 1.7\%; U.S.: 0.4\%.}\label{fig:foreign_aid_belief}
    \makebox[\textwidth][c]{\includegraphics[width=\textwidth]{../figures/country_comparison/foreign_aid_belief_agg.pdf}} 
\end{figure}

\begin{figure}[h!]
    \caption[Preferred foreign aid (without info on actual amount)]{Preferred foreign aid (without info on actual amount). \\ ``If you could choose the government spending, what percentage would you allocate
    to foreign aid?'' (Question \ref{q:foreign_aid_preferred})  \hfill (Back~to~Section~\ref{subsubsec:support_foreign_aid})}\label{fig:foreign_aid_preferred_no_info}
    \makebox[\textwidth][c]{\includegraphics[width=\textwidth]{../figures/country_comparison/foreign_aid_preferred_no_info_agg.pdf}} 
\end{figure}

\begin{figure}[h!]
    \caption[Preferred foreign aid (after info on actual amount)]{Preferred foreign aid (after info on actual amount). \\ ``Actually,
    [US1: 0.4\%; FR: 0.8\%; DE: 1.3\%; ES: 0.5\%; UK: 1.7\%] of [own country] government spending is allocated to foreign aid. \\
    If you could choose the government spending, what percentage would you allocate
    to foreign aid?'' (Question \ref{q:foreign_aid_preferred})  \hfill (Back~to~Section~\ref{subsubsec:support_foreign_aid})}\label{fig:foreign_aid_preferred_info}
    \makebox[\textwidth][c]{\includegraphics[width=\textwidth]{../figures/country_comparison/foreign_aid_preferred_info_agg.pdf}} 
\end{figure}

\begin{figure}[h!]
    \caption[Preferences for funding increased foreign aid]{Preferences for funding increased foreign aid. [Asked iff preferred foreign aid is strictly greater than [Info: actual; No info: perceived] foreign aid] \\ ``How would you like to finance such increase in foreign aid? (Multiple answers possible)'' (in percent) (Question \ref{q:foreign_aid_raise_how})  \hfill (Back~to~Section~\ref{subsubsec:support_foreign_aid})}\label{fig:foreign_aid_raise_how}
    \makebox[\textwidth][c]{\includegraphics[width=.75\textwidth]{../figures/country_comparison/foreign_aid_raise_positive.pdf}} 
\end{figure}

\begin{figure}[h!]
    \caption[Preferences of spending following reduced foreign aid]{Preferences of spending following reduced foreign aid. [Asked iff preferred foreign aid is strictly lower than [Info: actual; No info: perceived] foreign aid] \\ ``How would you like to use the freed budget? (Multiple answers possible)'' (in percent) (Question \ref{q:foreign_aid_reduce_how})  \hfill (Back~to~Section~\ref{subsubsec:support_foreign_aid})}\label{fig:foreign_aid_reduce_how}
    \makebox[\textwidth][c]{\includegraphics[width=.75\textwidth]{../figures/country_comparison/foreign_aid_reduce_positive.pdf}} 
\end{figure}

% \begin{figure}[h!]
%     \caption[Attitudes on the evolution of foreign aid]{Attitudes regarding the evolution of [own country] foreign aid. (Question \ref{q:foreign_aid_raise_support})}\label{fig:foreign_aid_raise_support}
%     \makebox[\textwidth][c]{\includegraphics[width=\textwidth]{../figures/country_comparison/foreign_aid_raise_support.pdf}} 
% \end{figure}

% \begin{figure}[h!]
%     \caption[Conditions at which foreign aid should be increased]{Conditions at which foreign aid should be increased (in percent). [Asked to those who wish an increase of foreign aid at some conditions.] (Question \ref{q:foreign_aid_condition})}\label{fig:foreign_aid_condition}
%     \makebox[\textwidth][c]{\includegraphics[width=\textwidth]{../figures/country_comparison/foreign_aid_condition_positive.pdf}} 
% \end{figure}

% \begin{figure}[h!]
%     \caption[Reasons why foreign aid should not be increased]{Reasons why foreign aid should not be increased (in percent). [Asked to those who wish a decrease or stability of foreign aid.] (Question \ref{q:foreign_aid_no})}\label{fig:foreign_aid_no}
%     \makebox[\textwidth][c]{\includegraphics[width=\textwidth]{../figures/country_comparison/foreign_aid_no_positive.pdf}} 
% \end{figure}

% \begin{figure}[h!]
%     \caption[Willingness to sign a real-stake petition]{Willingness to sign real-stake petition for the Global Climate Scheme or National Redistribution. (Question \ref{q:petition})}\label{fig:petition}
%     \makebox[\textwidth][c]{\includegraphics[width=.8\textwidth]{../figures/country_comparison/petition_only_positive.pdf}} 
% \end{figure}

\begin{figure}[h!]
    \caption[Willingness to sign a real-stake petition]{Willingness to sign real-stake petition for the Global Climate Scheme or National Redistribution, compared to stated support in corresponding subsamples (e.g. support for the GCS in the branch where the petition was about the GCS). (Question \ref{q:petition})}\label{fig:petition}
    \makebox[\textwidth][c]{\includegraphics[width=.8\textwidth]{../figures/country_comparison/petition_comparable_positive.pdf}} 
\end{figure}

\begin{figure}[h!] % TODO? More details?
    \caption[Absolute support for various global policies]{Absolute support for various global policies (Percent of (\textit{somewhat} or \textit{strong}) support). (Questions \ref{q:climate_policies} and \ref{q:other_policies}. See Figure \ref{fig:support} for the relative support.)}\label{fig:support_likert_positive}
    \makebox[\textwidth][c]{\includegraphics[width=\textwidth]{../figures/country_comparison/support_likert_positive.pdf}} 
\end{figure}

% \begin{figure}[h!]
%     \caption{label}\label{fig:climate_policies}
%     \makebox[\textwidth][c]{\includegraphics[width=\textwidth]{../figures/country_comparison/climate_policies.pdf}} 
% \end{figure}

% \begin{figure}[h!]
%     \caption{label}\label{fig:global_policies}
%     \makebox[\textwidth][c]{\includegraphics[width=\textwidth]{../figures/country_comparison/global_policies.pdf}} 
% \end{figure}

\begin{figure}[h!]
    \caption[Preferred approach for international climate negotiations]{Preferred approach of diplomats at international climate negotiations. \\ In international climate negotiations, would you prefer [U.S.] diplomats to defend [own country] interests or global justice? (Question \ref{q:negotiation})}\label{fig:negotiation}
    \makebox[\textwidth][c]{\includegraphics[width=\textwidth]{../figures/country_comparison/negotiation.pdf}} 
\end{figure}

\begin{figure}[h!]
    \caption[Importance of selected issues]{Percent of selected issues viewed as important.\\ ``To what extent do you think the following issues are a problem?'' (Question \ref{q:problem})}\label{fig:problem}
    \makebox[\textwidth][c]{\includegraphics[width=.75\textwidth]{../figures/country_comparison/problem_positive.pdf}} 
\end{figure}

\begin{figure}[h!]
    \caption[Group defended when voting]{Group defended when voting. \\ ``What group do you defend when you vote?'' (Question \ref{q:group_defended})}\label{fig:group_defended}
    \makebox[\textwidth][c]{\includegraphics[width=\textwidth]{../figures/country_comparison/group_defended_agg2.pdf}} 
\end{figure}

% \begin{figure}[h!]
%     \caption{label}\label{fig:group_defended}
%     \makebox[\textwidth][c]{\includegraphics[width=\textwidth]{../figures/country_comparison/group_defended.pdf}} 
% \end{figure}

\begin{figure}[h!] 
    \caption[Mean prioritization of policies]{Mean prioritization of policies. \\Mean number of points allocated policies to express intensity of support (among six policies chosen at random). Blue color means that the policy has been awarded more points than the average policy. (Question \ref{q:points})}\label{fig:points}
    \makebox[\textwidth][c]{\includegraphics[width=\textwidth]{../figures/country_comparison/points_mean.pdf}} 
\end{figure}

\begin{figure}[h!] 
    \caption[Positive prioritization of policies]{Positive prioritization of policies. \\ Percent of people allocating a positive number of points to policies, expressing their support (among six policies chosen at random). (Question \ref{q:points})}\label{fig:points_positive}
    \makebox[\textwidth][c]{\includegraphics[width=\textwidth]{../figures/country_comparison/points_positive.pdf}} 
\end{figure}

\begin{figure}[h!]
    \caption[Charity donation]{Charity donation. \\ ``How much did you give to charities in 2022?'' (Question \ref{q:donation_charities})}\label{fig:donation_charities}
    \makebox[\textwidth][c]{\includegraphics[width=.8\textwidth]{../figures/country_comparison/donation_charities.pdf}} 
\end{figure}

\begin{figure}[h!] 
    \caption[Interest in politics]{Interest in politics. \\ ``To what extent are you interested in politics?'' (Question \ref{q:interested_politics})}\label{fig:interested_politics}
    \makebox[\textwidth][c]{\includegraphics[width=.8\textwidth]{../figures/country_comparison/interested_politics.pdf}} 
\end{figure}

\begin{figure}[h!] 
    \caption[Desired involvement of government]{Desired involvement of government (from 1 to 5). (Question \ref{q:involvement_govt})}\label{fig:involvement_govt}
    \makebox[\textwidth][c]{\includegraphics[width=.9\textwidth]{../figures/country_comparison/involvement_govt.pdf}} 
\end{figure}

\begin{figure}[h!] 
    \caption[Political leaning]{Political leaning on economics (from 1: Left to 5: Right). (Question \ref{q:left_right})}\label{fig:left_right}
    \makebox[\textwidth][c]{\includegraphics[width=.8\textwidth]{../figures/country_comparison/left_right.pdf}} 
\end{figure}

\begin{figure}[h!] 
    \caption[Voted in last election]{Voted in last election. (Question \ref{q:vote_participation})}\label{fig:vote_participation}
    \makebox[\textwidth][c]{\includegraphics[width=.8\textwidth]{../figures/country_comparison/vote_participation.pdf}} 
\end{figure}

\begin{figure}[h!] 
    \caption[Vote in last election]{Vote in last election (aggregated). \textit{PNR} includes people who did not vote or prefer not to answer. (Question \ref{q:vote})}\label{fig:vote}
    \makebox[\textwidth][c]{\includegraphics[width=.75\textwidth]{../figures/country_comparison/vote.pdf}} 
\end{figure}

\begin{figure}[h!] 
    \caption[Perception that survey was biased]{Perception that survey was biased. \\ ``Do you feel that this survey was politically biased?'' (Question \ref{q:survey_biased})}\label{fig:survey_biased}
    \makebox[\textwidth][c]{\includegraphics[width=.7\textwidth]{../figures/country_comparison/survey_biased.pdf}} 
\end{figure}

% \begin{columns}
% \begin{column}{.5\textwidth}
% \begin{multicols}{2}
    \begin{figure}[h!]
        \caption[Classification of open-ended field on extreme poverty]{Opinion on the fight against extreme poverty. \\ ``According to you, what should high-income countries do to fight extreme poverty in low-income countries?'' (Question \ref{q:poverty_field})  \hfill (Back~to~Section~\ref{subsubsec:support_foreign_aid})}\label{fig:poverty_field}
    \begin{subfigure}{.34\textwidth}
        \caption{Elements found in the open-ended field on the question (manually recoded, in percent)}.
        \includegraphics[width=\textwidth]{../figures/country_comparison/poverty_field_positive.pdf}        
    \end{subfigure}
    \hspace{.02\textwidth}
    \begin{subfigure}{.64\textwidth}
        \caption{Keywords found in the open-ended field on the GCS (automatic search ignoring case, in percent).}
        \includegraphics[width=\textwidth]{../figures/country_comparison/poverty_field_contains_positive.pdf}    
    \end{subfigure}
    \end{figure}
% \end{column}
% \begin{column}{.5\textwidth}
    % \begin{figure}[h!]
    %     \caption[Topics of open-ended field on extreme poverty]{Opinion on the fight against extreme poverty. \\ ``According to you, what should high-income countries do to fight extreme poverty in low-income countries?'' \\ Keywords found in the open-ended field on the GCS (automatic search ignoring case, in percent). (Question \ref{q:poverty_field})}\label{fig:poverty_field_contains}
    %     \makebox[\textwidth][c]{\includegraphics[width=\columnwidth]{../figures/country_comparison/poverty_field_contains_positive.pdf}} 
    % \end{figure}
% \end{multicols}
% \end{column}
% \end{columns}


\begin{figure}[h!] 
    \caption[Main attitudes by vote]{Main attitudes by vote (``Right'' spans from Center-right to Far right). \\ (Relative support in percent in Questions \ref{q:gcs_support}, \ref{q:global_tax}, \ref{q:other_policies}, \ref{q:foreign_aid_raise_support}, \ref{q:negotiation}) \hfill (Back~to~Section~\ref{subsec:universalistic})}\label{fig:main_by_vote}
    \makebox[\textwidth][c]{\includegraphics[width=\textwidth]{../figures/country_comparison/main_all_by_vote_share.pdf}} 
\end{figure}

% \begin{figure}[h!] 
%     \caption[Interested to be interviewed]{Interested to be interviewed by a researcher for 30 min through videoconference. (Question \ref{q:interview})}\label{fig:interview}
%     \makebox[\textwidth][c]{\includegraphics[width=\textwidth]{../figures/country_comparison/interview.pdf}} 
% \end{figure}    

% \begin{figure}[h!]
%     \caption{label}\label{fig:share_policies_supported}
%     \makebox[\textwidth][c]{\includegraphics[width=\textwidth]{../figures/country_comparison/share_policies_supported.pdf}} 
% \end{figure} % TODO? uncomment?

% \begin{figure}[h!]
%     \caption{label}\label{fig:vars}
%     \makebox[\textwidth][c]{\includegraphics[width=\textwidth]{../figures/country_comparison/vars.pdf}} 
% \end{figure}

% In Denmark, France and the U.S., the questions with an asterisk were asked differently, asking ``To achieve a given reduction of greenhouse gas emissions globally, costly investments are needed. Ideally, how should countries bear the costs of fighting climate change?''. Instead of the equal right per capita, the item was ``Countries should pay in proportion to their current emissions'', historical responsibilities was worded as ``Countries should pay in proportion to their past emissions (from 1990 onwards)'', then there was an item ``The richest countries should pay it all'', and compensating vulnerable countries was worded as ``The richest countries should pay even more, to help vulnerable countries face adverse consequences: vulnerable countries would then receive money instead of paying''.

\clearpage 
\section{Questionnaire of the global survey (section on global policies)}\label{app:questionnaire_oecd}
%\subsection*{International burden-sharing}
\renewcommand{\theenumi}{\Alph{enumi}}
\begin{enumerate} \item \label{q:scale} At which level(s) do you think public policies to tackle climate change need to be put in place? (Multiple answers are possible) [\textit{Figures \ref{fig:oecd} and \ref{fig:oecd_absolute}}]
\\ \textit{Global; [Federal / European / ...]; [State / National]; Local}
\item Do you agree or disagree with the following statement: ``[country] should take measures to fight climate change.''% TODO! figure
	\\ \textit{Strongly disagree; Somewhat disagree; Neither agree nor disagree; Somewhat agree; Strongly agree}
\item How should [country] climate policies depend on what other countries do?% TODO! figure
 \begin{itemize}
\item If other countries do more, [country] should do...
\item If other countries do less, [country] should do...
\end{itemize}
\textit{Much less; Less; About the same; More; Much more}
\item ~[In all countries but the U.S., Denmark and France]  All countries have signed the Paris agreement that aims to contain global warming ``well below +2 \textdegree{}C\''. To limit global warming to this level, there is a maximum amount of greenhouse gases we can emit globally, called the carbon budget. Each country could aim to emit less than a share of the carbon budget. To respect the global carbon budget, countries that emit more than their national share would pay a fee to countries that emit less than their share. \\ 
Do you support such a policy? [\textit{Figures \ref{fig:oecd} and \ref{fig:oecd_absolute}}]
\\ \textit{Strongly oppose; Somewhat oppose; Neither support nor oppose; Somewhat support; Strongly support}
\item ~[In all countries but the U.S., Denmark and France] Suppose the above policy is in place. How should the carbon budget be divided among countries? [\textit{Figures \ref{fig:oecd} and \ref{fig:oecd_absolute}}]
\\ \textit{The emission share of a country should be proportional to its population, so that each human has an equal right to emit.; The emission share of a country should be proportional to its current emissions, so that those who already emit more have more rights to emit.; Countries that have emitted more over the past decades (from 1990 onwards) should receive a lower emission share, because they have already used some of their fair share.; Countries that will be hurt more by climate change should receive a higher emission share, to compensate them for the damages.}
\item \label{q:burden_sharing_asterisk} ~[In the U.S., Denmark, and France only] To achieve a given reduction of greenhouse gas emissions globally, costly investments are needed. % TODO! figure
Ideally, how should countries bear the costs of fighting climate change?
 \begin{itemize}
\item Countries should pay in proportion to their income
\item Countries should pay in proportion to their current emissions [Used as a substitute to the equal right per capita in Figure \ref{fig:oecd}]
\item Countries should pay in proportion to their past emissions (from 1990 onwards) [Used as a substitute to historical responsibilities in Figure \ref{fig:oecd}]
\item The richest countries should pay it all, so that the poorest countries do not have to pay anything
\item The richest countries should pay even more, to help vulnerable countries face adverse consequences: vulnerable countries would then receive money instead of paying [Used as a substitute to compensating vulnerable countries in Figures \ref{fig:oecd} and \ref{fig:oecd_absolute}]
\end{itemize} 
\textit{Strongly disagree; Somewhat disagree; Neither agree nor disagree; Somewhat agree; Strongly agree}
\item Do you support or oppose establishing a global democratic assembly whose role would be to draft international treaties against climate change? Each adult across the world would have one vote to elect members of the assembly. [\textit{Figures \ref{fig:oecd} and \ref{fig:oecd_absolute}}]
\\ \textit{Strongly oppose; Somewhat oppose; Neither support nor oppose; Somewhat support; Strongly support}
\item Imagine the following policy: a global tax on greenhouse gas emissions funding a global basic income. 
Such a policy would progressively raise the price of fossil fuels (for example, the price of gasoline would increase by [40 cents per gallon] in the first years). Higher prices would encourage people and companies to use less fossil fuels, reducing greenhouse gas emissions. Revenues from the tax would be used to finance a basic income of [\$30] per month to each human adult, thereby lifting the 700 million people who earn less than \$2/day out of extreme poverty. 
The average British person would lose a bit from this policy as they would face [\$130] per month in price increases, which is higher than the [\$30] they would receive.

Do you support or oppose such a policy?  [\textit{Figures \ref{fig:oecd} and \ref{fig:oecd_absolute}}]
\\ \textit{Strongly oppose; Somewhat oppose; Neither support nor oppose; Somewhat support; Strongly support}
\item \label{q:millionaire_tax} Do you support or oppose a tax on all millionaires around the world to finance low-income countries that comply with international standards regarding climate action? 
This would finance infrastructure and public services such as access to drinking water, healthcare, and education. [\textit{Figures \ref{fig:oecd} and \ref{fig:oecd_absolute}}]
\\ \textit{Strongly oppose; Somewhat oppose; Neither support nor oppose; Somewhat support; Strongly support}
\end{enumerate}

% \clearpage
% \section{Questionnaire of US1 %the first U.S. complementary 
% survey}\label{app:questionnaire_US1}

% \begin{figure}[h!]
%     \caption{US1 survey structure}\label{fig:flow_US1}
%     \makebox[\textwidth][c]{\includegraphics[width=\textwidth]{../questionnaire/survey_flow_US1.pdf}} 
% \end{figure}

\renewcommand{\theenumi}{\arabic{enumi}}
\clearpage
\section{Questionnaire of the complementary surveys}\label{app:questionnaire}
\input{app_questionnaire}


\clearpage
\section{Net gains from the Global Climate Scheme}\label{app:gain_gcs}

To specify the GCS, we use the IEA's 2DS scenario \citep{iea_energy_2017}, which is consistent with limiting the global average temperature increase to 2\textdegree{}C with a probability of at least 50\%. The paper by \citet{hood_input_2017} contributing to the Report of the High-Level Commission on Carbon Prices \citep{stern_report_2017} presents a price corridor compatible with this emissions scenario, from which we take the midpoint. The product of these two series provides an estimate of the revenues expected from a global carbon price. We then use the UN median scenario of future population aged over 15 years (\textit{adults}, for short). We derive the basic income that could be paid to all adults by recycling the revenues from the global carbon price: evolving between \$20 and \$30 per month, with a peak in 2030. Accounting for the lower price levels in low-income countries, an additional income of \$30 per month would allow \href{https://data.worldbank.org/indicator/SI.POV.DDAY}{670 million people} to escape extreme poverty, defined with the threshold of \$2.15 per day in purchasing power parity.\footnote{By taking the \href{https://data.worldbank.org/indicator/PA.NUS.PPPC.RF}{ratio} of the World Bank series relating the GDP per capita of Sub-Saharan Africa in \href{https://data.worldbank.org/indicator/NY.GDP.PCAP.PP.KD?locations=ZG&year_high_desc=true}{PPP} and \href{https://data.worldbank.org/indicator/NY.GDP.PCAP.KD?locations=ZG&year_high_desc=true}{nominal}, we obtain the purchasing power of \$1 in this region: \$2.4 in 2019. %See also the price level ratio of PPP conversion factor to market exchange rate.
} 

To estimate the increase in fossil fuel expenditures (or ``cost'') in each country by 2030, we make a key assumption concerning the evolution of the carbon footprints per adult: that they will decrease by the same proportion %$\rho$ 
in each country. We use data from the Global Carbon Project \citep{peters_synthesis_2012}. 
% Noting $e_c$ (resp. $e_c^b$) the carbon footprint per adult of a country $c$ in 2030 (resp. in baseline year $b$), we have $e_c = \rho e_c^b$. Noting $a_c$ (resp. $a_c^b$) the adult population of a country $c$ in 2030 (resp. in baseline year $b$) and $E = \sum_c e_c a_c$ global emissions in 2030, we find $\rho = \frac{E}{\sum_c e_c^b a_c}$. Finally, the average cost per adult in year $y$ is $p \cdot e_c \frac{a_c}{a^y_c}$. %Multiplying country $c$'s carbon footprint per capita with the carbon price $p$ yields its average cost per adult: $p \cdot e_c$. %$\frac{s_c^y}{p^y_c} R$. 
In 2030, the average carbon footprint of a country $c$, $e_c$, evolves from baseline year $b$ proportionally to the evolution of its adult population $\Delta p_c = p^{2030}_c/p^b_c$. Thus, the global share of country $c$'s carbon footprint, $s_c$, is proportional to $\sigma_c = e_c \Delta p_c$, and as countries' shares sum to 1, $s_c = \frac{\sigma_c}{\sum_k \sigma_k}$. Multiplying country $c$'s emission share with global revenues in 2030, $R$, and dividing by $c$'s adult population in year $y$, yields its average cost per adult: $R \cdot s_c / p^y_c$. %$\frac{s_c^y}{p^y_c} R$. 
Using findings from \citet{ivanova_unequal_2020} for Europe and \citet{fremstad_impact_2019} for the U.S., we approximate the median cost as 90\% of the average cost. Finally, the net gain is given by the basic income (\$30 per month) minus the cost. We provided consistent estimates of net gains in all surveys (using $y = b = 2015$), though in the global survey we gave the average net gains vs. the median ones in the complementary surveys. The latter are shown in Figure \ref{fig:median_gain_2015}. 
For the record, Table \ref{tab:gain_gcs.tex} also provides an estimate of \textit{average} net gains (computed with $b = 2019$ and $y = 2030$).\footnote{2015 was the last year of data available when the global questionnaire was conceived (\href{https://stats.oecd.org/Index.aspx?DataSetCode=IO_GHG_2019}{OECD data} was then used -- it does not cover all countries but give identical rounded estimates than those recomputed from the Global Carbon Project data for our complementary surveys). 2030 was chosen as the reference year as it is the date at which global carbon price revenues are expected to peak (and the GCS redistributive effects would be largest), and the GCS could not realistically enter into force before that date. In the surveys, we chose $y = b = 2015$ rather than $b = 2019$ and $y = 2030$ to get more conservative estimates of the monthly cost in the U.S. (\$20 higher than the other option) and in Europe (\euro{5} or £10 higher).}% TODO? remove footnote?
%  ((e/E)*(f/a)*A/F)*R/a

Estimates of the net gains from the Global Climate Scheme are necessarily imprecise, given the uncertainties surrounding the carbon price required to achieve emissions reductions as well as each country's trajectory in terms of emissions and population. These values are highly dependent on future (non-price) climate policies, technical progress, and economic growth of each country, which are only partially known. Integrated Assessment Models have been used to derive a Global Energy Assessment \citep{johansson_global_2012}, a 100\% renewable scenario \citep{greenpeace_energy_2015} as well as Shared Socioeconomic Pathways (SSPs), which include consistent trajectories of population, emissions, and carbon price \citep{riahi_shared_2017,bauer_shared_2017,van_vuuren_energy_2017,fricko_marker_2017}. Instead of using some of these modelling trajectories, we relied on a simple and transparent formula, for a number of reasons. First and foremost, those trajectories describe territorial emissions while we need consumption-based emissions to compute the incidence of the GCS. Second, the carbon price is relatively low in trajectories of SSPs that contain global warming below 2\textdegree{}C (less than \$35/tCO$_\text{2}$ in 2030), so we conservatively chose a method yielding a higher carbon price (\$90 in 2030). Third, modelling results are available only for a few macro regions, while we wanted country by country estimates. Finally, we have checked that the emissions per capita given by our method are broadly in line with alternative methods, even if it tends to overestimate net gains in countries which will decarbonize less rapidly than average.\footnote{Computations with alternative methods can be found on \href{https://github.com/bixiou/global_tax_attitudes/blob/main/code_global/map_GCS_incidence.R}{our public repository}.} For example, although countries' decarbonization plans should realign with the GCS in place, India might still decarbonize less quickly than the European Union, so India's gain and the EU's loss might be overestimated in our computations. For a more sophisticated version of the Global Climate Scheme which includes participation mechanisms preventing middle-income countries (like China) to lose from it and estimations of the Net Present Value by country, see \citet{fabre_global_2023}.  \hfill (Back~to~Section~\ref{box:GCS})

\begin{figure}[h!]
    \caption{Net gains from the Global Climate Scheme.}\label{fig:median_gain_2015}
    \makebox[\textwidth][c]{\includegraphics[width=\textwidth]{../figures/maps/median_gain_2015.pdf}} 
\end{figure}

% \begin{table}[h]\label{tab:gain_gcs}
%     \caption{Net gains from the Global Climate Scheme.} 
%     \makebox[\textwidth][c]{
        % \resizebox*{!}{.7\textheight}{
\clearpage
\begin{multicols}{2}
    \setbox\ltmcbox\vbox{
    \makeatletter\col@number\@ne
        
\begin{longtable}[t]{lrr}
\caption{\label{tab:gain_gcs.tex}Estimated net gain from the GCS in 2030 and carbon footprint by country.}\\
\toprule
  & \makecell{Mean\\net gain\\from\\the GCS\\(\$/month)} & \makecell{CO$_\text{2}$\\footprint\\per adult\\in 2019\\(tCO$_\text{2}$/y)}\\
\midrule
Saudi Arabia & -93 & 24.0\\
United States & -77 & 21.0\\
Australia & -60 & 17.6\\
Canada & -56 & 16.7\\
South Korea & -50 & 15.6\\
Germany & -30 & 11.7\\
Russia & -29 & 11.5\\
Japan & -28 & 11.3\\
Malaysia & -21 & 10.0\\
Iran & -19 & 9.5\\
Poland & -19 & 9.5\\
United Kingdom & -18 & 9.4\\
China & -14 & 8.6\\
Italy & -13 & 8.4\\
South Africa & -11 & 8.0\\
France & -10 & 7.8\\
Iraq* & -8 & 7.4\\
Spain & -6 & 7.0\\
Turkey & -2 & 6.2\\
Algeria* & -1 & 6.0\\
Mexico & 2 & 5.6\\
Ukraine & 2 & 5.6\\
Uzbekistan* & 4 & 5.1\\
Argentina & 5 & 4.9\\
Thailand & 6 & 4.6\\
Egypt & 12 & 3.6\\
Indonesia & 13 & 3.3\\
Colombia & 15 & 3.0\\
Brazil & 15 & 2.9\\
Vietnam & 15 & 2.9\\
Peru & 16 & 2.8\\
Morocco & 16 & 2.7\\
North Korea* & 17 & 2.5\\
India & 18 & 2.4\\
Philippines & 18 & 2.3\\
Pakistan & 22 & 1.6\\
Bangladesh & 24 & 1.1\\
Nigeria & 25 & 1.0\\
Kenya & 25 & 0.9\\
Myanmar* & 26 & 0.9\\
Sudan* & 26 & 0.9\\
Tanzania & 27 & 0.5\\
Afghanistan* & 27 & 0.5\\
Uganda & 28 & 0.4\\
Ethiopia & 28 & 0.3\\
Venezuela & 29 & 0.3\\
DRC* & 30 & 0.1\\
\bottomrule
\end{longtable}
    \unskip
    \unpenalty
    \unpenalty}
    \unvbox\ltmcbox
\end{multicols}
        % }
%     }
    {\footnotesize \textit{Note}: %Emission data is from \cite{peters_synthesis_2012}. 
    Asterisks denote countries where footprint is missing and territorial emissions is used instead. %Estimation of net gains is described in the text. 
    Values differ from Figure \ref{fig:median_gain_2015} as this table present estimates of \textit{mean} net gain per adult in \textit{2030}, not at the present. Only the countries with more than 20 million adults (covering 87\% of the global total) are shown. 
    }
% \end{table}

% \clearpage
% \section{Sources}\label{app:sources}

\clearpage
\section{Determinants of support}\label{app:determinants}

\begin{table}[h]\label{tab:gcs_determinant}
    \caption[Determinants of support for the GCS]{Determinants of support for the Global Climate Scheme. (Back to \ref{subsubsec:support_gcs})} 
    \makebox[\textwidth][c]{
\resizebox*{!}{.73\textheight}{ % 73 is the max when there is a title
        
\begin{tabular}{@{\extracolsep{5pt}}lccccccc} 
\\[-1.8ex]\hline 
\hline \\[-1.8ex] 
 & \multicolumn{7}{c}{\makecell{Supports the Global Climate Scheme}} \\ 
\cline{2-8} 
\\[-1.8ex] & All & United States & Europe & France & Germany & Spain & United Kingdom \\ 
\hline \\[-1.8ex] 
 Country: Germany & $-$0.157$^{***}$ &  & $-$0.144$^{***}$ &  &  &  &  \\ 
  & (0.022) &  & (0.022) &  &  &  &  \\ 
  Country: Spain & $-$0.044$^{*}$ &  & $-$0.026 &  &  &  &  \\ 
  & (0.024) &  & (0.024) &  &  &  &  \\ 
  Country: United Kingdom & $-$0.079$^{***}$ &  & $-$0.104$^{***}$ &  &  &  &  \\ 
  & (0.024) &  & (0.023) &  &  &  &  \\ 
  Country: United States & $-$0.375$^{***}$ &  &  &  &  &  &  \\ 
  & (0.019) &  &  &  &  &  &  \\ 
  Income quartile: 2 & 0.037$^{**}$ & 0.031 & 0.038 & 0.047 & 0.058 & 0.013 & 0.023 \\ 
  & (0.017) & (0.022) & (0.023) & (0.043) & (0.049) & (0.053) & (0.043) \\ 
  Income quartile: 3 & 0.042$^{**}$ & 0.033 & 0.049$^{**}$ & 0.080$^{**}$ & 0.059 & 0.074 & $-$0.052 \\ 
  & (0.017) & (0.024) & (0.024) & (0.040) & (0.052) & (0.056) & (0.052) \\ 
  Income quartile: 4 & 0.056$^{***}$ & 0.063$^{**}$ & 0.010 & 0.018 & $-$0.015 & $-$0.001 & $-$0.005 \\ 
  & (0.018) & (0.026) & (0.026) & (0.047) & (0.055) & (0.056) & (0.057) \\ 
  Diploma: Post secondary & 0.023$^{*}$ & 0.033$^{*}$ & 0.010 & 0.007 & 0.045 & 0.007 & $-$0.010 \\ 
  & (0.012) & (0.017) & (0.018) & (0.029) & (0.039) & (0.039) & (0.039) \\ 
  Age: 25-34 & $-$0.076$^{***}$ & $-$0.083$^{***}$ & $-$0.044 & $-$0.031 & $-$0.077 & $-$0.050 & $-$0.103 \\ 
  & (0.025) & (0.031) & (0.035) & (0.057) & (0.083) & (0.066) & (0.091) \\ 
  Age: 35-49 & $-$0.101$^{***}$ & $-$0.108$^{***}$ & $-$0.069$^{**}$ & $-$0.094$^{*}$ & $-$0.009 & $-$0.168$^{**}$ & $-$0.050 \\ 
  & (0.024) & (0.030) & (0.034) & (0.055) & (0.077) & (0.070) & (0.090) \\ 
  Age: 50-64 & $-$0.137$^{***}$ & $-$0.164$^{***}$ & $-$0.038 & $-$0.039 & $-$0.020 & $-$0.146$^{**}$ & $-$0.017 \\ 
  & (0.024) & (0.030) & (0.035) & (0.056) & (0.082) & (0.067) & (0.087) \\ 
  Age: 65+ & $-$0.116$^{***}$ & $-$0.140$^{***}$ & $-$0.056 & 0.003 & $-$0.045 & $-$0.258$^{***}$ & 0.011 \\ 
  & (0.028) & (0.034) & (0.044) & (0.076) & (0.094) & (0.091) & (0.105) \\ 
  Gender: Man & 0.019$^{*}$ & 0.023 & $-$0.010 & $-$0.014 & $-$0.018 & 0.042 & $-$0.005 \\ 
  & (0.011) & (0.015) & (0.016) & (0.029) & (0.033) & (0.038) & (0.034) \\ 
  Lives with partner & 0.029$^{**}$ & 0.022 & 0.058$^{***}$ & 0.070$^{**}$ & 0.082$^{**}$ & 0.017 & 0.040 \\ 
  & (0.013) & (0.017) & (0.018) & (0.033) & (0.038) & (0.038) & (0.039) \\ 
  Employment status: Retired & $-$0.020 & $-$0.047 & 0.056 & 0.087 & 0.096 & 0.040 & 0.001 \\ 
  & (0.024) & (0.030) & (0.038) & (0.081) & (0.075) & (0.082) & (0.073) \\ 
  Employment status: Student & 0.045 & 0.063 & 0.101$^{**}$ & 0.165$^{*}$ & 0.192$^{**}$ & 0.116 & $-$0.021 \\ 
  & (0.033) & (0.048) & (0.044) & (0.085) & (0.087) & (0.074) & (0.107) \\ 
  Employment status: Working & $-$0.016 & $-$0.021 & 0.011 & 0.082 & 0.006 & $-$0.050 & 0.036 \\ 
  & (0.019) & (0.024) & (0.028) & (0.064) & (0.056) & (0.056) & (0.051) \\ 
  Vote: Center-right or Right & $-$0.331$^{***}$ & $-$0.435$^{***}$ & $-$0.106$^{***}$ & $-$0.131$^{***}$ & $-$0.004 & $-$0.114$^{***}$ & $-$0.081$^{**}$ \\ 
  & (0.013) & (0.017) & (0.019) & (0.035) & (0.044) & (0.038) & (0.041) \\ 
  Vote: PNR/Non-voter & $-$0.184$^{***}$ & $-$0.198$^{***}$ & $-$0.136$^{***}$ & $-$0.196$^{***}$ & $-$0.034 & $-$0.116$^{**}$ & $-$0.108$^{***}$ \\ 
  & (0.016) & (0.022) & (0.021) & (0.039) & (0.043) & (0.046) & (0.040) \\ 
  Vote: Far right & $-$0.396$^{***}$ &  & $-$0.308$^{***}$ & $-$0.493$^{***}$ & $-$0.168$^{***}$ & $-$0.130 & $-$0.314$^{***}$ \\ 
  & (0.032) &  & (0.033) & (0.064) & (0.051) & (0.102) & (0.080) \\ 
  Urban & 0.049$^{***}$ & 0.074$^{***}$ & 0.006 & $-$0.002 & 0.019 & $-$0.014 & 0.017 \\ 
  & (0.012) & (0.018) & (0.016) & (0.029) & (0.032) & (0.036) & (0.033) \\ 
  Race: White &  & $-$0.030 &  &  &  &  &  \\ 
  &  & (0.019) &  &  &  &  &  \\ 
  Region: Northeast &  & 0.009 &  &  &  &  &  \\ 
  &  & (0.023) &  &  &  &  &  \\ 
  Region: South &  & 0.011 &  &  &  &  &  \\ 
  &  & (0.020) &  &  &  &  &  \\ 
  Region: West &  & 0.011 &  &  &  &  &  \\ 
  &  & (0.022) &  &  &  &  &  \\ 
  Swing State &  & $-$0.019 &  &  &  &  &  \\ 
  &  & (0.017) &  &  &  &  &  \\ 
 \hline \\[-1.8ex] 
Constant & 1.048 & 0.729 & 0.89 & 0.7 & 0.732 & 0.935 & 0.886 \\ 
Observations & 7,986 & 4,992 & 2,994 & 977 & 727 & 748 & 542 \\ 
R$^{2}$ & 0.160 & 0.180 & 0.064 & 0.116 & 0.067 & 0.043 & 0.063 \\ 
\hline 
\hline \\[-1.8ex] 
\textit{Note:}  & \multicolumn{7}{r}{$^{*}$p$<$0.1; $^{**}$p$<$0.05; $^{***}$p$<$0.01} \\ 
\end{tabular} 
        }
    }
    {\footnotesize %\textit{Note}: 
    }
\end{table}


\clearpage
\section{Representativeness of the surveys}\label{app:representativeness}


\begin{table}[h!]
    \caption[Sample representativeness of US1, US2, Eu]{Sample representativeness of the complementary surveys. (Back to \ref{par:surveys}) } \label{tab:representativeness_waves}
    \makebox[\textwidth][c]{
        \resizebox*{!}{.80\textheight}{% 73 without notes cf. https://tex.stackexchange.com/questions/13809/resizing-a-table-by-textheight 
        
\begin{tabular}[t]{llllllllll}
\toprule
\multicolumn{1}{c}{} & \multicolumn{3}{c}{US1} & \multicolumn{3}{c}{US2} & \multicolumn{3}{c}{EU} \\
\cmidrule(l{3pt}r{3pt}){2-4} \cmidrule(l{3pt}r{3pt}){5-7} \cmidrule(l{3pt}r{3pt}){8-10}
  & Pop. & Sample & \makecell{Weighted\\sample} & Pop. & Sample & \makecell{Weighted\\sample} & Pop. & Sample & \makecell{Weighted\\sample}\\
\midrule
Sample size &  & 3,000 & 3,000 &  & 678 & 678 &  & 3,000 & 3,000\\
\addlinespace
Gender: Woman & 0.51 & 0.52 & 0.51 & 0.51 & 0.67 & 0.57 & 0.51 & 0.49 & 0.51\\
Gender: Man & 0.49 & 0.47 & 0.49 & 0.49 & 0.32 & 0.43 & 0.49 & 0.51 & 0.49\\
\addlinespace
Income\_quartile: 1 & 0.25 & 0.27 & 0.25 & 0.25 & 0.55 & 0.34 & 0.25 & 0.28 & 0.25\\
Income\_quartile: 2 & 0.25 & 0.24 & 0.25 & 0.25 & 0.29 & 0.32 & 0.25 & 0.23 & 0.25\\
Income\_quartile: 3 & 0.25 & 0.25 & 0.25 & 0.25 & 0.12 & 0.23 & 0.25 & 0.25 & 0.25\\
Income\_quartile: 4 & 0.25 & 0.23 & 0.25 & 0.25 & 0.04 & 0.12 & 0.25 & 0.24 & 0.25\\
\addlinespace
Age: 18-24 & 0.12 & 0.12 & 0.12 & 0.12 & 0.14 & 0.12 & 0.10 & 0.11 & 0.10\\
Age: 25-34 & 0.18 & 0.15 & 0.18 & 0.18 & 0.16 & 0.17 & 0.15 & 0.17 & 0.15\\
Age: 35-49 & 0.24 & 0.25 & 0.24 & 0.24 & 0.25 & 0.25 & 0.24 & 0.25 & 0.24\\
Age: 50-64 & 0.25 & 0.27 & 0.25 & 0.25 & 0.22 & 0.24 & 0.26 & 0.24 & 0.26\\
Age: 65+ & 0.21 & 0.21 & 0.21 & 0.21 & 0.22 & 0.22 & 0.25 & 0.23 & 0.25\\
\addlinespace
Diploma\_25\_64: Below upper secondary & 0.06 & 0.02 & 0.05 & 0.06 & 0.08 & 0.07 & 0.13 & 0.14 & 0.13\\
Diploma\_25\_64: Upper secondary & 0.28 & 0.25 & 0.28 & 0.28 & 0.33 & 0.30 & 0.23 & 0.19 & 0.23\\
Diploma\_25\_64: Post secondary & 0.34 & 0.40 & 0.34 & 0.34 & 0.23 & 0.28 & 0.29 & 0.33 & 0.29\\
\addlinespace
Race: White only & 0.60 & 0.67 & 0.61 & 0.60 & 0.20 & 0.46 &  &  & \\
Race: Hispanic & 0.18 & 0.15 & 0.19 & 0.18 & 0.41 & 0.27 &  &  & \\
Race: Black & 0.13 & 0.16 & 0.14 & 0.13 & 0.36 & 0.20 &  &  & \\
\addlinespace
Region: Northeast & 0.17 & 0.20 & 0.17 & 0.17 & 0.15 & 0.16 &  &  & \\
Region: Midwest & 0.21 & 0.22 & 0.21 & 0.21 & 0.15 & 0.20 &  &  & \\
Region: South & 0.38 & 0.39 & 0.38 & 0.38 & 0.50 & 0.45 &  &  & \\
Region: West & 0.24 & 0.20 & 0.24 & 0.24 & 0.20 & 0.20 &  &  & \\
\addlinespace
Urban: TRUE & 0.73 & 0.78 & 0.74 & 0.73 & 0.73 & 0.69 &  &  & \\
\addlinespace
Employment\_18\_64: Inactive & 0.20 & 0.16 & 0.16 & 0.20 & 0.18 & 0.15 & 0.17 & 0.15 & 0.15\\
Employment\_18\_64: Unemployed & 0.02 & 0.07 & 0.08 & 0.02 & 0.15 & 0.11 & 0.03 & 0.06 & 0.05\\
\addlinespace
Vote: Left & 0.32 & 0.47 & 0.45 & 0.32 & 0.48 & 0.42 & 0.30 & 0.32 & 0.32\\
Vote: Center-right or Right & 0.30 & 0.31 & 0.31 & 0.30 & 0.15 & 0.24 & 0.28 & 0.32 & 0.32\\
Vote: Far right &  &  &  &  &  &  & 0.10 & 0.10 & 0.10\\
\addlinespace
Country: FR &  &  &  &  &  &  & 0.24 & 0.24 & 0.24\\
Country: DE &  &  &  &  &  &  & 0.33 & 0.33 & 0.33\\
Country: ES &  &  &  &  &  &  & 0.18 & 0.18 & 0.18\\
Country: UK &  &  &  &  &  &  & 0.25 & 0.25 & 0.25\\
\addlinespace
Urbanity: Cities &  &  &  &  &  &  & 0.43 & 0.49 & 0.43\\
Urbanity: Towns and suburbs &  &  &  &  &  &  & 0.33 & 0.32 & 0.33\\
Urbanity: Rural &  &  &  &  &  &  & 0.25 & 0.20 & 0.25\\
\bottomrule
\end{tabular}
        }
    }
    {\footnotesize \textit{Note}: This table displays summary statistics of the samples alongside actual population frequencies. %For \textit{Vote}, we regroup candidates or parties into three broad categories and we take abstention into account (but omit this category). 
    %For \textit{Inactivity rate (15-64)}, the sample statistics include the share of respondents aged between 15 and 64 years old who indicated being either ``\textit{Inactive (not searching for a job)},'' a ``\textit{Student},'' or ``\textit{Retired}.'' For \textit{Unemployment rate (15-64)}, the sample statistics include the share of respondents aged between 15 and 64 years old who indicated being ``\textit{Unemployed (searching for a job)}'', (`\textit{Unemployed (searching for a job)},'' ``\textit{Full-time employed},'' ``\textit{Part-time employed},'' or ``\textit{Self-employed}''). For	\textit{Employment rate (15-64)}, the sample statistics include the share of respondents aged between 15 and 64 years old who indicated being either ``\textit{Full-time employed},'' ``\textit{Part-time employed},'' or ``\textit{Self-employed}.'' 
    Detailed sources for each variable and country population frequencies, as well as the definitions of regions, diploma, urbanity, employment, and vote are available in \href{https://github.com/bixiou/global_tax_attitudes/raw/main/questionnaire/specificities.xlsx}{this spreadsheet}. % TODO! Appendix \ref{app:sources}.
    } % TODO add hline before Urbanity, move Country/Urbanity above and add in Notes that quotas are those above the line
\end{table}

\begin{table}[h]
    \caption[Sample representativeness of each European country]{Sample representativeness for each European country. (Back to \ref{par:surveys})} \label{tab:representativeness_EU}
    \makebox[\textwidth][c]{
        \resizebox*{!}{.50\textheight}{% 73 without notes cf. https://tex.stackexchange.com/questions/13809/resizing-a-table-by-textheight 
        
\begin{tabular}[t]{lllllllllllll}
\toprule
\multicolumn{1}{c}{} & \multicolumn{3}{c}{FR} & \multicolumn{3}{c}{DE} & \multicolumn{3}{c}{ES} & \multicolumn{3}{c}{UK} \\
\cmidrule(l{3pt}r{3pt}){2-4} \cmidrule(l{3pt}r{3pt}){5-7} \cmidrule(l{3pt}r{3pt}){8-10} \cmidrule(l{3pt}r{3pt}){11-13}
  & Pop. & Sample & \makecell{Weighted\\sample} & Pop. & Sample & \makecell{Weighted\\sample} & Pop. & Sample & \makecell{Weighted\\sample} & Pop. & Sample & \makecell{Weighted\\sample}\\
\midrule
Sample size &  & 620 & 620 &  & 757 & 757 &  & 543 & 543 &  & 644 & 644\\
\addlinespace
Gender: Woman & 0.52 & 0.49 & 0.54 & 0.51 & 0.53 & 0.58 & 0.51 & 0.55 & 0.60 & 0.50 & 0.26 & 0.32\\
Gender: Man & 0.48 & 0.51 & 0.46 & 0.49 & 0.47 & 0.42 & 0.49 & 0.45 & 0.40 & 0.50 & 0.74 & 0.68\\
\addlinespace
Income\_quartile: 1 & 0.25 & 0.30 & 0.27 & 0.25 & 0.28 & 0.23 & 0.25 & 0.27 & 0.23 & 0.25 & 0.32 & 0.28\\
Income\_quartile: 2 & 0.25 & 0.17 & 0.17 & 0.25 & 0.25 & 0.24 & 0.25 & 0.32 & 0.33 & 0.25 & 0.29 & 0.28\\
Income\_quartile: 3 & 0.25 & 0.22 & 0.22 & 0.25 & 0.29 & 0.30 & 0.25 & 0.25 & 0.25 & 0.25 & 0.20 & 0.21\\
Income\_quartile: 4 & 0.25 & 0.32 & 0.34 & 0.25 & 0.18 & 0.23 & 0.25 & 0.15 & 0.19 & 0.25 & 0.19 & 0.23\\
\addlinespace
Age: 18-24 & 0.12 & 0.08 & 0.06 & 0.09 & 0.18 & 0.15 & 0.08 & 0.17 & 0.15 & 0.10 & 0.02 & 0.02\\
Age: 25-34 & 0.15 & 0.17 & 0.16 & 0.15 & 0.21 & 0.20 & 0.12 & 0.15 & 0.14 & 0.17 & 0.10 & 0.09\\
Age: 35-49 & 0.24 & 0.33 & 0.37 & 0.22 & 0.20 & 0.22 & 0.28 & 0.23 & 0.26 & 0.24 & 0.12 & 0.15\\
Age: 50-64 & 0.24 & 0.20 & 0.19 & 0.28 & 0.23 & 0.26 & 0.27 & 0.25 & 0.27 & 0.25 & 0.28 & 0.33\\
Age: 65+ & 0.25 & 0.23 & 0.22 & 0.26 & 0.18 & 0.18 & 0.25 & 0.19 & 0.19 & 0.24 & 0.48 & 0.42\\
\addlinespace
Urbanity: Cities & 0.47 & 0.51 & 0.43 & 0.37 & 0.47 & 0.40 & 0.52 & 0.67 & 0.62 & 0.40 & 0.37 & 0.31\\
Urbanity: Towns and suburbs & 0.19 & 0.18 & 0.18 & 0.40 & 0.34 & 0.34 & 0.22 & 0.27 & 0.29 & 0.42 & 0.46 & 0.47\\
Urbanity: Rural & 0.34 & 0.30 & 0.39 & 0.23 & 0.18 & 0.25 & 0.26 & 0.06 & 0.08 & 0.18 & 0.17 & 0.22\\
\addlinespace
Diploma\_25\_64: Below upper secondary & 0.11 & 0.22 & 0.18 & 0.10 & 0.17 & 0.16 & 0.24 & 0.10 & 0.09 & 0.12 & 0.10 & 0.08\\
Diploma\_25\_64: Upper secondary & 0.26 & 0.15 & 0.24 & 0.27 & 0.11 & 0.18 & 0.16 & 0.15 & 0.23 & 0.21 & 0.18 & 0.29\\
Diploma\_25\_64: Post secondary & 0.26 & 0.33 & 0.30 & 0.29 & 0.36 & 0.33 & 0.28 & 0.38 & 0.33 & 0.33 & 0.23 & 0.20\\
\addlinespace
Employment\_18\_64: Inactive & 0.20 & 0.18 & 0.16 & 0.15 & 0.16 & 0.14 & 0.20 & 0.16 & 0.15 & 0.16 & 0.14 & 0.15\\
Employment\_18\_64: Unemployed & 0.04 & 0.05 & 0.05 & 0.02 & 0.04 & 0.04 & 0.07 & 0.10 & 0.10 & 0.02 & 0.03 & 0.03\\
\addlinespace
Vote: Left & 0.23 & 0.18 & 0.17 & 0.37 & 0.42 & 0.42 & 0.33 & 0.37 & 0.38 & 0.25 & 0.27 & 0.27\\
Vote: Center-right or Right & 0.26 & 0.31 & 0.32 & 0.28 & 0.26 & 0.27 & 0.18 & 0.22 & 0.22 & 0.36 & 0.50 & 0.50\\
Vote: Far right & 0.23 & 0.23 & 0.24 & 0.08 & 0.07 & 0.08 & 0.09 & 0.08 & 0.07 & 0.01 & 0.03 & 0.04\\
\bottomrule
\end{tabular}
        }
    }
    % TODO add explanatory note
    {\footnotesize \textit{Note}: This table displays summary statistics of the samples alongside actual population frequencies. In this Table, weights are defined at the country level.  %For \textit{Vote}, we regroup candidates or parties into three broad categories and we take abstention into account (but omit this category). 
    %For \textit{Inactivity rate (15-64)}, the sample statistics include the share of respondents aged between 15 and 64 years old who indicated being either ``\textit{Inactive (not searching for a job)},'' a ``\textit{Student},'' or ``\textit{Retired}.'' For \textit{Unemployment rate (15-64)}, the sample statistics include the share of respondents aged between 15 and 64 years old who indicated being ``\textit{Unemployed (searching for a job)}'', (`\textit{Unemployed (searching for a job)},'' ``\textit{Full-time employed},'' ``\textit{Part-time employed},'' or ``\textit{Self-employed}''). For	\textit{Employment rate (15-64)}, the sample statistics include the share of respondents aged between 15 and 64 years old who indicated being either ``\textit{Full-time employed},'' ``\textit{Part-time employed},'' or ``\textit{Self-employed}.'' 
    Detailed sources for each variable and country population frequencies, as well as the definitions of regions, diploma, urbanity, employment, and vote are available in \href{https://github.com/bixiou/global_tax_attitudes/raw/main/questionnaire/specificities.xlsx}{this spreadsheet}. % TODO Appendix \ref{app:sources}.
    }
\end{table}

Similar tables for the global surveys can be found in \citet{dechezlepretre_fighting_2022}.

\clearpage
\section{Attrition analysis}\label{app:attrition}

\begin{table}[h]\label{tab:attrition_US1}
    \caption[Attrition analysis: US1]{Attrition analysis for the US1 survey.} 
    \makebox[\textwidth][c]{
\resizebox*{!}{.73\textheight}{ % 73 is the max when there is a title
        
\begin{tabular}{@{\extracolsep{5pt}}lccccc} 
\\[-1.8ex]\hline 
\hline \\[-1.8ex] 
\\[-1.8ex] & \makecell{Dropped out} & \makecell{Dropped out\\after\\socio-eco} & \makecell{Failed\\attention test} & \makecell{Duration\\(in min)} & \makecell{Duration\\below\\4 min} \\ 
\\[-1.8ex] & (1) & (2) & (3) & (4) & (5)\\ 
\hline \\[-1.8ex] 
Mean & 0.08 & 0.059 & 0.082 & 21.198 & 0.016  \\ \hline \\[-1.8ex]
 Income quartile: 3 & 0.001 & 0.001 & $-$0.022$^{*}$ & $-$0.770 & $-$0.009 \\ 
  & (0.010) & (0.010) & (0.012) & (3.203) & (0.006) \\ 
  Income quartile: 4 & 0.004 & 0.004 & $-$0.029$^{**}$ & 0.775 & $-$0.004 \\ 
  & (0.012) & (0.012) & (0.012) & (2.737) & (0.007) \\ 
  Diploma: Post secondary & $-$0.012 & $-$0.012 & 0.011 & $-$4.141 & $-$0.004 \\ 
  & (0.012) & (0.012) & (0.014) & (2.803) & (0.007) \\ 
  Age: 25-34 & 0.006 & 0.006 & 0.001 & 1.004 & 0.004 \\ 
  & (0.009) & (0.009) & (0.009) & (2.509) & (0.005) \\ 
  Age: 35-49 & $-$0.058$^{***}$ & $-$0.058$^{***}$ & 0.001 & $-$0.859 & $-$0.032$^{**}$ \\ 
  & (0.015) & (0.015) & (0.019) & (2.503) & (0.013) \\ 
  Age: 50-64 & $-$0.053$^{***}$ & $-$0.053$^{***}$ & 0.001 & 4.431 & $-$0.033$^{***}$ \\ 
  & (0.015) & (0.015) & (0.017) & (2.945) & (0.013) \\ 
  Age: 65+ & $-$0.031$^{**}$ & $-$0.031$^{**}$ & $-$0.055$^{***}$ & 5.358$^{**}$ & $-$0.041$^{***}$ \\ 
  & (0.015) & (0.015) & (0.016) & (2.556) & (0.012) \\ 
  Race: Black & 0.034$^{*}$ & 0.034$^{*}$ & $-$0.061$^{***}$ & 8.417$^{**}$ & $-$0.050$^{***}$ \\ 
  & (0.018) & (0.018) & (0.016) & (4.117) & (0.012) \\ 
  Race: Hispanic & 0.026$^{**}$ & 0.026$^{**}$ & 0.017 & 7.964$^{***}$ & 0.003 \\ 
  & (0.010) & (0.010) & (0.014) & (2.759) & (0.008) \\ 
  Gender: Man & 0.007 & 0.007 & 0.120$^{**}$ & $-$2.808 & 0.031 \\ 
  & (0.024) & (0.024) & (0.047) & (1.804) & (0.029) \\ 
  Region: Northeast & $-$0.049$^{***}$ & $-$0.049$^{***}$ & 0.020$^{**}$ & $-$0.344 & 0.00003 \\ 
  & (0.007) & (0.007) & (0.009) & (2.339) & (0.005) \\ 
  Region: South & 0.0002 & 0.0002 & 0.010 & $-$4.919 & $-$0.004 \\ 
  & (0.011) & (0.011) & (0.013) & (4.796) & (0.007) \\ 
  Region: West & $-$0.004 & $-$0.004 & 0.009 & $-$0.945 & $-$0.004 \\ 
  & (0.009) & (0.009) & (0.011) & (4.520) & (0.006) \\ 
  Urban & 0.005 & 0.005 & $-$0.020 & $-$4.232 & $-$0.004 \\ 
  & (0.011) & (0.011) & (0.013) & (4.485) & (0.007) \\ 
  urban & 0.001 & 0.001 & 0.008 & 4.599$^{**}$ & $-$0.005 \\ 
  & (0.009) & (0.009) & (0.010) & (2.221) & (0.006) \\ 
 \hline \\[-1.8ex] 

Observations & 5,719 & 5,719 & 3,252 & 3,044 & 3,044 \\ 
R$^{2}$ & 0.023 & 0.023 & 0.030 & 0.006 & 0.016 \\ 
\hline 
\hline \\[-1.8ex] 
\end{tabular} 
        }
    }
    {\footnotesize %\textit{Note}: 
    }
\end{table}

\begin{table}[h]\label{tab:attrition_US2}
    \caption[Attrition analysis: US2]{Attrition analysis for the US2 survey.} 
    \makebox[\textwidth][c]{
\resizebox*{!}{.73\textheight}{ % 73 is the max when there is a title
        
\begin{tabular}{@{\extracolsep{5pt}}lccccc} 
\\[-1.8ex]\hline 
\hline \\[-1.8ex] 
\\[-1.8ex] & \makecell{Dropped out} & \makecell{Dropped out\\after\\socio-eco} & \makecell{Failed\\attention test} & \makecell{Duration\\(in min)} & \makecell{Duration\\below\\4 min} \\ 
\\[-1.8ex] & (1) & (2) & (3) & (4) & (5)\\ 
\hline \\[-1.8ex] 
Mean & 0.105 & 0.08 & 0.112 & 21.78 & 0.041  \\ \hline \\[-1.8ex]
 Income quartile: 2 & 0.007 & 0.007 & $-$0.053$^{***}$ & 1.441 & $-$0.043$^{***}$ \\ 
  & (0.022) & (0.022) & (0.020) & (3.244) & (0.015) \\ 
  Income quartile: 3 & 0.020 & 0.020 & $-$0.011 & 45.106 & $-$0.033 \\ 
  & (0.030) & (0.030) & (0.034) & (46.289) & (0.025) \\ 
  Income quartile: 4 & $-$0.002 & $-$0.002 & $-$0.003 & 1.041 & $-$0.079$^{***}$ \\ 
  & (0.043) & (0.043) & (0.061) & (10.058) & (0.019) \\ 
  Diploma: Post secondary & $-$0.043$^{**}$ & $-$0.043$^{**}$ & $-$0.043$^{**}$ & 9.394 & 0.026 \\ 
  & (0.021) & (0.021) & (0.020) & (9.764) & (0.016) \\ 
  Age: 25-34 & 0.053$^{*}$ & 0.053$^{*}$ & $-$0.045 & $-$7.393 & 0.017 \\ 
  & (0.030) & (0.030) & (0.042) & (6.961) & (0.033) \\ 
  Age: 35-49 & 0.052$^{**}$ & 0.052$^{**}$ & $-$0.042 & 17.468 & 0.006 \\ 
  & (0.026) & (0.026) & (0.039) & (16.385) & (0.029) \\ 
  Age: 50-64 & 0.066$^{**}$ & 0.066$^{**}$ & $-$0.071$^{*}$ & $-$7.421 & $-$0.042$^{*}$ \\ 
  & (0.029) & (0.029) & (0.040) & (9.109) & (0.025) \\ 
  Age: 65+ & 0.057$^{*}$ & 0.057$^{*}$ & $-$0.107$^{***}$ & $-$1.734 & $-$0.052$^{**}$ \\ 
  & (0.030) & (0.030) & (0.037) & (9.343) & (0.025) \\ 
  Race: Black & 0.100$^{***}$ & 0.100$^{***}$ & $-$0.011 & 20.168 & $-$0.016 \\ 
  & (0.021) & (0.021) & (0.033) & (14.147) & (0.023) \\ 
  Race: Hispanic & 0.062$^{***}$ & 0.062$^{***}$ & $-$0.054 & $-$4.035 & $-$0.028 \\ 
  & (0.019) & (0.019) & (0.033) & (7.283) & (0.023) \\ 
  Gender: Man & $-$0.050$^{***}$ & $-$0.050$^{***}$ & 0.015 & 13.563 & 0.017 \\ 
  & (0.018) & (0.018) & (0.023) & (16.255) & (0.017) \\ 
  Region: Northeast & $-$0.018 & $-$0.018 & 0.030 & $-$4.964 & 0.014 \\ 
  & (0.030) & (0.030) & (0.043) & (4.837) & (0.029) \\ 
  Region: South & 0.013 & 0.013 & $-$0.029 & 10.628 & 0.007 \\ 
  & (0.024) & (0.024) & (0.034) & (13.411) & (0.022) \\ 
  Region: West & 0.006 & 0.006 & $-$0.023 & 0.452 & 0.010 \\ 
  & (0.029) & (0.029) & (0.038) & (5.076) & (0.027) \\ 
  Urban & 0.050$^{**}$ & 0.050$^{**}$ & 0.007 & 8.278 & 0.001 \\ 
  & (0.019) & (0.019) & (0.026) & (6.513) & (0.018) \\ 
 \hline \\[-1.8ex] 

Observations & 946 & 946 & 777 & 706 & 706 \\ 
R$^{2}$ & 0.042 & 0.042 & 0.046 & 0.023 & 0.043 \\ 
\hline 
\hline \\[-1.8ex] 
\end{tabular} 
        }
    }
    {\footnotesize %\textit{Note}: 
    }
\end{table}

\begin{table}[h]\label{tab:attrition_EU}
    \caption[Attrition analysis: Eu]{Attrition analysis for the Eu survey.} 
    \makebox[\textwidth][c]{
\resizebox*{!}{.73\textheight}{ % 73 is the max when there is a title
        
\begin{tabular}{@{\extracolsep{5pt}}lccccc} 
\\[-1.8ex]\hline 
\hline \\[-1.8ex] 
\\[-1.8ex] & \makecell{Dropped out} & \makecell{Dropped out\\after\\socio-eco} & \makecell{Failed\\attention test} & \makecell{Duration\\(in min)} & \makecell{Duration\\below\\6 min} \\ 
\\[-1.8ex] & (1) & (2) & (3) & (4) & (5)\\ 
\hline \\[-1.8ex] 
Mean & 0.067 & 0.044 & 0.151 & 54.602 & 0.039  \\ \hline \\[-1.8ex]
 Income quartile: 3 & 0.001 & $-$0.001 & $-$0.031$^{**}$ & 27.825 & $-$0.015 \\ 
  & (0.013) & (0.012) & (0.013) & (20.371) & (0.010) \\ 
  Income quartile: 4 & 0.002 & 0.001 & $-$0.061$^{***}$ & 0.612 & $-$0.022$^{**}$ \\ 
  & (0.014) & (0.013) & (0.011) & (11.887) & (0.010) \\ 
  Diploma: Post secondary & $-$0.022 & $-$0.024$^{*}$ & $-$0.042$^{***}$ & 13.029 & $-$0.019$^{*}$ \\ 
  & (0.014) & (0.014) & (0.013) & (19.608) & (0.010) \\ 
  Age: 25-34 & $-$0.006 & $-$0.005 & $-$0.033$^{***}$ & 5.978 & $-$0.008 \\ 
  & (0.011) & (0.010) & (0.009) & (12.265) & (0.007) \\ 
  Age: 35-49 & 0.028$^{**}$ & 0.025$^{**}$ & 0.033$^{*}$ & 33.335 & $-$0.004 \\ 
  & (0.013) & (0.013) & (0.018) & (20.624) & (0.018) \\ 
  Age: 50-64 & 0.048$^{***}$ & 0.047$^{***}$ & $-$0.006 & 32.456$^{**}$ & $-$0.013 \\ 
  & (0.013) & (0.012) & (0.016) & (14.803) & (0.016) \\ 
  Age: 65+ & 0.074$^{***}$ & 0.073$^{***}$ & $-$0.010 & 41.300$^{**}$ & $-$0.063$^{***}$ \\ 
  & (0.014) & (0.014) & (0.017) & (20.533) & (0.015) \\ 
  Gender: Man & 0.142$^{***}$ & 0.140$^{***}$ & $-$0.011 & 26.513$^{**}$ & $-$0.063$^{***}$ \\ 
  & (0.016) & (0.016) & (0.017) & (12.755) & (0.015) \\ 
  Urban & $-$0.031$^{***}$ & $-$0.031$^{***}$ & 0.013 & $-$24.850$^{*}$ & 0.010 \\ 
  & (0.009) & (0.009) & (0.009) & (14.378) & (0.007) \\ 
  urban & $-$0.010 & $-$0.009 & 0.016$^{*}$ & 13.704 & $-$0.005 \\ 
  & (0.009) & (0.009) & (0.008) & (15.465) & (0.007) \\ 
 \hline \\[-1.8ex] 

Observations & 3,963 & 3,963 & 3,326 & 3,115 & 3,115 \\ 
R$^{2}$ & 0.026 & 0.026 & 0.021 & 0.003 & 0.024 \\ 
\hline 
\hline \\[-1.8ex] 
\end{tabular} 
        }
    }
    {\footnotesize %\textit{Note}: 
    }
\end{table} 

% \begin{itemize}
% \item[Acknowledgments] %I am grateful to 
% %  \item[Competing Interests] I declare that I also serve as president of Global Redistribution Advocates.
% \item[JEL codes] 
% \item[Keywords]  
% \item[Correspondence] Correspondence and requests for materials should be addressed to Adrien Fabre~(email: adrien.fabre@cnrs.fr).
% \end{itemize}
% % \end{addendum}

% \onehalfspacing
% \clearpage
\listoftables
\listoffigures

\appendix % NCCcomment
\renewcommand{\thetable}{A\arabic{table}}
\renewcommand{\thefigure}{A\arabic{figure}}
\setcounter{figure}{0}
\setcounter{table}{0}

\clearpage
\section{Appendix}


\subsection{Additional tables}
% \input{../tables/income.tex}


\end{document}
