\documentclass[12pt,english]{article}
\usepackage[utf8]{inputenc}
\usepackage{tgpagella} % Palatino text only
\usepackage{mathpazo}  % Palatino math & text
\usepackage[left=1.5in,right=1.5in,top=1.5in,bottom=1.5in]{geometry}
% \linespread{1.5}
\usepackage[super,comma,sort]{natbib}
% \usepackage[round,sort&compress]{natbib}
\usepackage{url} % [hyphens]
\usepackage[hyperpageref]{backref} % back references biblio. Needs latexmk at compilation.
\usepackage[pagebackref]{hyperref}
% \usepackage{multibib} % incompatible with backref
\hypersetup{
  colorlinks=true, % breaklinks=true,
  urlcolor=purple,    % color of external links
  linkcolor=blue,  % color of toc, list of figs etc.
  citecolor=violet,   % color of links to bibliography
}
\usepackage{bm}
\usepackage{indentfirst}
\usepackage{tocbibind}
\setcitestyle{aysep={}} 
\usepackage{amsmath}
\usepackage{amssymb}
\usepackage{eurosym}
\usepackage{amsfonts}
\usepackage{enumerate}
\usepackage{babel}
\usepackage{caption}
\usepackage{supertabular}
\usepackage{tabularx}
\usepackage{float}
\usepackage{dsfont}
\usepackage{fancyvrb}
\usepackage{verbatim}
\usepackage{enumitem}
\usepackage{setspace}
\usepackage{comment}
\usepackage{subcaption}
\usepackage{graphicx}
\usepackage{tikz}
\usepackage{gensymb}
\usepackage{textcomp}

\usepackage{tabulary}
\usepackage{tabularx}
\usepackage{booktabs}
\usepackage{fullpage}
\usepackage{morefloats}
\usepackage{makecell}
\usepackage{lscape}
\usepackage{pdflscape}
\usepackage{longtable}
\usepackage{rotating}
\usepackage{fancyhdr}
\usepackage{tocloft}
\usepackage{titletoc}
\usepackage[export]{adjustbox}
\usepackage[anythingbreaks]{breakurl} % for links
\usepackage{multicol}
\newsavebox\ltmcbox % For net gain table over two columns
%\usepackage[nomarkers,figuresonly]{endfloat} % Figures at the end
%\usepackage[section,below]{placeins} % Floats placed in the section they appear in.
\renewcommand{\floatpagefraction}{.9}

\title{Global Climate Scheme -- Policy Brief
} 

\author{Adrien Fabre\footnote{CNRS, CIRED. E-mail: fabre.adri1@gmail.com (corresponding author).}} 

\date{\today} 

\begin{document}

\maketitle

\begin{center}
{\textbf{\href{https://github.com/bixiou/global_tax_attitudes/raw/main/paper/policy_brief_GCS.pdf}{Link to most recent version}}}
\end{center}

Pistes pour éviter des transferts des pays pauvres vers les riches et que la Chine ne perde trop:
A. Que les pays à moyens (ou bas) revenus puissent se retirer (opt out) de la mutualisation des recettes et du revenu de base. La banque mondiale définit le seuil de hauts revenus à un GNI pc nominal de 13.2k (Chine: 11.9, Russie: 11.6, Arabie Saoudite: 21.5k, monde: 12k). On pourrait aussi choisir un seuil plus élevé (e.g. deux fois le GNI pc mondial soit 24k).
Pb: Sortir de la mutualisation casse la logique, difficile de calculer l'empreinte carbone. Effet de seuil. 

G. Comme ci-dessus mais on pour les pays s'étant retirés, on calcule leurs recettes sur la base des émissions territoriales (avec une borne max à +50\% des émissions mondiales moyennes). Et on ne les autorise à se retirer que s'ils participent à la taxation des millionaires en partie redistribuée aux pays pauvres. On calcule la taxe sur la fortune de sorte que ça compense le gain qu'ils obtiennent les recettes en fonction de leurs émissions territoriales plutôt que de leur empreinte. 
=> calculer le montant nécessaire -> Environ .1\% du PIB mondial pour la Chine (cf. policy_brief_tax). Si 1/3 est reversé aux paus à bas revenus, il faut une taxe qui rapporte .3\% du PIB mondial en Chine, soit \~2\% du PIB mondial en tout. Ce qui est faisable en ne s'attaquant qu'aux fortunes >5M et sans même les réduire (taux max de 7\%). Ça opérerait un transfert de \~.6\% du PIB mondial, du même ordre que le \~.75\% du GCS. À comparer au .85T\$ (surestimé, cf. calcul dans map_GCS_incidence.R) nécessaire pour résorber le poverty gap à 3.65\$ (il est de 4T pour le pg à 6.85\$/day)

B. Les pays à hauts revenus renonceraient aux recettes. Ça augmenterait le revenu de base de \~20\%, et diviserait par près de deux la perte de la Chine.
Pb: Pas sûr que les pays riches acceptent. Pas sûr que ça suffise pour convaincre la Chine.

C. Compléter la mesure par l'établissement d'une dette carbone due aux responsabilités historiques (entre 1990 et l'entrée en vigueur de l'accord). Le financement des émissions négatives devra être assuré au pro rata des dettes carbone.
Pb: Difficile de calculer l'empreinte carbone et de choisir une convention sur la population. Rompt avec la logique de taxer les individus. Pas sûr que la Chine accepte de payer en échange d'une promesse de transferts (non monétaire mais carbone) futurs. 

D. Les pays ayant excédé leur budget carbone 1.5°C ne toucheraient pas le revenu de base.
=> check lesquels c'est

E. Si un pays est dans le top 30\% du PIB pc mais en-dessous de la moyenne pour les émissions pc, alors il ne peut pas toucher le revenu de base, et cet argent est à la place reversé aux pays pauvres qui ont des émissions supérieures à la moyenne.
Pb: Effet de seuil. Sort de la logique du pollueur-payeur (en pratique, réduit le prix global des émissions pour les pays pauvresp polluants). Pas sûr que ça suffise à convaincre la Chine.

F. Compléter par une taxe sur les millionaires reversée aux pays à moyen et bas revenus du club (ou sous 2*GNIpc moyen), de façon dégressive de leur GNIpc. 
=> calculer quelle somme doit être versée pour compenser la Chine -> 0.1\% du PIB mondial (leur empreinte est +50\% de la moyenne, et à \~24\% du total, donc les recettes à compenser sont \~6\% du total, qui est 1.7\% du PIB mondial)
Pb: Les US n'accepteront peut-être pas de tels transferts vers la Chine.




\renewcommand{\url}[1]{\href{#1}{Link}} % NCCcomment
\bibliographystyle{plainnaturl_clean} % NCCcomment
\bibliography{global_tax_attitudes}
