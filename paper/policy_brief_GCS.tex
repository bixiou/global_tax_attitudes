\documentclass[12pt,english]{article}
\usepackage[utf8]{inputenc}
\usepackage{tgpagella} % Palatino text only
\usepackage{mathpazo}  % Palatino math & text
\usepackage[left=1.5in,right=1.5in,top=1.5in,bottom=1.5in]{geometry}
% \linespread{1.5}
% \usepackage[super,comma,sort]{natbib}
\usepackage[round,sort&compress]{natbib}
\usepackage{url} % [hyphens]
\usepackage[hyperpageref]{backref} % back references biblio. Needs latexmk at compilation.
\usepackage[pagebackref]{hyperref}
% \usepackage{multibib} % incompatible with backref
\hypersetup{
  colorlinks=true, % breaklinks=true,
  urlcolor=purple,    % color of external links
  linkcolor=blue,  % color of toc, list of figs etc.
  citecolor=violet,   % color of links to bibliography
}
\usepackage{bm}
\usepackage{indentfirst}
\usepackage{tocbibind}
\setcitestyle{aysep={}} 
\usepackage{amsmath}
\usepackage{amssymb}
\usepackage{eurosym}
\usepackage{amsfonts}
\usepackage{enumerate}
\usepackage{babel}
\usepackage{caption}
\usepackage{supertabular}
\usepackage{tabularx}
\usepackage{float}
\usepackage{dsfont}
\usepackage{fancyvrb}
\usepackage{verbatim}
\usepackage{enumitem}
\usepackage{setspace}
\usepackage{comment}
\usepackage{subcaption}
\usepackage{graphicx}
\usepackage{tikz}
\usepackage{gensymb}
\usepackage{textcomp}

\usepackage{tabulary}
\usepackage{tabularx}
\usepackage{booktabs}
\usepackage{fullpage}
\usepackage{morefloats}
\usepackage{makecell}
\usepackage{lscape}
\usepackage{pdflscape}
\usepackage{longtable}
\usepackage{rotating}
\usepackage{fancyhdr}
\usepackage{tocloft}
\usepackage{titletoc}
\usepackage{csquotes}
\usepackage[export]{adjustbox}
\usepackage[anythingbreaks]{breakurl} % for links
\usepackage{multicol}
\newsavebox\ltmcbox % For net gain table over two columns
%\usepackage[nomarkers,figuresonly]{endfloat} % Figures at the end
%\usepackage[section,below]{placeins} % Floats placed in the section they appear in.
\renewcommand{\floatpagefraction}{.9}

\title{The Global Climate Scheme -- Policy Brief
} 

\author{Adrien Fabre\footnote{CNRS researcher in economics at CIRED. E-mail: fabre.adri1@gmail.com.}} 

\date{\today} 

\begin{document}

\maketitle

% \begin{center}
% {\textbf{\href{https://github.com/bixiou/global_tax_attitudes/raw/main/paper/policy_brief_GCS.pdf}{Link to most recent version}}}
% \end{center}

% Summary: survey description + results
% Principle. Link to econ theory, CBAM/carbon market merge, SDGs.
% Distributive effects.
% Acceptation
% Details of the agreement
% Details of implementation (e.g. Aadhaar)

\section{Introduction}\label{sec:intro}

\begin{quote}
  ``At the Paris agreement in 2015, all countries have agreed to contain global warming `well below +2 $\mathrm{{}^\circ}$C'. To limit global warming to this level,~\textbf{there is a maximum amount of greenhouse gases we can emit globally}.\\
  To meet the climate target, a limited number of permits to emit greenhouse gases can be created globally. Polluting firms would be required to buy permits to cover their emissions. Such a policy would~\textbf{make fossil fuel companies pay}~for their emissions and progressively raise the price of fossil fuels.~\textbf{Higher prices would encourage people and companies to use less fossil fuels, reducing greenhouse gas emissions.}\\
  In accordance with the principle that each human has an equal right to pollute, the revenues generated by the sale of permits could finance a global basic income.~\textbf{Each adult in the world would receive } \$30/month, thereby lifting out of extreme poverty the 700 million people who earn less than \$2/day.\\
  \textbf{The typical }[\textbf{American}]\textbf{ would lose out financially }[\textbf{\$85}]\textbf{ per month}~(as he or she would face [\$115] per month in price increases, which is higher than the \$30 they would receive).\\
  The policy could be put in place as soon as countries totaling more than 60\% of global emissions agree on it. Countries that would refuse to take part in the policy could face sanctions (like tariffs) from the rest of the World and would be excluded from the basic income.''
\end{quote}

In a representative survey on 3,000 respondents, \citet{fabre_international_2023} show that 54\% of Americans support the Global Climate Scheme (GCS) as described above. Actually, \citet{fabre_international_2023} also run the survey on 3,000 Europeans (representative of France, Germany, Spain and the UK) and find that 76\% of them support the GCS. Moreover, in a survey on 40,680 respondents in 20 countries covering 72\% of global CO$_\text{2}$ emissions, \citet{dechezlepretre_fighting_2022} find strong majority support in each country for such a policy.

In this policy brief, we make the case for a Global Climate Scheme. We show that it is grounded on solid ethics and economics (Section \ref{sec:principles}), would operate a global redistribution from rich to poor (Section \ref{sec:distribution}), can be implemented with current technology (Section \ref{sec:implementation}), and is genuinely supported by the population across the World (Section \ref{sec:support}). Finally, we expand on the above description and formulate a well-specified plan (Section \ref{sec:details}).

\section{Principles}\label{sec:principles}
% Paris agreement, SDGs
The Global Climate Scheme would help achieving the internationally agreed agenda for a prosperous future. While the Paris agreement sets an unanimous climate objective, it does not establish binding rules, and current policies place the world on track to a temperature rise of 2.7\textdegree{}C in 2100 \citep{climate_action_tracker_warming_2022}. Likewise, the Sustainable Development Goals set different targets for 2030, the first one being to eradicate extreme poverty defined as living on less than \$1.90 a day (in 2011 PPP), and we are not on track to achieve this target as 8\% of the world population still live in extreme poverty \citep{un_sustainable_2022}. Meanwhile, the nominal GDP per capita (in 2021) is \href{https://data.worldbank.org/indicator/NY.GDP.PCAP.CD?end=2021&locations=EU-ZG-XD-XM-1W-IN-US-CD-BI-LU-CN&start=2021&view=bar}{62 times larger} in high-income countries (home to 1.2 billion people) than in low-income countries (700 million), meaning that a transfer of just 1\% of high-income countries' GDP would mechanically double low-income countries' national income. 

By design, the Global Climate Scheme (GCS) would stop global warming at a reasonable level, eradicate poverty, and make a dent on global inequalities. It relies on four principles:
\paragraph*{1. A cap on emissions to meet the 2\textdegree{}C target.} 

To limit global warming to 2\textdegree{}C with 67\% probability, we can deduce from the \citet{ipcc_climate_2021} and \href{https://ourworldindata.org/co2-emissions#global-co2-emissions-from-fossil-fuels-and-land-use-change}{current emissions} that the world has a remaining carbon budget of about 1,000GtCO$_\text{2}$ starting from 2024. 
% computed using 37.5G/year https://ourworldindata.org/co2-emissions#global-co2-emissions-from-fossil-fuels-and-land-use-change and IPCC SPM.2
Defining a global emissions trajectory and imposing a yearly quota on global CO$_\text{2}$ emissions would ensure that they decrease in line with the target. 
Emissions permits corresponding to the quota would then be auctioned ``upstream'' to industrial units that emit CO$_\text{2}$ or sell fossil fuels (like refineries, coal mines, or cementeries). 
% Trading of emissions allowances allows to reach efficiency in emissions reductions. 
In short, an Emissions Trading System (ETS) would be established to control CO$_\text{2}$ emissions at the global level. 
Implemented in various countries including the European Union, China, and South Korea, and being under consideration in others like India, Brazil or Nigeria, ETSs already cover 17\% of global GHG emissions. They can be successfully linked to one another, as California and Québec showed \citep{icap_emissions_2023}. 
% the EU, China, California, South Korea, Québec, Mexico, (Kazakhstan)

\paragraph*{2. Defending the interests of people rather than nations.}
Although global carbon pricing has long been discussed, it has stumbled upon the allocation of emissions entitlements between countries. 
For example, the U.S. has historically defended the free allocation of emissions permits to emitting sources while India has insisted on the historical responsibility of industralized countries to defend a redistributive solution \citep{bertram_tradeable_1992,michaelowa_report_2012}. 
An approach centered on individuals rather than countries helps escaping this impasse. Indeed, as shown in Section \ref{sec:support}, there is a worldwide consensus in favor of an equal right to emit for each human. 
Compared to other approaches, the egalitarian allocation has the merit of simplicity and provides a clear focal point. 
What is more, the individual approach can also be applied to address historical responsibilities, by redistributing individual wealth rather than attributing climate debts to industralized countries. In a separate policy brief, we propose a global wealth tax that would finance low-income countries as well as carbon removal. 
%To showcase the individual approach, it is worth introducing a thought experiment. Imagine if, in year 2000, all capital owners voluntarily redistributed their assets so that each human would get an equal wealth. In this utopian world, all countries now enjoy similar standards of living. Would it still make sense to transfer resources from the U.S. to India on the ground of historical responsibilities? Would it make sense to transfer resources from one person who gave away all their wealth to another person, now richer than the first, because the first one polluted more in the past?
Indeed, the best available approximation of the historical emissions of someone is arguably their wealth or, if the person died, the wealth of their descendents. Besides, ability to pay of individuals may be better suited than past emissions of countries to define fair shares of the decarbonization burden. % For example, should a person enjoy a lesser burden if they choose to live in France rather than Germany, on the ground that France historically decided to replace coal by nuclear in its electricity mix?
% Because carbon emissions were parallel to capital accumulation, countries that emitted more in the past tend to be richer today. But this is not always true (think of Ukraine). We argue that
% Past emissions disproportionally benefitted wealthy people, as carbon footprint increases with wealth. Attributing historical responsibilities to wealthy people

As the previous point exemplifies, the GCS is a good complement (rather than a substitute) to other climate or redistributive policies \citep{stiglitz_addressing_2019}. In particular, the GCS's negative effect on the purchasing power of an average emitter of a high-income country can be offset by national redistribution, through increased income taxes on the top 5\%. Furthermore, some decarbonization costs can be mutualized, e.g. through public investments in public transportation and subsidies to thermal insulation, to reduce the discrepancy in private costs between people with similar income but different carbon footprint. The GCS actually encourages complementary decarbonization policies, as countries decarbonizing faster will contribute less to the GCS revenues than countries entirely relying on the price mechanism. %which reduce the reliance on the carbon price incentive to decarbonize, 
% equal right to emit; efficiency of pricing, simplicity of rights allocation, complementarity with other climate & redistributive measures

\paragraph*{3. A global basic income that eradicates extreme poverty.}

The GCS revenues would be used to finance a global basic income. At their peak,  assuming a carbon price of \$90/tCO$_\text{2}$ in 2030, the GCS revenues are estimated to amount to 1.7\% of the Gross World Product. We use the price and emissions trajectories from the report by \citet{stern_report_2017} and estimate that the basic income would amount to \$30 per month for each human above 15 in 2030, enough to lift out of extreme poverty the 700 million people who live with less than \$2.15 a day. Conversely, high emitters like a typical German (with median German CO$_\text{2}$ emissions) would lose in net \euro{}25 per month, as they would face \euro{}55 per month in price increases. Overall, the GCS would operate substantial international transfers, estimated in Section \ref{sec:distribution}.  

Although distributing a basic income to every human is technically challenging, different options are available, reviewed in Section \ref{sec:implementation}. 

\paragraph*{4. A climate club to foster global cooperation.}

Building on insights from game theory \citep{mackay_price_2015,nordhaus_climate_2015}, the GCS should be launched by a club of voluntary countries, with carbon border adjustments and possibly sanctions to foster compliance by most countries. 
% 
% high CBAM for non-participants + possibly other sanctions (like on travel)
% USA: 15, In: 7, CN: 30, EU: 9, gain >0: 21 (gain >0 wo In: 14)
% a cap on emissions to meet the 2°C target
% granting an equal right to emit to each human / defending interests of people rather than nations
% a global basic income that eradicates extreme poverty
% a climate club (or sustainable union) to foster global cooperation

%Postulating %Assuming
% that each human has an equal right to emit CO$_\text{2}$, low emitters have a legitimate claim \textit{vis-à-vis} high emitters, that can be settled by monetary transfers. Coupling this burden-sharing principle to the carbon budget (remaining emissions that would be compatible with the Paris agreement) naturally defines a global climate policy. We call it the ``Global climate scheme'' (GCS); it consists of a global cap-and-trade system where emission rights are auctioned each year to polluting firms and the revenues finance a global basic income. 


\section{Distributive effects}\label{sec:distribution}


\section{Implementation}\label{sec:implementation}


\section{Support}\label{sec:support}


\section{Details of the scheme}\label{sec:details}

Vote with a weight corresponding to emissions(?) on emissions trajectory compatible with 2C with 66\% chance and without overshoot. Vote with same weight on details and sanctions for non-participants.

Pistes pour éviter des transferts des pays pauvres vers les riches et que la Chine ne perde trop:
$>$ A. Que les pays à moyens (ou bas) revenus puissent se retirer (opt out) de la mutualisation des recettes et du revenu de base. La banque mondiale définit le seuil de hauts revenus à un GNI pc nominal de 13.2k (Chine: 11.9, Russie: 11.6, Arabie Saoudite: 21.5k, monde: 12k). On pourrait aussi choisir un seuil plus élevé (e.g. deux fois le GNI pc mondial soit 24k).
Pb: Sortir de la mutualisation casse la logique, difficile de calculer l'empreinte carbone. Effet de seuil. 

$>$ G. Comme ci-dessus mais pour les pays s'étant retirés, on calcule leurs recettes sur la base des émissions territoriales (avec une borne max à +50\% des émissions mondiales moyennes). Et on ne les autorise à se retirer que s'ils participent à la taxation des millionaires en partie redistribuée aux pays pauvres. On calcule la taxe sur la fortune de sorte que ça compense le gain qu'ils obtiennent les recettes en fonction de leurs émissions territoriales plutôt que de leur empreinte. S'ils refusent la taxation des millionaires (et donc l'accord) et mettent en place une tarification des émissions unilatérale du même montant que dans l'accord, on met quand même en place un CBAM. En effet, on calcule le quota des pays de l'accord sur la base d'un égal droit à polluer pour chaque humain, ce qui laisse au pays hors accord exportateur d'émissions un droit à polluer inférieur à ses émissions. Comme le prix du carbon serait supérieur s'il rejoignait l'accord, on taxe ses émissions davantage. 
=$>$ calculer le montant nécessaire -$>$ Environ .1\% du PIB mondial pour la Chine (cf. policy\_brief\_tax). Si 1/3 est reversé aux paus à bas revenus, il faut une taxe qui rapporte .3\% du PIB mondial en Chine, soit $\approx 2\%$ du PIB mondial en tout. Ce qui est faisable en ne s'attaquant qu'aux fortunes $>$5M et sans même les réduire (taux max de 7\%). Ça opérerait un transfert de $\approx.6\%$ du PIB mondial, du même ordre que le $\approx$.75\% du GCS. À comparer au .85T\$ (surestimé, cf. calcul dans map\_GCS\_incidence.R) nécessaire pour résorber le poverty gap à 3.65\$ (il est de 4T pour le pg à 6.85\$/day)
=$>$ Pb: Effet de seuil + ça opèrerait un transfert des pays importateurs de la Chine à la Chine, puisque la Chine récupèrerait les recettes des émissions territoriales liées à ses exportations. Certes, mais c'est déjà comme ça que sont envisagés les fusions de marchés carbone ou CBAM (article 9), et le transfert ne serait pas énorme puisque les importateurs ne touchent qu'une part d'entre elles correspondant à leur population (les autres perdants seraient les pays à bas revenus qui verraient leur revenu de base diminué). Pour éviter l'effet de seuil, on peut dire que si le GNI pc du pays relatif au GNI pc mondial est de $1+y\in [1;2]$ alors une fraction y est mise en commun.

B. Les pays à hauts revenus renonceraient aux recettes. Ça augmenterait le revenu de base de \~20\%, et diviserait par près de deux la perte de la Chine.
Pb: Pas sûr que les pays riches acceptent. Pas sûr que ça suffise pour convaincre la Chine.

C. Compléter la mesure par l'établissement d'une dette carbone due aux responsabilités historiques (entre 1990 et l'entrée en vigueur de l'accord). Le financement des émissions négatives devra être assuré au pro rata des dettes carbone.
Pb: Difficile de calculer l'empreinte carbone et de choisir une convention sur la population. Rompt avec la logique de taxer les individus. Pas sûr que la Chine accepte de payer en échange d'une promesse de transferts (non monétaire mais carbone) futurs. 

D. Les pays ayant excédé leur budget carbone 1.5°C ne toucheraient pas le revenu de base.
=$>$ check lesquels c'est
=$>$ Pb: ça pourrait ptet résoudre le pb que les indiens émettraient plus que les Européens en 2050, mais ne résoudrait pas le pb de la Chine à court terme.

$>$ E. Si un pays est dans le top 30\% du PIB pc mais en-dessous de la moyenne pour les émissions pc, alors il ne peut pas toucher le revenu de base, et cet argent est à la place reversé aux pays pauvres qui ont des émissions supérieures à la moyenne.
Pb: Effet de seuil. Sort de la logique du pollueur-payeur (en pratique, réduit le prix global des émissions pour les pays pauvresp polluants). Pas sûr que ça suffise à convaincre la Chine.

F. Compléter par une taxe sur les millionaires reversée aux pays à moyen et bas revenus du club (ou sous 2*GNIpc moyen), de façon dégressive de leur GNIpc. 
=$>$ calculer quelle somme doit être versée pour compenser la Chine -$>$ 0.1\% du PIB mondial (leur empreinte est +50\% de la moyenne, et à \~24\% du total, donc les recettes à compenser sont \~6\% du total, qui est 1.7\% du PIB mondial)
Pb: Les US n'accepteront peut-être pas de tels transferts vers la Chine.




\renewcommand{\url}[1]{\href{#1}{Link}} % NCCcomment
\bibliographystyle{plainnaturl_clean} % NCCcomment
\bibliography{global_tax_attitudes}

\end{document}