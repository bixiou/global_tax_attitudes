% /!\ No footnote for NCC or NS.

% *. Main Figures
% 1. Methods
% A. Literature review
% B. Raw results
% C. Questionnaire (global survey)
% D. Questionnaire (complementary surveys)
% E. Net gains from the Global Climate Scheme
% F. Determinants of support
% G. Representativity of the surveys
% H. Attrition analysis
% I. Balance analysis
% J. Placebo tests

% TODO!! resolve "Section ??"
% TODO! more raw results, sources (cf. app.tex)
% TODO! Dechezlepretre as companion paper
% plot maps and compare distributive effects of equal pc, contraction & convergence, greenhouse dvlpt rights, historical respo, and each country retaining its revenues
% improve net gains with SSPs

% app_desc: some more figures (e.g. detailed OECD by country)
% conjoint d: by vote for all countries; merge with r; bug PDF
% TODO appendix sources
% autres to do!: n, additional figures/tables or details
% literature: list experiment, (elite surveys)
% Extra: translate country-specific appendices

% Now: 6200 words (excl. abstract, methods, appendix), incl. 5.2k in Results.
% Current: 5600 words (excl. abstract, methods, appendix), incl. 4.6k in Results.
% Old: 4300 words (excl. abstract, methods, appendix) + 3 medium-large figures (equivalent, I know it's 4) => correct size
% => write a 2-3 pager (1000 words with 2 figures and one-sentence abstract) on Word with Science's template, send it as submission enquiry to Nature (as one can't send full paper); then for Policy forum in Science.
% => write full paper as a 6-page (2500 words*) in LaTeX or Word so it can fit in PNAS, and even for NCC/NS it shouldn't exceed 8-9-page. (It's already too long for Science's max 5 page).
% *6-page is more 5000 than 2500 words. I've checked Steckel et al (NS, 21) and it's 5k words for 7p (excl. abstract and methods, dont 3.8k excl. the 6 medium-large figures/tables) + 2.2k in methods, data availability. Case & Deaton (PNAS, 21) is also 5k words for 5p (excl abstract and biblio, dont 4.7k excl. the 3 medium-large figures). Bruckner et al (NS, 22): 4.3k words for 6p (excl. abstract, methods, biblio, dont 3.8k excl the 5 medium-large figures) + 3.4k in methods. => about 1k per page without figure/table. 
% => Submission order: 1. Nature, 2. Science, 3. NCC, 4. Nature Sust, 5. PNAS, 6. Science Advances, 7. GEC, 8. ERL or JEEM. (or 3. NS (if: 27) and 4. NCC (if: 22)?)
% => Sample sizes should be given (only) on Figures

% Nature "abstract": Major sustainability objectives could be achieved by global approaches to mitigating climate change and inequality    . For instance, a global carbon price funding a global basic income, called the “Global Climate Scheme” (GCS), would be an effective way to jointly combat climate change and poverty. A key condition for the success of global cooperation is the support of citizens in affluent countries for such globally redistributive policies. Yet, few prior attitudinal surveys have examined support for global policies. To explore relevant public attitudes, we survey over 48,000 respondents from 20 high- and middle-income countries. The responses reveal strong support for global redistributive policies, including the GCS and a global wealth tax aimed at financing low-income countries. A list experiment shows no evidence of social desirability bias in survey responses, majorities are willing to sign a real-stake petition, and global redistribution ranks high in the prioritization of policies. Conjoint analyses reveal that a political platform is more likely to be preferred if it contains the GCS or a global tax on millionaires. In sum, our findings indicate that global redistributive policies are genuinely supported by a majority of the population, even in wealthy nations that would bear a significant burden. Public opinion is therefore not the reason that they do not prominently enter political debates. These results could help draw attention to global policies in the public debate and contribute to their increased prominence.

% Nature guidelines:
% Attention to the following details can help expedite publication if we invite a revision after external review.
% A fully referenced ~200 word summary paragraph; main text of 2,500 words and 4 modest display items (figures, tables) for a typical 6 page article and 4300 words and 5-6 modest display items for a typical 8 page article; as a guideline up to 50 references if needed and within the allocated page budget.

%%%%%%%%%%%%%%%%%%%%%%%%%%%%%%%%%%%%%%%%
%%%%% NATURE CLIMATE CHANGE FORMAT %%%%%
%%%%%%%%%%%%%%%%%%%%%%%%%%%%%%%%%%%%%%%%
%% Comment "% WPcomment" lines, uncomment "% NCCcomment" lines as well as the lines below, replace all citet/citep by cite

% \documentclass{nature}
% \usepackage{amsmath}
% \usepackage{amssymb}
% \usepackage{eurosym}
% % The following allows keeping figures within the text (otherwise nature.cls would ignore them)
% \usepackage{graphicx}
% \makeatletter
% \let\saved@includegraphics\includegraphics
% \AtBeginDocument{\let\includegraphics\saved@includegraphics}
% \renewenvironment*{figure}{\@float{figure}}{\end@float}
% \makeatother

% Nature guidelines (not NCC!)
% Sections can only be used in Articles.  Contributions should be organized in the sequence: title, text, methods, references, Supplementary Information line (if any), acknowledgements, interest declaration, corresponding author line, tables, figure legends.

% No subsubsection nor paragraph

% Spelling must be British English (Oxford English Dictionary)

%Each figure legend should begin with a brief title for the whole figure and continue with a short description of each panel and the symbols used. For contributions with methods sections, legends should not contain any details of methods, or exceed 100 words (fewer than 500 words in total for the whole paper). In contributions without methods sections, legends should be fewer than 300 words (800 words or fewer in total for the whole paper).

% Articles are restricted to 50 references,

% In addition, a cover letter needs to be written with the
% following:
% \begin{enumerate}
%  \item A 100 word or less summary indicating on scientific grounds
% why the paper should be considered for a wide-ranging journal like
% \textsl{Nature} instead of a more narrowly focussed journal.
%  \item A 100 word or less summary aimed at a non-scientific audience,
% written at the level of a national newspaper.  It may be used for
% \textsl{Nature}'s press release or other general publicity.
%  \item The cover letter should state clearly what is included as the
% submission, including number of figures, supporting manuscripts
% and any Supplementary Information (specifying number of items and
% format).
%  \item The cover letter should also state the number of
% words of text in the paper; the number of figures and parts of
% figures (for example, 4 figures, comprising 16 separate panels in
% total); a rough estimate of the desired final size of figures in
% terms of number of pages; and a full current postal address,
% telephone and fax numbers, and current e-mail address.
% \end{enumerate}

% See \textsl{Nature}'s website
% (\texttt{http://www.nature.com/nature/submit/gta/index.html}) for
% complete submission guidelines.

%%%%%%%%%%%%%%%%%%%%%%%%%%%%%%%%
%%%%% WORKING PAPER FORMAT %%%%%
%%%%%%%%%%%%%%%%%%%%%%%%%%%%%%%%
%% Comment "% NCCcomment" lines, uncomment "% WPcomment" lines as well as the lines below
\documentclass[12pt,english]{article}
\usepackage[utf8]{inputenc}
\usepackage{tgpagella} % Palatino text only
\usepackage{mathpazo}  % Palatino math & text
\usepackage[left=1.5in,right=1.5in,top=1.5in,bottom=1.5in]{geometry}
% \linespread{1.5}
% \usepackage[super,comma,sort]{natbib} % WPcomment
\usepackage[round,sort&compress]{natbib} % NCCcomment
\usepackage{url} % [hyphens]
\usepackage[hyperpageref]{backref} % back references biblio. Needs latexmk at compilation.
\usepackage[pagebackref]{hyperref}
% \usepackage{multibib} % incompatible with backref
\hypersetup{
  colorlinks=true, % breaklinks=true,
  urlcolor=purple,    % color of external links
  linkcolor=blue,  % color of toc, list of figs etc.
  citecolor=violet,   % color of links to bibliography
}
\usepackage{bm}
\usepackage{indentfirst}
\usepackage{tocbibind}
\setcitestyle{aysep={}} 
\usepackage{amsmath}
\usepackage{tcolorbox}
\usepackage{amssymb}
\usepackage{eurosym}
\usepackage{amsfonts}
\usepackage{enumerate}
\usepackage{babel}
\usepackage{graphicx}
\usepackage{caption}
\usepackage{supertabular}
\usepackage{tabularx}
\usepackage{float}
\usepackage{dsfont}
\usepackage{fancyvrb}
\usepackage{verbatim}
\usepackage{enumitem}
\usepackage{setspace}
\usepackage{comment}
\usepackage{subcaption}
\usepackage{tikz}
\usepackage{gensymb}
\usepackage{textcomp}
\usepackage{lineno}
\linenumbers

\usepackage{tabulary}
\usepackage{tabularx}
\usepackage{booktabs}
\usepackage{fullpage}
\usepackage{morefloats}
\usepackage{makecell}
\usepackage{lscape}
\usepackage{pdflscape}
\usepackage{longtable}
\usepackage{rotating}
\usepackage{fancyhdr}
\usepackage{tocloft}
\usepackage{titletoc}
\usepackage[export]{adjustbox}
\usepackage[anythingbreaks]{breakurl} % for links
\usepackage{multicol}
\newsavebox\ltmcbox % For net gain table over two columns
%\usepackage[nomarkers,figuresonly]{endfloat} % Figures at the end
%\usepackage[section,below]{placeins} % Floats placed in the section they appear in.
\renewcommand{\floatpagefraction}{.99}
\newenvironment{stretchpars}{\par\setlength{\parfillskip}{0pt}}{\par} % to justify a line

% % Getting landscape page and page number/footer on bottom of page (instead of to the left)
% \fancypagestyle{mylandscape}{
% \fancyhf{} %Clears the header/footer
% \fancyfoot{% Footer
% \makebox[\textwidth][r]{% Right
%   \rlap{\hspace{1.5cm}% Push out of margin by \footskip
%     \smash{% Remove vertical height
%       \raisebox{13.6cm}{% Raise vertically
%         \rotatebox{90}{\thepage}}}}}}% Rotate counter-clockwise
% \renewcommand{\headrulewidth}{0pt}% No header rule
% \renewcommand{\footrulewidth}{0pt}% No footer rule
% }

% \fancypagestyle{page_left}{%
% 	\renewcommand{\headrulewidth}{0pt}
%   \fancyhf{}
%   \fancyfoot[OC]{%
%       \begin{tikzpicture}[remember picture,overlay]
%           \node[xshift=1cm] (number) at (current page.west) {\thepage};
%       \end{tikzpicture}
%   }%
% }
% \renewcommand{\thesubfigure}{\Alph{subfigure}}

% \newcites{App}{Appendix References}

% \captionsetup[table]{skip=-10pt}
% \begin{document}

% \maketitle

% \clearpage
% % \startcontents
% % \printcontents{ }{1}{\section{\contentsname}}
% % \clearpage
% \section{Introduction\label{sec:intro}}

% % \clearpage
% \renewcommand{\bibsection}{\section{\refname}}
% \bibliographystyle{naturemag}
% \bibliography{global_tax_attitudes}
% % \stopcontents

% \end{document}


\title{International Attitudes Toward Global Policies \\ Supplementary Material %\\ Addressing Climate Change and Inequality 
} 

% \author{Adrien Fabre$^{1,2}$, Thomas Douenne$^3$ and Linus Mattauch$^{4,5,6}$} % WPcomment
\author{Adrien Fabre\footnote{CNRS, CIRED. E-mail: adrien.fabre@cnrs.fr (corresponding author).}, Thomas Douenne\footnote{University of Amsterdam}\; and Linus Mattauch\footnote{Technical University Berlin, Potsdam Institute for Climate Impact Research -- Member of the Leibniz Association and University of Oxford}}%~~\thanks{The project %is approved by IRB at Harvard University (IRB21-0137), and 
% was preregistered in the Open Science Foundation registry (\href{https://osf.io/fy6gd}{osf.io/fy6gd}). \\ We are grateful for financial support from the University of Amsterdam and TU Berlin. Mattauch also thanks the Robert Bosch Foundation. %We are grateful for financial support from the OECD, the French Ministry of Foreign Affairs, the French Conseil d’Analyse Economique and the Spanish Ministry for the Ecological Transition and Demographic Challenge. We also acknowledge support from the Grantham Foundation for the Protection of the Environment and the Economic and Social Research Council through the Centre for Climate Change Economics and Policy. 
% We thank Antoine Dechezleprêtre, Tobias Kruse, Bluebery Planterose, Ana Sanchez Chico, and Stefanie Stantcheva for their invaluable inputs for the project. We thank Auriane Meilland for feedback. We further thank Jakob Niemann, Laura Schepp, Martín Fernández-Sánchez, Samuel Gervais, Samuel Haddad, and Guadalupe Manzo for assistance. Fabre declares that he also serves as president of Global Redistribution Advocates.}} % NCCcomment

\date{\today} % NCCcomment

\begin{document}

\maketitle

\begin{center}
\end{center}


% WPcomment
% \begin{affiliations}
% \item CNRS
% \item CIRED
% \item University of Amsterdam
% \item Technical University Berlin
% \item Potsdam Institute for Climate Impact Research 
% \item University of Oxford
% \end{affiliations}

% \begin{small} % NCCcomment


\tableofcontents

\onehalfspacing % NCCcomment

\clearpage
\section*{Main figures}\label{sec:figures}
\addcontentsline{toc}{section}{\nameref{sec:figures}}

\renewcommand{\thefigure}{S\arabic{figure}}
\renewcommand{\thetable}{S\arabic{table}}

\begin{figure}[h!]
  % MAJOR figure
  \caption[Relative support for global climate policies]{Relative support for global climate policies. (Reproduced from \cite{dechezlepretre_fighting_2022}, Figure A21.)}  % TODO!? put back?
  \makebox[\textwidth][c]{\includegraphics[width=1.2\textwidth]
  {../figures/OECD/Heatplot_global_tax_attitudes_share.pdf}}\label{fig:oecd} % with dependence on others (absent from OECD): Heatplot_burden_share_all_share_countries
  {\footnotesize \\ $\quad$ \\ Note 1: The numbers represent the share of \textit{Somewhat} or \textit{Strongly support} among non-\textit{indifferent} answers (in percent, $n$ = 40,680). The color blue denotes a relative majority. See Figure \ref{fig:oecd_absolute} for the absolute support. (Questions \ref{q:scale}-\ref{q:millionaire_tax}). %Reproduced from \cite{dechezlepretre_fighting_2022}, Figure A21.) % TODO!? put back?
  \\ Note 2: *In Denmark, France and the U.S., the questions with an asterisk were asked differently, cf. Question F.% \ref{q:burden_sharing_asterisk}. 
  } 
\end{figure}

\begin{figure}
  % MAJOR figure
  \caption[Relative support for further global policies]{Relative support for various global policies (percentage of \textit{somewhat} or \textit{strong support}, after excluding \textit{indifferent} answers). (Questions \ref{q:climate_policies} and \ref{q:other_policies}; See Figure \ref{fig:support_likert_positive} for the absolute support.)% $n_\text{US} = n_\text{Eu} = 3,000,\, n_\text{FR} = 729,\, n_\text{DE} = 929,\, n_\text{ES} = 543,\, n_\text{UK} = 749, n_\text{US, global/national wealth tax} = 2,000$
  }
  \makebox[\textwidth][c]{\includegraphics[width=\textwidth]{../figures/country_comparison/support_likert_share.pdf}}\label{fig:support}
\end{figure} 

\begin{table}[h]
  % MAJOR figure % TODO! same table for NR in appendix
  % TODO table by country
  \caption[List experiment: tacit support for the GCS]{Number of supported policies in the list experiment depending on the presence of the Global Climate Scheme (GCS) in the list.%in function of the composition of the list. GCS stands for the Global Climate Scheme and NR for the National Redistribution Scheme.} % Beware, this question is quite unusual. \\ Among the policies below, how many do you support?  \\ Coal exit, Marriage only for opposite-sex couples 
  }\label{tab:list_exp}
  \makebox[\textwidth][c]{
\begin{tabular}{@{\extracolsep{5pt}}lccc} 
\\[-1.8ex]\hline 
\hline \\[-1.8ex] 
 & \multicolumn{3}{c}{Number of supported policies} \\ 
\cline{2-4} 
\\[-1.8ex] & All & US & Eu \\ 
\hline \\[-1.8ex] 
 List contains: GCS & 0.624$^{***}$ & 0.524$^{***}$ & 0.724$^{***}$ \\ 
  & (0.028) & (0.041) & (0.036) \\ 
\hline  \\[-1.8ex] \textit{Support for GCS} & 0.617  &  0.542  &  0.757 \\
\textit{Social desirability bias} & \textit{$ -0.026 $} & \textit{$ -0.018 $} & \textit{$ -0.033 $}\\
\textit{80\% C.I. for the bias} & \textit{ $[ -0.06 ; 0.01 ]$ } & \textit{ $[ -0.07 ; 0.01 ]$} & \textit{ $[ -0.08 ; 0.01 ]$}\\
 \hline \\[-1.8ex] 
Constant & 1.317 & 1.147 & 1.486 \\ 
Observations & 6,000 & 3,000 & 3,000 \\ 
R$^{2}$ & 0.089 & 0.065 & 0.125 \\ 
\hline 
\hline \\[-1.8ex] 
\textit{Note:}  & \multicolumn{3}{r}{$^{*}$p$<$0.1; $^{**}$p$<$0.05; $^{***}$p$<$0.01} \\ 
\end{tabular} 
  }  
  % {\footnotesize \textit{Note:} $^{*}p<0.1$; $^{**} p<0.05$; $^{***} p<0.01$.}
\end{table}

\begin{table}[h]
  % MAJOR figure
  \caption[Influence of the GCS on electoral prospects]{Preference for a progressive platform depending on whether it includes the GCS or not. (Question \ref{q:conjoint_c}) 
  %Imagine if the [Democratic and Republican presidential candidates in 2024] campaigned with the following policies in their platforms. [Credible Progressive and Conservative platforms] \\ % TODO See More
% Which of these candidates would you vote for? \textit{A; B; None of them} \\
% ~[FR: second round of presidential; DE, ES, UK: two favorite candidates in one's constituency]
} % Beware, this question is quite unusual. \\ Among the policies below, how many do you support?  \\ Coal exit, Marriage only for opposite-sex couples 
  \makebox[\textwidth][c]{
\begin{tabular}{@{\extracolsep{5pt}}lcccccc} 
\\[-1.8ex]\hline 
\hline \\[-1.8ex] 
 & \multicolumn{6}{c}{Prefers the Progressive platform} \\ 
\cline{2-7} 
\\[-1.8ex] & All & United States & France & Germany & UK & Spain \\ 
\hline \\[-1.8ex] 
 GCS in Progressive platform & 0.028$^{*}$ & 0.029 & 0.112$^{***}$ & 0.015 & 0.008 & $-$0.015 \\ 
  & (0.014) & (0.022) & (0.041) & (0.033) & (0.040) & (0.038) \\ 
 \hline \\[-1.8ex] 
Constant & 0.623 & 0.604 & 0.55 & 0.7 & 0.551 & 0.775 \\ 
Observations & 5,202 & 2,619 & 605 & 813 & 661 & 504 \\ 
R$^{2}$ & 0.001 & 0.001 & 0.013 & 0.0003 & 0.0001 & 0.0003 \\ 
\hline 
\hline \\[-1.8ex] 
\end{tabular} 
}\label{tab:conjoint_c}
  {\footnotesize \textit{Note:} The 14\% of \textit{None of them} answers have been excluded from the regression samples. GCS has no significant influence on them. $^{*}p<0.1$; $^{**} p<0.05$; $^{***} p<0.01$. 
  }
\end{table}

\begin{figure}[h!]
  \caption[Support for the Global Climate Scheme]{Support for the GCS, NR and the combination of GCS, NR and C. \\(p. \pageref{subsec:questionnaire_GCS}, Questions \ref{q:global_tax}, \ref{q:national_tax}, \ref{q:gcs_support}, \ref{q:nr_support} and \ref{q:crg_support}).%; $n_\text{US} = n_\text{Eu} = 3,000,\, n_\text{FR} = 729,\, n_\text{DE} = 929,\, n_\text{ES} = 543,\, n_\text{UK} = 749$)
  }\label{fig:support_binary}
  \makebox[\textwidth][c]{\includegraphics[width=.9\textwidth]{../figures/country_comparison/support_binary_positive.pdf}} 
\end{figure}


\begin{figure}
  \centering 
  \caption[Preferred share of wealth tax for low-income countries]{Percent of global wealth tax that should finance low-income countries (\textit{mean}). (Question \ref{q:global_tax_global_share})} % TODO? n
  \includegraphics[width=1\textwidth]{../figures/country_comparison/global_tax_global_share_mean.pdf} \label{fig:global_share_mean}
\end{figure}

% \begin{figure}
%   % MAJOR figure
%   \caption[Relative support for further global policies]{Relative support for various global policies (percentage of \textit{somewhat} or \textit{strong support}, after excluding \textit{indifferent} answers). (Questions \ref{q:climate_policies} and \ref{q:other_policies}; See Figure \ref{fig:support_likert_positive} for the absolute support.)% $n_\text{US} = n_\text{Eu} = 3,000,\, n_\text{FR} = 729,\, n_\text{DE} = 929,\, n_\text{ES} = 543,\, n_\text{UK} = 749, n_\text{US, global/national wealth tax} = 2,000$
%   }
%   \makebox[\textwidth][c]{\includegraphics[width=\textwidth]{../figures/country_comparison/support_likert_share.pdf}}\label{fig:support}
% \end{figure} 

\begin{figure}[h!]
\caption[Attitudes on the evolution of foreign aid]{Attitudes regarding the evolution of [own country] foreign aid. (Question \ref{q:foreign_aid_raise_support})}\label{fig:foreign_aid_raise_support}
\makebox[\textwidth][c]{\includegraphics[width=\textwidth]{../figures/country_comparison/foreign_aid_raise_support.pdf}} 
\end{figure}

\begin{figure}[h!]
\caption[Conditions at which foreign aid should be increased]{Conditions at which foreign aid should be increased (in percent). [Asked to those who wish an increase of foreign aid at some conditions.] (Question \ref{q:foreign_aid_condition})}\label{fig:foreign_aid_condition}
\makebox[\textwidth][c]{\includegraphics[width=\textwidth]{../figures/country_comparison/foreign_aid_condition_positive.pdf}} 
\end{figure}

\begin{figure}[h!]
\caption[Reasons why foreign aid should not be increased]{Reasons why foreign aid should not be increased (in percent). [Asked to those who wish a decrease or stability of foreign aid.] (Question \ref{q:foreign_aid_no})}\label{fig:foreign_aid_no}
\makebox[\textwidth][c]{\includegraphics[width=\textwidth]{../figures/country_comparison/foreign_aid_no_positive.pdf}} 
\end{figure}



\begin{figure}[h] 
\caption[Preferences for various policies in political platforms]{Effects of the presence of a policy (rather than none from this domain) in a random platform on the likelihood that it is preferred to another random platform. (See English translations in Figure \ref{fig:ca_r_en}; Question \ref{q:conjoint_r}%; in the U.S., asked only to non-Republicans.
)}\label{fig:ca_r}
\begin{subfigure}{\textwidth}
  \subcaption{U.S. (Asked only to non-Republicans)}
  \includegraphics[width=\textwidth]{../figures/US1/ca_r.png}
\end{subfigure}
\begin{subfigure}{\textwidth}
  \subcaption{France}
  \includegraphics[width=\textwidth]{../figures/FR/ca_r.png}
\end{subfigure}
\end{figure}%
\clearpage
\begin{figure}[h!]\ContinuedFloat % if bugs try b! instead of h!
\begin{subfigure}{\textwidth}
  \subcaption{Germany}
  \includegraphics[width=\textwidth]{../figures/DE/ca_r.png}
\end{subfigure}
\begin{subfigure}{\textwidth}
  \subcaption{Spain}
  \includegraphics[width=\textwidth]{../figures/ES/ca_r.png}
\end{subfigure}
\begin{subfigure}{\textwidth}
  \subcaption{UK}
  \includegraphics[width=\textwidth]{../figures/UK/ca_r.png}
\end{subfigure}
%\makebox[\textwidth][c]{} 
\end{figure}


\begin{figure}[h!]
  \caption[Influence of the GCS on preferred platform]{Influence of the GCS on preferred platform:\\ Preference for a random platform A that contains the Global Climate Scheme rather than a platform B that does not (in percent). (Question \ref{q:conjoint_d}; in the U.S., asked only to non-Republicans.)}\label{fig:conjoint_left_ag_b}
  \makebox[\textwidth][c]{\includegraphics[width=\textwidth]{../figures/country_comparison/conjoint_left_ag_b_binary_positive.pdf}} 
\end{figure}


\begin{figure}[h!]
  \caption[Beliefs about support for the GCS and NR]{Beliefs regarding the support for the GCS and NR. (Questions \ref{q:gcs_belief} and \ref{q:nr_belief})}\label{fig:belief}
  \makebox[\textwidth][c]{\includegraphics[width=.7\textwidth]{../figures/country_comparison/belief_all_mean.pdf}} 
\end{figure}


\clearpage
% \chapter*{Materials and Methods}\label{ch:material}
% \addcontentsline{toc}{chapter}{\nameref{ch:material}}
\section{Methods}\label{sec:methods}
% \addcontentsline{toc}{section}{\nameref{sec:methods}}

\subsection{Data and questionnaires}
\paragraph{Data collection} % WPcomment

The paper utilizes two sets of surveys: the \textit{Global} survey and the \textit{Complementary} surveys. The \textit{Complementary} surveys consist of two U.S. surveys, \textit{US1} and \textit{US2}, and one European survey, \textit{Eu}. The \textit{Global} survey was conducted from March 2021 to March 2022 on 40,680 respondents from 20 countries (with 1,465 to 2,488 respondents per country). \textit{US1} collected responses from 3,000 participants between January and March 2023, while \textit{US2} gathered data from 2,000 respondents between March and April 2023. \textit{Eu} included 3,000 participants and was conducted from February to March 2023. We used the survey companies \emph{Dynata} and \emph{Respondi}. To ensure representative samples, we employed stratified quotas based on gender, age (5 brackets), income (4), region (4), and education level (3), as well as ethnicity (3) for the U.S. We also incorporated survey weights throughout the analysis to account for any remaining imbalances. These weights were constructed using the quota variables as well as the degree of urbanity, and trimmed between 0.25 and 4. By applying weights, the results are fully representative of the respective countries. Results at the European level apply different weights which ensure  representativeness of the combined four European countries. 
Supplementary Section \ref{app:representativeness} confirms that our samples closely match population frequencies in high-income countries. In middle-income countries, the samples are only representative of the online population (young, graduated and urban people are over-represented). %due to the over-representation of young, educated, and urban populations in the online sample
% Supplementary Section G confirms that our samples are representative of the population.%\footnote{We trim weights so that no respondent receives a weight below 0.25 or above 4. Overall, trimming changes the weights for xx\% of the respondents.} % /!\ leave commented


\paragraph{Data quality} % WPcomment % TODO attrition analysis
% \paragraph{\small Data quality.} % NCCcomment
The median duration is 28 minutes for the \textit{Global} survey, 14 min for \textit{US1}, 11 min for \textit{US2}, and 20 min for \textit{Eu}. To ensure the best possible data quality, we exclude respondents who fail an attention test or rush through the survey (i.e., answer in less than 11.5 minutes in the \textit{Global} survey, 4 minutes in \textit{US1} or \textit{US2}, 6 minutes in \textit{Eu}). We study the determinants of attrition in Supplementary Section \ref{app:attrition}. Some socio-demographics drop out significantly more frequently than others, but the coefficients remain small, indicating that our results are not driven by selection bias.  %All the results and analyses use survey weights, defined to make the results fully representative of the country (or of Eu in the case of results at the Eu level) along the quota variables. Weights are trimmed to be between .25 and 4. 
%\textit{Ex post}, we checked that there were only a few careless response patterns (such as choosing the same answer for all items in a matrix of questions; see Appendix \ref{app:data_quality}). At the end of the survey, we ask whether respondents thought that our survey was politically biased and provide some feedback. 67\% of the respondents found the survey unbiased. 25\% found it left-wing biased, and 8\% found it right-wing biased.

\paragraph{Questionnaires and raw results} % WPcomment
% \paragraph{\small Questionnaires and raw results.} % NCCcomment
% Possible confusion in the questionnaire: people confuse GCS with the four policies together (so support for GCS can suffer from dislike of death penalty), although this confusion is mitigating by the fact that we right after ask about NR; people may answer about revenue-use rather than the whole measure for ETS2 support; people may answer that GCS will or will not have the effects proposed rather than these effects being important or not in their attitude toward GCS; they may answer that it's important that others (not them) get more information; the minimum wage could be reduced at 50% of local median wage.
The questionnaire and raw results of the \textit{Global} survey can be found in the Appendix of the companion paper \citep{dechezlepretre_fighting_2022}. %.\cite{dechezlepretre_fighting_2022} the companion paper TODO? put back?
The raw results are reported in Supplementary Section \ref{app:raw_results} %\footnote{Country-specific raw results are also available as supplementary material files:  \href{https://github.com/bixiou/global_tax_attitudes/raw/main/paper/app_desc_stats_US.pdf}{US}, \href{https://github.com/bixiou/global_tax_attitudes/raw/main/paper/app_desc_stats_EU.pdf}{EU}, \href{https://github.com/bixiou/global_tax_attitudes/raw/main/paper/app_desc_stats_FR.pdf}{FR}, \href{https://github.com/bixiou/global_tax_attitudes/raw/main/paper/app_desc_stats_DE.pdf}{DE}, \href{https://github.com/bixiou/global_tax_attitudes/raw/main/paper/app_desc_stats_ES.pdf}{ES}, \href{https://github.com/bixiou/global_tax_attitudes/raw/main/paper/app_desc_stats_UK.pdf}{UK}.} 
while the surveys' structures and questionnaires are given in Supplementary Sections \ref{app:questionnaire_oecd} and \ref{app:questionnaire}. The questionnaires are the same as the ones given \textit{ex ante} in the registration plan (\href{https://osf.io/fy6gd}{osf.io/fy6gd}).


\paragraph{Incentives} % WPcomment
% \paragraph{\small Incentives.} % NCCcomment
To encourage accurate and truthful responses, several questions of the \textit{US1} survey use incentives. For each of the three comprehension questions that follow the policy descriptions, we randomly select and reward three respondents who provide correct answers with a \$50 gift certificate. Similarly, for questions involving estimating support shares for the GCS and NR, three participants with the closest guesses to the actual values receive a \$50 gift certificate. In the donation lottery question, we randomly select one respondent and split the \$100 prize between the NGO GiveDirectly and the winner according to the winner's choice. In total, our incentives scheme distributes gift certificates (and donations) for a value of \$850. Finally, respondents have an incentive to answer truthfully to the petition question, as they are aware that the results for that question (the share of respondents supporting the policy) will be transmitted to the head of state's office.

%To encourage respondents to answer accurately and truthfully, several questions of the \textit{US1} survey use incentives. For each of the three comprehension questions that follow the policies' descriptions, we reward three (randomly drawn) respondents with the correct answer with a \$50 gift certificate. For each of the questions asking respondents to guess the share of support for the GCS and NR, we reward three people who are closest to the true value with a \$50 gift certificate. For the donation lottery question, we randomly draw one respondent and split the \$100 prize between the NGO GiveDirectly and the winner according to the winner's choice. In total, our incentives scheme distributes gift certificates (and donation) for a value of \$850. Finally, respondents have an incentive to answer truthfully to the petition question, given that they know that the results to that question (the share of respondents supporting the policy) will be transmitted to the U.S. President's office.




% Here is a description of a specific method used.  Note that the subsection heading ends with a full stop (period) and that the command is \verb|\subsection{}| not \verb|\subsection**{}|.
% \subsection{\small Details on sources, methodology and results} % WPcomment
\subsection{Methodology}
\paragraph{National Redistribution scheme} 
After describing the Global Climate Scheme (GCS) to the respondents, we assess respondents' understanding of the GCS with incentivized questions to test their comprehension of the expected financial outcome for typical individuals in high-income countries (loss) and the poorest individuals globally (gain), followed by the provision of correct answers (Figures \ref{fig:understood_each}-\ref{fig:understood_score}). The same approach is then applied to a National Redistribution scheme. NR targets the top 5\% (in the U.S.) or top 1\% (in Europe) with the aim of financing cash transfers to all adults,\footnote{The wider base in the U.S. was chosen because emissions are larger in the U.S. than in Europe, and it would hardly be feasible to offset the median American's loss by taxing only the top 1\%.} calibrated to offset the monetary loss of the GCS for the median emitter in their country. We evaluate respondents' understanding that the richest would lose and the typical fellow citizens would gain from that policy. Subsequently, we summarize both schemes to enhance respondents' recall. Additionally, we present a final incentivized comprehension question and provide the expected answer that the combined GCS and NR would result in no net gain or loss for a typical fellow citizen. 

We introduced NR in the questionnaire because we formulated the hypothesis that the GCS would be more supported if complemented with NR. As shown in conjoint analyses (see below), this hypothesis turned out to be false. 

We also used NR at several occasions in the questionnaire (the list experiment, the petition, and second-order beliefs) as a point of comparison with the GCS (see Supplementary Section \ref{app:questionnaire}). The rationale was to test whether we found higher or smaller effects for the GCS compared to a benchmark policy: NR. Overall, we find similar effects for the GCS and for NR.

\paragraph{Support for the GCS} The 95\% confidence intervals are $[52.4\%, 55.9\%]$ in the U.S. and $[74.2\%, 77.2\%]$ in Europe. The average support is computed with survey weights, employing weights based on quota variables, which exclude vote. Another method to reweigh the raw results involves running a regression of the support for the GCS on sociodemographic characteristics (including vote) and multiplying each coefficient by the population frequencies. This alternative approach yields similar figures: 76\% in Europe and 52\% or 53\% in the U.S. (depending on whether individuals who did not disclose their vote are classified as non-voters or excluded). Notably, the average support excluding non-voters is 54\% in the U.S. 

Though the level of support for the GCS is significantly lower in swing States (at 51\%) that are key to win U.S. elections, the electoral effect of endorsing the GCS remains non-significantly different from zero (at +1.2 p.p.) in these States. Note that we define swing states as the 8 states with less than 5 p.p. margin of victory in the 2020 election (MI, NV, PA, WI, AZ, GA, NC, FL). The results are robust to using the 3 p.p. threshold (that excludes FL) instead. 

\paragraph{Global wealth tax estimates}
A 2\% tax on net wealth exceeding \$5 million would annually raise \$816 billion, leaving unaffected 99.9\% of the world population. More specifically, it would collect \euro{}5 billion in Spain, \euro{}16 billion in France, £20 billion in the UK, \euro{}44 billion in Germany, \$430 billion in the U.S., and \$1 billion collectively in all low-income countries (28 countries, home to 700 million people). These Figures come from the \href{https://wid.world/world-wealth-tax-simulator/}{WID wealth tax simulator} of \cite{chancel_world_2022}.

\paragraph{List experiment}
We utilize the difference-in-means estimator, and confidence intervals are computed using Monte Carlo simulation with the R package \textit{list} by \cite{imai_multivariate_2011}.

\paragraph{Petition}
Paired weighted \textit{t}-tests are conducted to test the equality in support for a policy among respondents who were questioned about the policy in the petition.

\paragraph{Conjoint analysis}
In order to assess the public support for the GCS in conjunction with other policies, we conduct a series of conjoint analyses. We ask respondents to make five choices between pairs of political platforms. In the main text, we do not present the first two conjoint analyses.

The first conjoint analysis suggests that the GCS is supported independently of being complemented by the National Redistribution Scheme and a national climate policy (``Coal exit'' in the U.S., ``Thermal insulation plan'' in Europe, denoted C).\footnote{Indeed, 54\% of %($n$ = 3,000) 
U.S. respondents and 74\% of %($n$ = 3,000) 
European ones prefer the combination of C, NR and the GCS to the combination of C and NR alone, indicating similar support for the GCS conditional on NR and C than for the GCS alone (Figure \ref{fig:conjoint}).} % (as it does not significantly differ from the direct support of 53\%). 

For the second analysis, we split the sample into four random branches.\footnote{Results from the first branch show that the support for the GCS conditional on NR, at 55\% in the U.S. ($n$ = 757) and 77\% in Europe ($n$ = 746), is not significantly different from the support for the GCS alone. This suggests that rejection of the GCS is not driven by the cost of the policy on oneself. The second branch shows that the support for C conditional on NR is somewhat higher, at 62\% in the U.S. ($n$ = 751) and 84\% in Europe ($n$ = 747). However, the third one shows no significant preference for C compared to GCS (both conditional on NR), neither in Europe, where GCS is preferred by 52\% ($n$ = 741) nor in the U.S., where C is preferred by 53\% ($n$ = 721). The fourth branch shows that 55\% in the U.S. ($n$ = 771) and 77\% in Europe ($n$ = 766) prefer the combination of C, NR and the GCS to NR alone.} The outcome is that there is majority support for the GCS and for C, which are seen as neither complement nor substitute. A minor share of respondents like a national climate policy and dislike a global one, but as many people prefer a global rather than a national policy; and there is no evidence that implementing NR would increase the support for the GCS.

The effects reported as the changes in likelihood that a platform is preferred are the Average Marginal Component Effects \citep{hainmueller_causal_2014}. The policies studied are progressive policies prominent in the country. Except for the category \textit{foreign policy}, which features the GCS 42\% of the time, they are drawn uniformly.

\paragraph{Pros and cons}
Surprisingly, the support for National Redistribution also decreased by 7 p.p. following the closed question about the GCS. This suggests that some individuals may lack attention and confuse the two policies, or that contemplating the pros and cons alters the mood of some people, moving them away from their initial positive impression.

\subsection{Sources}
Detailed sources for the questionnaires and the figures are given in the \href{https://github.com/bixiou/international_attitudes_toward_global_policies/raw/main/questionnaire/specificities.xlsx}{Supplementary Spreadsheet}.


\appendix % NCCcomment
\renewcommand{\thetable}{A\arabic{table}}
\renewcommand{\thefigure}{A\arabic{figure}}
\setcounter{figure}{0}
\setcounter{table}{0}

\input{literature_review_nature.tex}

% \clearpage
% \section{Raw results% from the complementary surveys
% }\label{app:raw_results}

\input{app_nature} 

\clearpage
\renewcommand{\url}[1]{\href{#1}{Link}} % NCCcomment
\bibliographystyle{plainnaturl_clean} % NCCcomment
\bibliography{global_tax_attitudes}

\clearpage
\listoftables
\listoffigures

\end{document}
