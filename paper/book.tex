\documentclass[12pt,french]{book}
\usepackage[T1]{fontenc}
\usepackage[utf8]{inputenc}
\usepackage{tgpagella} % Palatino text only
\usepackage{mathpazo}  % Palatino math & text
\usepackage[left=1.5in,right=1.5in,top=1.5in,bottom=1.5in]{geometry}
% \linespread{1.5}
% \usepackage[super,comma,sort]{natbib}
\usepackage[round,sort&compress]{natbib}
\usepackage{url} % [hyphens]
\usepackage[hyperpageref]{backref} % back references biblio. Needs latexmk at compilation.
\usepackage[pagebackref]{hyperref}
% \usepackage{multibib} % incompatible with backref
\hypersetup{
  colorlinks=true, % breaklinks=true,
  urlcolor=purple,    % color of external links
  linkcolor=blue,  % color of toc, list of figs etc.
  citecolor=violet,   % color of links to bibliography
}
\usepackage{bm}
\usepackage{indentfirst}
\usepackage{tocbibind}
\setcitestyle{aysep={}} 
\usepackage{amsmath}
\usepackage{amssymb}
\usepackage{eurosym}
\usepackage{amsfonts}
\usepackage{enumerate}
\usepackage{babel}
\usepackage{caption}
\usepackage{supertabular}
\usepackage{tabularx}
\usepackage{float}
\usepackage{dsfont}
\usepackage{fancyvrb}
\usepackage{verbatim}
\usepackage{enumitem}
\usepackage{setspace}
\usepackage{comment}
\usepackage{subcaption}
\usepackage{graphicx}
\usepackage{tikz}
\usepackage{gensymb}
\usepackage{textcomp}

\usepackage{tabulary}
\usepackage{tabularx}
\usepackage{booktabs}
\usepackage{fullpage}
\usepackage{morefloats}
\usepackage{makecell}
\usepackage{lscape}
\usepackage{pdflscape}
\usepackage{longtable}
\usepackage{rotating}
\usepackage{fancyhdr}
\usepackage{tocloft}
\usepackage{titletoc}
\usepackage[export]{adjustbox}
\usepackage[anythingbreaks]{breakurl} % for links
\usepackage{multicol}
\newsavebox\ltmcbox % For net gain table over two columns
%\usepackage[nomarkers,figuresonly]{endfloat} % Figures at the end
%\usepackage[section,below]{placeins} % Floats placed in the section they appear in.
\renewcommand{\floatpagefraction}{.9}

% \renewcommand{\thechapter}{\Roman{chapter}}

\title{Un plan mondial pour le climat \\et contre l'extrême pauvreté} 
% Pour une révolution fiscale: 42k mots, 290 mots par page.

\author{Adrien Fabre\footnote{CNRS, CIRED. E-mail: adrien.fabre@cnrs.fr.}} 

\date{\today} 

\begin{document}

\maketitle

\tableofcontents

\chapter{Un statu quo insupportable\label{ch:statu_quo}}
% Un statu quo insupportable : l'extrême pauvreté, le changement climatique (chiffres, constat)

Plusieurs fléaux affligent l'humanité. Dans ce livre, nous nous préoccupons de deux d'entre eux : le changement climatique et l'extrême pauvreté. La lenteur des progrès effectués est une honte pour notre société, qui ne semble pas se soucier des personnes vulnérables ni des générations futures. Le constat est insupportable.

\section{Le changement climatique}

Le climat est un système complexe, mais les travaux du GIEC ont prouvé qu'on pouvait l'approximer avec une règle simple : le réchauffement climatique est proportionnel aux émissions de CO$_\text{2}$ cumulées depuis la révolution industrielle\footnote{La Figure SPM.10 in \citet{ipcc_climate_2021} montre qu'un degré de plus correspond à 2~000 GtCO$_\text{2}$.}. Pour mettre fin au réchauffement climatique dû à l'accumulation de CO$_\text{2}$ dans l'atmosphère, il faut donc atteindre la neutralité carbone. En d'autres termes, il faut amener les émissions de CO$_\text{2}$ à zéro --- ou plus exactement zéro net, dans la mesure où des émissions résiduelles peuvent être compensées par une captation équivalente grâce à la reforestation ou la séquestration artificielle du carbone. La température à laquelle l'humanité choisit de stabiliser le climat détermine le budget carbone, c'est-à-dire les émissions qu'il nous reste à émettre. Par exemple, pour avoir deux chances sur trois de limiter le réchauffement à +2\textdegree{}C, le budget carbone est de 1~000 milliards de tonnes (Gt) de CO$_\text{2}$ à partir de 2024\footnote{Cf. Table SPM.2 in \citet{ipcc_climate_2021}. L'usage d'une probabilité (<<~deux chances sur trois~>>) vient du fait que les modèles climatiques comportent une marge d'erreur sur la température atteinte par un budget carbone donné.}. Le budget carbone pourrait être respecté en réduisant linéairement les émissions de CO$_\text{2}$, en partant de leur valeur actuelle de 38 Gt jusqu'à zéro en 2077. 

Si, au contraire, les émissions continuent de croître, le réchauffement pourrait atteindre +4\textdegree{}C en 2100, et jusqu'à +7-8\textdegree{}C entre 2300 et 5000\footnote{\citet{montenegro_long_2007}.}. La fonte de l'Antarctique pourrait élever le niveau de la mer de 15 mètres d'ici 2500 et submerger d'ici 2300 des zones côtières où vivent actuellement près d'un milliard de personnes\footnote{\citet{deconto_contribution_2016,kopp_evolving_2017}.}. De vastes zones de Chine, d'Asie du Sud et du Moyen-Orient seraient rendues inhabitables au XXII$^\text{e}$ siècle du fait d'une combinaison létale de chaleur et d'humidité\footnote{\citet{pal_future_2016,im_deadly_2017,kang_north_2018}.}. Même dans un scénario d'émissions moins extrême, avec une température de +2\textdegree{}C en 2100, le niveau de la mer submergerait (en l'absence de digues) des zones où vivent actuellement 250 millions de personnes\footnote{\citet{kulp_new_2019}.}. De manière générale, nos infrastructures (et nos usages des sols) sont adaptées au climat actuel. Le changement climatique en rendra de nombreuses obsolètes, lorsqu'elles ne seront pas tout simplement détruites. Pour résumer, la continuation des émissions de gaz à effet de serre mettrait en péril de multiples pans de la société, multipliant les sécheresses, réduisant les rendements agricoles, accroissant la probabilité de conflit violent, et entraînant d'importants déplacements de population\footnote{Ce paragraphe reprend des éléments du préambule de ma thèse \citep{fabre_is_2020}, et repose sur de nombreux travaux \citep{burke_warming_2009,cattaneo_human_2019,carleton_social_2016,dell_temperature_2012,elliott_constraints_2014,schlenker_robust_2010,moore_new_2017}.}. 
% Il est donc nécessaire de mettre fin au réchauffement climatique dès que possible, et le limiter à +1.5°C.

% Climat et distribution
% Net zéro est possible

\section{L'extrême pauvreté} % Les pauvres ont faim / La faim de l'extrême pauvreté

La Banque mondiale définit l'extrême pauvreté par une consommation inférieure à 2\$ par jour (en parité de pouvoir d'achat\footnote{Le seuil de 2\$ est exprimé en parité de pouvoir d'achat (2,15\$ en dollar constant de 2017 pour être exact) : il correspond à ce que 2\$ permet d'acheter aux États-Unis. Dans un pays comme l'Inde, il faut ainsi moins de 1\$ pour se procurer l'équivalent de 2\$ aux États-Unis.}). % https://data.worldbank.org/indicator/NY.GDP.PCAP.KD?end=2021&locations=EU-ZG-XD-XM-1W-IN-US-CD-BI-LU-CN&start=2021&view=bar / https://data.worldbank.org/indicator/NY.GDP.PCAP.PP.KD?end=2021&locations=EU-ZG-XD-XM-1W-IN-US-CD-BI-LU-CN&start=2021&view=bar
Ce seuil permet de satisfaire les besoins nutritionnels minimaux\footnote{\citet{allen_absolute_2017-1} calcule que, dans les pays à bas revenus, le seuil d'extrême pauvreté permet de payer 3 m² dans un logement chauffé à 15\textdegree{}C ainsi qu'un régime alimentaire constitué uniquement d'huile et d'une céréale (parfois complété par des lentilles), qui assure un apport journalier de 2100 kcalories, 50 g de protéines et 34 g de lipides.}. Ainsi, le nombre de personnes en situation d'extrême pauvreté recoupe celui des 700 millions de personnes sous-alimentées\footnote{\citet{fao_state_2023}, \href{https://data.worldbank.org/indicator/SI.POV.DDAY?end=2019&locations=MW-1W&start=1990&view=chart}{Banque mondiale}.}. 

Bien que la proportion d'humains vivant avec moins de 2\$ par jour ait été divisée par quatre dans les trente dernières années, cela concerne encore deux tiers de la population dans un pays comme le Malawi. En fait, avec l'augmentation de la population, il y a davantage d'Africains extrêmement pauvres aujourd'hui qu'il y a trente ans. Si l'extrême pauvreté s'est réduite durant la période, c'est uniquement grâce au développement de l'Asie, et en particulier de la Chine. % https://ourworldindata.org/poverty?insight=hundreds-of-millions-will-remain-in-extreme-poverty-on-current-trends#key-insights

La Chine a désormais un PIB par habitant autour de la moyenne mondiale, soit 960\euro{} par mois. % world current GDP pc 2022 (current $): 12703 
En comparaison, le PIB par habitant est trois fois plus élevé dans les pays à hauts revenus et dix fois plus faible dans les pays à bas revenus. L'écart de niveau de vie entre pays est difficile à exagérer. En effet, un transfert de seulement 1\% du PIB des pays à hauts revenus (1,2 milliard de personnes) doublerait mécaniquement le revenu national des pays à bas revenus (700 millions de personnes). 

\begin{figure}[h!]
  \caption{PIB par habitant par rapport à la moyenne mondiale, ajustés en parité de pouvoir d'achat (2021, Banque mondiale). %Selected GDP per capita in PPP relative to the World's (2021, World Bank).
  }\label{fig:GDPpc}
  \makebox[\textwidth][c]{\includegraphics[width=.56\textwidth]{../figures/policies/GDP_pc_PPP_few.pdf}} % https://data.worldbank.org/indicator/NY.GDP.PCAP.PP.CD?contextual=default&end=2021&locations=EU-ZG-XD-XM-1W-IN-US-CD-BI-LU-CN&start=2021&view=bar
\end{figure}

\chapter{La nécessité de redistribution mondiale\label{ch:redistribution_necessaire}}

Qu'elle soit religieuse, philosophique ou intuitive, la morale prescrit généralement des transferts des personnes à hauts revenus vers les personnes à bas revenus, et donc des pays à hauts revenus vers les pays à bas revenus. C'est le cas de l'utilitarisme, la théorie éthique de référence utilisée en économie. L'utilitarisme attribue le même poids à chaque personne et justifie ainsi le transfert d'un euro d'une personne riche à une personne pauvre, puisqu'un euro procurera plus de satisfaction à cette dernière. D'après la théorie de la taxation optimale, ce raisonnement est valable tant qu'une augmentation des prélèvements n'incite pas les plus riches à réduire, expatrier ou dissimuler leur activité au point de diminuer les recettes obtenues. Des économistes ont calculé le système fiscal optimal en tenant compte de ces effets. Celui-ci réduirait drastiquement les inégalités entre pays et procurerait un revenu minimum de 250\$ par mois au niveau mondial\footnote{Dans ces calculs, \citet{kopczuk_limitations_2005} se limitent à un taux unique (une \textit{flat tax}) et ne s'autorisent pas un barème progressif. Sans cette restriction, le véritable optimum serait encore plus redistributif.}. La théorie de la taxation optimale ne peut rationaliser la situation actuelle qu'en tordant le cou à la morale. En effet, la quasi-absence de transferts internationaux n'est optimale que si on attribue un poids 2~000 fois plus élevé à un Américain qu'à un Congolais (ou bien, si on attribue une valeur 100 fois supérieure à l'Américain et qu'on considère que seul un vingtième de l'argent transféré arrivera à son destinataire, le reste étant détourné par la corruption). %de distordre les poids de sorte que la satisfaction d'un Américain vaille autant que celle de 2~000 Malgaches. 

Au-delà des considérations éthiques, la redistribution mondiale a des fondements juridiques. En 2015, l'ensemble des pays a adopté les Objectifs de développement durable (ODD), au premier rang duquel se trouve l'élimination de l'extrême pauvreté d'ici à 2030. Or, les pays à bas revenus n'ont pas les ressources domestiques suffisantes pour éliminer l'extrême pauvreté. En effet,  dans les 19 pays les plus pauvres, exproprier tous les revenus à partir de 13\$ par jour ne suffirait pas à financer des transferts suffisants pour faire passer leurs 700 millions d'habitants au-dessus de 2\$ par jour d'ici à 2030. Même en faisant l'hypothèse très optimiste d'une croissance du revenu moyen de 7\% par an d'ici à 2030% (soit le maximum observé dans le monde dans les cinq années qui ont précédé la Covid)
, exproprier tous les revenus au-delà de 7\$ par jour ne suffirait pas à éliminer l'extrême pauvreté dans un pays tel que Madagascar\footnote{Ces calculs sont inspirés de \citet{bolch_arithmetics_2022}, reposent sur les données \textit{Poverty and Inequality Platform} de la Banque mondiale, et sont reproductibles sur \href{https://github.com/bixiou/domestic\_poverty\_eradication/code\_poverty/main.R}{github.com/bixiou/domestic\_poverty\_eradication}.}. En d'autres termes, il est impossible d'atteindre le premier ODD sans transferts internationaux. Et ce, alors que le premier ODD se borne à assurer un revenu juste suffisant pour ne plus avoir faim. Le transfert nécessaire correspond à 0,1\% du PIB mondial, soit autant que les dépenses de nourriture pour les animaux de compagnie. % 2019 poverty gap = 2.6% of $2.15 / world GDP of 16865 https://data.worldbank.org/indicator/SI.POV.GAPS; https://www.grandviewresearch.com/industry-analysis/pet-food-industry

Pour s'assurer une vie décente, qui garantit l'accès à l'eau, l'assainissement, l'éducation, à un système de santé, à une capacité minimale à se déplacer et socialiser, on estime qu'il faut un revenu d'au moins 7\$ par jour\footnote{Cf. \citet{kikstra_decent_2021}.}. Près de la moitié des humains vit sous ce seuil de pauvreté\footnote{Cf. \href{https://ourworldindata.org/grapher/distribution-of-population-between-different-poverty-thresholds-up-to-30-dollars}{ourworldindata.org}.}. Combler l'écart qui les sépare de ce seuil coûterait 2 à 3\% du PIB mondial\footnote{En parité de pouvoir d'achat, cet écart (le \textit{poverty gap}, qu'on peut traduire par \textit{l'étendue de la pauvreté}) est de 4 billions de dollar et le PIB mondial de 140 billions.}. % https://data.worldbank.org/indicator/SI.POV.UMIC.GP https://data.worldbank.org/indicator/NY.GDP.MKTP.PP.KD
En outre, 500 millions de personnes vivent dans un pays où le PIB par habitant est inférieur à ce seuil, et où il est donc rigoureusement impossible d'assurer une vie décente à chacun en mobilisant les seules ressources domestiques. 

En 1970, les pays industrialisés ont pris l'engagement d'allouer 0,7\% de leur PIB à l'aide publique au développement, dont 0,2\% du PIB pour les pays les moins avancés. Cet engagement, renouvelé en 2005 et 2015, n'a été jamais été tenu\footnote{Plus exactement, seule une poignée de pays respecte son engagement : la Suède, la Norvège, le Danemark, le Luxembourg et le Royaume-Uni.}. On estime que l'essentiel des ODD pourraient être atteints si les pays industrialisés respectaient enfin cet engagement\footnote{\citet{sdsn_sdg_2019}.}. Pour atteindre une version maximaliste des ODD (y compris assurer l'accès à une énergie propre) ou un autre objectif ambitieux (tel qu'assurer 7\$ par jour à chacun) au regard du statu quo, les pays à hauts revenus devraient transférer davantage de ressources, probablement entre 2 et 6\% de leur PIB. 
% Il va sans dire que pour développer des services de santé, d'éducation, une énergie décarbonée, et plus généralement atteindre l'ensemble des ODD, la solidarité internationale est indispensable. Pour assurer 
% Plus généralement, 12 ODD sur les 17 concernent la pauvreté, les inégalités et le changement climatique. Si les pays à bas revenus n'ont déjà pas les ressources nécessaires pour éliminer l'extrême pauvreté (estimées à 0,1\% du PIB mondial), % 2019 poverty gap = 2.6% of $2.15 / world GDP of 16865 https://data.worldbank.org/indicator/SI.POV.GAPS
% il va sans dire qu'une redistribution internationale est nécessaire pour développer des services de santé, d'éducation, une énergie décarbonée, et plus généralement atteindre l'ensemble des ODD.

De tels transferts seraient colossaux. Mais ils pourraient être intégralement supportés par les millionnaires. En se limitant au millième d'humains les plus riches, qui ont une fortune supérieure à 5 millions d'euros, et en ne taxant que leur fortune au-delà de ce seuil, avec un taux effectif progressant de 1\% pour une fortune de 10 millions d'euros à 10\% pour une fortune de 100 milliards, on récolterait environ 2\% du PIB mondial, et leur fortune ne baisserait même pas (puisque la plupart des milliardaires ont des rendements supérieurs à 10\%). Avec une taxation plus progressive, qui démarrerait à 500~000 euros, avec un taux de 0,25\% pour une fortune d'un million d'euro (soit une taxe de 2~500\euro{} par an), et progresserait jusqu'à un taux de 20\% sur les plus grosses fortunes, les recettes pourraient atteindre 6\% du PIB mondial. Si une telle redistribution était mise en place, les classes moyennes seraient largement épargnées. Certes, des emplois seraient détruits dans le secteur du luxe, puisque les plus fortunés consommeraient un peu moins, mais d'autres secteurs seraient portés par le développement des pays du Sud et créeraient des emplois orientés vers la production de biens exportés, notamment dans l'industrie. 

% TODO? Comment distribuer ? Créer système de protection sociale, cf. ILO Kenya.
% Si les pays de l'OCDE tenaient leur promesse d'allouer 0,7\% de leur PIB à l'aide publique au développement, dont 0,2\% du PIB pour les pays les moins avancés, les ODD pourraient être remplis\footnote{Cf. \citet{sdsn_sdg_2019}.}. 
% Flux dans l'autre sens

Enfin, les transferts internationaux sont une condition sine qua none pour que les pays à bas revenus se décarbonent. D'une part, ces pays font face à d'autres priorités que la décarbonation et déploient donc le système énergétique le plus abordable --- reposant souvent sur le charbon. D'autre part, ces pays font valoir --- à juste titre --- qu'ils sont les plus vulnérables au changement climatique et qu'ils n'y ont contribué que marginalement\footnote{L'Afrique et l'Asie du Sud sont responsables de 6\% des émissions de CO$_\text{2}$ cumulées. % https://ourworldindata.org/contributed-most-global-co2
}. Dans les négociations internationales, ces pays annoncent généralement deux objectifs de réductions d'émissions : un objectif inconditionnel peu ambitieux et un objectif ambitieux conditionné à des financements extérieurs. Par exemple, l'Éthiopie s'est engagé à réduire inconditionnellement ses émissions de 14\% en 2030 par rapport à un scénario sans action climatique, et conditionne une réduction de 69\% à un financement de 250 milliards de dollar. % https://www.climatewatchdata.org/custom-compare/overview?section=fairness_and_ambition&targets=NGA-revised_first_ndc%2CETH-revised_first_ndc%2CIDN-

Dans les prochains chapitres, nous proposons un Plan mondial pour mettre fin au changement climatique et à l'extrême pauvreté, impliquant d'importants transferts Nord-Sud, tout en étant acceptable pour les populations des pays du Nord.
% ODA for LDCs: only Luxembourg is above the objective of 0.2% of GNI (SDG 17.2) https://data.worldbank.org/indicator/DC.ODA.TLDC.GN.ZS?locations=US-GR-LU-SE-GB&most_recent_value_desc=true
% 19 pays (570M en 2022, 700M en 2030) ne pourraient pas éradiquer l'extrême pauvreté même en expropriant tous les revenus au-delà de 13$/jour. Pays comme la RDC ne pourrait pas mettre fin à l'extrême pauvreté en 2030, même avec une croissance de 6% et en expropriant tous les revenus au-delà de 7$/jour.

% Addressing global poverty, inequalities and climate change are at the heart of the universally agreed Sustainable Development Goals (SDG). % 12 out of  17
% It has been pointed out that low-income countries generally do not have enough domestic resources to eliminate the poverty gap in the short run.\cite{bolch_arithmetics_2022} % In other words, it would hardly be possible to achieve the first SDG and end extreme poverty by 2030 without international transfers. => Careful, Bolch use a poverty line above the SDG one.



\chapter{Les grands principes du Plan mondial pour le climat\label{ch:principes}}
% Le plan mondial pour le climat : les grands principes (description de la mesure, des trajectoires)

La proposition développée dans les prochains chapitres ne résout pas tous les problèmes de l'humanité, et ne constitue pas non plus une réponse complète au changement climatique. Bien qu'elle soit désignée <<~Plan mondial pour le climat~>> (parce que ça sonne bien), <<~Cadre international de sortie des énergies fossiles~>> aurait été plus fidèle.  % Une désignation telle que <<~Cadre international de sortie des énergies fossiles~>> aurait été plus fidèle, mais il faut dire que <<~Plan mondial pour le climat~>> sonne mieux. %C'est donc peut-être exagéré de le nommer <<~Plan mondial pour le climat~>>, mais ça sonne mieux qu'une désignation plus fidèle telle que <<~Cadre international de sortie des énergies fossiles~>>. 
En effet, ce plan couvre uniquement les émissions de CO$_\text{2}$ fossiles et industrielles, pas celles liées à l'usage des terres, à la forêt ou aux autres gaz à effet de serre. % TODO? Inclure toutes émissions ?
Sa portée se limite à établir un traité-cadre qui garantit les réductions d'émissions et détermine les transferts internationaux. Charge ensuite à chaque État ou collectivité de prendre les mesures (climatiques et sociales) appropriées pour que la décarbonation se passe bien sur son territoire. 

\section{Le cœur du Plan}

On a vu au chapitre \ref{ch:statu_quo} que l'humanité disposait d'un budget carbone à ne pas dépasser pour maintenir le réchauffement sous une cible donnée. L'accord de Paris fournit cette cible. En effet, l'intégralité des pays a signé cet accord en 2015, et visent ainsi à contenir le réchauffement <<~nettement en dessous de 2\textdegree{}C (...) en poursuivant l'action menée pour limiter l'élévation de température à 1,5\textdegree{}C~>>. 

Comment garantir une trajectoire d'émissions conforme à ce budget carbone~? Le plus sûr serait de plafonner les émissions mondiales, avec un plafond annuel qui décroît en conformité avec l'objectif. 

Comment alors allouer les permis d'émissions de CO$_\text{2}$~? Le plus naturel % simple, élémentaire
est d'allouer un même permis d'émissions à chaque humain. 

Faut-il autoriser la revente des permis d'émissions~? Oui, % Passer en FAQ la suite ? TODO paragraphe indigeste, à réécrire
instaurer un marché du carbone est préférable à un système de quota carbone non échangeable pour plusieurs raisons, détaillées dans la FAQ en postface. 
%Déjà, si les permis d'émissions ne sont pas échangeables, cela signifie soit que des centaines millions de personnes (notamment dans les pays du Nord) devraient diviser leurs émissions par deux ou trois du jour au lendemain (les mettant dans l'impossibilité de poursuivre leurs activités), soit qu'on allouerait (dans un premier temps) davantage de permis d'émissions aux personnes qui polluent davantage (ce qui romprait avec le principe d'égalité cher aux défenseurs des quotas non échangeables). A contrario, si les permis sont échangeables, les pollueurs auraient une certaine latitude pour choisir leurs émissions et du temps pour adapter progressivement leurs activités et changer leur équipement, tandis que les personnes avec une faible empreinte carbone pourraient revendre leurs permis d'émissions inutilisés et ainsi gagner du pouvoir d'achat. Ainsi, tant les pollueurs que les frugaux bénéficieraient de la flexibilité permise par le marché. D'ailleurs, le bénéfice potentiel serait tellement important que l'émergence d'un marché noir serait difficile à empêcher dans le cas d'un système de quotas non échangeables. Vous vous dites peut-être qu'un marché du carbone serait immoral ou injuste, car il permettrait aux plus riches de continuer à polluer. Pourtant, un tel système opérerait une redistribution des pollueurs vers les frugaux : ceux-là devant payer pour acheter des permis d'émissions à ceux-ci. En outre, mettre en place un marché du carbone n'empêche pas d'interdire par ailleurs les consommations jugées superflues, telles que les yachts, les jets privés, voire les SUV. Enfin, si on considère injuste que les plus riches soient capables de préserver un mode de vie dispendieux dans un tel système, n'est-ce pas parce qu'on considère l'extrême richesse comme injuste~? Si c'est le cas, autant s'attaquer à la fortune directement, plutôt que de passer par des moyens détournés. En effet, plafonner les émissions de Rupert Murdoch ne l'empêcherait pas d'utiliser son empire médiatique pour minimiser, voire nier le changement climatique. %d'Elon Musk ne l'empêcherait pas d'acheter Twitter et d'en contrôler l'algorithme. 

% Un dernier argument contre le marché est qu'il permettrait des enrichissements illégitimes à cause de la fraude ou la spéculation. 

Pour l'instant, j'ai fait comme si les individus auraient la responsabilité d'acheter ou vendre des permis d'émissions, alors qu'on ne sait même pas mesurer précisément les émissions de quelqu'un. En réalité, on peut obtenir des effets équivalents au système esquissé précédemment avec un fonctionnement bien plus simple~: 

Chaque année, un nombre limité de permis d'émissions est créé, en conformité avec la trajectoire d'émissions qu'on s'est fixée. Ces permis d'émissions sont mis aux enchères auprès des entreprises à la source des émissions de CO$_\text{2}$, et en particulier celles qui extraient du charbon, du pétrole ou du gaz. Ces entreprises doivent se procurer des permis correspondant à leurs émissions. Enfin, les recettes générées par la vente de permis sont redistribuées en un revenu de base égal pour tous les humains. 

D'un point de vue théorique, ce système est équivalent au précédent. En effet, dans ce système, les entreprises polluantes répercutent le coût des permis d'émissions en hausse de prix, si bien que les consommateurs font face à une hausse de dépenses égale au prix du carbone multiplié par leur empreinte carbone. Quant aux recettes générées, elles correspondent au produit du prix du carbone par l'ensemble des émissions, de sorte que le revenu de base est égal au prix du carbone multiplié par l'empreinte carbone moyenne. Dans le système précédent, les permis d'émissions sont alloués de façon égalitaire aux individus, donc les permis d'émissions d'un individu correspondent à l'empreinte carbone moyenne. Ainsi, si une personne revendait tous ses permis d'émissions, elle toucherait un montant équivalent au revenu de base. Par ailleurs, cette personne devrait alors acheter des permis pour couvrir ses émissions, pour un montant égal au prix du carbone multiplié par son empreinte. Dans les deux cas, on retombe sur le même calcul. Le système reposant sur les individus est intéressant d'un point de vue théorique, puisqu'il permet de montrer l'équivalence entre allocation des permis d'émissions et allocation des recettes d'une tarification carbone. En revanche, le système d'enchères aux entreprises est le seul qui soit réaliste à administrer. 
% TODO: la formule

Pour résumer, on peut mettre fin au réchauffement climatique en plafonnant les émissions et éliminer l'extrême pauvreté à l'aide d'un revenu de base. Un système simple et efficace pour traiter ces deux problèmes est de combiner ces deux solutions. Voici le cœur du Plan mondial pour le climat, % TODO changer cette fin?
qui constituent les deux premiers principes détaillés ci-dessous. Pour des considérations de justice et de géopolitique, quelques ajustements sont nécessaires pour compléter notre proposition : je les décris ci-après aux principes 3 et 4. Enfin, ce Plan mondial pour le climat doit être complémenté par d'autres mesures : je les esquisse au chapitre \ref{ch:premier_pas}.

\section{1$^\text{er}$ principe~: Un quota annuel d'émissions}

Le budget carbone --- et avec lui le climat futur --- est l'élément décisif que les États devront négocier. Dans ce livre, nous interprétons l'accord de Paris comme permettant un dépassement temporaire de la cible de 1,5\textdegree{}C dès lors que le réchauffement n'excède pas 2\textdegree{}C. En d'autres termes, des émissions négatives nettes (à partir de la deuxième moitié de ce siècle) permettront de redescendre jusqu'à 1,5\textdegree{}C, seuil qui sera très probablement déjà franchi en 2040\footnote{\citet{diffenbaugh_data-driven_2023} estiment que le réchauffement dépassera 1,5\textdegree{}C en 2035 dans un scénario de décarbonation ambitieuse, ce qui est cohérent avec la Table 4.2 du rapport de l'\citet{ipcc_climate_2021}.}. 

Notre proposition repose sur deux budgets carbone~: un budget d'émissions positives, conforme à l'objectif de ne pas dépasser les 2\textdegree{}C de réchauffement, et un budget d'émissions négatives, permettant de réduire la température. Pour fixer les idées, disons que le budget d'émissions positives serait de 1~000 GtCO$_\text{2}$ à partir de 2025, et le budget d'émissions négatives serait de 500 GtCO$_\text{2}$, ce qui signifie que le budget d'émissions nettes serait de $1~000-500=500$ GtCO$_\text{2}$. Pour l'instant, ne nous préoccupons pas des émissions négatives, qui ne prendront leur essor quand dans quelques décennies. 
% Faire différemment pour gérer le plastique : budget carbone jusqu'à net zéro, puis double budget (positif et négatif)
% Pb du plastique: si on le fait payer à l'extraction, on a intérêt à extraire, stocker sous forme de plastiques, et brûler plus tard quand la tonne est plus chère. Comment faire ?

L'organe décisionnaire du Plan définit le quota annuel d'émissions (positives), selon une trajectoire conforme au budget carbone. En début d'année, ce quota est mis aux enchères. %sous la forme de permis d'émissions d'une tonne de CO$_\text{2}$. 
Les sociétés assujetties sont les entreprises ``upstream'', c'est-à-dire celles qui extraient des hydrocarbures, importent des biens depuis des pays non participants, ou produisent du ciment. Chaque société assujettie transmet la quantité de permis qu'elle s'engage à acheter pour chaque niveau de prix. Pour chaque prix possible, on obtient ainsi une quantité agrégée demandée par les sociétés assujetties. Les permis d'émissions sont alors vendus au prix pour lequel la quantité demandée correspond au quota. Les sociétés assujetties et des sociétés financières agréées peuvent ensuite échanger des permis d'émissions sur un marché secondaire. Au bout de quelques années (probablement un ou deux), le prix d'équilibre aura été découvert, de sorte que le prix sur le marché secondaire sera relativement stable et égal à celui des enchères. En fin d'année, les émissions issues des entreprises assujetties seront contrôlées, et celles-ci devront délivrer des permis correspondant à ces émissions. Des sanctions dissuasives garantiront le bon fonctionnement du système. Par exemple, pour toute tonne de CO$_\text{2}$ non couverte par un permis d'émissions, la société assujettie devrait payer une amende égale à trois fois le prix d'une tonne de CO$_\text{2}$, et devrait toujours délivrer le permis manquant l'année suivante. 

En résumé, un système d'échange de permis d'émission (ETS, pour \textit{emissions trading system}) serait mis en place pour plafonner les émissions de CO$_\text{2}$ au niveau mondial. 
Un tel système a déjà fait ses preuves dans plusieurs pays, dont l'Union Européenne\footnote{L'ETS européen est souvent décrié. Pourtant, il a bel et bien permis de réduire les émissions couvertes (celles de l'industrie et de la production d'électricité) conformément à l'objectif fixé, tandis que les émissions non couvertes (mais qui vont l'être à partir de 2027) ont continué de croître. En réalité, l'ETS européen a été critiqué pour deux (bonnes) raisons. D'une part, l'objectif fixé n'était pas assez ambitieux (c'est ce qui explique le prix très faible jusqu'à une réforme du système en 2019). D'autre part, les permis d'émissions étaient attribués gratuitement aux entreprises polluantes, plutôt que vendus aux enchères. Ces deux écueils seront évités dans le Plan mondial pour le climat.}, la Chine et la Corée du Sud, et est à l'étude dans d'autres comme l'Inde, le Brésil ou le Nigeria. 17\% des émissions mondiales sont déjà couvertes par un ETS. Différents ETS peuvent être fusionnés avec succès, comme l'ont montré la Californie et le Québec \citep{icap_emissions_2023}. 
% Plan en +, pas fusion
% Détails on sait faire, pas besoin d'en parler ici

\section{2$^\text{e}$ principe~: Un revenu de base mondial}

Les recettes du GCP serviraient à financer un revenu de base mondial~: un même transfert à tous les humains de 15 ans ou plus. Nous estimons que le revenu de base s'élèverait à environ 50\euro{} par mois entre 2030 et 2050, ce qui suffirait à sortir de l'extrême pauvreté les 700 millions de personnes qui vivent avec moins de 2 dollars par jour. À leur apogée en 2030, les recettes de l'ETS sont estimées à 5\% du PIB mondial. Le Plan entraînerait des transferts internationaux nets d'environ 1\% % 0,9\%
du PIB mondial, le reste des recettes étant reversé dans le pays où elles sont perçues. Les effets distributifs du Plan sont décrits au chapitre \ref{ch:effets_distributifs}. %En utilisant un scénario limitant le réchauffement climatique à 1,8\textdegree{}C, n
Même si les recettes diminueront lorsque la décarbonation sera presque achevée (vers les années 2060), le revenu de base mondial devrait être maintenu grâce à de nouvelles sources de financement (par exemple, un impôt mondial sur les sociétés). Bien que la distribution d'un revenu de base à chaque être humain soit techniquement difficile, différentes options sont disponibles~: soit les outils administratifs existants, soit des smartphones potentiellement connectés à l'internet par satellite (cf. chapitre \ref{ch:details}).

\begin{figure}[h!]
  \caption{Trajectoires estimées des émissions, du prix du carbone et du revenu de base.}\label{fig:trajectory}
  \makebox[\textwidth][c]{\includegraphics[width=.8\textwidth]{../figures/policies/GCP_trajectoires.pdf}} 
\end{figure}

\section{3$^\text{e}$ principe~: Un club climatique}

Le Plan serait lancé par un club de pays volontaires et mis en œuvre dès que 60\% des émissions mondiales de CO$_\text{2}$ seraient couvertes par les entités participantes. Ce seuil peut être atteint par l'union de la Chine (30\% des émissions mondiales), des États-Unis (15\%), de l'Inde (7\%), de l'UE et du Royaume-Uni (9\%). Si les États-Unis ne participent pas\footnote{Notons que certains États américains pourraient rejoindre le club même si le niveau fédéral ne le fait pas}, ce seuil serait quand même atteint dans un scénario \textit{Prudent} où le club serait formé par l'UE, les pays qui bénéficieraient financièrement du Plan (23\%, dont l'Inde) et ceux qui ne seraient ni gagnant ni perdant financièrement (35\%, dont la Chine). Dans un scénario \textit{Optimiste} où on ajoute à cela les autres États susceptibles de rejoindre le club\footnote{En plus des pays qui ne seraient pas perdants financièrement et de l'UE, on peut s'attendre à une participation des États suivants~: Royaume-Uni, Japon, Corée du Sud, Norvège, Suisse, Nouvelle-Zélande, Canada, ainsi que les 12 États états-uniens où le parti démocrate a remporté les dernières élections avec plus de 10 points d'écart (en particulier la côte Ouest, l'Illinois, et le Nord-Est à l'exception de la Pennsylvanie).}, 93\% de la population et 76\% des émissions mondiales seraient couvertes (cf. Table \ref{scenarios_table_fr}). 

\begin{table}[h]

    \caption{\label{tab:scenarios_table_fr}Principales caractéristiques des différents scénarios de club climatique.}
    \makebox[\textwidth][c]{
    \begin{tabular}[t]{ccccc}
    \toprule
    \makecell{Scenario\\de club} & \makecell{Émissions\\mondiales\\couvertes} & \makecell{Population\\mondiale\\couverte} & \makecell{Revenu de base\\en 2040\\(\$/mois)} & \makecell{Contribution de l'UE\\en 2040\\(fraction de son PIB)}\\
    \midrule
    Tous les pays & 100\% & 100\% & 47 & 0.6\%\\
    Tous sauf OPEP+ & 90\% & 97\% & 44 & 0.7\%\\
    Optimiste & 76\% & 93\% & 39 & 0.9\%\\
    Prudent & 65\% & 87\% & 30 & 1.0\%\\
    UE + Chine + gagnants & 61\% & 84\% & 27 & 1.1\%\\
    UE + Afrique & 12\% & 24\% & 31 & 1.0\%\\
    \bottomrule
    \end{tabular}}
\end{table}

Les importations vers le club seraient taxées en proportion de leur contenu carbone~: c'est la fameuse tarification carbone aux frontières (que l'UE met en place à son échelle).

\section{4$^\text{e}$ principe~: Des mécanismes de participation}

Une clause dérogatoire (dite d'\textit{opt out}) à la mutualisation des recettes permettrait à des pays (comme la Chine, l'Afrique du Sud ou l'Algérie) dont le PIB par habitant n'excède pas la moyenne mondiale de plus de 50\% de conserver les recettes prélevées sur leur territoire. Cette clause éviterait à des pays aux revenus moyens d'être contributeur net malgré leur empreinte carbone supérieure à la moyenne.
Le traité permettrait également à des États tels que la Californie ou l'État de New York de rejoindre le club indépendamment du niveau fédéral, notamment en les exemptant de la tarification aux frontières.

% 1.	Plafonner les émissions en conformité avec une trajectoire +2°C, à l’aide d’un ETS
% •	Le traité définirait un budget carbone, ensuite décliné en quotas annuels.
% •	Des permis d’émissions de CO2 seraient vendus aux enchères aux entreprises émettrices, comme dans le marché du carbone (ETS) européen.	
% •	Le prix du carbone inciterait les ménages et entreprises à se décarboner. 
% •	Il serait in fine payé par les individus en proportion de leur empreinte carbone.	

% 2.	Utiliser les recettes pour verser un revenu de base mondial et résorber la pauvreté
% •	Les recettes seraient reversées sous la forme d’un revenu de base à tous les adultes.
% •	Chaque humain recevrait ainsi environ 50€/mois.
% •	Les versements sont techniquement possibles (smartphones, internet par satellites).

% 3.	Former un club climatique avec une tarification carbone aux frontières
% •	Le traité entrerait en vigueur dès que les entités signataires couvrent 60% des émissions mondiales (Chine : 30%, pays bénéficiaires nets : 23%, UE+UK : 9%, U.S. : 15%).
% •	Les importations vers le club seraient taxées en proportion de leur contenu carbone.

% 4.	Inclure des mécanismes qui encouragent la participation de la Chine, la Californie… 
% •	Le traité permettrait à des États de rejoindre l’accord indépendamment du niveau fédéral, notamment en les exemptant de la tarification aux frontières.
% •	Une clause d’opt out de la mutualisation des recettes permettrait à des pays (comme la Chine, l’Afrique du Sud ou l’Algérie) dont le PIB par habitant n’excède pas la moyenne mondiale de plus de 50% de conserver les recettes prélevées sur leur territoire, leur évitant d’être contributeur net malgré leur empreinte carbone supérieure à la moyenne.

% 5.	Complémenter ce Plan par d’autres mesures pour le climat et la justice sociale
% •	Une fiscalité plus progressive pour compenser les classes moyennes des pays riches.
% •	Un ISF mondial finançant les pays pauvres pour traiter les responsabilités historiques.
% •	Des mesures climatiques nationales pour faciliter la décarbonation.


\chapter{Les détails du Plan\label{ch:details}}
% Le plan mondial pour le climat : les détails (implémentation, mécanismes de participation)

\chapter{Un transfert massif vers les pays du Sud\label{ch:effets_distributifs}}
% Le plan mondial pour le climat : les effets distributifs (carte des pays gagnants et perdants)


\begin{figure}[h!]
  \caption{Gains ou pertes suite au Plan mondial pour le climat en 2030.}\label{fig:median_gain_2015}
  \makebox[\textwidth][c]{\includegraphics[width=\textwidth]{../figures/maps/gain_adj_2030_fr.pdf}} % mean_gain_over_gdp_2019 mean_gain_2030
\end{figure}

\begin{figure}[b!]
  \caption{Gains ou pertes suite au Plan mondial pour le climat sur le XXI$^\text{e}$ siècle}\label{fig:median_gain_adj}
  \makebox[\textwidth][c]{\includegraphics[width=\textwidth]{../figures/maps/npv_over_gdp_gcs_adj_fr.pdf}} % mean_gain_over_gdp_2019
  {\footnotesize \textit{Note:} La valeur nette actualisée est calculée avec un taux d'actualisation de 4\% sur la période 2020 -- 2100.}
\end{figure}

% Une redistribution de 1% de PIB mondial des pays riches vers les pays pauvres.
% •	Le Plan opérerait une redistribution des individus avec une empreinte carbone élevée vers ceux avec une empreinte faible. L’effet serait nul au niveau de la moyenne mondiale.
% •	Lors de leur plateau, entre 2030 et 2050, les recettes correspondraient à environ 5% du PIB mondial, dont 1% en transferts nets entre pays.
% •	Le Français moyen perdrait 14€/mois au pic, soit une perte de 0,4% du revenu national.
% •	Le revenu de base sortirait de l’extrême pauvreté les 700 millions vivant sous les 2$/jour.
% •	Ces estimations ont été faites avec le scénario +2°C central du GIEC, qui implique un prix du carbone d’environ 150$/tCO2 en 2030, 200$ en 2040, 400$ en 2050, et net zéro en 2070.


\chapter{Un Plan largement soutenu\label{ch:soutien}}

% Un plan largement soutenu (soutien majoritaire dans 20 pays, avantage électoral aux partis de gauche qui l'incluraient dans leur programme, consensus en faveur de la répartition égalitaire de l'effort de décarbonation)

\chapter{Un pas vers un monde soutenable\label{ch:premier_pas}}
% Un pas vers un monde soutenable (le Plan n'est pas suffisant, il doit être complémenté d'autres mesures climatiques (nationales, sectorielles), d'autres mesures de redistribution mondiale (impôt sur la fortune), et l'objectif de long terme doit être d'assurer les conditions nécessaires au bien-être à chaque humain)

\chapter{L'appel pour la redistribution mondiale\label{ch:appel}}
% L'appel pour la redistribution mondiale (description de l'activité de Global Redistribution Advocates, lien vers la lettre ouverte aux dirigeants pour la redistribution mondiale - que nous prévoyons de publier dans les grands journaux type The Guardian, Le Monde..., appel à une manifestation mondiale en soutien à cet appel le 1er décembre 2024 lors de la COP29).

\chapter{Foire Aux Questions\label{ch:faq}}


% \begin{center}
% {\textbf{\href{https://github.com/bixiou/global_tax_attitudes/raw/main/paper/book.pdf}{Link to most recent version}}}
% \end{center}

% 0. Préface : j'envisage Thomas Piketty, Jean Tirole, Gaël Giraud ou Esther Duflo pour la version française et Greta Thunberg ou Joseph Stiglitz pour la version anglaise (mais ça reste à définir)
% 1. Un statu quo insupportable : l'extrême pauvreté, le changement climatique (chiffres, constat)
% 2. La nécessité de redistribution mondiale (objectifs de développement durable, justifications théoriques)
% 3. Le plan mondial pour le climat : les grands principes (description de la mesure, des trajectoires)
% 4. Le plan mondial pour le climat : les détails (implémentation, mécanismes de participation)
% 5. Le plan mondial pour le climat : les effets distributifs (carte des pays gagnants et perdants)
% 6. Un plan largement soutenu (soutien majoritaire dans 20 pays, avantage électoral aux partis de gauche qui l'incluraient dans leur programme, consensus en faveur de la répartition égalitaire de l'effort de décarbonation)
% 7. Un pas vers un monde soutenable (le Plan n'est pas suffisant, il doit être complémenté d'autres mesures climatiques (nationales, sectorielles), d'autres mesures de redistribution mondiale (impôt sur la fortune), et l'objectif de long terme doit être d'assurer les conditions nécessaires au bien-être à chaque humain)
% 8. L'appel pour la redistribution mondiale (description de l'activité de Global Redistribution Advocates, lien vers la lettre ouverte aux dirigeants pour la redistribution mondiale - que nous prévoyons de publier dans les grands journaux type The Guardian, Le Monde..., appel à une manifestation mondiale en soutien à cet appel le 1er décembre 2024 lors de la COP29).
% 9. Postface : Foire Aux Questions (description et réponse aux objections habituelles)

% Un plan mondial pour le climat et contre l'extrême pauvreté (version courte illustrée / version longue) :
% - situation pauvreté, csq CC
% - SDGs, justifications redistr. mondiale
% - description GCS, y.c. trajectoires
% - détails implémentation (e.g. Aadhaar)
% - acceptation
% - autres mesures mondiales possibles
% - appel signé par milliers + manif mondiale un an après lancement
% - préface Piketty puis Greta
% - FAQ, y.c. critiques d'une page (et réponses) de Piketty, etc.
% Opuscule Cepremap au pire

% FAQ
% - Les émissions vont augmenter si on distribue un revenu de base et les revenus augmentent
% - Qui paient : entreprises ou consommateurs ?
% - Les quotas profitent aux plus riches qui peuvent s'acheter plus de quotas, les plus pauvres bradent leurs quotas pour des rétributions immédiates en obérant des potentialités pour le futur 
% - pas sûr que ça ait fait ses preuves 
% - il y a certaines pratiques qui vont devoir s'arrêter, il faut interdire
% => Le Plan fournit un cadre qui assure des réductions d'émissions adéquates; qui incite les États à prendre des mesures complémentaires type interdiction; on peut remplacer le système d'échanges de permis par une taxe, c'est équivalent. 
% Pourquoi pas taxe progressive ?
% Problème de la fraude / du monitoring.
% Problème de la transparence sur les implications (les gens réalisent-ils les changements de mode de vie nécessaires ?)
% Qu'est-ce qu'on fait de l'EU ETS? On le laisse en place. 
% Les taxes affectées ne sont-elles pas interdites ? => no, e.g. Chirac tax on flying

% Wealth tax: 
% taxe universelle ? exit tax ?

\renewcommand{\url}[1]{\href{#1}{Link}} % NCCcomment
\bibliographystyle{plainnaturl_clean} % NCCcomment
\bibliography{global_tax_attitudes}

\end{document}