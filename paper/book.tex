\documentclass[12pt,english]{article}
\usepackage[utf8]{inputenc}
\usepackage{tgpagella} % Palatino text only
\usepackage{mathpazo}  % Palatino math & text
\usepackage[left=1.5in,right=1.5in,top=1.5in,bottom=1.5in]{geometry}
% \linespread{1.5}
\usepackage[super,comma,sort]{natbib}
% \usepackage[round,sort&compress]{natbib}
\usepackage{url} % [hyphens]
\usepackage[hyperpageref]{backref} % back references biblio. Needs latexmk at compilation.
\usepackage[pagebackref]{hyperref}
% \usepackage{multibib} % incompatible with backref
\hypersetup{
  colorlinks=true, % breaklinks=true,
  urlcolor=purple,    % color of external links
  linkcolor=blue,  % color of toc, list of figs etc.
  citecolor=violet,   % color of links to bibliography
}
\usepackage{bm}
\usepackage{indentfirst}
\usepackage{tocbibind}
\setcitestyle{aysep={}} 
\usepackage{amsmath}
\usepackage{amssymb}
\usepackage{eurosym}
\usepackage{amsfonts}
\usepackage{enumerate}
\usepackage{babel}
\usepackage{caption}
\usepackage{supertabular}
\usepackage{tabularx}
\usepackage{float}
\usepackage{dsfont}
\usepackage{fancyvrb}
\usepackage{verbatim}
\usepackage{enumitem}
\usepackage{setspace}
\usepackage{comment}
\usepackage{subcaption}
\usepackage{graphicx}
\usepackage{tikz}
\usepackage{gensymb}
\usepackage{textcomp}

\usepackage{tabulary}
\usepackage{tabularx}
\usepackage{booktabs}
\usepackage{fullpage}
\usepackage{morefloats}
\usepackage{makecell}
\usepackage{lscape}
\usepackage{pdflscape}
\usepackage{longtable}
\usepackage{rotating}
\usepackage{fancyhdr}
\usepackage{tocloft}
\usepackage{titletoc}
\usepackage[export]{adjustbox}
\usepackage[anythingbreaks]{breakurl} % for links
\usepackage{multicol}
\newsavebox\ltmcbox % For net gain table over two columns
%\usepackage[nomarkers,figuresonly]{endfloat} % Figures at the end
%\usepackage[section,below]{placeins} % Floats placed in the section they appear in.
\renewcommand{\floatpagefraction}{.9}

\title{Un plan mondial pour le climat et contre l'extrême pauvreté} 

\author{Adrien Fabre\footnote{CNRS, CIRED. E-mail: fabre.adri1@gmail.com (corresponding author).}} 

\date{\today} 

\begin{document}

\maketitle

% \begin{center}
% {\textbf{\href{https://github.com/bixiou/global_tax_attitudes/raw/main/paper/book.pdf}{Link to most recent version}}}
% \end{center}

% Un plan mondial pour le climat et contre l'extrême pauvreté (version courte illustrée / version longue) :
% - situation pauvreté, csq CC
% - SDGs, justifications redistr. mondiale
% - description GCS, y.c. trajectoires
% - détails implémentation (e.g. Aadhaar)
% - acceptation
% - autres mesures mondiales possibles
% - appel signé par milliers + manif mondiale un an après lancement
% - préface Piketty puis Greta
% - FAQ, y.c. critiques d'une page (et réponses) de Piketty, etc.
% Opuscule Cepremap au pire

% FAQ
% - Les émissions vont augmenter si on distribue un revenu de base et les revenus augmentent
% - Qui paient : entreprises ou consommateurs ?
% - Les quotas profitent aux plus riches qui peuvent s'acheter plus de quotas, les plus pauvres bradent leurs quotas pour des rétributions immédiates en obérant des potentialités pour le futur 
% - pas sûr que ça ait fait ses preuves 
% - il y a certaines pratiques qui vont devoir s'arrêter, il faut interdire
% => Le Plan fournit un cadre qui assure des réductions d'émissions adéquates; qui incite les États à prendre des mesures complémentaires type interdiction; on peut remplacer le système d'échanges de permis par une taxe, c'est équivalent. 
% - Pourquoi pas taxe progressive ?
% - Problème de la fraude / du monitoring.
% - Problème de la transparence sur les implications (les gens réalisent-ils les changements de mode de vie nécessaires ?)
% - Qu'est-ce qu'on fait de l'EU ETS? On le laisse en place. 

% Wealth tax: 
% taxe universelle ? exit tax ?

\renewcommand{\url}[1]{\href{#1}{Link}} % NCCcomment
\bibliographystyle{plainnaturl_clean} % NCCcomment
\bibliography{global_tax_attitudes}
