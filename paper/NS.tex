% TODO! Cite Oxfam (23) that millionaires want wealth tax?
% TODO? show that support decrease between global and complementary surveys due to personal costs
% TODO? say more on method of list experiment in Table 1
% TODO? say that ETS doesn't specify allocation of emissions shares?
% TODO? when we cite Carattini (" global carbon taxes"), say our relative contribution
% TODO? rewrite "By employing a list experiment..." for a broader audience?
% TODO? Cite Gampfer et al. (14) & Bechtel et al. (22) as more examples that people prefer global/multilateral response
% TODO? Cite Gaikwad et al. (23): show Americans prefer to decarbonize the U.S. rather than India => BOF
% TODO! Authors contributions, numbering lines, word count
% TODO! cite Young-Brun et al. (23), blacksands (23)?
% TODO? add figshare data: https://doi.org/10.6084/m9.figshare.24274702
% TODO! add variables names to questionnaire
% /!\ No footnote for NCC or NS.
% TODO! cite Andre et al. (24)
% TODO! more raw results, sources (cf. app.tex)
% plot maps and compare distributive effects of equal pc, contraction & convergence, greenhouse dvlpt rights, historical respo, and each country retaining its revenues
% improve net gains with SSPs

% Words in text (excl. abstract, figures, references, methods): 3652, incl. 3487 outside box + 43 in headers => 3,540 (vs. 3500)
% Abstract: 150 (vs 150)
% Display items: 2 large figures, 2 medium tables + 1 box (vs 6)
% References: 21
% Methods: 1162 
% Legend: 145

% TODO!! standard class, bibliography in .tex
% Each Supplementary Figure should fit, along with its legend, on a single PDF page. 

% app_desc: some more figures (e.g. detailed OECD by country)
% conjoint d: by vote for all countries; merge with r; bug PDF
% TODO appendix sources
% autres to do!: n, additional figures/tables or details
% literature: list experiment, (elite surveys)
% Extra: translate country-specific appendices

% Nature guidelines:
% Attention to the following details can help expedite publication if we invite a revision after external review.
% A fully referenced ~200 word summary paragraph; main text of 2,500 words (incl. summary paragraph but excl. methods) and 4 modest display items (figures, tables, i.e. 1/4 page ~270 words) for a typical 6 page article and 4300 words and 5-6 modest display items for a typical 8 page article; as a guideline up to 50 references if needed and within the allocated page budget.

% Nature: 4800 words (excl. methods, appendix) + 1160 words in methods + 6eq modest figures + 15 references. (Total 5965) => we need to cut ~500 words. I have identified ~200 words that can be easily removed (TODO? remove?). We still need to find 150-450 more. This could be the Tables (especially the electoral gains). 
% Draft: 6200 words (excl. abstract, methods, appendix), incl. 5.2k in Results.
% Current: 5600 words (excl. abstract, methods, appendix), incl. 4.6k in Results.
% Old: 4300 words (excl. abstract, methods, appendix) + 3 medium-large figures (equivalent, I know it's 4) => correct size
% => write a 2-3 pager (1000 words with 2 figures and one-sentence abstract) on Word with Science's template, send it as submission enquiry to Nature (as one can't send full paper); then for Policy forum in Science.
% => write full paper as a 6-page (2500 words*) in LaTeX or Word so it can fit in PNAS, and even for NCC/NS it shouldn't exceed 8-9-page. (It's already too long for Science's max 5 page).
% *6-page is more 5000 than 2500 words. I've checked Steckel et al (NS, 21) and it's 5k words for 7p (excl. abstract and methods, dont 3.8k excl. the 6 medium-large figures/tables) + 2.2k in methods, data availability. Case & Deaton (PNAS, 21) is also 5k words for 5p (excl abstract and biblio, dont 4.7k excl. the 3 medium-large figures). Bruckner et al (NS, 22): 4.3k words for 6p (excl. abstract, methods, biblio, dont 3.8k excl the 5 medium-large figures) + 3.4k in methods. => about 1k per page without figure/table. 
% => Submission order: 1. Nature, 2. Science, 3. NCC, 4. Nature Sust, 5. PNAS, 6. Science Advances, 7. GEC, 8. ERL or JEEM. (or 3. NS (if: 27) and 4. NCC (if: 22)?)
% => Sample sizes should be given (only) on Figures

% Nature "abstract": Major sustainability objectives could be achieved by global approaches to mitigating climate change and inequality    . For instance, a global carbon price funding a global basic income, called the “Global Climate Scheme” (GCS), would be an effective way to jointly combat climate change and poverty. A key condition for the success of global cooperation is the support of citizens in affluent countries for such globally redistributive policies. Yet, few prior attitudinal surveys have examined support for global policies. To explore relevant public attitudes, we survey over 48,000 respondents from 20 high- and middle-income countries. The responses reveal strong support for global redistributive policies, including the GCS and a global wealth tax aimed at financing low-income countries. A list experiment shows no evidence of social desirability bias in survey responses, majorities are willing to sign a real-stake petition, and global redistribution ranks high in the prioritization of policies. Conjoint analyses reveal that a political platform is more likely to be preferred if it contains the GCS or a global tax on millionaires. In sum, our findings indicate that global redistributive policies are genuinely supported by a majority of the population, even in wealthy nations that would bear a significant burden. Public opinion is therefore not the reason that they do not prominently enter political debates. These results could help draw attention to global policies in the public debate and contribute to their increased prominence.


%%%%%%%%%%%%%%%%%%%%%%%%%%%%%%%%%%%%%%%%
%%%%% NATURE CLIMATE CHANGE FORMAT %%%%%
%%%%%%%%%%%%%%%%%%%%%%%%%%%%%%%%%%%%%%%%
%% Comment "% WPcomment" lines, uncomment "% NCCcomment" lines as well as the lines below, replace all citet/citep by cite

% \documentclass{nature}
% \usepackage{amsmath}
% \usepackage{amssymb}
% \usepackage{eurosym}
% % The following allows keeping figures within the text (otherwise nature.cls would ignore them)
% \usepackage{graphicx}
% \makeatletter
% \let\saved@includegraphics\includegraphics
% \AtBeginDocument{\let\includegraphics\saved@includegraphics}
% \renewenvironment*{figure}{\@float{figure}}{\end@float}
% \makeatother

% Nature guidelines (not NCC!)
% Sections can only be used in Articles.  Contributions should be organized in the sequence: title, text, methods, references, Supplementary Information line (if any), acknowledgements, interest declaration, corresponding author line, tables, figure legends.

% No subsubsection nor paragraph

% Spelling must be British English (Oxford English Dictionary)

%Each figure legend should begin with a brief title for the whole figure and continue with a short description of each panel and the symbols used. For contributions with methods sections, legends should not contain any details of methods, or exceed 100 words (fewer than 500 words in total for the whole paper). In contributions without methods sections, legends should be fewer than 300 words (800 words or fewer in total for the whole paper).

% Articles are restricted to 50 references,

% In addition, a cover letter needs to be written with the
% following:
% \begin{enumerate}
%  \item A 100 word or less summary indicating on scientific grounds
% why the paper should be considered for a wide-ranging journal like
% \textsl{Nature} instead of a more narrowly focussed journal.
%  \item A 100 word or less summary aimed at a non-scientific audience,
% written at the level of a national newspaper.  It may be used for
% \textsl{Nature}'s press release or other general publicity.
%  \item The cover letter should state clearly what is included as the
% submission, including number of figures, supporting manuscripts
% and any Supplementary Information (specifying number of items and
% format).
%  \item The cover letter should also state the number of
% words of text in the paper; the number of figures and parts of
% figures (for example, 4 figures, comprising 16 separate panels in
% total); a rough estimate of the desired final size of figures in
% terms of number of pages; and a full current postal address,
% telephone and fax numbers, and current e-mail address.
% \end{enumerate}

% See \textsl{Nature}'s website
% (\texttt{http://www.nature.com/nature/submit/gta/index.html}) for
% complete submission guidelines.

%%%%%%%%%%%%%%%%%%%%%%%%%%%%%%%%
%%%%% WORKING PAPER FORMAT %%%%%
%%%%%%%%%%%%%%%%%%%%%%%%%%%%%%%%
%% Comment "% NCCcomment" lines, uncomment "% WPcomment" lines as well as the lines below
\documentclass[12pt,english]{article}
% \usepackage[utf8]{inputenc}
% \usepackage{tgpagella} % Palatino text only
% \usepackage{mathpazo}  % Palatino math & text
% \usepackage[left=1.5in,right=1.5in,top=1.5in,bottom=1.5in]{geometry}
% \linespread{1.5}
\usepackage[super,comma,sort]{natbib} % WPcomment NCCcomment
% \usepackage[round,sort&compress]{natbib} % NCCcomment
\usepackage{url} % [hyphens]
\usepackage[hyperpageref]{backref} % back references biblio. Needs latexmk at compilation.
\usepackage[pagebackref]{hyperref}
% \usepackage{multibib} % incompatible with backref
\hypersetup{
  colorlinks=true, % breaklinks=true,
  urlcolor=purple,    % color of external links
  linkcolor=blue,  % color of toc, list of figs etc.
  citecolor=violet,   % color of links to bibliography
}
\usepackage{bm}
\usepackage{indentfirst}
\usepackage{tocbibind}
% \setcitestyle{aysep={}} 
\usepackage{amsmath}
\usepackage{tcolorbox}
\usepackage{amssymb}
\usepackage{eurosym}
\usepackage{amsfonts}
\usepackage{enumerate}
\usepackage{babel}
\usepackage{graphicx}
\usepackage{caption}
\usepackage{supertabular}
\usepackage{tabularx}
\usepackage{float}
\usepackage{dsfont}
\usepackage{fancyvrb}
\usepackage{verbatim}
\usepackage{enumitem}
% \usepackage{setspace} % NCCcomment
\usepackage{comment}
\usepackage{subcaption}
\usepackage{tikz}
\usepackage{gensymb}
\usepackage{textcomp}
\usepackage{lineno}
\linenumbers

\usepackage{tabulary}
\usepackage{tabularx}
\usepackage{booktabs}
\usepackage{fullpage}
\usepackage{morefloats}
\usepackage{makecell}
\usepackage{lscape}
\usepackage{pdflscape}
\usepackage{longtable}
\usepackage{rotating}
\usepackage{fancyhdr}
\usepackage{tocloft}
\usepackage{titletoc}
\usepackage[export]{adjustbox}
\usepackage[anythingbreaks]{breakurl} % for links
\usepackage{multicol}
\newsavebox\ltmcbox % For net gain table over two columns
\usepackage[nomarkers,tablesonly]{endfloat} % Tables at the end
%\usepackage[section,below]{placeins} % Floats placed in the section they appear in.
\renewcommand{\floatpagefraction}{.99}
\newenvironment{stretchpars}{\par\setlength{\parfillskip}{0pt}}{\par} % to justify a line

\makeatletter
\newcommand{\fakesection}[1]{%
  % go to vertical mode and don't allow a page break here
  \par\nopagebreak
  % step up the counter
  \refstepcounter{section}
  % teach nameref the title
  \def\@currentlabelname{#1}
  % add to TOC
  \addcontentsline{toc}{section}{\protect\numberline{\thesection}#1}
}
\makeatother

% % Getting landscape page and page number/footer on bottom of page (instead of to the left)
% \fancypagestyle{mylandscape}{
% \fancyhf{} %Clears the header/footer
% \fancyfoot{% Footer
% \makebox[\textwidth][r]{% Right
%   \rlap{\hspace{1.5cm}% Push out of margin by \footskip
%     \smash{% Remove vertical height
%       \raisebox{13.6cm}{% Raise vertically
%         \rotatebox{90}{\thepage}}}}}}% Rotate counter-clockwise
% \renewcommand{\headrulewidth}{0pt}% No header rule
% \renewcommand{\footrulewidth}{0pt}% No footer rule
% }

% \fancypagestyle{page_left}{%
% 	\renewcommand{\headrulewidth}{0pt}
%   \fancyhf{}
%   \fancyfoot[OC]{%
%       \begin{tikzpicture}[remember picture,overlay]
%           \node[xshift=1cm] (number) at (current page.west) {\thepage};
%       \end{tikzpicture}
%   }%
% }
% \renewcommand{\thesubfigure}{\Alph{subfigure}}

% \newcites{App}{Appendix References}

% \captionsetup[table]{skip=-10pt}
% \begin{document}

% \maketitle

% \clearpage
% % \startcontents
% % \printcontents{ }{1}{\section{\contentsname}}
% % \clearpage
% \section{Introduction\label{sec:intro}}

% % \clearpage
% \renewcommand{\bibsection}{\section{\refname}}
% \bibliographystyle{naturemag}
% \bibliography{global_tax_attitudes}
% % \stopcontents

% \end{document}


\title{International Attitudes Toward Global Policies %\\ Addressing Climate Change and Inequality 
} 

% \author{Adrien Fabre$^{1,2}$, Thomas Douenne$^3$ and Linus Mattauch$^{4,5,6}$} % WPcomment
\author{Adrien Fabre\footnote{CNRS, CIRED. E-mail: adrien.fabre@cnrs.fr (corresponding author).}, Thomas Douenne\footnote{University of Amsterdam}\; and Linus Mattauch\footnote{Technical University Berlin, Potsdam Institute for Climate Impact Research -- Member of the Leibniz Association and University of Oxford}} 
%~~\thanks{The project %is approved by IRB at Harvard University (IRB21-0137), and 
% was preregistered in the Open Science Foundation registry (\href{https://osf.io/fy6gd}{osf.io/fy6gd}). \\ We are grateful for financial support from the University of Amsterdam and TU Berlin. Mattauch also thanks the Robert Bosch Foundation. %We are grateful for financial support from the OECD, the French Ministry of Foreign Affairs, the French Conseil d’Analyse Economique and the Spanish Ministry for the Ecological Transition and Demographic Challenge. We also acknowledge support from the Grantham Foundation for the Protection of the Environment and the Economic and Social Research Council through the Centre for Climate Change Economics and Policy. 
% We thank Antoine Dechezleprêtre, Tobias Kruse, Bluebery Planterose, Ana Sanchez Chico, and Stefanie Stantcheva for their invaluable inputs for the project. We thank Antonio Bento, Dietmar Fehr, and Auriane Meilland for feedback. We further thank Jakob Niemann, Laura Schepp, Martín Fernández-Sánchez, Samuel Gervais, Samuel Haddad, and Guadalupe Manzo for assistance. Fabre declares that he also serves as president of Global Redistribution Advocates.}} % NCCcomment

\date{\today} % NCCcomment

\begin{document}

\maketitle

\begin{center}
%{\textbf{\href{https://github.com/bixiou/global_tax_attitudes/raw/main/paper/draft.pdf}{Link to most recent version}}}
\end{center}


% WPcomment
% \begin{affiliations}
% \item Centre National de la Recherche Scientifique
% \item Centre International de Recherches sur l'Environnement et le Développement
% \item University of Amsterdam
% \item Technical University Berlin
% \item Potsdam Institute for Climate Impact Research 
% \item University of Oxford
% \end{affiliations}

% \begin{small} % NCCcomment
\begin{abstract}
  % We document majority support for policies entailing global redistribution and climate mitigation. Recent surveys on 40,680 respondents in 20 countries covering 72\% of global carbon emissions show strong support for an effective way to jointly combat climate change and poverty: a global carbon price funding a global basic income, called the ``Global Climate Scheme'' (GCS). Using complementary surveys on 8,000 respondents in the U.S., France, Germany, Spain, and the UK, we test several hypotheses that could reconcile strong stated support with a lack of salience in policy circles. A list experiment shows no evidence of social desirability bias, majorities are willing to sign a real-stake petition, and global redistribution ranks high in the prioritization of policies. Conjoint analyses reveal that a platform is more likely to be preferred if it contains the GCS or a global tax on millionaires. Universalistic attitudes are confirmed by an incentivized donation. In sum, our findings indicate that global policies are genuinely supported by a majority of the population. Public opinion is therefore not the reason that they do not prominently enter political debates. % 179 words
  % Major sustainability objectives could be achieved by global approaches to mitigating climate change and inequality. For instance, a global carbon price funding a global basic income, called the ``Global Climate Scheme'' (GCS), would be an effective way to jointly combat climate change and poverty. %A key condition for the success of global cooperation is the support of citizens in affluent countries for such globally redistributive policies. 
  % Yet, few prior attitudinal surveys have examined support for global policies. To explore relevant public attitudes, we 
  % %comment on the results of surveys conducted by Dechezleprêtre et al. (2022) % 
  % analyse surveys  % TODO!back
  % over 40,000 respondents from 20 high- and middle-income countries,\cite{dechezlepretre_fighting_2022} and conduct 
  % %our own % TODO!back
  % complementary surveys on 8,000 respondents from the U.S. and four European countries. % TODO!? "We survey" put back /  conducted by Dechezleprêtre et al. (2022)
  % We find that there exists substantial support for global policies addressing climate change and global inequality, even in high-income countries. The GCS is supported by three quarters of Europeans and half of Americans. 
  % Further responses reveal strong support for global redistributive policies, including the GCS and a global wealth tax aimed at financing low-income countries. We test whether support of the expressed preference is sincere: a list experiment shows no evidence of social desirability bias in survey responses, majorities are willing to sign a real-stake petition, and global redistribution ranks high in the prioritization of policies. Conjoint analyses reveal that a political platform is more likely to be preferred if it contains the GCS or a global tax on millionaires. In sum, our findings indicate that global redistributive policies are genuinely supported by a majority of the population, even in wealthy nations that would bear a significant burden. %Public opinion is therefore not the reason that they do not prominently enter political debates. These results could help draw attention to global policies in the public debate and contribute to their increased prominence. % 232 words
  % TODO? mention more the global tax or the assembly?

  Major sustainability objectives could be achieved by global approaches to climate change and inequality. For instance, a global carbon price funding a global basic income, called the ``Global Climate Scheme'' (GCS), would jointly combat climate change and poverty. %yet evidence on the support for global policies is scarce. 
  We analyse surveys 
  over 40,000 respondents from 20 high- and middle-income countries,\cite{dechezlepretre_fighting_2022} and conduct 
  %our own % TODO!back
  complementary surveys on 8,000 respondents from the U.S. and four European countries. %Using surveys over 48,000 respondents from 20 countries, 
  The GCS is supported by three quarters of Europeans and half of Americans, even as they understand the policy's cost to them. % even as the policy costs are salient in the survey
  Using different experiments, we show that the support for the GCS is sincere and that electoral candidates could win votes by endorsing it. More generally, we document widespread support for other global redistribution policies, such as a wealth tax funding low-income countries. 
  % Our findings indicate that 
  In sum, we provide evidence that global policies are genuinely supported by majorities, even in wealthy nations that would bear the burden. % Overall, we provide evidence that
\end{abstract}
% Conf submissions: AFSE, EAERE, JMA, AFEP, Earth System governance, Philo Éco
% \end{small} % NCCcomment
% TODO add results 66% willing to adopt sustainable behavior under conditions

% \textbf{JEL codes:} P48, Q58, H23, Q54 % NCCcomment
% Q54 Climate • Natural Disasters and Their Management • Global Warming
% Q58 Government Policy (Q is Environmental econ)
% D78 Positive Analysis of Policy Formulation and Implementation
% H23 Externalities • Redistributive Effects • Environmental Taxes and Subsidies (H is public econ)
% P48 Political Economy • Legal Institutions • Property Rights • Natural Resources • Energy • Environment • Regional Studies (P4 is Other economic systems)
% H41 Public Goods
% H54 Infrastructures • Other Public Investment and Capital Stock

% \textbf{Keywords:} Climate change, global policies, cap-and-trade, attitudes, survey.%, inequality, wealth tax. % NCCcomment

% \tableofcontents % NCCcomment

% \onehalfspacing % NCCcomment

%\clearpage

% \section{Introduction}% NCCcomment
% TODO!? Now at 62? 
% Inequality between countries: https://data.worldbank.org/indicator/NY.GDP.PCAP.PP.CD?contextual=default&end=2021&locations=EU-ZG-XD-XM-1W-IN-US-CD-BI-LU-CN&start=2021&view=bar (PPP current $) / https://data.worldbank.org/indicator/NY.GDP.PCAP.CD?end=2021&locations=EU-ZG-XD-XM-1W-IN-US-CD-BI-LU-CN&start=2021&view=bar (current $) / https://data.worldbank.org/indicator/NY.GDP.PCAP.PP.KD?end=2021&locations=EU-ZG-XD-XM-1W-IN-US-CD-BI-LU-CN&start=2021&view=bar (PPP 2017 $) / Pop high-income 1.2G, low-income 700M https://data.worldbank.org/indicator/SP.POP.TOTL?locations=XD-XM

% TD change the intro, e.g. Global poverty and climate change are among the most critical issues faced by the world today", and then explain that the first could be solved by transfer, the second by capping pollution, hence an effective policy to tackle these two problems is the GCS. Yet, this policy is nowhere to be seen in policy debates. Why? In this paper we provide evidence from surveys showing that people all over the world support this policy. To explain this paradox (people stated support vs absence of the policy), we further investigate the sincerity of these claims and rationales behind the support, etc.
% Ballard-Rosas et al. 2017 JOP: conjoint Compare different income tax schedules 
% Bechtel and Liesch 2020 POQ: conjoint American people care about the effects of a generic policy on the poorest fellow citizens about half as much as on their income
% Scheve/Stasavage 2022: three explanations for why we do not observe more wealth redistribution

Major sustainability objectives could be achieved by global cooperation policies involving transfers from high- to lower-income countries.\cite{budolfson_climate_2021,franks_mobilizing_2018,dennig_inequality_2015,soergel_combining_2021,bauer_quantification_2020,cramton_global_2017,fehr_your_2022} 
Yet international negotiations have not led %are not conducive 
to ambitious global policies. % Yet, disagreements on burden-sharing and differing priorities among nations often hinder effective collaboration. 
We examine a key condition for achieving sustainability objectives: the support of citizens for globally redistributive policies%, studied only by scant prior attitudinal surveys (8–10)
. 
Recent surveys administered 
%by Dechezleprêtre et al. (2022)\cite{dechezlepretre_fighting_2022} % TODO!back
to over 40,000 respondents from 20 high- and middle-income countries reveal substantial support for those policies, especially global climate policies and a global tax on the wealthiest aimed at financing low-income countries %.\cite{dechezlepretre_fighting_2022}
(other questions from these surveys are analyzed in a companion paper\cite{dechezlepretre_fighting_2022}). 
% Using surveys over 40,000 respondents from 20 high- and middle-income countries, we document substantial support for global policies. 

To gain insights into the factors shaping public support for global policies in high-income countries, we conduct complementary surveys among 8,000 respondents from France, Germany, Spain, the U.S., and the UK. The focus of our approach is a specific policy aimed at addressing both climate change and poverty, referred to as the ``Global Climate Scheme'' (GCS). It implements a cap on carbon emissions to limit global warming below 2\textdegree{}C. The emission rights are auctioned each year to polluting firms and fund a global basic income, alleviating extreme poverty. 
In the wording of the question, respondents are made aware of the cost to themselves of such global redistribution. 
The GCS is supported by three quarters of Europeans and half of Americans. We test whether support of the expressed preference is sincere: a list experiment shows no evidence of social desirability bias in survey responses, majorities are willing to sign a real-stake petition, and global redistribution ranks high in the prioritization of policies. Conjoint analyses reveal that a political platform is more likely to be preferred if it contains the GCS or a global tax on millionaires. 
% In sum, our findings indicate that global redistributive policies are genuinely supported by a majority of the population. 

% Major sustainability objectives could be achieved by global approaches to mitigating climate change and poverty.\cite{budolfson_climate_2021,franks_mobilizing_2018,dennig_inequality_2015,soergel_combining_2021} For example, an equal per capita dividend paid out of 2 degree compatible carbon prices can improve well-being as well as reduce inequality and poverty at a national level.\cite{budolfson_climate_2021} Global carbon pricing is even more redistributive.\cite{bauer_quantification_2020} 
% However, disagreements on burden-sharing, differing priorities, and lack of institutional capacity are commonly seen as obstacles to effective global collaboration on these objectives.\cite{cramton_global_2017} We examine a key condition for the success of global cooperation, neglected in social science research so far: the support of citizens in affluent countries for globally redistributive policies.% which can deliver on poverty reduction and climate change mitigation. This article investigates public attitudes toward such global policies.

% %Recent TODO!
% % Earlier 
% Recent surveys administered 
% %by Dechezleprêtre et al. (2022)\cite{dechezlepretre_fighting_2022} % TODO!back
% to over 40,000 respondents from 20 high- and middle-income countries reveal substantial support for those policies, especially global climate policies and a global tax on the wealthiest aimed at financing low-income countries %.\cite{dechezlepretre_fighting_2022}
% (other questions from these surveys are analyzed in a companion paper\cite{dechezlepretre_fighting_2022}). % TODO!? put back
 
% In particular, a global 2\% tax on individual wealth in excess of \$5 million would effectively reduce poverty as it would, to first order, increase low-income countries' national income by 50\%, if merely 35\% of the revenue were allocated for this purpose.\cite{chancel_world_2022} %\footnote{Figures derived from \cite{chancel_world_2022}, the \href{https://wid.world/world-wealth-tax-simulator/}{WID wealth tax simulator}, and the World Bank.} 
% Surprisingly, even in wealthy nations that would bear a significant burden, majorities of citizens express support for such globally redistributive measures.% I have added the following lines. TODO? put in a footnote?
% % Using the price and emissions trajectories from the report by \cite{stern_report_2017}, %Stern-Stiglitz report,\cite{stern_report_2017} 
% % we estimate that the basic income would amount to \$30 per month for each human above 15 in 2030, enough to lift out of extreme poverty the 700 million people who live with less than PPP \$2 per day. Conversely, assuming a carbon price of \$90/tCO$_\text{2}$ in 2030, high emitters like a typical American (with median U.S. CO$_\text{2}$ emissions) would lose in net \$85 per month, as they would face \$115 per month in price increases (see details in Appendix \ref{app:gain_gcs}). 

% To gain insights into the factors shaping public support for global policies in high-income countries, we conducted complementary surveys among 8,000 respondents from France, Germany, Spain, the U.S., and the UK. The focus of our approach is a specific policy aimed at addressing both climate change and poverty, referred to as the ``Global Climate Scheme'' (GCS). It implements a cap on carbon emissions to limit global warming below 2\textdegree{}C. The emission rights are auctioned each year to polluting firms and fund a global basic income, alleviating extreme poverty. Although the GCS may seem idealistic, we focus on this % TODO? concrete?
% policy as its key features allow us to expose respondents in a concise and simple way with the key trade-off between the costs and benefits of globally redistributive climate policies. %\footnote{Although the GCS may seem idealistic, we focus on this policy as its key features allow us to expose respondents in a concise and simple way with the key trade-off between the costs and benefits of globally redistributive climate policies.} 
% By employing a list experiment, a real-stake petition (with the question results communicated to the heads of state), and conjoint analyses, our study indicates genuine and robust support for the GCS among respondents. For example, the conjoint analyses provide evidence that political parties would not lose vote intention by endorsing the GCS, and that majorities prefer platforms that contain GCS to those that do not.% https://data.worldbank.org/indicator/NY.GDP.MKTP.CD?locations=XD-XM

% % On top of addressing both global poverty and climate change, we provide evidence from surveys showing that people all over the world support this policy. Yet, the GCS is nowhere to be seen in policy debates. Why? To explain this paradox (absence of the policy despite majority stated support), we further investigate rationales behind the support for the GCS and the sincerity of these claims, as well as attitudes toward other global policies, global redistribution, and universalistic values. % TODO: rework

% These findings underscore a strong demand for globally redistributive climate policies, even in the absence of significant policy proposal. In our discussion we offer potential explanations behind this policy implementation gap, indicating that public opinion does not seem to be the reason why they are rarely mentioned in public debates.


% Global cooperation is necessary to solve sustainability challenges such as poverty and climate change. Disagreements on burden-sharing, differing priorities and lack of institutional capacity are obstacles to effective global collaboration. One further obstacle, neglected in social science research so far, is the willingness of citizens in affluent countries to bear the costs of the global redistribution associated with effective poverty reduction and climate change mitigation. This article investigates public attitudes toward global policies delivering on climate change mitigation and poverty reduction.

% To explore relevant public attitudes, we use surveys administered to over 40,000 respondents from 20 high- and middle-income countries covering 72\% of global carbon emissions. The responses reveal substantial support for global policies, including global emissions trading and a global wealth tax aimed at financing low-income countries. Surprisingly, even in wealthy nations that would bear a significant burden, majorities of citizens express support for globally redistributive measures.

% To gain insights into the factors shaping public support in these countries, we conducted complementary surveys among 8,000 respondents from France, Germany, Spain, the U.S., and the UK. The focus is a specific policy aimed at addressing both climate change and extreme poverty, referred to as the ``Global Climate Scheme'' (GCS). It consists of global emissions trading, which limits total carbon emissions. The emission rights are auctioned each year to polluting firms and fund a global basic income. By employing a list experiment, a real-stake petition, and conjoint analysis, our study demonstrates the genuine and robust support for the GCS among respondents. For example, the conjoint analysis provides evidence that policymakers do not lose public support by endorsing the GCS.

% These findings underscore a strong demand for globally redistributive climate policies, even in the absence of significant policy implementation. In our discussion we offer potential explanations behind this policy implementation gap.


% Ethical theories often warrant transfers from high- to low-income people, hence from high- to low-income countries. This is the case of utilitarianism, the benchmark ethical theory used in economics. Utilitarianism assigns the same weight to each person and thus considers that a dollar is better allocated to a low-income person, which has a higher marginal utility than a high-income person.\cite{mill_utilitarianism_1861} 

% Addressing global poverty, inequalities and climate change are at the heart of the universally agreed Sustainable Development Goals (SDG). % 12 out of  17
% It has been pointed out that low-income countries generally do not have enough domestic resources to eliminate the poverty gap in the short run.\cite{bolch_arithmetics_2022} % In other words, it would hardly be possible to achieve the first SDG and end extreme poverty by 2030 without international transfers. => Careful, Bolch use a poverty line above the SDG one.

% Climate change is another issue that calls for a global response and in particular international transfers. Postulating %Assuming
% that each human has an equal right to emit CO$_\text{2}$, low emitters have a legitimate claim \textit{vis-à-vis} high emitters, that can be settled by monetary transfers. Coupling this burden-sharing principle to the carbon budget (remaining emissions that would be compatible with the Paris agreement) naturally defines a global climate policy. We call it the ``Global climate scheme'' (GCS); it consists of a global cap-and-trade system where emission rights are auctioned each year to polluting firms and the revenues finance a global basic income. Using the price and emissions trajectories from the Stern-Stiglitz report,\cite{stern_report_2017} we estimate that the basic income would amount to \$30 per month for each human above 15 in 2030, enough to lift out of extreme poverty the 700 million people who live with less than PPP \$2 per day. Conversely, high emitters like a typical American (with median U.S. CO$_\text{2}$ emissions) would lose in net \$85 per month, as they would face \$115 per month in price increases (assuming a carbon price of \$90/tCO$_\text{2}$ in 2030). % TD Give the numbers in the Results section
% % G default policy for economists, we focus on it; transfers at heart of COP; global wealth tax proposed by Piketty, Saez (fair and effective); democratisation of int'l institutions recurring topic.
% % Few studies on CC burden-sharing, all compatible with G
% % Few studies on global policies, but they show support (Ghassim, Carattini 19)
% % Here, two sets of results. First, twenty countries. Second, dig deeper using complementary survey.

% If high emitters share universalistic ethical values, we expect strong support for the GCS, even in high-income countries. On the contrary, if people defend their own financial interest, we expect low support for the GCS in high-income countries. 

% In this paper, we study attitudes toward global policies that address climate change, global poverty or inequalities, with a focus on the GCS. We measure stated support for different global policies using unpublished results from a survey\cite{dechezlepretre_fighting_2022} on climate attitudes conducted in 2021 on 40,680 respondents from 20 countries covering 72\% of global CO$_\text{2}$ emissions. We then conduct a representative survey on 3,000 U.S. respondents to study in detail the sincerity and rationales behind the support for the GCS, the attitudes toward various global policies, global redistribution, and universalistic values.

\paragraph{Literature} A wealth of studies have examined public support for national carbon pricing policies.\cite{kotchen_public_2017,klenert_making_2018,douenne_yellow_2022,dechezlepretre_fighting_2022} %  or willingness to contribute to climate action.\cite{andre_actual_2024} % TODO!
Yet, few prior attitudinal surveys have examined policies for global redistribution. They find agreement close to 50\% in high-income countries for global carbon taxes with international per capita redistribution;\cite{carattini_how_2019} %  (but do not examine an emissions cap nor the sincerity of support) shortened
and near consensus that ``present economic differences between rich and poor countries are too large'' (overall, 78\% agree and 5\% disagree) % TODO? remove?
in each of 29 countries.\cite{issp_international_2019} Furthermore, correcting misperceptions concerning one's position in the world's income distribution does not affect the support for global redistributive policies.\cite{fehr_your_2022} Besides, an international study of the support for global democracy finds that, in countries governed by a coalition, voting shares would shift by 8 (Brazil) to 12 p.p. (Germany) % about 10 p.p. TODO? remove?
from parties that are said to oppose global democracy to parties that supposedly support it.\cite{ghassim_public_2022} Supplementary Section A summarises attitudinal surveys on global policies; prior work on attitudes toward climate burden sharing, attitudes toward foreign aid; global carbon pricing, global redistribution, basic income, and global democracy.
%Appendix \ref{sec:literature} contains a broader literature review including further attitudinal surveys on global policies (\ref{subsubsec:literature_attitudes_policies}); prior work on attitudes toward climate burden sharing (Appendix \ref{subsubsec:literature_attitudes_burden_sharing}), attitudes toward foreign aid (Appendix \ref{subsubsec:literature_foreign_aid}); global carbon pricing (Appendix \ref{subsubsec:literature_pricing}), global redistribution (Appendix \ref{subsubsec:literature_redistribution}), basic income (Appendix \ref{subsubsec:literature_basic_income}), and global democracy (Appendix \ref{subsubsec:literature_democracy}).


% \paragraph{Literature.} The literature review is relegated to Appendix \ref{sec:literature}. It includes references to the few other attitudinal surveys on global policies (e.g. \cite{carattini_how_2019,issp_international_2019,ghassim_public_2022}, see Appendix \ref{subsubsec:literature_attitudes_policies}); a critical review of the literature on attitudes toward climate burden sharing (Appendix \ref{subsubsec:literature_attitudes_burden_sharing}); references to the large literature on attitudes toward foreign aid (Appendix \ref{subsubsec:literature_foreign_aid}); and introduction to the literatures on global carbon pricing (Appendix \ref{subsubsec:literature_pricing}), global redistribution (Appendix \ref{subsubsec:literature_redistribution}), basic income (Appendix \ref{subsubsec:literature_basic_income}), and global democracy (Appendix \ref{subsubsec:literature_democracy}). 


\section*{Results}
% % 4 most important figures: heatmap OECD, heatmap support, prioritization or conjoint (r), list exp (table)
% The presentation of results proceeds as follows: after briefly describing the survey data (\ref{subsec:data}), we first document broad international support for global approaches to climate policy that lead to global redistribution (\ref{subsubsec:global_support}). Subsequently, we present specific findings from surveys in the U.S. and Europe that document support for the GCS, wealth taxes, and foreign aid in those countries (\ref{subsubsec:support_gcs}-\ref{subsubsec:support_foreign_aid}). We proceed to study the support for the Global Climate Scheme in more detail, by means of a list experiment, petition, conjoint analyses, prioritization task, and by eliciting pros and cons (\ref{subsec:robustness_sincerity}). To understand the gap between support for global policies and their appearance in public discussion, we conclude by reporting results on underlying universalistic values (\ref{subsec:universalistic}) and beliefs about the support of others (\ref{subsec:second_order_beliefs}). 



\subsection*{Data}\label{subsec:data}


% Stated support for different global policies has been measured in % TD better way to sell these results?
% a survey on climate attitudes conducted in 2021 on 40,680 respondents from 20 countries covering 72\% of global CO$_\text{2}$ emissions (\cite{dechezlepretre_fighting_2022}, which focuses on questions related to national policies). %(the questions of this survey on national policies are analysed in another paper: \cite{dechezlepretre_fighting_2022}). 
% We conduct complementary surveys in the U.S. and four European countries to study in detail the sincerity and rationales behind the support for the GCS, the attitudes toward various global policies, global redistribution, and universalistic values. The U.S. survey has been divided in two waves: \textit{US1} and \textit{US2}, with respectively 3,000 and 2,000 respondents. The European survey, called \textit{Eu}, combines the two U.S. waves (just without one question of US2 that uses results from US1). The Eu survey comprises 3,000 respondents representative of France, Germany, Spain and the UK, along the dimensions of gender, income, age, highest diploma, country, and degree of urbanisation. The U.S. samples are representative along the same dimensions (with \textit{region} instead of \textit{country}) as well as along ethnicity. Tables \ref{tab:representativeness_waves}-\ref{tab:representativeness_EU} confirm that our samples closely match population frequencies. The questionnaires are given in Appendices \ref{app:questionnaire_oecd} and \ref{app:questionnaire}.

The study relies on two sets of representative surveys. 
% The study relies on a figure from an earlier survey and on new representative surveys. % TODO? put back?
The figure on global policies originates from a % The %figure from the 
\textit{Global} survey, 
conducted for % is tied to %relies on 
another paper that focuses on attitudes toward climate change and national climate policies.\cite{dechezlepretre_fighting_2022} % TODO!? put back
The \textit{Global} survey was conducted in 2021--2022 on 40,680 respondents from 20 countries covering 72\% of global CO$_\text{2}$ emissions. 
We conducted 
\textit{Complementary} surveys in the U.S. (\textit{US1}: N=3,000, \textit{US2}: N=2,000) and four European countries (\textit{Eu}: N=3,000) %were conducted 
in 2023 (See \nameref{sec:methods} for details on data collection and data quality).% (see Table \ref{tab:survey_summary}).

% \begin{table}[h!]
% \renewcommand{\arraystretch}{1.5}
% \caption[Surveys summary]{Summary of the surveys used in the analysis.}
% \label{tab:survey_summary}
% \centering
% \footnotesize
% \begin{tabular}{ |p{2.5cm}|p{3.5cm}|p{2.5cm}|p{2.5cm}|p{3.5cm}|  }
% \cline{2-5}
% \multicolumn{1}{c|}{} & \textbf{Global survey} & \multicolumn{3}{|c|}{\textbf{Complementary surveys}} \\
% \hline
% \textbf{\textit{Name}} & \textit{Global} & \textit{US1} & \textit{US2} & \textit{Eu} \\
% \hline
% \textbf{Region} & 20 high and middle-income countries & U.S. & U.S. & France, Germany, Spain, UK \\
% \hline
% \textbf{Sample size} & 40,680 & 3,000 & 2,000 & 3,000 \\
% \hline
% \textbf{Main purpose} & Stated support for global policies & \multicolumn{3}{|p{8.5cm}|}{Focus on GCS (sincerity, rationales, etc.) + Support for global redistribution + Universalistic values} \\
% % Median duration (in min) & 28 & 20 & 14 & 11 \\ 
% % Date of administration & 03/21--03/22 & 02--03/23 & 01--03/23 & 03--04/23 \\ 
% \hline
% \end{tabular}
% \end{table}

% \begin{table}%[h] % NCCcomment [h]
%   \caption[Surveys summary]{Summary of the surveys used in the analysis.}
%   % \caption[Surveys summary]{Characteristics of the different surveys.} 
%   \label{tab:survey_summary}
%   \centering
% \begin{tabular}
%   {@{\extracolsep{5pt}}lcccc} 
%   \\[-1.8ex]\hline 
%   \hline \\[-1.8ex] 
%    & \textit{Global survey} & \multicolumn{3}{c}{\textit{Complementary surveys}} \\
%   \\[-1.8ex] Survey & \textit{Global} & \textit{Eu} & \textit{US1} & \textit{US2} \\ 
%   \hline \\[-1.8ex]   
%   Country coverage & 20 countries & FR, DE, ES, UK & U.S. & U.S. \\ 
%   Sample size & 40,680 & 3,000 & 3,000 & 2,000 \\ 
%   Main purpose & \makecell{Stated support \\for global policies} & \multicolumn{3}{c}{\makecell{Focus on GCS (sincerity, rationales, etc.) \\+ Support for global redistribution \\+ Universalistic values}} \\
%   % Median duration (in min) & 28 & 20 & 14 & 11 \\ 
%   % Date of administration & 03/21--03/22 & 02--03/23 & 01--03/23 & 03--04/23 \\ 
%   \hline 
%   \hline \\[-1.8ex] 
% \end{tabular}
% \end{table}

% \paragraph{Global Survey}

% The \textit{Global} survey, conducted in 2021, involved 40,680 respondents from 20 countries, representing approximately 72\% of global CO$_\text{2}$ emissions. This survey serves as the basis for measuring stated support for various global policies worldwide. Detailed information about the data collection process, sample representativeness, and analysis of questions on national policies can be found in our companion paper.\cite{dechezlepretre_fighting_2022}

% \paragraph{Complementary Surveys}\label{par:surveys}

% To delve deeper into the sincerity and rationales behind support for the GCS and attitudes toward global policies, global redistribution, and universalistic values, complementary surveys were conducted in 2023. These surveys are based on a sample of 8,000 respondents from France, Germany, Spain, the UK, and the U.S. The European survey (\textit{Eu}) comprises 3,000 respondents, while the U.S. sample was collected in two separate waves: \textit{US1} with 3,000 respondents and \textit{US2} with 2,000 respondents. The survey questions in both the European and U.S. surveys are identical, except for an additional question in \textit{US2} that uses results from \textit{US1} to assess the bandwagon effect.

% The complementary surveys ensured representativeness along key dimensions such as gender, income, age, highest diploma, and degree of urbanization. The \textit{Eu} survey is also representative of its four countries in terms of population size, while the \textit{US1} and \textit{US2} surveys are representative in terms of region and ethnicity. Supplementary Section G confirm that our samples closely match population frequencies. More detail on data collection is given in the Methods. %Section \nameref{sec:methods}. 
% The questionnaires used in the surveys are provided in Supplementary Sections C and D.
% % Details regarding the representativeness of the samples are provided in Tables \ref{tab:representativeness_waves}-\ref{tab:representativeness_EU}.



\subsection*{Stated support for global policies}\label{subsec:stated_support}

% The results from both the \textit{Global} and \textit{Complementary} surveys demonstrate robust support for global policies across all surveyed countries, with a clear preference for addressing climate action on a global scale over regional, national, or local approaches.

% \subsubsection{Global support}\label{subsubsec:global_support} % NCCcomment
\paragraph{Global support}\label{subsubsec:global_support} % WPcomment

The Global survey shows strong support for climate policies enacted at the global level (Figure \ref{fig:oecd}%, reproduced from Dechezleprêtre et al. (2022)\cite{dechezlepretre_fighting_2022}
). % TODO!back
When asked ``At which level(s) do you think public policies to tackle climate change need to be put in place?'', 70\% (U.S.) to 94\% (Japan) choose the global level. 
%The next most popular choice is the federal or continental level, favored by 52\% of U.S. and less than half of European respondents. Local policies receive the least support. % shortened
% This preference for climate policies implemented at the global scale is in line with the literature\cite{beiser-mcgrath_could_2019} and consistent with individuals' concerns for the fairness and effectiveness of such policies, which have been identified as two of the three key determinants of support, besides self-interest.\cite{klenert_making_2018,douenne_yellow_2022,dechezlepretre_fighting_2022} % TODO? remove? 43 w


\begin{figure}[h!]
  % MAJOR figure
  \caption[Relative support for global climate policies]{Relative support for global climate policies (Reproduced from Dechezleprêtre et al. (2022),\cite{dechezlepretre_fighting_2022} Figure A21.).} % TODO!? put back
  \makebox[\textwidth][c]{\includegraphics[width=1.2\textwidth]
  {../figures/OECD/Heatplot_global_tax_attitudes_share.pdf}}\label{fig:oecd} % with dependence on others (absent from OECD): Heatplot_burden_share_all_share_countries
  {\footnotesize %\\ $\quad$ \\ 
  Note 1: The numbers represent the share of \textit{Somewhat} or \textit{Strongly support} among non-\textit{indifferent} answers (in percent, $n$ = 40,680). The color blue denotes a relative majority. See Supplementary Figure A3 for the absolute support. (Questions A-I in Supplementary Section C).
  %See Figure \ref{fig:oecd_absolute} for the absolute support. (Questions \ref{q:scale}-\ref{q:millionaire_tax}). %Reproduced from \cite{dechezlepretre_fighting_2022}, Figure A21.) % TODO!? put back?
  \\ Note 2: *In Denmark, France and the U.S., the questions with an asterisk were asked differently, cf. Question F in Supplementary Section C.% \ref{q:burden_sharing_asterisk}. 
  } 
\end{figure}

Among the four global climate policies examined in the \textit{Global} survey, three policies garner high support across all countries (Figure \ref{fig:oecd}): a global democratic assembly on climate change, a global tax on millionaires to finance low-income countries contingent on their climate action, and a global carbon budget of +2\textdegree{}C divided among countries based on tradable shares (or ``global quota''), with the allocation of country shares unspecified. %\footnote{The policies were all described with further details to make sure people understood them. Specifically, the policies were presented as follows: an international emissions trading system where ``countries that emit more than their national share would pay a fee to countries that emit less than their share''; ``a tax on all millionaires in dollars around the world to finance low-income countries that comply with international standards regarding climate action [which] would finance infrastructure and public services such as access to drinking water, healthcare, and education''; ``a global democratic assembly whose role would be to draft international treaties against climate change [where] each adult across the world would have one vote to elect members of the assembly''.} 
The three policies garner a majority of absolute support (i.e., ``somewhat'' or ``strong'' support) in all countries (except in the U.S. for the global assembly, 48\% absolute support). In high-income countries, the global quota obtains 64\% absolute support and 84\% relative support (i.e., excluding ``indifferent'' answers). 
% Support for this policy is even higher in middle-income countries, however their samples are only representative of the online population (young, graduated and urban people are over-represented).%due to the over-representation of young, educated, and urban populations in the online sample. % TODO? remove? 27 w
% done Several global policies obtain an absolute majority (i.e. \textit{somewhat} or \textit{strong}) %more than 70\% relative % support in all countries

Following the support for the global quota, respondents are asked about their preferences for dividing the carbon budget among countries (see third block of Figure \ref{fig:oecd}). Consistent with the existing literature (see Supplementary Section A.1.2), an equal per capita allocation of emission rights emerges as the preferred burden-sharing principle, garnering absolute majority support in all countries and never below 84\% relative support. Taking into account historical responsibilities or vulnerability to climate damages is also popular, albeit with less consensus, while grandfathering (i.e., allocation of emission shares in proportion to current emissions) receives the least support in all countries.

A global quota with equal per capita emission rights should produce the same distributional outcomes as a global carbon tax that funds a global basic income. %\footnote{Similarly,  a global quota with grandfathering is equivalent to a global carbon tax where each country keeps the revenues it collects.} 
The support for the global carbon tax is also tested and its redistributive effects --  the average increase in expenditures along with the amount of the basic income -- are specified to the respondents explicitly. %: the \$30 per month basic income would lift the 700 million people who earn less than \$2/day out of extreme poverty, and fossil price increases would cost the typical person in their country a certain amount (that is provided).  % The average British person would lose a bit from this policy as they would face £42 per month in price increases, which is higher that the £22 they would receive.
The support for the carbon tax is lower than for the quota, particularly in high-income countries, and there is no relative majority for the tax in Anglo-Saxon countries. %\footnote{The levels of support are consistent with the findings of \cite{carattini_how_2019}, the only previous study that tested a global carbon tax.} 
Two possible reasons %for this lower support 
are that distributive effects are made salient in the case of the tax, and that citizens may prefer a quota, perhaps because they find it more effective than a tax to reduce emissions. This interpretation is consistent with the level of support for the global quota once we make the distributive effects salient, as we do in the complementary surveys.

% One possible reason for this discrepancy is the explicit emphasis on the redistributive effects associated with the tax in the survey. The survey informs respondents that the \$30 per month basic income would lift 700 million people earning less than \$2/day out of extreme poverty, while fossil price increases would impose costs (specified in the survey) on the typical person in their country. Although other factors such as perceptions of effectiveness may also influence support for a quota versus a tax, this interpretation aligns with the level of support for the global quota when distributive effects are made salient, as we do in the complementary surveys. 

% The remainder of the paper analyzes the results from the complementary surveys in the U.S. and in Europe. This Section covers the stated support for different global redistributive policies. %the Global Climate Scheme, a global wealth tax, other global policies, and foreign aid.

%\subsubsection{Complementary surveys}

% \subsubsection{Global Climate Scheme}\label{subsubsec:support_gcs} % NCCcomment
\paragraph{Global Climate Scheme}\label{subsubsec:support_gcs} % WPcomment

The complementary surveys (\textit{US1}, \textit{US2}, \textit{Eu}) consist of a comprehensive exploration of citizens' attitudes toward the GCS. We present to respondents a detailed description of the GCS and explain its distributive effects, including specific amounts at stake (see box below and Supplementary Section D). 
Furthermore, we assess respondents' understanding of the GCS with incentivized questions to test their comprehension of the expected financial outcome for typical individuals in their country (loss) and the poorest individuals globally (gain), followed by the provision of correct answer. %63\% understand the policy will make people in their country poorer, before even we indicate this as the right answer. 
That is, respondents are made aware and we find they understand that the policy will make people in their country poorer. 
The same approach is applied to a National Redistribution scheme (NR) targeting the top 5\% (in the U.S.) or top 1\% (in Europe) with the aim of financing cash transfers to all adults, %\footnote{The wider base in the U.S. was chosen because emissions are larger in the U.S. than in Europe, and it would hardly be feasible to offset the median American's loss by taxing only the top 1\%.} 
calibrated to offset the monetary loss of the GCS for the median emitter in their country. We evaluate respondents' understanding that the richest would lose and the typical fellow citizens would gain from that policy. % done We proceed the same way for a National Redistribution Scheme (NR) that would tax the top 5\% (in the U.S.) or the top 1\% (in Europe) to finance cash transfers to all adults (calibrated to offset the monetary loss of the GCS for the median emitter), expecting people to find out at the comprehension question that the richest would lose and the typical people in their country would win.
Subsequently, we summarize both schemes to enhance respondents' recall. Additionally, we present a final incentivized comprehension question and provide the expected answer that the combined GCS and NR would result in no net gain or loss for a typical fellow citizen. Finally, respondents are directly asked to express their support for the GCS and NR using a simple \textit{Yes}/\textit{No} question. % TODO? remove? shorten? 169 w


The stated support for the GCS is 54\% in the U.S. and 76\% in Europe (Figure \ref{fig:support}), %\footnote{The 95\% confidence intervals are $[52.4\%, 55.9\%]$ in the U.S. and $[74.2\%, 77.2\%]$ in Europe. The average support is computed with survey weights, employing weights based on quota variables, which exclude vote. Another method to reweigh the raw results involves running a regression of the support for the GCS on sociodemographic characteristics (including vote) and multiplying each coefficient by the population frequencies. This alternative approach yields similar figures: 76\% in Europe and 52\% or 53\% in the U.S. (depending on whether individuals who did not disclose their vote are classified as non-voters or excluded). Notably, the average support excluding non-voters is 54\% in the U.S.} 
while the support for NR is very similar: 56\% and 73\% respectively (Supplementary Figure 3). Supplementary Section F presents the sociodemographic determinants of GCS support, showing, for instance, stronger support among young people.

\begin{tcolorbox}\label{box:GCS}
  \paragraph{The Global Climate Scheme} The GCS consists of global emissions trading with emission rights being auctioned each year to polluting firms, and of a global basic income, funded by the auction revenues. Using the price and emissions trajectories from the Stern-Stiglitz report,\cite{stern_report_2017} and in particular a carbon price of \$90/tCO$_\text{2}$ in 2030, we estimate that the basic income would amount to \$30 per month for every human over the age of 15 (see details in Supplementary Section E). %, enough to lift out of extreme poverty the 700 million people who live with less than PPP \$2 per day. Conversely, assuming a carbon price of \$90/tCO$_\text{2}$ in 2030, high emitters like a typical American (with median U.S. CO$_\text{2}$ emissions) would lose in net \$85 per month, as they would face \$115 per month in price increases (see details in Appendix \ref{app:gain_gcs}). 
  We describe the GCS to the respondents as a climate club and we specify its redistributive effects: The 700 million people with less than \$2/day [in Purchasing Power Parity] % PPP TODO? remove?
  would be lifted out of extreme poverty, and fossil fuel price increases would cost the typical person in their country a specified amount (see Supplementary Section D). The monthly median net cost is \$85 in the U.S., \euro{}10 in France, \euro{}25 in Germany, \euro{}5 in Spain, £20 in the UK.
\end{tcolorbox}

% \begin{figure}[h!]
%     \caption[Support for the Global Climate Scheme]{Support for the GCS, NR and the combination of GCS, NR and C. \\(Questions \ref{q:global_tax}, \ref{q:national_tax}, \ref{q:gcs_support}, \ref{q:nr_support} and \ref{q:crg_support}).%; $n_\text{US} = n_\text{Eu} = 3,000,\, n_\text{FR} = 729,\, n_\text{DE} = 929,\, n_\text{ES} = 543,\, n_\text{UK} = 749$)
%     }\label{fig:support_binary}
%     \makebox[\textwidth][c]{\includegraphics[width=.9\textwidth]{../figures/country_comparison/support_binary_positive.pdf}} 
% \end{figure} % TODO!

% TODO?: add note to the figure above, to explain what is the national climate policy.


% \subsubsection{Global wealth tax}\label{subsubsec:support_global_wealth_tax} % NCCcomment
\paragraph{Global wealth tax}\label{subsubsec:support_global_wealth_tax} % WPcomment

Consistent with the results of the global survey, a ``tax on millionaires of all countries to finance low-income countries'' garners absolute majority support of over 67\% in each country, only 5 p.p. lower than a national millionaires tax overall (Figure \ref{fig:support}). In random subsamples, we inquire about respondents' preferences regarding the redistribution of revenues from a global tax on individual wealth exceeding \$5 million, after providing information on the revenue raised by such a tax in their country compared to low-income countries. %\footnote{A 2\% tax on net wealth exceeding \$5 million would annually raise \$816 billion, leaving unaffected 99.9\% of the world population. More specifically, it would collect \euro{}5 billion in Spain, \euro{}16 billion in France, £20 billion in the UK, \euro{}44 billion in Germany, \$430 billion in the U.S., and \$1 billion collectively in all low-income countries (28 countries, home to 700 million people).%Figures come from \cite{chancel_world_2022}, the \href{https://wid.world/world-wealth-tax-simulator/}{WID wealth tax simulator}, and the World Bank. % TODO: do they?
% } 
We ask certain respondents ($n$ = 1,283) what percentage of global tax revenues should be pooled to finance low-income countries. In each country, at least 88\% of respondents indicate a positive amount, with an average ranging from 30\% (Germany) to 36\% (U.S., France) (Supplementary Figure 4). To other respondents ($n$ = 1,233), we inquire whether they would prefer each country to retain all the revenues it collects or that half of the revenues be pooled to finance low-income countries. Approximately half of the respondents opt to allocate half of the tax revenues to low-income countries.


% \begin{figure}
%     % \centering 
%     \caption[Preferred share of wealth tax for low-income countries]{Percent of global wealth tax that should finance low-income countries (\textit{mean}). (Question \ref{q:global_tax_global_share})} % TODO? n
%     \includegraphics[width=1\textwidth]{../figures/country_comparison/global_tax_global_share_mean.pdf} \label{fig:global_share_mean}
% \end{figure}


% \subsubsection{Other global policies}\label{subsubsec:support_other_global_policies} % NCCcomment
\paragraph{Other global policies}\label{subsubsec:support_other_global_policies} % WPcomment

Other global redistributive policies garner majority support across all countries (Figure \ref{fig:support}).
% We also assess support for other global policies (Figure \ref{fig:support}). Most policies garner relative majority support in each country, with two exceptions: the ``cancellation of low-income countries' public debt'' and ``a maximum wealth limit'' for each individual. The latter policy obtains relative majority support in Europe but not in the U.S., despite the cap being set at \$10 billion in the U.S. compared to \euro{}/£100 million in Europe. Notably, climate-related policies enjoy significant popularity, with ``high-income countries funding renewable energy in low-income countries'' receiving absolute majority support across all surveyed countries. Additionally, relative support for loss and damages compensation, as approved in principle at the international climate negotiations in 2022 (``COP27''), ranges from 55\% (U.S.) to 81\% (Spain), with absolute support ranging from 41\% to 62\%.

\begin{figure}[h!]
  % MAJOR figure
  \caption[Relative support for further global policies]{Relative support for various global policies. %(percentage of \textit{somewhat} or \textit{strong support}, after excluding \textit{indifferent} answers). 
  (p. 58, Questions 20, 44 and 45 in Supplementary Section D; See Figure A25 for the absolute support.)% $n_\text{US} = n_\text{Eu} = 3,000,\, n_\text{FR} = 729,\, n_\text{DE} = 929,\, n_\text{ES} = 543,\, n_\text{UK} = 749, n_\text{US, global/national wealth tax} = 2,000$
  }
  \makebox[\textwidth][c]{\includegraphics[width=\textwidth]{../figures/country_comparison/support_likert_gcs_share.pdf}}\label{fig:support}
  {\footnotesize Note: The numbers represent the percentage of \textit{somewhat} or \textit{strong support}, after excluding \textit{indifferent} answers. \\
  *Except for the GCS: Share of ``Yes'' in a simple \textit{Yes/No} question. } 
\end{figure} 


% \subsubsection{Foreign aid}\label{subsubsec:support_foreign_aid} % NCCcomment
\paragraph{Foreign aid}\label{subsubsec:support_foreign_aid} % WPcomment
% 395 w

We provide respondents with information about the actual amount ``spent on foreign aid to reduce poverty in low-income countries'' relative to their country's government spending and GDP. Less than 16\% of respondents state that their country's foreign aid should be reduced, while 62\% express support for increasing it, including 17\% who support an unconditional increase (Supplementary Figure 5). Among the 45\% who think aid should be increased under certain conditions, we subsequently ask them to specify the conditions they deem necessary (Supplementary Figure 6). The three most commonly selected conditions are: ``we can be sure the aid reaches people in need and money is not diverted'' (73\% chose this condition), ``that recipient countries comply with climate targets and human rights'' (67\%), and ``that other high-income countries also increase their foreign aid'' (48\%). %\footnote{It is worth noting that these conditions align closely with the principles of the GCS.} 
On the other hand, respondents who do not wish to increase their country's foreign aid primarily justify their view by prioritizing the well-being of their fellow citizens or by perceiving each country as responsible for its own fate (Supplementary Figure 7). In response to an open-ended question regarding measures high-income countries should take to fight extreme poverty, a large majority of Americans expressed that more help is needed (Supplementary Figure A38). 
The most commonly suggested form of aid is financial support, closely followed by investments in education. % TOD? remove? 81 w


We also inquire about the perceived amount of foreign aid. Consistent with prior research (see Supplementary Section A.1.3), most people overestimate the actual amount of foreign aid (Supplementary Figures A17, A19). 
We then elicit respondents' preferred amount of foreign aid, after randomly presenting them with either the actual amount or no information. Most of the respondents choose a bracket at least as high as the actual amount (when the receive the information) or the perceived one (when they do not), %who learn the actual amount choose a bracket at least as high as the actual one, and most of those without the information choose a bracket at least as high as the perceived one 
see Supplementary Figures A17--A21. 
Finally, we ask a last question to the respondents who received the information. To those who prefer an increase of foreign aid, we ask how they would finance it and find that the preferred source of funding is overwhelmingly higher taxes on the wealthiest (Supplementary Figure A22). 
To those who prefer a reduction, we ask how they would use the funds becoming available: %resulting savings: 
In every country, more people choose higher spending on education or healthcare rather than lower taxes (Supplementary Figure A23). % TODO? remove?



% \begin{figure}[h!]
%   \caption[Attitudes on the evolution of foreign aid]{Attitudes regarding the evolution of [own country] foreign aid. (Question \ref{q:foreign_aid_raise_support})}\label{fig:foreign_aid_raise_support}
%   \makebox[\textwidth][c]{\includegraphics[width=\textwidth]{../figures/country_comparison/foreign_aid_raise_support.pdf}} 
% \end{figure}

% \begin{figure}[h!]
%   \caption[Conditions at which foreign aid should be increased]{Conditions at which foreign aid should be increased (in percent). [Asked to those who wish an increase of foreign aid at some conditions.] (Question \ref{q:foreign_aid_condition})}\label{fig:foreign_aid_condition}
%   \makebox[\textwidth][c]{\includegraphics[width=\textwidth]{../figures/country_comparison/foreign_aid_condition_positive.pdf}} 
% \end{figure}

% \begin{figure}[h!]
%   \caption[Reasons why foreign aid should not be increased]{Reasons why foreign aid should not be increased (in percent). [Asked to those who wish a decrease or stability of foreign aid.] (Question \ref{q:foreign_aid_no})}\label{fig:foreign_aid_no}
%   \makebox[\textwidth][c]{\includegraphics[width=\textwidth]{../figures/country_comparison/foreign_aid_no_positive.pdf}} 
% \end{figure}

\subsection*{Robustness and sincerity of support for the GCS}\label{subsec:robustness_sincerity} % TODO!? Reliability of the stated support?

We use several methods to assess the sincerity of the support for the GCS. %: a list experiment, a real-stake petition, conjoint analyses, and the prioritization of policies.  % shortened
All methods suggest that the support is either completely sincere, or the share of insincere answers is limited. % TODO? remove? 45 w

% \subsubsection{List experiment}\label{subsubsec:list_exp} % NCCcomment
\paragraph{List experiment}\label{subsubsec:list_exp} % WPcomment

% By asking how many policies within a list respondents support, and adding for some respondents the GCS in the list, we identify the tacit support for that policy. It is not significantly different from the stated support (table S1). Hence, we do not find a social desirability bias: people faithfully report their opinion on the GCS.  % shortened

% A list experiment is employed to gauge tacit support for a specific policy of interest by asking respondents how many policies within a given list they support. By varying the list among respondents, the difference in number of policies supported can be used to estimate tacit support, revealing potential social desirability biases \cite{hainmueller_causal_2014}.
We use a list experiment to identify the tacit support for the GCS. To do so, we ask \textit{how many} policies within a list respondents support, and vary the list among respondents. The tacit support is estimated as the difference in the average number of policies supported between two random subsamples, whose list differ only by the inclusion of the GCS.\cite{hainmueller_causal_2014} %\footnote{For example, say a first subsample faces the list of policies A, B, and C, while a second subsamples faces the list A, B, C, and GCS. We do not need to know which policies each respondent support to estimate the average (tacit) support for the GCS, we simply need to compute the difference in the average number of supported policies between the two random subsamples.} % respondents who face a list containing the policy, and respondents who face the same list without it. 
% List experiments have been used to reveal social desirability bias, which silences racism in the Southern U.S.\cite{kuklinski_racial_1997} or opposition to the invasion of Ukraine in Russia.\cite{chapkovski_solid_2022} % TODO? remove? 26 w
In our case, as shown in Table \ref{tab:list_exp}, the tacit support for the GCS measured through the list experiment is not significantly lower than the direct stated support. %\footnote{We utilize the difference-in-means estimator, and confidence intervals are computed using Monte Carlo simulation with the R package \textit{list} \cite{imai_multivariate_2011}.} 
Hence, we do not find a social desirability bias in our study.

% The tacit support for the GCS measured through the list experiment is as high as  the direct question in Eu but significantly lower by 5 p.p. in the U.S. This may be the sign of a social norm pushing some Americans to state that they support the GCS although they secretly do not. Still, if there is a social norm in favor of the GCS, there is a similar norm in favor of the National Redistribution Scheme, as the gap between the tacit and direct support for it is comparable (at 6 p.p.). %However, two observations qualify this interpretation. First, the gap between the tacit and direct support for the National Redistribution Scheme is comparable (at 7 p.p.) though we did not expect such a social norm in the case of the national redistribution, as the 95\% who would benefit from it should not feel ashamed to oppose a policy that would benefit them. Second, while we tested the questionnaire on random people in cafés, we noticed that some were confused by the question of the list experiment (asking how many policies from the list they supported), upset with the conservative societal policy (``Marriage only for opposite-sex couples in the U.S.'', ``Death penalty for major crimes'' in Europe), to the point that they did not answer attentively.

\begin{table} % NCCcomment [h]
  % MAJOR figure % TODO! same table for NR in appendix
  % TODO table by country
  \caption[List experiment: tacit support for the GCS]{Number of supported policies in the list experiment depending on the presence of the Global Climate Scheme (GCS) in the list. The tacit support for the GCS is estimated by regressing the number of supported policies on the presence of the GCS in the list of policies. The social desirability is estimated as the difference between the tacit and stated support, and it is not significantly different from zero even at a 20\% threshold (see \nameref{sec:methods}). 
  %Confidence intervals are computed using Monte Carlo simulation with the R package \textit{list}.\cite{imai_multivariate_2011} %in function of the composition of the list. GCS stands for the Global Climate Scheme and NR for the National Redistribution Scheme.} % Beware, this question is quite unusual. \\ Among the policies below, how many do you support?  \\ Coal exit, Marriage only for opposite-sex couples 
  }\label{tab:list_exp}
  \makebox[\textwidth][c]{
\begin{tabular}{@{\extracolsep{5pt}}lccc} 
\\[-1.8ex]\hline 
\hline \\[-1.8ex] 
 & \multicolumn{3}{c}{Number of supported policies} \\ 
\cline{2-4} 
\\[-1.8ex] & All & US & Eu \\ 
\hline \\[-1.8ex] 
 List contains: GCS & 0.624$^{***}$ & 0.524$^{***}$ & 0.724$^{***}$ \\ 
  & (0.028) & (0.041) & (0.036) \\ 
\hline  \\[-1.8ex] \textit{Support for GCS} & 0.617  &  0.542  &  0.757 \\
\textit{Social desirability bias} & \textit{$ -0.026 $} & \textit{$ -0.018 $} & \textit{$ -0.033 $}\\
\textit{80\% C.I. for the bias} & \textit{ $[ -0.06 ; 0.01 ]$ } & \textit{ $[ -0.07 ; 0.01 ]$} & \textit{ $[ -0.08 ; 0.01 ]$}\\
 \hline \\[-1.8ex] 
Constant & 1.317 & 1.147 & 1.486 \\ 
Observations & 6,000 & 3,000 & 3,000 \\ 
R$^{2}$ & 0.089 & 0.065 & 0.125 \\ 
\hline 
\hline \\[-1.8ex] 
\textit{Note:}  & \multicolumn{3}{r}{$^{*}$p$<$0.1; $^{**}$p$<$0.05; $^{***}$p$<$0.01} \\ 
\end{tabular} 
  }  
  % {\footnotesize \textit{Note:} $^{*}p<0.1$; $^{**} p<0.05$; $^{***} p<0.01$.}
\end{table}

% Donation addresses experimenter demand
% \subsubsection{Petition}\label{subsubsec:petition} % Addresses hypothetical bias  % NCCcomment
\paragraph{Petition}\label{subsubsec:petition}% WPcomment

% H1: Petition: Small effect against GCS: -4pp
% We ask respondents whether they are willing to sign a petition in support of either the GCS or NR policy. We inform them that the petition results will be sent to the head of state's office, highlighting the proportion of fellow citizens endorsing the respective scheme. Even when framed as a real-stake petition, both policies continue to receive majority support.  % shortened
In a real-stake question, we ask respondents whether they are willing to sign a petition in support of either the GCS or NR policy, informing them that the results of that question will be sent to the head of state's office.
In the U.S., we find no significant difference between the support in the real-stake petitions and the simple questions (GCS: $p=.30$; NR: $p=.76$). %\footnote{Paired weighted \textit{t}-tests are conducted to test the equality in support for a policy among respondents who were questioned about the policy in the petition.} 
In Europe, the petition leads to a comparable lower support for both the GCS (7 p.p., $p=10^{-5}$) and NR (4 p.p., $p = .008$), but the support remains strong, at 69\% for the GCS. 
% While some European respondents are unwilling to sign a petition for policies they are expected to support, this effect is not specific to the GCS, and the overall willingness to sign a real-stake petition remains strong, with 69\% expressing support for the GCS and 67\% for NR.

% \subsubsection{Conjoint analyses}\label{subsubsec:conjoint} % Addresses acquiescence bias  % NCCcomment
\paragraph{Conjoint analyses}\label{subsubsec:conjoint}% WPcomment
% H1, H2: Conjoint analysis: G|C+R 56%, G|R 59%, G 48% ~ C (|R), G+C|R 56%, C|R 64%, Left+G - Left = -3pp, A+G vs. B 59%
% => G is supported for itself, rather independently from R or C, with similar support to both, and it doesn't significantly penalize the Left, and would help a Democratic candidate

To assess the public support for the GCS in conjunction with other policies, we %conduct a series of conjoint analyses. We 
ask respondents to make five choices between pairs of political platforms.

The first conjoint analysis suggests that the GCS is supported independently of being complemented by the National Redistribution Scheme and a national climate policy, denoted C.  
% (``Coal exit'' in the U.S., ``Thermal insulation plan'' in Europe, denoted C). %\footnote{Indeed, 54\% of %($n$ = 3,000) % TODO? remove? 12 w
% U.S. respondents and 74\% of %($n$ = 3,000) 
% European ones prefer the combination of C, NR and the GCS to the combination of C and NR alone, indicating similar support for the GCS conditional on NR and C than for the GCS alone (Figure \ref{fig:conjoint}).} % (as it does not significantly differ from the direct support of 53\%). 
For the second analysis, we split the sample into four random branches. %\footnote{Results from the first branch show that the support for the GCS conditional on NR, at 55\% in the U.S. ($n$ = 757) and 77\% in Europe ($n$ = 746), is not significantly different from the support for the GCS alone. This suggests that rejection of the GCS is not driven by the cost of the policy on oneself. The second branch shows that the support for C conditional on NR is somewhat higher, at 62\% in the U.S. ($n$ = 751) and 84\% in Europe ($n$ = 747). However, the third one shows no significant preference for C compared to GCS (both conditional on NR), neither in Europe, where GCS is preferred by 52\% ($n$ = 741) nor in the U.S., where C is preferred by 53\% ($n$ = 721). The fourth branch shows that 55\% in the U.S. ($n$ = 771) and 77\% in Europe ($n$ = 766) prefer the combination of C, NR and the GCS to NR alone.} 
The outcome is that there is majority support for the GCS and for C, which are seen as neither complement nor substitute (Supplementary Figure A7). A minor share of respondents like a national climate policy and dislike a global one, but as many people prefer a global rather than a national policy; and there is no evidence that implementing NR would increase the support for the GCS.

In the third analysis, we present two random branches of the sample with hypothetical progressive and conservative platforms that differ only by the presence (or not) of the GCS in the progressive platform. Table \ref{tab:conjoint_c} shows that a progressive candidate would not significantly lose voting share by endorsing the GCS in any country, and may even gain 11 p.p. ($p = .005$) in voting intention in France and 3 p.p. ($p = .13$) in the U.S. 
% The effect is also positive at 3 p.p. ($p = .13$) in the U.S., although not significant at the 5\% threshold. % France holds multiple hypotheses testing  % shortened
% Though the level of support for the GCS is significantly lower in swing States (at 51\%) that are key to win U.S. elections, the electoral effect of endorsing the GCS remains non-significantly different from zero (at +1.2 p.p.) in these States.%\footnote{We define swing states as the 8 states with less than 5 p.p. margin of victory in the 2020 election (MI, NV, PA, WI, AZ, GA, NC, FL). The results are robust to using the 3 p.p. threshold (that excludes FL) instead.} % TODO? remove? 41 w

%The third analysis suggests that a progressive candidate would not significantly lose voting share if he or she were to endorse the GCS, and that he or she may even gain 11 p.p. vote intention in France (see Table \ref{tab:conjoint_c}). To estimate this, we present to two random branches of the sample hypothetical progressive and conservative platforms that differ only by the presence (or not) of the GCS in the progressive platform. 

\begin{table} % NCCcomment [h]
  % MAJOR figure
  \caption[Influence of the GCS on electoral prospects]{Preference for a progressive platform depending on whether it includes the GCS or not. (Question 28 in Supplementary Section D) 
  %Imagine if the [Democratic and Republican presidential candidates in 2024] campaigned with the following policies in their platforms. [Credible Progressive and Conservative platforms] \\ % TODO See More
Which of these candidates would you vote for? \textit{A; B; None of them} \\
% ~[FR: second round of presidential; DE, ES, UK: two favorite candidates in one's constituency]
} % Beware, this question is quite unusual. \\ Among the policies below, how many do you support?  \\ Coal exit, Marriage only for opposite-sex couples 
  \makebox[\textwidth][c]{
\begin{tabular}{@{\extracolsep{5pt}}lcccccc} 
\\[-1.8ex]\hline 
\hline \\[-1.8ex] 
 & \multicolumn{6}{c}{Prefers the Progressive platform} \\ 
\cline{2-7} 
\\[-1.8ex] & All & United States & France & Germany & UK & Spain \\ 
\hline \\[-1.8ex] 
 GCS in Progressive platform & 0.028$^{*}$ & 0.029 & 0.112$^{***}$ & 0.015 & 0.008 & $-$0.015 \\ 
  & (0.014) & (0.022) & (0.041) & (0.033) & (0.040) & (0.038) \\ 
 \hline \\[-1.8ex] 
Constant & 0.623 & 0.604 & 0.55 & 0.7 & 0.551 & 0.775 \\ 
Observations & 5,202 & 2,619 & 605 & 813 & 661 & 504 \\ 
R$^{2}$ & 0.001 & 0.001 & 0.013 & 0.0003 & 0.0001 & 0.0003 \\ 
\hline 
\hline \\[-1.8ex] 
\end{tabular} 
}\label{tab:conjoint_c}
  {\footnotesize \textit{Note:} Simple OLS model. The 14\% of \textit{None of them} answers have been excluded from the regression samples. GCS has no significant influence on them. $^{*}p<0.1$; $^{**} p<0.05$; $^{***} p<0.01$. 
  }
\end{table}
% \begin{stretchpars}
Our last two analyses  make respondents choose between two random platforms (in the U.S., these questions are framed as a Democratic primary and asked only to non-Republicans).  % shortened
% In Europe, respondents are prompted to imagine that a left- or center-left coalition will win the next election and are asked what platform they would prefer that coalition to have campaigned on. In the U.S., the question is framed as a hypothetical duel in a Democratic primary, and asked only to non-Republicans ($n$ = 2,218), i.e. the respondents who choose \textit{Democrat}, \textit{Independent}, \textit{Non-Affiliated} or \textit{Other} for their political affiliation. 
In the fourth analysis, a policy (or an absence of policy) is randomly drawn for each platform in each of five categories.%: \textit{economic issues}, \textit{societal issues}, \textit{climate policy}, \textit{tax system}, \textit{foreign policy} (Supplementary Figure 8). % TODO? remove? 10 w
% \end{stretchpars}
 
% \begin{figure}[h] 
%   \caption[Preferences for various policies in political platforms]{Effects of the presence of a policy (rather than none from this domain) in a random platform on the likelihood that it is preferred to another random platform. (Question \ref{q:conjoint_r}%; in the U.S., asked only to non-Republicans.
%   )}\label{fig:ca_r}
%   \begin{subfigure}{\textwidth}
%     \subcaption{U.S. (Asked only to non-Republicans)}
%     \includegraphics[width=\textwidth]{../figures/US1/ca_r.png}
%   \end{subfigure}
%   \begin{subfigure}{\textwidth}
%     \subcaption{France}
%     \includegraphics[width=\textwidth]{../figures/FR/ca_r.png}
%   \end{subfigure}
% \end{figure}%
% \clearpage
% \begin{figure}[h!]\ContinuedFloat % if bugs try b! instead of h!
%   \begin{subfigure}{\textwidth}
%     \subcaption{Germany}
%     \includegraphics[width=\textwidth]{../figures/DE/ca_r.png}
%   \end{subfigure}
%   \begin{subfigure}{\textwidth}
%     \subcaption{Spain}
%     \includegraphics[width=\textwidth]{../figures/ES/ca_r.png}
%   \end{subfigure}
%   \begin{subfigure}{\textwidth}
%     \subcaption{UK}
%     \includegraphics[width=\textwidth]{../figures/UK/ca_r.png}
%   \end{subfigure}
%   %\makebox[\textwidth][c]{} 
% \end{figure}
% \clearpage 
% \noindent

% Except for the category \textit{foreign policy}, which features the GCS 42\% of the time, the policies are prominent progressive policies and they are drawn uniformly. % except for tax1: .35 vs. tax2: .4 in EU. % TODO? remove? 25 w
In the UK, Germany, and France, a platform is about 9 to 13 p.p. more likely to be preferred if it includes the GCS rather than no foreign policy (Supplementary Figure 8). %\footnote{This is the Average Marginal Component Effect computed following \cite{hainmueller_causal_2014}.} 
This effect is between 1 and 4 p.p. and no longer significant in the U.S. and in Spain. Moreover, a platform that includes a global tax on millionaires rather than no foreign policy is 5 to 13 p.p. more likely to be preferred in all countries (the effect is significant and at least 9 p.p. in all countries but Spain). 
% Moreover, a platform that includes a global tax on millionaires rather that no foreign policy is 9 to 13 percentage points (p.p.) more likely to be preferred in all countries but Spain (not significant, at +5 p.p.). 
Similarly, a global democratic assembly on climate change has a significant effect of 8 to 12 p.p. in the U.S., Germany, and France. 
%In each country, a platform is more likely to be preferred if it includes the GCS rather than no foreign policy. This effect is significant in France, Germany and the UK, where a platform is about 10 p.p. more likely to be preferred. 
These effects are large, and not far from the effects of the policies most influential on the platforms, which range between 15 and 18 p.p. in most countries (and 27 p.p. in Spain), and all relate to improved public services. %(in particular healthcare, housing, and education).  % shortened

The fifth analysis draws random platforms similarly, except that candidate A's platform always contains the GCS while B's includes no foreign policy. In this case, A is chosen by 60\% in Europe %($n$ = 3,000) 
and 58\% in the U.S. (Supplementary Figure 9). %($n$ = 2,218). 
Overall, taking the U.S. as an example, our conjoint analyses indicate that a candidate at the Democratic primary would have more chances to obtain the nomination by endorsing the GCS, and this endorsement would not penalize her or him at the presidential election. 
% This result reminds the finding that 12\% of Germans shift their voting intention from SPD and CDU/CSU to the Greens and the Left when they are told that the latter parties support global democracy.\cite{ghassim_who_2020} % TODO? remove? 34 w

% \begin{figure}[h!]
%     \caption[Influence of the GCS on preferred platform]{Influence of the GCS on preferred platform:\\ Preference for a random platform A that contains the Global Climate Scheme rather than a platform B that does not (in percent). (Question \ref{q:conjoint_d}; in the U.S., asked only to non-Republicans.)}\label{fig:conjoint_left_ag_b}
%     \makebox[\textwidth][c]{\includegraphics[width=\textwidth]{../figures/country_comparison/conjoint_left_ag_b_binary_positive.pdf}} 
% \end{figure}


% \begin{figure}
%   % Imagine that at the 2024 Democratic party presidential primaries, the two main candidates campaign with the following key policies in their platforms. \\ Which of these candidates do you prefer?

%   \caption{Conjoint analysis. Average Marginal Component Effects (relative to the baseline: an absence of policy of that category) of policies in the choice between two platforms, where policies in each platform are randomly drawn ($n$ = 6,000). In Eu, it is framed as two potential platforms of a left-wing coalition that would win the next elections; in the U.S., it is framed as a hypothetical duel in the 2024 Democratic primary and asked only to non-Republicans.}\label{fig:ca_r} % TODO: add ref to Question
%   \makebox[\textwidth][c]{\includegraphics[width=\textwidth]{../figures/all/ca_r.png}}
% \end{figure}

% \subsubsection{Prioritization}\label{subsubsec:prioritization} % Addresses acquiescence bias and social desirability bias % NCCcomment
\paragraph{Prioritization}\label{subsubsec:prioritization} % WPcomment
% H1: Prioritization: G has mean only slightly lower than average, makes better than ban of cars and coal exit; global tax on millionaires does as well as wealth tax and almost as good as $15 minimum wage

Toward the end of the survey, we ask respondents to allocate 100 points among six randomly selected policies from the previous conjoint analyses, using sliders. The instruction was to distribute the points based on their level of support, with a higher allocation indicating greater support for a policy. %At the end of the survey, we pick six policies at random (and uniformly) among the policies used in the last conjoint analyses, and ask respondents to allocate 100 points among them (using sliders), with the instruction that ``the more you give points to a policy, the more you support it''. 
% As a result, the average support across policies is 16.67 points.  % shortened
%(Figure \ref{fig:points}). % TODO! figures for each country
In each country, the GCS ranks in the middle of all policies or above, with an average number of points from 15.4 in the U.S. to 22.9 in Germany (Supplementary Figure A29).%The GCS ranked in the middle or higher among all policies in each country, receiving an average of 15.4 points in the U.S. and 22.9 points in Germany. 

Interestingly, in Germany, the most prioritized policy is the global tax on millionaires, while the GCS is the second most prioritized policy. The global tax on millionaires consistently ranks no lower than fifth position (out of 15 or 17 policies) in every country, garnering an average of 18.3 points in Spain to 22.9 points in Germany.

% This question sheds light on a potential discrepancy between the policy priorities of the public and those enacted by legislators. For instance, while the European Union and California have enacted plans to phase out new combustion-engine cars by 2035, the proposal to ``ban the sale of new combustion-engine cars by 2030'' emerged as one of the three least prioritized policies in each country, with an average allocation of 7.8 points in France to 11.4 points in the UK. % TODO? remove? 78 w
%It is higher than to ``ban the sale of new combustion-engine cars by 2030'' (13.4) and ``coal exit'' (10.0), but lower than the third climate policy: ``trillion dollar investment in clean transportation infrastructure and building insulation'' (20.3). The support for other globally redistributive policies is variable: ``Doubling foreign aid'' is the least supported policy (8.4), while the ``Global tax on millionaires'' is one of the five policies with more than 20 points (20.2), and the ``global democratic assembly on climate change'' is just below the GCS (14.5). The most supported policies are ``Funding affordable housing'' (28.5), ``\$15 minimum wage'' (23.8), and ``Universal childcare/pre-K'' (22.1). % TODO? share that allocated at least 1

% \begin{figure}[h!]
%   \caption{Prioritization of policies. Each respondent faces six policies taken at random from the ones below and allocates 100 points among them to signal the strength of their support for each one ($n$ = 3,000).} % Imagine you have 100 points that you can allocate to different policies. The more you give points to a policy, the more you support it.  \\  How do you allocate the points among the following policies?  
  
%   \makebox[\textwidth][c]{\includegraphics[width=\textwidth]{../figures/US1/points_us.pdf}}\label{fig:points}
% \end{figure}

% \subsubsection{Pros and Cons}\label{subsubsec:pros_cons} % NCCcomment
\paragraph{Pros and Cons}\label{subsubsec:pros_cons} % WPcomment

We survey respondents to gather their perspectives on the pros and cons of the GCS, utilizing either an open-ended or a closed question. 
% In the closed question format, respondents tend to consider every argument as important in determining their support or opposition to the GCS (see Supplementary Figure A8). Notably, the least important aspect was the negative impact on their household, with 60\% in Europe ($n$=1,505) and 75\% in the U.S. ($n$=493) finding it important. The most important elements differ between Europe and the U.S. In Europe, the key factors are the GCS's potential to limit climate change and reduce poverty in low-income countries, both deemed important by 85\% of respondents. In the U.S., having sufficient information about the scheme ranks highest at 89\%, followed by its potential to foster global cooperation at 82\%. % TODO? remove? 111 w
% However, d
Due to the limited variation in the ratings for each element, the closed question format is inconclusive (Supplementary Figure A9). % TODO? distinguish between supporters and opponents?

% The open-ended question provides more insights into what people associate with the GCS when prompted to think about it.  % shortened % The open-ended question gives interesting insight into ``what comes to [people's] mind'' when ``thinking of the GCS''. 
Analyzing keywords in the responses (automatically translated into English), the most frequently mentioned topics are the international aspect and the environment, 
while obstacles to implementation or agreement on the proposal are relatively infrequently mentioned (Supplementary Figure A11). 
% each appearing in approximately one-quarter of the answers (Supplementary Figure A11). This is followed by discussions on the effects of the GCS on poverty and prices, each mentioned by about one-tenth of the respondents. We also manually classified each answer into different categories (Supplementary Figure A10). This exercise confirms the findings from the automatic search: the environmental benefit of the GCS is the most commonly discussed topic, while obstacles to implementation or agreement on the proposal are relatively infrequently mentioned.  % shortened
%the most frequent topic is the environmental benefit of the GCS, while the obstacles to implement it or to get %people or countries agreement on it are relatively seldom mentioned.
%\footnote{Moreover, around one in four respondents explicitly cites pros or cons. Few individuals explicitly express support or opposition, and misunderstandings are rare. Only 11\% of the responses are empty or express a lack of opinion, though one-quarter are unclassifiable due to the rarity, nonsensical nature, or irrelevance of the conveyed idea.}% TODO n

% In the \textit{US2} survey, we presented these questions to random subsamples \textit{before} inquiring about support for the GCS or NR. The sample was divided into four branches: two branches with questions on pros and cons (either in closed or open form), one branch providing information on the actual level of support for the GCS and NR (estimated in \textit{US1}), and one control group without these questions.\footnote{Consistent with Americans accurately perceiving the levels of support for the GCS or NR, providing information on the actual level had no substantial effect on their support. In the closed question regarding pros and cons, we intentionally included more cons (6) than pros (3) to conservatively estimate the potential campaign effect on the GCS, which refers to the shift in opinion resulting from media coverage of the proposal. Interestingly, the support for the GCS decreased by 11 p.p. after participants viewed a list of its pros and cons. Surprisingly, the support for National Redistribution also decreased by 7 p.p. following the closed question about the GCS. This suggests that some individuals may lack attention and confuse the two policies, or that contemplating the pros and cons alters the mood of some people, moving them away from their initial positive impression. Notably, the support also decreased by 7 p.p. after participants were asked to consider the pros and cons in an open-ended question.} % /!\ leave commented
%Despite some significant effects of pondering the pros and cons (see Table \ref{tab:branch_gcs}), approximately half of the Americans expressed support for the GCS across all treatment branches. If similar effects were observed in Europe, it suggests that the GCS would still enjoy strong majority support among Europeans once it enters the public debate (Table \ref{tab:branch_gcs}).


In the \textit{US2} survey, we divided the sample into four random branches. Two branches were presented the pros and cons questions (either in open or closed format) \textit{before} being asked about their support for the GCS or NR. Another branch received information on the actual level of support for the GCS and NR (estimated in \textit{US1}, see Section \ref{subsec:second_order_beliefs}), and one control group received none of these treatments. % TODO? Clarify that the information was also before the support? Write here (rather than in second-order beliefs) that ``Consistent with Americans correctly perceiving the levels of support for the GCS or NR, providing information on the actual level has no substantial effect on their support.''
The objective of this ``pros and cons treatment'' was to simulate a ``campaign effect'',\cite{anderson_can_2023} which refers to the shift in opinion resulting from media coverage of the proposal. To conservatively estimate the effect of a (potentially negative) campaign, we intentionally included more cons (6) than pros (3). Interestingly, the support for the GCS decreased by 11 p.p. after respondents viewed a list of its pros and cons. %\footnote{Surprisingly, the support for National Redistribution also decreased by 7 p.p. following the closed question about the GCS. This suggests that some individuals may lack attention and confuse the two policies, or that contemplating the pros and cons alters the mood of some people, moving them away from their initial positive impression.} 
Notably, the support also decreased by 7 p.p. after respondents were asked to consider the pros and cons in an open-ended question. Although support remains significant, % relatively high (it would remain majoritarian in Europe if similar effects were observed)
%\footnote{Despite some significant effects of pondering the pros and cons, approximately half of the Americans express support for the GCS across all treatment branches (see Table \ref{tab:branch_gcs}).} 
these results suggest that the support for the GCS is context-dependent, sensitive to the content of the debate about it, and subject to the discourse adopted by interest groups.

% where respondents are prompted to think about the advantages and drawbacks of the policy before making a decision

% On the campaign effect: Funk (16) shows that there is a survey bias of 5-6 p.p.: in post-referendum surveys, Swiss people approve left-wing policies 5 pp less. Jenning & Wlezien (18) show that between seven to one month before the election, polls mean absolute error about the result is ~4 p.p. (goes down to ~2pp the days before).

\subsection*{Universalistic values}\label{subsec:universalistic}
% H4: A strong majority is universalist/cosmopolitan (TODO: which word?), even a majority for non-Republican
% TD It is not obvious how these answers are informative of malleable opinions. So I don't think we should state the hypothesis and sell this as a test.
%Another hypothesis to explain the discrepancy between the lack of interest for global policies in the public debate despite a strong stated support is that opinions on the topic are weak and malleable. A way to test this is to

We also elicit underlying values, to test whether values are consistent with people's support for specific policies. Most people express some degree of universalism, consistently with the support for specific policies. %To better understand people's support for specific policies, we also elicit underlying values. %We ask broad questions on people's values to see whether their core values are consistent with universalism. 

When we ask respondents which group they defend when they vote, % ($n$ = 6,000)
20\% choose ``sentient beings (humans and animals),'' 22\% choose ``humans,'' 33\% select their ``fellow citizens'' (or ``Europeans''), 15\% choose ``My family and myself,'' and the remaining 10\% choose another group (mainly ``My State or region'' or ``People sharing my culture or religion''). 
% The first two categories, representing close to one out of two people, can be described as universalist in their vote. % TODO? remove? 20 w
Notably, a majority of left-wing choose ``humans'' or ``sentient beings'' 
% voters can even be considered universalist voters 
(see Supplementary Figure A39 for main attitudes by vote).% The share of universalist even constitutes a majority of left-wing voters

%Regarding the priorities of their country's diplomats in international climate negotiations, 
When asked what their country's diplomats should defend in international climate negotiations, only 11\% prefer their country's ``interests, even if it goes against global justice'' (Supplementary Figure A26). In contrast, 30\% prefer global justice (with or without consideration of national interests), and the bulk of respondents (38\%) prefer their country's ``interests, to the extent it respects global justice.''

Furthermore, when we ask respondents to assess the extent to which climate change, global poverty, and inequality in their country are issues, climate change is generally viewed as the most significant problem (with a mean score of 0.59 after recoding answers between $-$2 and 2). This is followed by global poverty (0.42) and national inequality (0.37). %, $n$ = 6,000).

Finally, we conduct a lottery experiment. % to elicit universalistic values.  % shortened
Respondents were automatically enrolled in a lottery with a \$100 prize and had to choose the proportion of the prize they would keep for themselves versus give to a person living in poverty. The charity donation is directed either to an African individual or a fellow citizen, depending on the respondent's random assignment. In Europe, we observe no significant variation in the willingness to donate based on the recipient's origin. In the U.S., the donations to Africans are 3 p.p. lower% (with an average donation of 34\%) % shortened
, but the slightly lower donations to Africans are entirely driven by Trump voters and non-voters (Supplementary Table A2). 

% TODO! put back?
% Overall, answers to these broad value questions are consistent with half of Americans and three quarters of Europeans supporting global policies like the GCS: people are almost as much willing to give to poor Africans than to poor fellow citizens, find that global issues are among the biggest problems, almost half of them are universalist when they vote, and most of them wish that their diplomats take into account global justice.

\subsection*{Second-order Beliefs}\label{subsec:second_order_beliefs}
% H3 belief: No pluralistic ignorance
To explain the strong support for the GCS despite its absence from political platforms and public debate, we hypothesized pluralistic ignorance, i.e. that the public and policymakers mistakenly perceive the GCS as unpopular. 
% As a result, individuals might conceal their support for such globally redistributive policies, believing that advocating for them would be futile. However, the  % shortened
The evidence for pluralistic ignorance is limited based on an incentivized question about perceived support (Supplementary Figure 10).

Beliefs about the level of support for the GCS are fairly accurate for U.S. subjects. The mean perceived support is 52\% (with quartiles of 36\%, 52\%, and 68\%), which closely aligns with the actual support of 53\%. Europeans, on the other hand, underestimate the support by 17 p.p. Nonetheless, 65\% of them correctly estimate that the GCS garners majority support, and the mean perceived support is 59\% (and quartiles of 43\%, 61\%, and 74\%). 
% , compared to the actual support of 76\%. Second-order beliefs are equally accurate for NR in the U.S. and similarly underestimated in Europe. %, with mean (resp. quartiles) perceived support of 54.7\% (resp. 40, 55, 71\%, $n$ = 3,000) vs. 56\%.  % shortened
Finally, consistent with U.S. subjects accurately perceiving the levels of support for the GCS or NR, providing information on the actual level had no significant effect on their support in the \textit{US2} survey. % Consistent with Americans correctly perceiving the levels of support for the GCS or NR, providing information on the actual level has no substantial effect on their support.

% \begin{figure}[h!]
%     \caption[Beliefs about support for the GCS and NR]{Beliefs regarding the support for the GCS and NR. (Questions \ref{q:gcs_belief} and \ref{q:nr_belief})}\label{fig:belief}
%     \makebox[\textwidth][c]{\includegraphics[width=.7\textwidth]{../figures/country_comparison/belief_all_mean.pdf}} 
% \end{figure}


\section*{Discussion} % Summary, conclusion

We provide some cautious evidence that certain global climate and redistribution policies may be popular in the Global North. We find no evidence for insincerity or underestimation of fellow citizens' support as potential explanations for the scarcity of global policies in the public debate, though support for them might decay once they are debated. New hypotheses could explain the seeming inconsistency of the high support for global policies with their peripheral appearance in the policy sphere: First, policymakers may be unaware that global policies are popular. Second, policymakers may believe that global redistribution is politically infeasible in some key countries. Third, with political discourse focused on the national level, shaped by media and elections, national framing may suppress universalistic values. 

% Having ruled out insincerity and underestimation of fellow citizens' support as potential explanations for the scarcity of global policies in the public debate, we propose different alternative explanations. 
% First, there may be pluralistic ignorance among policymakers. Second, policymakers may believe that globally redistributive policies are politically infeasible in some key countries like the U.S. Third, political discourse centrally happens at the national level, shaped by media and institutions such as voting. In turn, national framing may suppress universalistic values. 
% In any case, our findings indicate that public opinion is not the reason why global redistributive policies do not prominently enter political debates.
% National framing by political voices may create biases and suppress universalistic values. Uncovering evidence to support these hypotheses could %help
% draw attention to global policies in the public debate and contribute to their increased prominence.

% TODO? give less space to OECD paper?
% Our point of departure are recent surveys %by \cite{dechezlepretre_fighting_2022} 
% %  conducted by Dechezleprêtre et al. (2022) % TODO!back
% in 20 of the largest countries, % \cite{dechezlepretre_fighting_2022}
% as they reveal robust majority support for global redistributive and climate policies, even in high-income countries that would financially lose from them.\cite{dechezlepretre_fighting_2022} The results from %our  TODO? put back?
% complementary surveys conducted in the U.S. and four European countries %presented here 
% reinforce these findings. We find strong support for global taxes on the wealthiest individuals, as well as majority support for our main policy of interest -- the Global Climate Scheme (GCS). The GCS encompasses carbon pricing at a global level through an emissions trading system, accompanied by a global basic income funded by the scheme's revenues. Additional experiments, such as a list experiment and a real-stake petition, demonstrate that the support for the GCS is real. 
% Such genuine support is further substantiated by the prioritization of the GCS over prominent national climate policies and aligned with a significant portion of the population holding universalistic values rather than nationalistic or egoistic ones. Moreover, the conjoint analyses indicate that a progressive candidate would not lose voting shares by endorsing the GCS, and may even gain 11 p.p. in voting shares in France. Similarly, a candidate endorsing the GCS would gain votes in a U.S. Democratic primary, while in Europe, a progressive platform that includes the GCS would be preferred over one that does not.

% %Having ruled out insincerity and underestimation of fellow citizens' support as potential explanations for the scarcity of global policies in the public debate, we propose alternative explanations. %As we ruled out all hypotheses of our registration plan,\footnote{The project was preregistered in the Open Science Foundation registry (\href{https://osf.io/fy6gd}{osf.io/fy6gd}).} we now need to study new explanations. % /!\ leave commented
% % The first two are variations of pluralistic ignorance, and the last three represent complementary explanations. % TODO? remove? 15 w

% What could explain the gap between sincere support of citizens and the scarce mention in public debate? First, there may be pluralistic ignorance \textit{among policymakers} regarding universalistic values, support for global redistribution, or the electoral advantage of endorsing it. Second, people or policymakers may believe that globally redistributive policies are technically impossible or politically infeasible in some key (potentially foreign) countries. % like the U.S. % We intend to test these hypotheses by running a survey on Congress staffers and Members of the European Parliament.  %Second, there may be a more subtle form of pluralistic ignorance: although most people correctly predict what people would answer to a survey question, they may view globally redistributive policies as unrealistic, perhaps because they have never reflected upon the fact that many people across the world hold univeralistic values and are supportive of global solidarity. Third, most people and perhaps even most policy makers may have simply never heard of the GCS, let alone built their political ideas upon it. 
% Third, political discourse centrally happens at the national level, shaped by national media and institutions such as voting. 
% National framing by political voices may create biases and suppress universalistic values. Uncovering evidence to support these hypotheses could %help
% draw attention to global policies in the public debate and contribute to their increased prominence.% Third, most institutions are national: the largest scale votes take place at the national level% so political platforms are devised at this level, most media target a national audience, most commentators frame their discourse from a national perspective and portray relations to foreign countries as conflictual. 
% Fourth, many individuals, including policymakers, may perceive global redistributive policies as ill-defined or technically infeasible, ultimately dismissing them as unrealistic. In particular, policymakers may have insider information about the feasibility of such policies. Alternatively, the perception of unrealism may stem from an unawareness of specific proposals. % cautiously doubt that they are well-specified
% TODO? rewrite as a fifth hypothesis that the policies may indeed not be well specified nor implementable?
% Fourth, many individuals, including policymakers, may simply be unaware of specific global redistributive policies, let alone base their political ideas on them. This lack of awareness may lead people to % cautiously doubt that they are well-specified
% perceive such policies as ill-defined or technically infeasible, ultimately dismissing them as unrealistic. % TODO? rewrite as a fifth hypothesis that the policies may indeed not be well specified nor implementable?
% The latter hypothesis is supported by ignorance of the GCS expressed in the feedback fields, where a common response is a variation of ``thank you for this interesting, thought-provoking survey.'' 
% Fifth, just as policy is disproportionately influenced by the economic elites,\cite{gilens_testing_2014,persson_rich_2023} public debate may be shaped by the wealthiest, who have vested interests in preventing global redistribution.
% TODO? remove? Fourth & Fifth: 103 w

% Also: vested interest influence on elites (a la Gilens and Page for the US, Persson & Sundell 23), cf. Boone story
% decline of support the more specific a measure gets / enters the public debate => addressed in conclusion of pros and cons
% people not believing that the policy is technically implementable => addressed
% politicians seeing defects or lack of specification of the policy that warrant opposition, while ignorant citizens would naively like it => addressed

% LM This are all legtimiate and important points. For a global transfer (rather than say a EU transfer), I am really missing remarks about (lack) of quality of governance, rent creation and capture, corruption etc. To play devil's advocate, as much as I am sympathetic to the idea, I'm not sure I would currently vote in favor of such policies. Why? Not because I dislike the idea, but I am not sure that my carbon tax rent as a German would actually reach the poor population of Indonesia or Nigeria rather than the pockets of cleptocratic elites :) I'd like the draft to reflect that and could make an attempt to pre-empt that objection based on some references later in the project.
%In any case, 

%Confirmation of any of these hypotheses would lead to a common conclusion: there exists substantial support for global policies addressing climate change and global inequality, even in high-income countries, and the perceived boundaries of political realism on this issue may soon shift. Uncovering evidence to support these hypotheses could %help
%draw attention to global policies in the public debate and contribute to their increased prominence. % Uncovering evidence for this might actually contribute itself to garner more attention to global policies in the public debate. % TODO? remove? 65 w

% Also: no short term outcome (before the end of the mandate) so no incentive to start it

% TODO!? alternative conclusion: strong and sincere support for global wealth tax in both continents => probable reason why it's not more discussed is that it's not the determining issue for voters. Strong and sincere support for the GCS in Europe but half-half in the U.S. (where there is a risk that support decreases in a campaign) => not endorsed by the Democrats because they seek policies more strongly supported, and not discussed in Europe probably because the U.S. won't support it.

% For warm glow and how to test it, see my 28/6, 30/6,  (2023) emails (elicit beliefs on feasibility/obstacles, ask support for Bridgetown / 100G€ with vs. without saying it may be approved at the COP, redo conjoint analysis without support question before, asking for approval of global redistr. policy costing 50€/month with/without asking for a similar one costing 20€/month before). 

\section*{Methods}\label{sec:methods} % NCCcomment
% \begin{methods}  % WPcomment
  \begin{small} % NCCcomment
%Put methods in here.  If you are going to subsection it, use \subsection commands.  Methods section should be less than 800 words and if it is less than 200 words, it can be incorporated into the main text.
% \fakesection{Methods}\label{sec:methods}
% \addcontentsline{toc}{section}{Methods}
\subsection*{\small Data collection.} % WPcomment
% \paragraph{\small Data collection.} % NCCcomment
% The paper relies on two different sets of surveys. The first set of surveys was conducted between March 2021 and March 2022 on 40,680 respondents from 20 countries  (between 1,465 and 2,488 respondents per country). The first U.S. complementary, denoted \textit{US1}, was conducted on 3,000 U.S. respondents between January and March 2023, while the second, \textit{US2}, was conducted on 2,000 respondents between March and April 2023. The \textit{Eu} complementary survey was conducted on 3,000 respondents between February and March 2023. We used the survey companies \emph{Dynata} and \emph{Respondi}. Stratified quotas ensure that the samples are representative along the dimensions of gender, age (5 brackets), income (4), region (4), education level (3), as well as ethnicity (3) for the U.S. To correct for small remaining imbalances, we apply survey weights throughout the analysis, constructed using the quotas variables as well as the degree of urbanity, and trimmed between 0.25 and 4. Weights make the results fully representative of the country (or of the four European countries combined in the case of results at the European level, where different weights are used). Appendix \ref{app:representativeness} confirms that our samples are representative of the population. % TODO! and urbanity instead of education for OECD, TODO table representativeness
%\footnote{We trim weights so that no respondent receives a weight below 0.25 or above 4. Overall, trimming changes the weights for xx\% of the respondents.} % /!\ leave commented

The paper utilizes two sets of surveys: the \textit{Global} survey and the \textit{Complementary} surveys. The \textit{Complementary} surveys consist of two U.S. surveys, \textit{US1} and \textit{US2}, and one European survey, \textit{Eu}. The \textit{Global} survey was conducted from March 2021 to March 2022 on 40,680 respondents from 20 countries (with 1,465 to 2,488 respondents per country). \textit{US1} collected responses from 3,000 respondents between January and March 2023, while \textit{US2} gathered data from 2,000 respondents between March and April 2023. \textit{Eu} included 3,000 respondents and was conducted from February to March 2023. We used the survey companies \emph{Dynata} and \emph{Respondi}. To ensure representative samples, we employed stratified quotas based on gender, age (5 brackets), income (4), region (4), and education level (3), as well as ethnicity (3) for the U.S. We also incorporated survey weights throughout the analysis to account for any remaining imbalances. These weights were constructed using the quota variables as well as the degree of urbanity, and trimmed between 0.25 and 4. By applying weights, the results are fully representative of the respective countries. Results at the European level apply different weights which ensure  representativeness of the combined four European countries. 
Supplementary Section G confirms 
that our samples closely match population frequencies in high-income countries. In middle-income countries, the samples are only representative of the online population (young, graduated and urban people are over-represented). Supplementary Material I shows that the treatment branches are balanced. Supplementary Material J runs placebo tests of the effects of each treatment on unrelated outcomes. We do not find effects of earlier treatments on unrelated outcomes arriving later in the survey. %due to the over-representation of young, educated, and urban populations in the online sample
% Supplementary Section G confirms that our samples are representative of the population.%\footnote{We trim weights so that no respondent receives a weight below 0.25 or above 4. Overall, trimming changes the weights for xx\% of the respondents.} % /!\ leave commented


\subsection*{\small Data quality.} % WPcomment % TODO attrition analysis
% \paragraph{\small Data quality.} % NCCcomment
The median duration is 28 minutes for the \textit{Global} survey, 14 min for \textit{US1}, 11 min for \textit{US2}, and 20 min for \textit{Eu}. To ensure the best possible data quality, we exclude respondents who fail an attention test or rush through the survey (i.e., answer in less than 11.5 minutes in the \textit{Global} survey, 4 minutes in \textit{US1} or \textit{US2}, 6 minutes in \textit{Eu}). %All the results and analyses use survey weights, defined to make the results fully representative of the country (or of Eu in the case of results at the Eu level) along the quota variables. Weights are trimmed to be between .25 and 4. 
%\textit{Ex post}, we checked that there were only a few careless response patterns (such as choosing the same answer for all items in a matrix of questions; see Appendix \ref{app:data_quality}). At the end of the survey, we ask whether respondents thought that our survey was politically biased and provide some feedback. 67\% of the respondents found the survey unbiased. 25\% found it left-wing biased, and 8\% found it right-wing biased.



\subsection*{\small Questionnaires and raw results.} % WPcomment
% \paragraph{\small Questionnaires and raw results.} % NCCcomment
% Possible confusion in the questionnaire: people confuse GCS with the four policies together (so support for GCS can suffer from dislike of death penalty), although this confusion is mitigating by the fact that we right after ask about NR; people may answer about revenue-use rather than the whole measure for ETS2 support; people may answer that GCS will or will not have the effects proposed rather than these effects being important or not in their attitude toward GCS; they may answer that it's important that others (not them) get more information; the minimum wage could be reduced at 50% of local median wage.
The questionnaire and raw results of the \textit{Global} survey can be found in the Appendix of the companion paper, Dechezleprêtre et al. (2022).\cite{dechezlepretre_fighting_2022} %.\cite{dechezlepretre_fighting_2022} the companion paper TODO? put back?
The raw results are reported in Supplementary Section B %\footnote{Country-specific raw results are also available as supplementary material files:  \href{https://github.com/bixiou/global_tax_attitudes/raw/main/paper/app_desc_stats_US.pdf}{US}, \href{https://github.com/bixiou/global_tax_attitudes/raw/main/paper/app_desc_stats_EU.pdf}{EU}, \href{https://github.com/bixiou/global_tax_attitudes/raw/main/paper/app_desc_stats_FR.pdf}{FR}, \href{https://github.com/bixiou/global_tax_attitudes/raw/main/paper/app_desc_stats_DE.pdf}{DE}, \href{https://github.com/bixiou/global_tax_attitudes/raw/main/paper/app_desc_stats_ES.pdf}{ES}, \href{https://github.com/bixiou/global_tax_attitudes/raw/main/paper/app_desc_stats_UK.pdf}{UK}.} 
while the surveys' structures and questionnaires are given in Supplementary Sections C and D. The questionnaires are the same as the ones given \textit{ex ante} in the registration plan (\href{https://osf.io/fy6gd}{osf.io/fy6gd}).


\subsection*{\small Incentives.} % WPcomment
% \paragraph{\small Incentives.} % NCCcomment
To encourage accurate and truthful responses, several questions of the \textit{US1} survey use incentives. For each of the three comprehension questions that follow the policy descriptions, we randomly select and reward three respondents who provide correct answers with a \$50 gift certificate. Similarly, for questions involving estimating support shares for the GCS and NR, three respondents with the closest guesses to the actual values receive a \$50 gift certificate. In the donation lottery question, we randomly select one respondent and split the \$100 prize between the NGO GiveDirectly and the winner according to the winner's choice. In total, our incentives scheme distributes gift certificates (and donations) for a value of \$850. Finally, respondents have an incentive to answer truthfully to the petition question, as they are aware that the results for that question (the share of respondents supporting the policy) will be transmitted to the head of state's office.

%To encourage respondents to answer accurately and truthfully, several questions of the \textit{US1} survey use incentives. For each of the three comprehension questions that follow the policies' descriptions, we reward three (randomly drawn) respondents with the correct answer with a \$50 gift certificate. For each of the questions asking respondents to guess the share of support for the GCS and NR, we reward three people who are closest to the true value with a \$50 gift certificate. For the donation lottery question, we randomly draw one respondent and split the \$100 prize between the NGO GiveDirectly and the winner according to the winner's choice. In total, our incentives scheme distributes gift certificates (and donation) for a value of \$850. Finally, respondents have an incentive to answer truthfully to the petition question, given that they know that the results to that question (the share of respondents supporting the policy) will be transmitted to the U.S. President's office.




% Here is a description of a specific method used.  Note that the subsection heading ends with a full stop (period) and that the command is \verb|\subsection*{}| not \verb|\subsection**{}|.
% \subsection*{\small Details on sources, methodology and results} % WPcomment

\subsection*{\small Support for the GCS} The 95\% confidence intervals are $[52.4\%, 55.9\%]$ in the U.S. and $[74.2\%, 77.2\%]$ in Europe. The average support is computed with survey weights, employing weights based on quota variables, which exclude vote. Another method to reweigh the raw results involves running a regression of the support for the GCS on sociodemographic characteristics (including vote) and multiplying each coefficient by the population frequencies. This alternative approach yields similar figures: 76\% in Europe and 52\% or 53\% in the U.S. (depending on whether individuals who did not disclose their vote are classified as non-voters or excluded). Notably, the average support excluding non-voters is 54\% in the U.S. 

Though the level of support for the GCS is significantly lower in swing States (at 51\%) that are key to win U.S. elections, the electoral effect of endorsing the GCS remains non-significantly different from zero (at +1.2 p.p.) in these States. Note that we define swing states as the 8 states with less than 5 p.p. margin of victory in the 2020 election (MI, NV, PA, WI, AZ, GA, NC, FL). The results are robust to using the 3 p.p. threshold (that excludes FL) instead. 

\subsection*{\small Global wealth tax estimates}
A 2\% tax on net wealth exceeding \$5 million would annually raise \$816 billion, leaving unaffected 99.9\% of the world population. More specifically, it would collect \euro{}5 billion in Spain, \euro{}16 billion in France, £20 billion in the UK, \euro{}44 billion in Germany, \$430 billion in the U.S., and \$1 billion collectively in all low-income countries (28 countries, home to 700 million people). These Figures come from the \href{https://wid.world/world-wealth-tax-simulator/}{WID wealth tax simulator}.\cite{chancel_world_2022}

\subsection*{\small List experiment}
We utilize the difference-in-means estimator, and confidence intervals are computed using Monte Carlo simulation with the R package \textit{list}.\cite{imai_multivariate_2011}

\subsection*{\small Petition}
Paired weighted \textit{t}-tests are conducted to test the equality in support for a policy among respondents who were questioned about the policy in the petition.

\subsection*{\small Conjoint analysis}
The effects reported in the fourth analyses are the Average Marginal Component Effects.\cite{hainmueller_causal_2014} This method is common in the literature that explores attitudes towards taxation.\cite{bechtel_reforms_2020,ballard-rosa_structure_2017} The policies studied are progressive policies prominent in the country. Except for the category \textit{foreign policy}, which features the GCS 42\% of the time, they are drawn uniformly.

\subsection*{\small Prioritization}
The prioritization (Question 58, Supplementary Figures A29 and A30) allows inferring individual-level preferences for one policy over another. This slightly differs from a conjoint analysis, which only allows inferring individual-level preferences for one platform over another or collective-level preferences for one policy over another. Also, by comparing platforms, conjoint analyses may be subject to interaction effects between policies of a platform (which can be seen as complementary, subsitute, or antagonistic) while the prioritization frames the policies as independent.

\subsection*{\small Pros and cons}
Surprisingly, the support for National Redistribution also decreased by 7 p.p. following the closed question about the GCS. This suggests that some individuals may lack attention and confuse the two policies, or that contemplating the pros and cons alters the mood of some people, moving them away from their initial positive impression.

\subsection*{\small Determinants of support}
Supplementary Section F examines the sociodemographic determinants of support for the GCS as well as the beliefs correlated with the support for a global tax on GHG financing a global basic income. The strongest correlates are political leaning, trust in the government and perceptions that the policy is effective at reducing emissions or in one's self-interest.

\subsection*{\small Sources}
Detailed sources for the questionnaires and the figures are given in the \href{https://github.com/bixiou/international_attitudes_toward_global_policies/raw/main/questionnaire/specificities.xlsx}{Supplementary Spreadsheet}.

\subsection*{\normalsize Data and code availability}

All data and code of the \textit{Complementary} surveys as well as figures of the paper are available on \href{https://github.com/bixiou/international_attitudes_toward_global_policies}{github.com/bixiou/international\_attitudes\_toward\_global\_policies}. Data and code for the \textit{Global} survey will be made public upon publication.

% \end{methods} % WPcomment
\end{small}  % NCCcomment

\fakesection{Bibliography}\label{sec:bib}
% \bibliographystyle{naturemag_noURL} % nature class works only with style naturemag or naturemag_noURL, and naturemag bugs if there are certain URLs (like .pdf). Also, nature class only works with \cite, not \citet or \citep.  % WPcomment
% % \renewcommand{\url}[1]{\href{#1}{Link}} % NCCcomment
% % \bibliographystyle{plainnaturl_clean} % NCCcomment
% \bibliography{global_tax_attitudes}
\begin{thebibliography}{10}
  \expandafter\ifx\csname url\endcsname\relax
    \def\url#1{\texttt{#1}}\fi
  \expandafter\ifx\csname urlprefix\endcsname\relax\def\urlprefix{URL }\fi
  \providecommand{\bibinfo}[2]{#2}
  \providecommand{\eprint}[2][]{\url{#2}}
  
  \bibitem{dechezlepretre_fighting_2022}
  \bibinfo{author}{Dechezlepr{\^e}tre, A.} \emph{et~al.}
  \newblock \bibinfo{title}{Fighting climate change: {{International}} attitudes
    toward climate policies}.
  \newblock \emph{\bibinfo{journal}{NBER Working Paper}}
    \textbf{\bibinfo{volume}{30265}} (\bibinfo{year}{2022}).
  
  \bibitem{budolfson_climate_2021}
  \bibinfo{author}{Budolfson, M.} \emph{et~al.}
  \newblock \bibinfo{title}{Climate action with revenue recycling has benefits
    for poverty, inequality and well-being}.
  \newblock \emph{\bibinfo{journal}{Nature Climate Change}}
    \textbf{\bibinfo{volume}{11}}, \bibinfo{pages}{1111--1116}
    (\bibinfo{year}{2021}).
  
  \bibitem{franks_mobilizing_2018}
  \bibinfo{author}{Franks, M.}, \bibinfo{author}{Lessmann, K.},
    \bibinfo{author}{Jakob, M.}, \bibinfo{author}{Steckel, J.~C.} \&
    \bibinfo{author}{Edenhofer, O.}
  \newblock \bibinfo{title}{Mobilizing domestic resources for the {{Agenda}} 2030
    via carbon pricing}.
  \newblock \emph{\bibinfo{journal}{Nature Sustainability}}
    \textbf{\bibinfo{volume}{1}}, \bibinfo{pages}{350--357}
    (\bibinfo{year}{2018}).
  
  \bibitem{dennig_inequality_2015}
  \bibinfo{author}{Dennig, F.}, \bibinfo{author}{Budolfson, M.~B.},
    \bibinfo{author}{Fleurbaey, M.}, \bibinfo{author}{Siebert, A.} \&
    \bibinfo{author}{Socolow, R.~H.}
  \newblock \bibinfo{title}{Inequality, climate impacts on the future poor, and
    carbon prices}.
  \newblock \emph{\bibinfo{journal}{Proceedings of the National Academy of
    Sciences}} \textbf{\bibinfo{volume}{112}}, \bibinfo{pages}{15827--15832}
    (\bibinfo{year}{2015}).
  
  \bibitem{soergel_combining_2021}
  \bibinfo{author}{Soergel, B.} \emph{et~al.}
  \newblock \bibinfo{title}{Combining ambitious climate policies with efforts to
    eradicate poverty}.
  \newblock \emph{\bibinfo{journal}{Nature Communications}}
    \textbf{\bibinfo{volume}{12}}, \bibinfo{pages}{2342} (\bibinfo{year}{2021}).
  
  \bibitem{bauer_quantification_2020}
  \bibinfo{author}{Bauer, N.} \emph{et~al.}
  \newblock \bibinfo{title}{Quantification of an
    efficiency{\textendash}sovereignty trade-off in climate policy}.
  \newblock \emph{\bibinfo{journal}{Nature}} \textbf{\bibinfo{volume}{588}},
    \bibinfo{pages}{261--266} (\bibinfo{year}{2020}).
  
  \bibitem{cramton_global_2017}
  \bibinfo{editor}{Cramton, P.~C.}, \bibinfo{editor}{MacKay, D. J.~C.} \&
    \bibinfo{editor}{Ockenfels, A.} (eds.) \emph{\bibinfo{title}{Global Carbon
    Pricing: The Path to Climate Cooperation}} (\bibinfo{publisher}{{MIT Press}},
    \bibinfo{address}{{Cambridge, MA}}, \bibinfo{year}{2017}).
  
  \bibitem{fehr_your_2022}
  \bibinfo{author}{Fehr, D.}, \bibinfo{author}{Mollerstrom, J.} \&
    \bibinfo{author}{{Perez-Truglia}, R.}
  \newblock \bibinfo{title}{Your {{Place}} in the {{World}}: {{Relative Income}}
    and {{Global Inequality}}}.
  \newblock \emph{\bibinfo{journal}{American Economic Journal: Economic Policy}}
    \textbf{\bibinfo{volume}{14}}, \bibinfo{pages}{232--268}
    (\bibinfo{year}{2022}).
  
  \bibitem{kotchen_public_2017}
  \bibinfo{author}{Kotchen, M.~J.}, \bibinfo{author}{Turk, Z.~M.} \&
    \bibinfo{author}{Leiserowitz, A.~A.}
  \newblock \bibinfo{title}{Public willingness to pay for a {{US}} carbon tax and
    preferences for spending the revenue}.
  \newblock \emph{\bibinfo{journal}{Environmental Research Letters}}
    \textbf{\bibinfo{volume}{12}}, \bibinfo{pages}{094012}
    (\bibinfo{year}{2017}).
  
  \bibitem{klenert_making_2018}
  \bibinfo{author}{Klenert, D.} \emph{et~al.}
  \newblock \bibinfo{title}{Making carbon pricing work for citizens}.
  \newblock \emph{\bibinfo{journal}{Nature Climate Change}}
    \textbf{\bibinfo{volume}{8}}, \bibinfo{pages}{669} (\bibinfo{year}{2018}).
  
  \bibitem{douenne_yellow_2022}
  \bibinfo{author}{Douenne, T.} \& \bibinfo{author}{Fabre, A.}
  \newblock \bibinfo{title}{Yellow {{Vests}}, {{Pessimistic Beliefs}}, and
    {{Carbon Tax Aversion}}}.
  \newblock \emph{\bibinfo{journal}{American Economic Journal: Economic Policy}}
    (\bibinfo{year}{2022}).
  
  \bibitem{carattini_how_2019}
  \bibinfo{author}{Carattini, S.}, \bibinfo{author}{Kallbekken, S.} \&
    \bibinfo{author}{Orlov, A.}
  \newblock \bibinfo{title}{How to win public support for a global carbon tax}.
  \newblock \emph{\bibinfo{journal}{Nature}} \textbf{\bibinfo{volume}{565}},
    \bibinfo{pages}{289} (\bibinfo{year}{2019}).
  
  \bibitem{issp_international_2019}
  \bibinfo{author}{ISSP}.
  \newblock \bibinfo{title}{International {{Social Survey Programme ISSP}} 2019 -
    {{Social Inequality V}}}  (\bibinfo{year}{2019}).
  
  \bibitem{ghassim_public_2022}
  \bibinfo{author}{Ghassim, F.}, \bibinfo{author}{{Koenig-Archibugi}, M.} \&
    \bibinfo{author}{Cabrera, L.}
  \newblock \bibinfo{title}{Public {{Opinion}} on {{Institutional Designs}} for
    the {{United Nations}}: {{An International Survey Experiment}}}.
  \newblock \emph{\bibinfo{journal}{International Studies Quarterly}}
    \textbf{\bibinfo{volume}{66}}, \bibinfo{pages}{sqac027}
    (\bibinfo{year}{2022}).
  
  \bibitem{stern_report_2017}
  \bibinfo{author}{Stern, N.} \& \bibinfo{author}{Stiglitz, J.~E.}
  \newblock \bibinfo{title}{Report of the {{High-Level Commission}} on {{Carbon
    Prices}}}.
  \newblock \bibinfo{type}{Tech. Rep.}, \bibinfo{institution}{{Carbon Pricing
    Leadership Coalition}} (\bibinfo{year}{2017}).
  
  \bibitem{hainmueller_causal_2014}
  \bibinfo{author}{Hainmueller, J.}, \bibinfo{author}{Hopkins, D.~J.} \&
    \bibinfo{author}{Yamamoto, T.}
  \newblock \bibinfo{title}{Causal {{Inference}} in {{Conjoint Analysis}}:
    {{Understanding Multidimensional Choices}} via {{Stated Preference
    Experiments}}}.
  \newblock \emph{\bibinfo{journal}{Political Analysis}}
    \textbf{\bibinfo{volume}{22}}, \bibinfo{pages}{1--30} (\bibinfo{year}{2014}).
  
  \bibitem{anderson_can_2023}
  \bibinfo{author}{Anderson, S.}, \bibinfo{author}{Marinescu, I.} \&
    \bibinfo{author}{Shor, B.}
  \newblock \bibinfo{title}{Can {{Pigou}} at the {{Polls Stop Us Melting}} the
    {{Poles}}?}
  \newblock \emph{\bibinfo{journal}{Journal of the Association of Environmental
    and Resource Economists}} \textbf{\bibinfo{volume}{10}},
    \bibinfo{pages}{903--945} (\bibinfo{year}{2023}).
  
  \bibitem{chancel_world_2022}
  \bibinfo{author}{Chancel, L.}, \bibinfo{author}{Piketty, T.},
    \bibinfo{author}{Saez, E.} \& \bibinfo{author}{Zucman, G.}
  \newblock \bibinfo{title}{World {{Inequality Report}} 2022}
    \bibinfo{pages}{236} (\bibinfo{year}{2022}).
  
  \bibitem{imai_multivariate_2011}
  \bibinfo{author}{Imai, K.}
  \newblock \bibinfo{title}{Multivariate {{Regression Analysis}} for the {{Item
    Count Technique}}}.
  \newblock \emph{\bibinfo{journal}{Journal of the American Statistical
    Association}} \textbf{\bibinfo{volume}{106}}, \bibinfo{pages}{407--416}
    (\bibinfo{year}{2011}).
  
  \bibitem{bechtel_reforms_2020}
  \bibinfo{author}{Bechtel, M.~M.} \& \bibinfo{author}{Liesch, R.}
  \newblock \bibinfo{title}{Reforms and {{Redistribution}}: {{Disentangling}} the
    {{Egoistic}} and {{Sociotropic Origins}} of {{Voter Preferences}}}.
  \newblock \emph{\bibinfo{journal}{Public Opinion Quarterly}}
    \textbf{\bibinfo{volume}{84}}, \bibinfo{pages}{1--23} (\bibinfo{year}{2020}).
  
  \bibitem{ballard-rosa_structure_2017}
  \bibinfo{author}{{Ballard-Rosa}, C.}, \bibinfo{author}{Martin, L.} \&
    \bibinfo{author}{Scheve, K.}
  \newblock \bibinfo{title}{The {{Structure}} of {{American Income Tax Policy
    Preferences}}}.
  \newblock \emph{\bibinfo{journal}{The Journal of Politics}}
    \textbf{\bibinfo{volume}{79}}, \bibinfo{pages}{1--16} (\bibinfo{year}{2017}).
  
  \end{thebibliography}
  
% \appendix % NCCcomment
% \renewcommand{\thetable}{A\arabic{table}}
% \renewcommand{\thefigure}{A\arabic{figure}}
% \setcounter{figure}{0}
% \setcounter{table}{0}

% \clearpage
\section{Literature review}\label{sec:literature}

\subsection{Attitudes and perceptions}\label{subsec:literature_attitudes}

\subsubsection{Population attitudes on global policies}\label{subsubsec:literature_attitudes_policies}
Our surveys fill gaps in the knowledge of attitudes toward global policies. 
We are not aware of any other survey on a global wealth tax. 
\citet{carattini_how_2019} test the support for different variants of a global carbon tax, but their samples are representative only along gender and age, and as respondents face only one variant, the sample size for a given variant is about 167 respondents per country. They find more than 80\% of support for any variant in India, between 50 and 65\% in Australia, the UK and South Africa, and 43 to 59\% of support in the U.S., depending on the variant. The support for a global carbon tax funding an equal dividend for each human is close to 50\% in high-income countries (e.g. at 44\% in the U.S.), consistently with what we find in the OECD survey (see Figure \ref{fig:oecd}). 
Using a conjoint analysis in the U.S. and Germany, \citet{beiser-mcgrath_could_2019} find that the support for a carbon tax increases by up to 50\% % e.g. in their Fig. 4 the DE support for $70/t jumps from 26 to 39% with extension to all industrialized countries
if it applies to all industralized countries rather than just one's own country. % Variant of carbon tax is 8 (US) - 17 (DE) p.p. more likely to be preferred if tax is extended to all industrialized countries

In surveys in Brazil, Germany, Japan, the UK and the U.S., \citet{ghassim_who_2020} finds 55 to 74\% of support for ``a global democracy including both a global government and a global parliament, directly elected by the world population, to recommend and implement policies on global issues''. % (for example, international peace, world poverty, and climate change)''
Using an experiment, he also finds that, in countries where the government stems from a coalition, voting shares would shift by 8 (Brazil) to 12 p.p. (Germany) from parties who are said to oppose global democracy to parties that supposedly support it. For example, the Greens and the Left gained respectively 9 and 3 p.p. in vote intentions while the SPD and the CDU-CSU each lost 6 p.p., when Germans respondents were told that (only) the former parties support global democracy. 
\citet{ghassim_who_2020} also document survey results which show strong majorities support in each of 18 countries for the direct election of one's country's UN representative. % GlobeScan 2005; also: half/half (majorities or not depend on the country) for “Global Parliament, where votes are based on country population sizes, and the global parliament is able to make binding policies” (Synovate 2007); also: (GlobeScan 22, not from Ghassim) in 31 countries: 77% agree that “Rich countries must pay for poorer countries do deal with the effects of CC”
Similarly, in each of 10 countries, there are clear majorities in favor of ``a new supranational entity [taking] enforceable global decisions in order to solve global risks'' \citep{global_challenges_foundation_attitudes_2018}. Actually, already in 1946, 54\% of Americans agreed (and 24\% disagreed) that ``the UN should be strengthened to make it a world government with the power to control the armed forces of all nations'' \citep{gallup_seventy_1946}. 
In surveys in Argentina, China, India, Russia, Spain, and the U.S., \citet{ghassim_public_2022} find support for UN reform that would make United Nations' decisions binding, give veto powers at the Security Council to a few other major countries, and complement the highest body of the UN with a chamber of directly elected representatives. 
% TODO Schleich 16: international agreements are important but current ones are unsuccessful, people find themselves poorly represented in climate negotiations

These specific questions are in line with the answers to more general questions. In each of 36 countries, \citet{issp_research_group_international_2010} find near consensus that ``for environmental problems, there should be international agreements that [their country] and other countries should be made to follow'' (overall, 82\% agree and 4\% disagree). % No question like this in the next Envi wave in 2022
In each of 29 countries, \citet{issp_international_2019} find near consensus that ``resent economic differences between rich and poor countries are too large'' (overall, 78\% agree and 5\% disagree). 
%* Also in ISSP (19): slight minorities (in rich countries) that “People in wealthy countries should make an additional tax contribution to help people in poor countries.” p. 104, but strong majorities everywhere that “People from poor countries should be allowed to work in wealthy countries.” p. 106
Relatedly, \citet{meilland_international_2023} find that Americans and French people prefer an international settlement of climate justice even if it empedes on sovereignty. In a 2013 survey in China, Germany and the U.S., \citet{schleich_citizens_2016} show that more than 73\% of people find important future international climate agreements, while less than 26\% think that international reached so far are successful. 

%* ISSP (19): Near consensus that “Present economic differences between rich and poor countries are too large.” p. 102, slight minorities (in rich countries) that “People in wealthy countries should make an additional tax contribution to help people in poor countries.” p. 104, but strong majorities everywhere that “People from poor countries should be allowed to work in wealthy countries.” p. 106
%* Ghassim et al. (22): support for stronger UN with more direct elections.
%* Ghassim (20):  in Germany those two parties that supposedly endorse global democracy – the Greens and the Left – benefitted, gaining nine and three percentage points respectively in terms of voting intentions. Meanwhile, the traditional centrist parties – SPD and CDU – each lost six percentage points due to their supposed opposition to global democracy.
%* Beiser-McGrath & Bernauer (19): Conjoint analysis in US, DE. Variant of carbon tax is 8 (US) - 17 (DE) p.p. more likely to be preferred and 50% more likely to be supported if tax is extended to all industrialized countries (Fig 1, 4). (Unfortunately, don't test extension to global level).
%- Çarkoğlu.. (15) International Social Survey Program 2010 data reveal that people in LDCs are less supportive of international agreements forcing their country to take necessary environmental measures than are citizens in the developed world [80% instead of 85%]. (‘for environmental problems, there should be international agreements that [their country] and other countries should be made to follow.’)
%* Carattini et al. (Nature, 19): 1k in US, IA, ZA, AU, UK. Each respondent receives one variant at random of global carbon price of 40/60/80 $/t redistributed as international dividend / national dividend / mitigation in all countries / mitigation in developing countries / domestic mitigation / reduced labour tax. Immense majorities for any scheme in India, small majorities for each elsewhere except US international dividend (44%) or mitigation in developing (43%), and AU mitigation in developing (49,6%). PB: very low sample size (~167) for a given redistribution, even lower (~55) for a given variant (that also specifies the price). Appendix also contains estimation of distributive impacts. Representative only along the two quotas: gender and age. Don't give the representativeness in terms of income (the third socio-demos that they ask) so it's probably bad.

\subsubsection{Population attitudes on climate burden sharing}\label{subsubsec:literature_attitudes_burden_sharing}

Despite their differences in the description of the fairness principles, the surveys burden-sharing rules show consistent attitudes. Or at least, their various results can be made compatible with the following interpretation. 
Concerning emissions reductions, most people want that every country engage in strong decarbonization effort together, with a global quota converging to climate neutrality in the medium run. Concerning the financial effort, most people support high-emitting countries paying and low-income countries receive funding. The most supported rules are those that appear equitable, in particular an equal right to emit per person. 
% When the rankings between rules differ, it can be due to the difference in countries surveyed, but it is most often due to differences in definitions and wording. 

This interpretation helps understanding the apparent differences between articles, which approach burden sharing from different angles: cost sharing (i.e. in money terms), effort sharing (in terms of emissions reductions), or resource sharing (in terms of rights to emit). Most papers adopt the cost sharing or effort sharing approaches and preclude any country being a net receiver of money. Also, by focusing on either the financial or the decarbonization effort, these surveys miss the other half of the picture, which can explain why some papers find strong support for the ability-to-pay principle while others find strong support for grandfathering (defined as emissions reductions being the same in every country). The literature follow these approaches to stick to the terms used by the UNFCCC. Yet, we argue that the resource sharing approach is preferable to uncover attitudes, as it unambiguously describes the distributive implications of each rule while achieving an efficient location of emissions reductions and explicitly allowing for monetary gains for some countries. % TODO? say more simply that the location of emissions reductions is flexible in resource sharing
% TODO? appendix with the definitions for each author, incl. us

Now, let us summarize the different papers' results in the light of this clarification. 
\citet{schleich_citizens_2016} find an identical ranking in the support for the burden-sharing principles in China, Germany, and the U.S.: polluter-pays followed by ability-to-pay, equal emissions per capita, and grandfathering. 
% \footnote{The survey of \citet{schleich_citizens_2016} defines these rules as follows: \\
% \textit{Polluter-pays}: ``Every country has to bear costs according to the emissions it causes (hence countries causing higher emissions have a higher share of the costs).'' \\
% \textit{Ability-to-pay}: ``Every country has to bear costs according to its economic strength (hence richer countries have a higher share of the costs).''
% \textit{Egalitarianism}: ``Every country is allowed to produce the same amount of emissions per capita (hence countries with currently high emissions per capita have higher costs).''
% \textit{Sovereignty} (i.e. grandfathering): ``Every country is allowed to produce the same share of global emissions as in the past (hence the proportional reduction of emissions is the same for every country).''} 
Note that the authors do not allow for emissions trading in their description of equal \textit{emissions per capita}, which may explain its relatively low support. 
Yet, the relative support for egalitarianism also depends on how \textit{the other} rules are described. Indeed, \citet{carlsson_is_2011} find that Swedes prefer that ``all countries are allowed to emit an equal amount per capita'' rather than options where emissions are reduced in relation to current or historical emissions for which it is explicitly written that high-emitting countries ``will continue to emit more than others''. 
\citet{bechtel_mass_2013} find agreement that rich countries should pay more and historical emissions matter, but that rich countries should not be the only one to make the efforts. More precisely, their conjoint analysis in France, Germany, the UK and the U.S. shows that a climate agreement is 15 p.p. more likely to be preferred  (to a random alternative) if it includes 160 countries rather than 20, and 5 p.p. less likely to be preferred if ``only rich countries pay'' comapred other burden-sharing rules: ``rich countries pay more than poor'', ``countries pay proportional to current emissions'' or ``countries pay proportional to historical emissions''. %=> confirms preference for global policies (rather than only partial coverage). Finds that costs is what matters most: preference decreases by 30pp if it’s 2.5\% of GDP compared to 0.5\%.
Using a choice experiment, \citet{carlsson_fair_2013} find that the least preferred option in China and the U.S. is when low-emitting countries are exempted from any effort. Ability-to-pay is appreciated in both countries, though the preferred option in China is another one, which accounts for historical responsibility. %that Americans prefer capacity to pay > current responsibility > historical responsibility > equal emissions per capita while Chinese prefer historical > capacity > current > equal emissions.
%   Capacity to pay: Countries with high income levels must pay a larger share of the costs than countries with low income levels. This option says that countries with greater ability to pay should pay more
%   Current responsibility: Countries with currently high emissions levels must pay a larger share of the costs than countries with currently low emissions levels. This option says that those countries that are currently a larger part of the problem should pay more.
%   Historical responsibility: Countries with a history of high emissions levels must pay a larger share of the costs than countries with a history of lower emissions. This option recognizes that CO2 builds up in the atmosphere over many years. Thus, countries with a history of high emissions should pay more because they caused more of the problem.
%   Equal emissions pc: Countries with emissions per person greater than an agreed amount must pay, and they must pay more the higher their emissions per person are.
% > "equal emissions" is a misnomer as this is about costs (not emissions) and it's just a more progressive version of current responsibility / polluter-pay, where high-emitting pay more and low-emitting don't pay. The result for US is compatible with the other papers as Americans agree that rich countries (or high-emitting, the diff is small) should pay more. The Chinese position could also be reconciliable once we define responsibility from footprint rather than territorial and that there will be transfers from rich to poor countries.
In the U.S. and France, \citet{meilland_international_2023} find that the most favored fairness principle is that ``all countries commit to converge to the same average of total emissions per inhabitant, compatible with a controlled climate change''. Furthermore, in each country, 73\% disagree with grandfathering defined as ``countries which emitted a lot of carbon in the past have a right to continue emitting more than others in the future''. \citet{meilland_international_2023} contain many other results, for example majorities prefers to hold countries accountable for their consumption-based rather than territorial emissions, and the median choice regarding historical responsibility is to hold a country accountable for their post-1990 emissions (rather than post-1850 or just their current emissions). 
% - Meilland et al. (23) find that in US and France, most favored fairness principle is Equality in per capita emissions: "all countries commit to converge to the same average of total emissions per inhabitant, compatible with a controlled climate change" and second-most (which closely follows) is grandfathering: "all countries commit to reduce their emissions by a same proportion". 73% in each disagree with grandfathering when defined as "countries which emitted a lot of carbon in the past have a right to continue emitting more than others in the future". To rationalize these contrasted views with grandfathering, we can interpret them as: equal rights, equal emission reductions, and transfers. 
%   convergence per capita (70%): all countries commit to converge to the same average of total emissions per inhabitant, compatible with a controlled climate change
%   grandfathering (60%): all countries commit to reduce their emissions by a same proportion
%   past emissions (20% choose it among their two favorite): countries which emitted less in the past commit to reduce their emissions less than other countries
%   poor countries (20%): poorer countries commit to reduce their emissions less than richer countries
%   cost-efficiency (20%): countries where reducing emissions is more costly commit to reduce their emissions less than other countries
% - Meilland et al. (23) Other findings: people prefer international settlement on CC even if it empedes on sovereignty, a majority prefers to target footprint rather than territorial emissions, median is that countries should be held accountable for post-1990 emissions, self-serving bias when judging e.g. India vs. EU, no shared understanding of fairness when asked to coordinate between French and Americans
Finally, in each of 28 (among the largest) countries, \citet{dabla-norris_public_2023} find strong majority for ``all countries'' to the question ``Which countries do you think should be paying to reduce carbon emissions?''. Asked to choose between a cost sharing based on \textit{current} vs. \textit{accumulated historic emissions}, a majority prefers \textit{current emissions} in all countries but China and Saudi Arabia (where the two options are close to equally preferred). 


\subsubsection{Population attitudes on foreign aid}\label{subsubsec:literature_foreign_aid}

\subsubsection{Population attitudes on wealth tax}\label{subsubsec:literature_wealth_tax}

\subsubsection{Population attitudes on ethical norms}\label{subsubsec:literature_wealth_tax}
\paragraph{Universalism}
% TODO WVS on world citizenship (e.g. Bayram 15), Reysen and Katzarska-Miller 2018
%- Buntaine & Prather (18), Diedrich & Goeschl (18) Willingness to act for domestic vs. international climate action (lab experiment) READ
\paragraph{Free-riding}

\subsubsection{Second-order beliefs}\label{subsubsec:literature_beliefs}

\subsubsection{Elite attitudes}\label{subsubsec:literature_beliefs}


\subsection{Proposals and analyses of global policy-making}\label{subsec:literature_policies}

\subsubsection{Global carbon pricing}\label{subsubsec:literature_pricing}

\subsubsection{Climate burden sharing}\label{subsubsec:literature_burden_sharing}

\subsubsection{Global redistribution}\label{subsubsec:literature_redistribution}

\subsubsection{Global democracy}\label{subsubsec:literature_democracy}


% Burden-sharing
%- Agarwal & Narain (91) first to defend an equal right to emit per capita (equal to the absorbing capacity of the Earth)
%- Gampfer (14): lab experiment (ultimatum game) to test whether preferences respect fairness principles
%- Chancel & Piketty (15): global progressive carbon tax
% cf. bottom

% Global policies attitudes 
%* ISSP (19): Near consensus that “Present economic differences between rich and poor countries are too large.” p. 102, slight minorities (in rich countries) that “People in wealthy countries should make an additional tax contribution to help people in poor countries.” p. 104, but strong majorities everywhere that “People from poor countries should be allowed to work in wealthy countries.” p. 106
%* Ghassim et al. (22): support for stronger UN with more direct elections.
%* Ghassim (20):  in Germany those two parties that supposedly endorse global democracy – the Greens and the Left – benefitted, gaining nine and three percentage points respectively in terms of voting intentions. Meanwhile, the traditional centrist parties – SPD and CDU – each lost six percentage points due to their supposed opposition to global democracy.
%* Beiser-McGrath & Bernauer (19): Conjoint analysis in US, DE. Variant of carbon tax is 8 (US) - 17 (DE) p.p. more likely to be preferred and 50% more likely to be supported if tax is extended to all industrialized countries (Fig 1, 4). (Unfortunately, don't test extension to global level).
%- Çarkoğlu.. (15) International Social Survey Program 2010 data reveal that people in LDCs are less supportive of international agreements forcing their country to take necessary environmental measures than are citizens in the developed world [80% instead of 85%]. (‘for environmental problems, there should be international agreements that [their country] and other countries should be made to follow.’)
%* Carattini et al. (Nature, 19): 1k in US, IA, ZA, AU, UK. Each respondent receives one variant at random of global carbon price of 40/60/80 $/t redistributed as international dividend / national dividend / mitigation in all countries / mitigation in developing countries / domestic mitigation / reduced labour tax. Immense majorities for any scheme in India, small majorities for each elsewhere except US international dividend (44%) or mitigation in developing (43%), and AU mitigation in developing (49,6%). PB: very low sample size (~167) for a given redistribution, even lower (~55) for a given variant (that also specifies the price). Appendix also contains estimation of distributive impacts. Representative only along the two quotas: gender and age. Don't give the representativeness in terms of income (the third socio-demos that they ask) so it's probably bad.


% Global policies
% Pottier et al (17): A survey of global climate justice 
%* Hickel (17): The Divide: A Brief Guide to Global Inequality and its Solutions
%* Kopczuk et al (EER, 17) Compute optimal linear tax rate for all countries in two ways: decentralized or globally. Shows that the tax rate increases with inequality of skills (calibrated with the gini). The average decentralized rate is 0.41 The global one 0.62, with a global demogrant of 250$/month (higher than 73 countries' GDP). Show that within decentralized/country optimal taxation would not decrease global inequality by much (gini from 0.695 to 0.69, but down to 0.25 with global income tax). Show that USA don't give a damn of poor countries' people. citizens in the US (one of the richest) attach only 1/(2,000\*a) of the weight to the welfare of citizens in poorest countries, where a is the share  of transfer (supposedly) effectively arriving to the recipients. e.g. if half of aid is wasted by corrupt politicians, the weight is 1/1000.
% Carthy & Walsh (Oxfam, 22) propose various sources of funding for damages.
% Piketty (2014) "At what rate would [a global wealth tax] be levied? One might imagine a rate of 0 percent for net assets below 1 million euros, 1 percent between 1 and 5 million, and 2 percent above 5 million. Or one might prefer a much more steeply progressive tax on the largest fortunes (for example, a rate of 5 or 10 percent on assets above 1 billion euros). There might also be advantages to having a minimal rate on modest-to-average wealth (for example, 0.1 percent below 200,000 euros and 0.5 percent between 200,000 and 1 million)" He doesn't explicitly talk about revenue use, but implicitly they would be retained by each collecting country: "Le rôle principal de l'impôt sur le capital n'est pas de financer l'État social, mais de réguler le capitalisme.", "En principe, chaque pays de l'Union européenne pourrait obtenir des recettes du même ordre en appliquant seul un tel système."

% Global carbon pricing TODO! find current advocates of GCS
%* Grubb (90), Betram (92) advocate for global market with equal pc right
%* Bergh et al. (20) call for a "dual-track transition to global carbon pricing": an expanding climate club, and "a reorientation of UNFCCC negotiations creates room for talking seriously about a global carbon price schedule, including redistribution-of-revenues rules." They don't specify which equity rules to use.
%* Jamieson (01) advocates of equal pc burden-sharing (after the precursors Agarwal & Narain (91))
%* Bear et al (Science, 00), Bear (02), Athanasiou & Baer (02) advocate for equal pc burden-sharing (although weirdly, Bear & Athanasiou then change mind and advocate for the Greenhouse Development Rights, accounting for capacity and responsibility)
%* Cramton et al (17): Livre de pontes. Tout le monde est d'accord : un prix mondial du carbone est requis, il ne peut être obtenu que par la réciprocité des engagements (style climate club), et il faut quelques transferts des riches vers les pauvres ainsi que des sanctions commerciales pour aligner les incitations. Ch 4 (also Cramton et al 15) propose la formule suivante de transfert (positif ou négatif) à un fonds climat : générosité*émissions en excès (par rapport à la cible)*prix du carbone. On demanderait aux États autour de la moyenne d'émission de fixer ce paramètre de générosité, pour qu'il soit fixé de sorte à maximiser le prix, puis on fixerait le prix comme le prix minimum proposé (après avoir éjecté qqs pays récalcitrants des négos). Puis, sanctions commerciales pour ceux qui ne respectent pas le prix. Ch Gollier & Tirole proposent une formule aussi simple que l'autre : quota global*((1-g)*part des émissions à t=0 + g*part de la population), où g joue le même rôle de paramètre de générosité/éthique (que je voudrais mettre à 1, mais qu'ils disent tous de mettre < 1 pour que les pays riches acceptent. Le livre argumente bcp sur prix vs. quantité (TLM préfère prix sauf Gollier & Tirole), l'argument le plus convaincant en faveur du prix c'est qu'avec la procédure proposée le prix négocié serait le plus élevé possible, alors qu'avec la quantité c'est le budget carbone qui serait le point focal et ça aboutirait à une impasse (objectif trop ambitieux).
%* MacKay et al (Nature, 15) summarizes the above
%* Weitzman (17) advocates for a World Climate Assembly, choosing the price level with the median voter, and each country retaining the revenues.
% Fleurbaey & Zuber (13): The discount rate converges to the worst-off (affected by the measure) to the worst-off (beneficiary of the measure) discount rate, which depends on the growth between both agents. Applied to real data, we can consider that the worst-off affected by a global tax on CO_2 is the average-earner on earth (around 75% centile i.e. ~1000€/month, cf. Chancel & Piketty, Lakner & Milanovic, Chakravorty) while the worst-off beneficiary is the worst-off person in the future (among those less affected by CC thanks to the measure), probably below 1000€/month => negative discount rate.
% Stanton (11): Negishi weights obviate the IAMs’ equalization of income. 4 ways to solve this problem: 1. be more transparent, 2. stop weighting, 3. take linear utility (i.e. maximize global GDP), 4. stop optimizing. 
% Hoel (91): Shows that an international tax can be designed so that it is both efficient and satisfies whatever distributional objectives one might have.
% IMF (2019): global pricing (with either differentiated prices or international transfers) or, as a first step, a carbon price floor. 25% of revenues should be rebated to the bottom 40%, the rest used to reduce distortionary taxes or for green investments. Estimate that $75/t is needed in 2030 for 2°C.
% Parry et al (21): Proposal for an International Carbon Price Floor Among Large Emitters. Acknowledges that transfers could be necessary to induce climate action in low/middle-income countries, talks about transferring 1% of carbon revenues.
%- Sager: distributive effects of global pricing without int'l transfers.
%- Budolfson et al. (incl. Fleurbaey, Méjean, Zuber, Dennig) (21): global carbon price with within-country per capita dividend. Acknowledge that "The overall benefits to society are even greater if total carbon tax revenues are returned on an equal per capita basis globally, which directs more of the revenues towards the poorest populations in the world (rather than the poorest within each country or region)." Very short (3p, no appendix, no suppl. info)

% Foreign aid TODO: find more recent, check .lyx for already written paragraph
%* Kaufmann et al (12) Shows the level of perceived and desired aid in 26 countries between 2005 and 2008 (cf. Table 1). In most countries (incl. UK, DE, FR, ES but not U.S.) desired aid is larger than perceived. Argue that this is due to political influence efforts/possibilities of the rich, as they prefer less aid due to vested interests (support this by a theoretical model + correlations between level of lobbying and actual aid level, controling for desired aid). In most countries the gap between the two is small, except in the U.S. where perceived is 7.5% of GDP and preferred is 3%.Use WVS and Gallup (like Chong & Gradstein, Paxton & Knack) but have more waves and the others don't use the question on perceived aid. Shows that richer want less aid ("those in the top income quintile favour ODA (as a share of GNI) that is 0.13 percentage points lower than the preferred share for individuals in the bottom 40\% of the income distribution" after controling for perceived aid - our regression results are sensibly the same.). from 0 to higher than 25%: threshold at 0.05; 0.15; 0.35; 0.75; 1.5; 2.5; 4; 7.5; 17.5; 25, i.e. same number of thresholds but small than ours below 2.5 and higher above. 
%*? Milner & Tingley (13): (highly cited but no original data, don't think we need to cite it) In 2008, 44% of American wanted foreign aid cut (american elections study, 08). fraction of federal budget going to foreign aid (mean: 27%, median: 25%) / should go (mean: 13%, median: 10%) (WorldPublicOpinion, 10)
% PIPA (01): Overwhelming majorities support a multilateral effort to cut hunger in half by the year 2015 and say that they would be willing to pay for the costs of such a program. However, most do not think that the average American would be as willing to pay the necessary costs. when PIPA asked respondents to estimate how much of the federal budget was devoted to foreign aid, the median estimate was 15% -- 15 times the actual amount, which was just under 1%. More dramatically, when asked what an appropriate percentage would be, the median response was 5% -- 5 times the actual amount. And when asked to imagine that they heard the real amount was only 1%, only 18% of respondents said they thought that would be too much--as compared to the 75% who had initially said that the US was spending too much. what percentage of their "tax dollars that go to help poor people at home and abroad...should go to help poor people in other countries." The mean response was 16% (down a bit from 22% in response to this question in a 1996 PIPA poll). Strikingly, this turns out to be a far higher percentage than is currently given. In 1999, a bit less than 4% of the total spent on the poor went to the poor abroad. Sixty percent of respondents proposed a percentage that was higher than 4%.
%- DFID (10): Priorities: 1 NHS, 2 education, 3 support to poor countries, 4 police, 5 defence (p. 19). Show majority support for increased aid until 07, then median is to support stable aid (due to crisis?). It seems they don't give the info on actual amount though.
%* PIPA (08): Across 20 countries, 81% support that "developed countries have a moral responsibility to help reduce hunger ansevere poverty in poor countries (majority in every country). “the World Bank (Shantayanan et al, 2002) has estimated that it will require an extra US$39-54 billion per year to meet Millennium Development Goal 1 (MDG1). (…) The per person cost of meeting MDG1 came to £25 for the UK, $56 for the US, €27 for Germany, and so on. On average 77 per cent of respondents are in favour of contributing towards meeting the goal (provided that all others do too). To take the US example, 75 per cent of people supported paying an extra $56 per year to meet MDG1. What is significant about this figure is that it is only slightly below the support for the ‘cost free’ question as to whether the US should be willing to share a  small portion of its wealth with those who are in great need (79%).” Hudson & van Heerde (12)
%* Hudson & van Heerde (12):Reviews literature on foreign aid and criticizes it on a number of points (e.g. not uncovering the determinants, and not asking well the questions). Shows strong support for poverty alleviation, (at least partly) out of intrinsic altruism. Use 4 main sources: PIPA (01, 08) UK DIDP, Eurobarometer; cf. Table 1 for all surveys on foreign aid / Public support for development has been famously described as a mile wide and and inch deep (Smillie, 1996: ref impossible to find). Hard times at home have meant that public support appears to have turned against international development efforts (Henson and Lindstrom, 2010). / Monitor public support: (Fransman and Solignac Lacomte, 2004; McDonnell et al, 2003), Paxton and Knack, 2008; Chong & Gradstein 2006. Review surveys on aid. / ~75% support aid in developed countries (stable) but ‘84 per cent agreed with the assertion that ‘taking care of problems at home is more important than giving aid to foreign countries’ (PIPA, 2001:9).” / References on covariates of aid support / PIPA 2001, "On average, Americans thought just under 25 per cent of the US budget was allocated to foreign aid, and government should allocate less than 14 per cent of the national budget. However, when told that US spends approximately 1 per cent of the federal budget on foreign aid, 37 per cent of respondents thought this was too little, 44 per cent thought it was about right, and 13 per cent thought it too much."  Think that only 23% of aid really goes to the poor / “The 2009 UK survey, Public Attitudes towards Development, reports ‘public support for overseas aid’ at 72 per cent (DFID, 2009); while in the US support was a comparable 79 per cent (PIPA, 2001); and average support across the EU trends slightly higher than in the US and UK with 91 per cent saying it was either very (53%) or fairly (38%) important to provide aid to poor countries (Eurobarometer, 2005).” / “DFID has now begun asking questions that provide relative measures of the salience of development aid vis-à-vis other competing policy issues (DFID, 2009; IDC, 2009). / "high proportion (61%) of US citizens who felt that the US spends too much on foreign aid. [from another source]” / “The distinction between foreign aid, which includes military spending, and development aid/assistance is an important one” / “81 per cent of respondents believed that developed countries do have a moral responsibility to work towards reducing hunger and severe poverty (WorldPublicOpinion.org, 2008). (…) there are a good number of people who support aid despite the fact they do not think it works. What this suggests – but cannot show in any detail – is that people have nonutilitarian motives for supporting aid.” / “support for development assistance is highly contingent on respondents’ perceptions of the effectiveness of aid, especially with regard to corruption (Henson et al, 2010). For example, in the UK, 47 per cent of respondents thought that aid was wasted, with sizable majorities citing corruption and poor management and/or delivery as primary factors (DFID, 2008). More disconcertingly, US respondents thought that only 23 per cent of US aid money that goes to poor countries ends up helping the people who really need it and 54 per cent of US aid money that goes to poor countries ends up in the pockets of corrupt government officials (PIPA, 2001). (…) international charities and NGOs are deemed best suited/most effective compared to donor countries” / UK ‘MyAid’ plan – where the public gets to vote on how a pot of money should be distributed – / "public engagement should be about ‘opening up the political and wider societal space to the possibility of deeper change’ (Darnton and Kirk, 2011:14).”
%* Gilens (01) 17% fewer American with high political knowledge want to cut foreign aid when we provide them specific information about aid amount.
%- Chong & Gradstein (16): from WVS 95-99, 58% want that their country give more foreign aid (but misperceptions are not taken into account)
%* Bauhr et al (13): Support for aid is reduced by perception of corruption in recipient countries. However, this effect is reduced by the aid-corruption paradox (and other things): most corrupt countries need more help.
%- Nair (18): (lack of) Aid support in US driven by information on global distribution, because people underestimate their rank by 27 centiles and overestimate global median income by a factor 10.
%- Williamson (19): Public Ignorance or Elitist Jargon? Reconsidering Americans’ Overestimates of Government Waste and Foreign Aid. "Foreign aid" encompasses military spending, in the mind of American.
%- McDonnell et al (03) Public Opinion and the Fight against Poverty
%- Nair (16): preferences driven by worldviews rather than self-interest
%- Bodenstein & Faust (17): Determinants of support for aid conditionality. They are: perceived corruption in donor country, right-wing.
%- Scotto et al (17): We Spend How Much? Misperceptions, Innumeracy, and Support for the Foreign Aid in the United States and Great Britain. Less American and British want aid cut when information on current aid is given in % of GDP rather than in $.
%* Paxton & Knack (12): Majorities want more aid, and main determinants are trust, ideology, interest in politics, and female (all positive). Gallup 02: in US 45% want more aid (rather than stable) vs. 68-91 in DE-UK-ES. Like Chong & Gradstein, find that desired aid increases with income, contrary to Kaufmann et al. but the latter contains more datasets.
%- Wood (15): Determinants for aid support in Australia. Wood (18) Examine Australian support for aid: although there is support to help foreign poor, people back recent aid cuts.
%- Bayram (17): Aid support associated with trust, i.e. seeing integrity and trustworthiness in others.
%- Cheng & Smyth (16): Why Give it Away When You Need it Yourself? Understanding Public Support for Foreign Aid in China. Political ideology and patriotism main explaining variables for aid support. People in poorer provinces less supportive.
%- Milner & Tingley (10) theory + empirics: who supports aid and why. owners of capital in donor countries tend to gain from aid and thus are more likely to support giving aid
%- Easterly (JEP, 03) Can Foreign Aid Buy Growth? No (disproves Hansen & Tarp).
%- Hansen & Tarp (01) Aid increases growth (empirical evidence)
%- Tresch et al. (22): 66% of Swiss people want to increase their foreign aid
%- Harris (17): majority of French want to decrease foreign aid


% Universalism
%- Enke et al. (Manag. Science, 23): measures universalism by asking to split donation to domestic and foreigner of same absolute income (US).
%- Enke et al. (ReStud, 23): unviersalism more correlated to policy attitudes than income, education, religiosity or beliefs about government efficiency (West).
%- Cappelen et al. (NBER, 22): how unviversalism (as measured above) varies across countries. Comparable in Europe and US (lower in China, higher in Africa)
%- Cherry et al (17) show in the lab that some people prefer policies detrimental to them due to their worldview.


% Free-riding
%- Mildenberg (2019): people are not free riders
%- McGrath & Bernauer (17): review paper. people are not free riders. Preferences concerning climate policy tend to be driven primarily by a range of personal predispositions and cost considerations, which existing research has already explored quite extensively, rather than by considerations of what other countries do
%- Bernauer & Gampfer (15): US and IA people are not free riders. They each overestimate their country's emissions at one third of global total.


% Social norms
%- Bursztyn et al. (AER, 20): social norms can change following new public information such as unexpected election outcome. After Trump election, people express more xenophobic views and judge less severely those who do.
%- Farrow et al. (17): review of effect of social norm intervention on environmental attitudes

% Incentive compatibility
%- Danz et al


% Second-order beliefs
%* Mildenberg & Tingley (19): survey elites (Congress staffers, scholars) and public in U.S. and China and show pluralistic ignorance of pro-climate attitudes, egocentric bias, and increasing support after beliefs are updated.
%- Bursztyn & Yang (21): Review of the field. Misperceptions about others are widespread, asymmetric, much larger when about out-group members, and positively associated with one’s own attitudes.
%- Drews et al. (22): in Spain, supporters (resp. opponents) of carbon tax overestimate (resp. underestimate) support. Providing information doesn't change the overall support.
%* Falk et al. (21): Respondents vastly underestimate the prevalence of climate- friendly behaviors and norms among their fellow citizens. Providing respondents with correct information causally raises individual willingness to fight climate change as well as individual support for climate policies. The effects are strongest for individuals who are skeptical about the existence and threat of global warming.
%- Di Tella et al. (AER, 15): The results of the lab experiment favor the hypothesis that people avoid altruistic actions by distorting beliefs about others' altruism
%- Allport (1924): first book on pluralistic ignorance
%- Allport (40): function of poll is to correct pluralistic ignorance
%- Studies on pluralistic ignorance: business (Buckley et al. 00), against affirmative action (Van Boven 00), political correctness (Braghieri, AER 21), alcohol (Suls & Green, 03), white support for racial segregation (O'Gorman 75), CC (Geiger & Swim 16), hooking up (Lambert et al 03, cf. note for paragraph of pluralistic ignorance), women working outside home in Saudi Arabia (Bursztyn et al. 20)
%- Geiger & Swim (16) Shows that pluralistic ignorance of others' concern about CC leads people to talk less about CC and self-silence themselves.
%- Miller & MacFarland (87) Shows that pluralistic ignorance emerges because individuals believe that fear of embarrassment is a sufficient cause for their own behavior but not for the behavior of others.


% Elite surveys TODO find more
%* Mildenberg & Tingley (19): Congress staffers, cf. second-order beliefs
%- Hertel-Fernandez et al. (2019): Survey on US Congress staffers (not on climate)
%- Milner & Tingley (10) (not sure it's a survey) owners of capital in donor countries tend to gain from aid and thus are more likely to support giving aid
%- Lange et al. (Energy Econ, 2007): climate negotiators
%- Lange et al. (EER, 2010): same data as Lange et al. (10)
%- Dannenberg et al. (ERE, 2010): elicit climate negotiators’ equity preferences using Fehr & Schmidt (99) method => regional differences in addressing climate change are driven more by national interests than by different equity concerns
%- Kesternich et al. (EEPS, 2020): survey on climate negotiators about their preferred burden-sharing rules: we observe tendencies for a more harmonized view among key groups towards the ability-to-pay rule in a setting of weighted burden sharing rules
%- Lange & Schwirplies (ERE, 2017): combines Lange et al. (10) and Schleich et al.
%* Hjerpe et al. (2011): Delegates at COP2009. The results indicate that voluntary contribution, indicated as willingness to contribute, was the least preferred principle among both negotiators and observers. Three of the four principles for allocating mitigation commitments were recognized widely across the major geographical regions: historic 1990, capacity to pay, and equal per capita emissions. The difference was never below 25 percentage units, and the opponent share never exceeded 16%.
%- Scholte et al. (2020)
%- Bayram (17): cosmopolitanism of German politicians and their respect of international law


% Global poverty gap
%* Bolch et al. (22)
%- Zhang (16) estimates the poverty gap in each country. Global one is at $80G/year.


% Basic income TODO find more
%* Egger et al. (19): positive gen eq effects. We provided one-time cash transfers of about USD 1000 to over 10,500 poor households across 653 randomized villages in rural Kenya. The implied fiscal shock was over 15 percent of local GDP. We find large impacts on consumption and assets for recipients. Importantly, we document large positive spillovers on non-recipient households and firms, and minimal price inflation.
%* Haushofer & Shapiro (16): The Short-term Impact of Unconditional Cash Transfers to the Poor: Experimental Evidence from Kenya. Monthly transfers are more likely than lump-sum transfers to improve food security


% Unequal exchange / embodided labour
%- Reyes et al (17)
%- Sakai et al (17)
%- Alsamawi et al. 2014


% NDCs assessments or burden-sharing computations. TODO check the contraction & convergence scheme proposed by France
%- Bourban (18): Soutient un marché du carbone avec droits en proportion des émissions cumulées depuis 1990. Et des “mesures volontaires de contrôle de la population mondiale”.
%- Raupach et al (NCC, 14) 
%- Grasso (2012)
%- van den Berg et al (20)
%- Meyer (04) Contraction and Convergence (i.e. grandfathering converging to equal pc, within an ETS)
%- >Baer et al (08)< (cite this one, others don't give more info), Baer (13), Athanasiou et al (22), Holz et al (19) https://calculator.climateequityreference.org/ Athanasiou, Greenhouse Development Rights, EcoEquity calculator, US fair share. Effort-sharing approach based on splitting emissions reductions in function of capacity to pay (~ share of global income in top 30%) and responsibility (share of emissions since 1950), weighted equally. Corresponds to UNFCCC wording. Pb of this method (applying to any choice of parameters): A country with relatively low incomes (e.g. equal distribution slightly above the p70) and that has few historical responsibility would have a relatively low effort. Even more problematic, the **poorest countries would have virtually 0% of the effort, hence they would be allowed to emit following the baseline trajectory… but this baseline is not fair; it amounts to grandfathering**. It is computed as the “product of the projected GDP and CO2 emission intensity”. ([https://climateequityreference.org/calculator-information/gdp-and-emissions-baselines/](https://climateequityreference.org/calculator-information/gdp-and-emissions-baselines/)), and give for example 0.8tCO2e/cap for RDC in 2030 (16% more than in 2020, but lot lower than the objective of ~4t). => Compared to an equal right to emit pc, this method favors countries like China (allowed to remain stable over 2020-30 vs. reduced by 35-40%) and penalizes countries like the U.S. and Africa. 
%  in Athanasiou et al (22) Justification of Greenhouse Development Rights instead of Equal per capita right is on p. 36. It is weak, and basically that historical responsibility should be taken into account. Conversely, justification against historical resp. is that the latter doesn’t take into account capacity to pay (it is not said like this, but we can think of ex-USSR).
%- Pachauri et al. (Science, 2022) 
%- Robiou du Pont et al. (NCC, 2016)
%- Robiou du Pont et al. (ERL, 2016)
%- Höhne et al. (Climate Policy, 2014): review of 40 papers
%- Gao et al. (FEM, 2019)
%- Gignac & Matthews (ERL, 15)
%- Matthews (16) Quantifying carbon debts among nations
%- https://climateequitymonitor.in/ computes carbon debt based on equal per capita cumulative emissions. contact@climateequitymonitor.in https://twitter.com/equity4climate


% Mismatch between preferences and climate action
%- McCright & Dunlap (03) show that it's an organized conservative movement that succeeded in the U.S. not ratifying Kyoto, through lobbying and disinformation.


% Wealth tax attitudes
% look for surveys on global tax => I've found no result with survey or attitudes + "global tax" or "global wealth tax" in google scholar
% Fisman et al (17): Americans want a 3% tax on inherited wealth
%- Christensen et al. (Oxfam, 23) p. 32 gives references on rich tax attitudes, with always strong majority support:
%* OECD (19): 52-80% of absolute support for "government tax the rich more than they currently do in order to support the poor" in 21 OECD countries
%* Isbell (22): 34 African countries
%- Patriotic Millionaires (22), UK
%- Americans for Tax Fairness (21), US
%- Gallup (22), US
%- Fight Inequality Alliance India (22), IA

% Different framing of burden-sharing, depending on what should be split:
% - mitigation costs: this is the most used as it is easiest to explain. The issue is that it is not specified how agents pay (or if some agents receive payments) and implicitly, there is no negative costs (transfers exceeding the costs) and the carbon price is not uniform. Used in .
% - emission: this one is vague as it doesn't state at which date emissions pc converge (if they do) and whether there are side payments.
% - emission rights: this one is the most accurate as there is no need of a BAU scenario to compute the mitigation needed and its cost.

% Different fairness principles:
% - equal emission right per capita: using this as a baseline, we can call 'grandfathering' any principle that is more regressive and 'historical responsibility' any principle that is more progressive
% - equal emission reduction (in share of current emission) per capita: grandfathering
% - emission rights proportional to current emissions: grandfathering
% - costs proportional to current emissions: polluter-pay principle
% - costs proportional to cumulative emissions: so-called historical responsibility but may actually have a grandfathering component

% Surveys of population:
% - Schleich et al. (Climate Policy, 16) ask for ranking (TODO check) and find an identical ranking of fairness principles in China, Germany, and the US: accountability (costs according to emissions) followed by capability (according to economic strength), egalitarianism (equal emission per capita), and sovereignty (constant share of global emission) (see Lange & Schiwplies (17) for the computations). 
%   Polluter-pays: Every country has to bear costs according to the emissions it causes (hence countries causing higher emissions have a higher share of the costs).
%   Ability-to-pay: Every country has to bear costs according to its economic strength (hence richer countries have a higher share of the costs).
%   Egalitarian: Every country is allowed to produce the same amount of emissions per capita (hence countries with currently high emissions per capita have higher costs).
%   Sovereignty: Every country is allowed to produce the same share of global emissions as in the past (hence the proportional reduction of emissions is the same for every country).
% other findings: international agreements are important but current ones are unsuccessful, people find themselves poorly represented in climate negotiations
% - Bechtel & Scheve (PNAS, 13) find with a conjoint analysis on FR, DE, UK, US that a climate agreement is 5 p.p. less likely to be preferred (to a random alternative) if only rich countries pay (other burden-sharing are: pay prop. to current emissions / historical emissions / rich countries pay more than poor countries) [TODO: check SI that these are the verbatim] and 15 p.p. more likely to be preferred if it includes 160 (out of 192) countries rather than 20 => confirms preference for global policies (rather than only partial coverage). Finds that costs is what matters most: preference decreases by 30pp if it’s 2.5% of GDP compared to 0.5%.
% - Carlsson et al. (REE, 13) find using a 09 choice experiment that Americans prefer capacity to pay > current responsibility > historical responsibility > equal emissions per capita while Chinese prefer historical > capacity > current > equal emissions.
%   Capacity to pay: Countries with high income levels must pay a larger share of the costs than countries with low income levels. This option says that countries with greater ability to pay should pay more
%   Current responsibility: Countries with currently high emissions levels must pay a larger share of the costs than countries with currently low emissions levels. This option says that those countries that are currently a larger part of the problem should pay more.
%   Historical responsibility: Countries with a history of high emissions levels must pay a larger share of the costs than countries with a history of lower emissions. This option recognizes that CO2 builds up in the atmosphere over many years. Thus, countries with a history of high emissions should pay more because they caused more of the problem.
%   Equal emissions pc: Countries with emissions per person greater than an agreed amount must pay, and they must pay more the higher their emissions per person area.
% > "equal emissions" is a misnomer as this is about costs (not emissions) and it's just a more progressive version of current responsibility / polluter-pay, where high-emitting pay more and low-emitting don't pay. The result for US is compatible with the other papers as Americans agree that rich countries (or high-emitting, the diff is small) should pay more. The Chinese position could also be reconciliable once we define responsibility from footprint rather than territorial and that there will be transfers from rich to poor countries.
% - Carlsson et al. (Ecol Eco, 11) find that Swedes prefer that "all countries are allowed to emit an equal amount per capita" rather than options where emissions reduce in relation to current or historical emissions and continue to be higher in high-emitting countries. 
% - Meilland et al. (23) find that in US and France, most favored fairness principle is Equality in per capita emissions: "all countries commit to converge to the same average of total emissions per inhabitant, compatible with a controlled climate change" and second-most (which closely follows) is grandfathering: "all countries commit to reduce their emissions by a same proportion". 73% in each disagree with grandfathering when defined as "countries which emitted a lot of carbon in the past have a right to continue emitting more than others in the future". To rationalize these contrasted views with grandfathering, we can interpret them as: equal rights, equal emission reductions, and transfers. 
%   convergence per capita (70%): all countries commit to converge to the same average of total emissions per inhabitant, compatible with a controlled climate change
%   grandfathering (60%): all countries commit to reduce their emissions by a same proportion
%   past emissions (20% choose it among their two favorite): countries which emitted less in the past commit to reduce their emissions less than other countries
%   poor countries (20%): poorer countries commit to reduce their emissions less than richer countries
%   cost-efficiency (20%): countries where reducing emissions is more costly commit to reduce their emissions less than other countries
% Other findings: people prefer international settlement on CC even if it empedes on sovereignty, a majority prefers to target footprint rather than territorial emissions, median is that countries should be held accountable for post-1990 emissions, self-serving bias when judging e.g. India vs. EU, no shared understanding of fairness when asked to coordinate between French and Americans
% - Dechezleprêtre et al. (WP, 22) find that equal per capita right > historical responsability, capabilities > grandfathering; that global CC policies are needed; 50% support for global T&D; strong support for global tax on millionaires; no free-riding. TODO: check FR, US wording
% - Dabla-Norris et al. (WP, 23) find strong majority for “all countries” everywhere in “Which countries do you think should be paying to reduce carbon emissions?”, and majority for current rather than historical in all countries but China and Saudi Arabia in “Should countries be paying to reduce carbon emissions based on their current or accumulated historic levels of emissions?”

% > Position making all this compatible: people want that every country engage in strong decarbonization effort together, with a global quota, converging to climate neutrality in the medium run, based on an equal right to emit per person, implying that rich countries pay and low-emitting countries receive funding. Where the rankings differ, it is likely because the definitions or wordings are different, and also because it involves different countries (Sweden != US != China).
% - Schleich find support for costs according to emissions and against immediate equalization of emissions (but nothing against convergence to equal emissions per capita).
% - This is just in contradiction with Carlsson (11) which finds that Swedes prefer the equalization (with a similar wording) to other reduction options. TODO: check wording of the latter.
% - Bechtel find agreement that rich countries should pay more and historical emissions matter, but just that they should not be the only one to make the efforts. 
% - Carlsson (13) find that the least preferred option in China and US is when low-emitting countries don't participate to the effort. Ability to pay is liked in both countries.
% - Meilland find that convergence is the most preferred, followed by emission reductions of same proportion, disagreement with grandfathering expressed in terms of emission rights.
% - Dechezleprêtre find support for equal right is strongest, although historical responsibility and capabilities are also supported. The quota system is strongly supported.

% Surveys of negotiators:
% - Hjerpe et al. (WP, 11)
% - Dannenberg et al. (ERE, 10): measuring negotiators' equity preferences, regional differences in addressing climate change are driven more by national interests than by different equity concerns.
% - Lange et al. (Energy Econ, 07): Mix of self-serving bias and support for egalitarian principle.
% - Kesternich et al. (EEPS, 21): kind of convergence on ability-to-pay.

% Other papers:
% - Lange & Schwirplies (ERE, 17) develop a theoretical model (building on Buchholz et al. (05)), supported by data, justifying that climate negotiators (chosen by the citizens) have lower environmental preferences than their citizens and equity views more aligned with the other negotiators. 
% \clearpage
\section{Raw results% from the complementary surveys
}\label{app:raw_results}
% /!\ Do not replace by app_desc_stats_US1 as the latter also contains figures that are already in the main text
% TODO? add country-specific prioritization? No, it's in (separate) country appendices.
% TODO! add share who click on info or reminder
% TODO! Appendix Sources or at least clean up specificities.xlsx

Country-specific raw results are also available as supplementary material files:  \href{https://github.com/bixiou/global_tax_attitudes/raw/main/paper/app_desc_stats_US.pdf}{US}, \href{https://github.com/bixiou/global_tax_attitudes/raw/main/paper/app_desc_stats_EU.pdf}{EU}, \href{https://github.com/bixiou/global_tax_attitudes/raw/main/paper/app_desc_stats_FR.pdf}{FR}, \href{https://github.com/bixiou/global_tax_attitudes/raw/main/paper/app_desc_stats_DE.pdf}{DE}, \href{https://github.com/bixiou/global_tax_attitudes/raw/main/paper/app_desc_stats_ES.pdf}{ES}, \href{https://github.com/bixiou/global_tax_attitudes/raw/main/paper/app_desc_stats_UK.pdf}{UK}.

\begin{figure}[h!]
    \caption[Absolute support for global climate policies]{Absolute support for global climate policies. \\ Share of \textit{Somewhat} or \textit{Strongly support} (in percent, $n$ = 40,680). The color blue denotes an absolute majority. See Figure \ref{fig:oecd} for the relative support. (Questions \ref{q:scale}-\ref{q:millionaire_tax} of the global survey. Reproduced from \citealp{dechezlepretre_fighting_2022}, Figure A20.)} 
    \makebox[\textwidth][c]{\includegraphics[width=1.2\textwidth]{../figures/OECD/Heatplot_global_tax_attitudes_positive.pdf}}\label{fig:oecd_absolute}% with dependence on others (absent from OECD): Heatplot_burden_share_all_positive_countries
    {\footnotesize *In Denmark, France and the U.S., the questions with an asterisk were asked differently, cf. Question \ref{q:burden_sharing_asterisk}. } 
\end{figure}

\begin{figure}[h!]
    \caption[Comprehension]{Correct answers to comprehension questions (in percent). (Questions \ref{q:understood_gcs}-\ref{q:understood_both})}\label{fig:understood_each}
    \makebox[\textwidth][c]{\includegraphics[width=\textwidth]{../figures/country_comparison/understood_each_positive.pdf}} 
\end{figure}

\begin{figure}[h!]
    \caption[Comprehension score]{Number of correct answers to comprehension questions (mean). (Questions \ref{q:understood_gcs}-\ref{q:understood_both})}\label{fig:understood_score}
    \makebox[\textwidth][c]{\includegraphics[width=\textwidth]{../figures/country_comparison/understood_score_mean.pdf}} 
\end{figure}

% \begin{figure}[h!]
%     \caption[Support for the Global Climate Scheme]{Support for the GCS, NR and the combination of GCS, NR and C. (Questions \ref{q:gcs_support}, \ref{q:nr_support} and \ref{q:crg_support})}\label{fig:support_binary}
%     \makebox[\textwidth][c]{\includegraphics[width=.9\textwidth]{../figures/country_comparison/support_binary.pdf}} 
% \end{figure}

% \begin{figure}[h!]
%     \caption[Beliefs about support for the GCS and NR]{Beliefs regarding the support for the GCS and NR. (Questions \ref{q:gcs_belief} and \ref{q:nr_belief})}\label{fig:belief}
%     \makebox[\textwidth][c]{\includegraphics[width=.8\textwidth]{../figures/country_comparison/belief.pdf}} 
% \end{figure}

\begin{figure}[h!]
    \caption[List experiment]{List experiment: mean number of supported policies. (Section \ref{subsubsec:list_exp}, Question \ref{q:list_exp})}\label{fig:list_exp}
    \makebox[\textwidth][c]{\includegraphics[width=.7\textwidth]{../figures/country_comparison/list_exp_mean.pdf}} 
\end{figure}

\begin{figure}[h!]
    \caption[Conjoint analyses 1 and 2]{Conjoint analyses 1 and 2. (Questions \ref{q:conjoint_a}-\ref{q:conjoint_b}, Back to Section \ref{subsubsec:conjoint})}\label{fig:conjoint}
    \makebox[\textwidth][c]{\includegraphics[width=.8\textwidth]{../figures/country_comparison/conjoint_ab_all_positive.pdf}} 
\end{figure}

% \begin{figure}[h!] % already in text
%     \caption{[Asked only to non-Republicans] Conjoint analysis n°4: random programs at the Democratic primary. (Question \ref{q:conjoint_r})}\label{fig:ca_r}
%     \makebox[\textwidth][c]{\includegraphics[width=\textwidth]{../figures/country_comparison/ca_r.png}} 
% \end{figure}

% \begin{figure}[h!]
%     \caption[Influence of the GCS on preferred platform]{Influence of the GCS on preferred platform:\\ Preference for a random platform A that contains the Global Climate Scheme rather than a platform B that does not (in percent). (Question \ref{q:conjoint_d}; in the U.S., asked only to non-Republicans.)}\label{fig:conjoint_left_ag_b}
%     \makebox[\textwidth][c]{\includegraphics[width=\textwidth]{../figures/country_comparison/conjoint_left_ag_b_binary_positive.pdf}} 
% \end{figure}

\begin{figure}[h!]
    \caption[Perceptions of the GCS]{Perceptions of the GCS. Elements seen as important for supporting the GCS in a 4-Likert scale (in percent). (Question \ref{q:gcs_important})  \hfill (Back~to~Section~\ref{subsubsec:pros_cons})}\label{fig:gcs_important}
    \makebox[\textwidth][c]{\includegraphics[width=\textwidth]{../figures/country_comparison/gcs_important_positive.pdf}} 
\end{figure}

\begin{figure}[h!]
    \caption[Classification of open-ended field on the GCS]{Perceptions of the GCS. Elements found in the open-ended field on the GCS (manually recoded, in percent). (Question \ref{q:gcs_field}) \hfill (Back~to~Section~\ref{subsubsec:pros_cons})}\label{fig:gcs_field}
    \makebox[\textwidth][c]{\includegraphics[width=.75\textwidth]{../figures/country_comparison/gcs_field_positive.pdf}} 
\end{figure}

\begin{figure}[h!]
    \caption[Topics of open-ended field on the GCS]{Perceptions of the GCS. Keywords found in the open-ended field on the GCS (automatic search ignoring case, in percent). (Question \ref{q:gcs_field}) \hfill (Back~to~Section~\ref{subsubsec:pros_cons})}\label{fig:gcs_field_contains}
    \makebox[\textwidth][c]{\includegraphics[width=\textwidth]{../figures/country_comparison/gcs_field_contains_positive.pdf}} 
\end{figure}

\begin{table}[h]
    \caption[Campaign and bandwagon effects on the support for the GCS.]{Effects on the support for the GCS of a question on its pros and cons and on information about the actual support, in the U.S. (See Section \ref{subsec:questionnaire_perceptions} in the US2 Questionnaire)  \hfill (Back~to~Section~\ref{subsubsec:pros_cons})} \label{tab:branch_gcs}
    \makebox[\textwidth][c]{
        
\begin{tabular}{@{\extracolsep{5pt}}lcccc} 
\\[-1.8ex]\hline 
\hline \\[-1.8ex] 
 & \multicolumn{4}{c}{Support} \\ 
\cline{2-5} 
\\[-1.8ex] & \multicolumn{2}{c}{Global Climate Scheme} & \multicolumn{2}{c}{National Redistribution} \\ 
\\[-1.8ex] & (1) & (2) & (3) & (4)\\ 
\hline \\[-1.8ex] 
Control group mean & 0.557 & 0.557 & 0.569 & 0.569  \\ \hline \\[-1.8ex]
 Treatment: Open\mbox{-}ended field on GCS pros \& cons & $-$0.073$^{**}$ & $-$0.073$^{**}$ & $-$0.035 & $-$0.031 \\ 
  & (0.035) & (0.031) & (0.035) & (0.032) \\ 
  Treatment: Closed questions on GCS pros \& cons & $-$0.109$^{***}$ & $-$0.096$^{***}$ & $-$0.065$^{*}$ & $-$0.062$^{**}$ \\ 
  & (0.034) & (0.031) & (0.034) & (0.031) \\ 
  Treatment: Info on actual support for GCS and NR & $-$0.021 & $-$0.017 & 0.048 & 0.054$^{*}$ \\ 
  & (0.034) & (0.031) & (0.033) & (0.031) \\ 
 \hline \\[-1.8ex] 
Includes controls &  & \checkmark &  & \checkmark \\

Observations & 2,000 & 1,995 & 2,000 & 1,995 \\ 
R$^{2}$ & 0.007 & 0.169 & 0.007 & 0.153 \\ 
\hline 
\hline \\[-1.8ex] 
\end{tabular} 
    }
    {\footnotesize %\textit{Note}: 
    }
\end{table}

\begin{figure}[h!]
    \caption[Donation to Africa vs. own country]{Donation in case of lottery win, depending on the recipient's (randomly drawn) nationality (mean). (Question \ref{q:donation})\hfill (Back~to~Section~\ref{subsec:universalistic})}\label{fig:donation}
    \makebox[\textwidth][c]{\includegraphics[width=.8\textwidth]{../figures/country_comparison/donation_mean.pdf}} 
\end{figure}

\begin{table}[h]
    \caption[Donation to Africa vs. own country]{Donation in case of lottery win, depending on the recipient's (randomly drawn) nationality. (Question \ref{q:donation})\hfill (Back~to~Section~\ref{subsec:universalistic})} \label{tab:donation}
    \makebox[\textwidth][c]{\input{../tables/continents/donation_interaction.tex}}
\end{table}

\begin{figure}[h!]
    \caption[Support for a global wealth tax]{Support for a global wealth tax. \\
    ``Do you support or oppose a tax on millionaires of all countries to finance low-
    income countries? \\
    Such tax would finance infrastructure and public services such as access to drinking water, healthcare, and education.'' (Question \ref{q:global_tax})}\label{fig:global_tax}
    \makebox[\textwidth][c]{\includegraphics[width=\textwidth]{../figures/country_comparison/global_tax_support.pdf}} 
\end{figure}

\begin{figure}[h!]
    \caption[Support for a national wealth tax]{Support for a national wealth tax financing public services like healthcare, education, and social housing. (Question \ref{q:national_tax})}\label{fig:national_tax}
    \makebox[\textwidth][c]{\includegraphics[width=\textwidth]{../figures/country_comparison/national_tax_support.pdf}} 
\end{figure}

\begin{figure}[h!]
    \caption[Preferred share of global tax for low-income countries]{Preferred share of global wealth tax revenues that should be pooled to finance low-income countries. (Question \ref{q:global_tax_global_share})}\label{fig:global_tax_global_share}
    \makebox[\textwidth][c]{\includegraphics[width=\textwidth]{../figures/country_comparison/global_tax_global_share.pdf}} 
\end{figure}

\begin{figure}[h!]
    \caption[Support for sharing half of global tax revenues with low-income countries]{Support for sharing half of global tax revenues with low-income countries, rather that each country retaining all the revenues it collects (in percent). (Question \ref{q:global_tax_sharing})}\label{fig:global_tax_sharing}
    \makebox[\textwidth][c]{\includegraphics[width=\textwidth]{../figures/country_comparison/global_tax_sharing_positive.pdf}} 
\end{figure}

\begin{figure} 
    \caption[Actual, perceived and preferred amount of foreign aid (mean)]{Actual, perceived and preferred amount of foreign aid, with random info (or not) on actual amount. (\textit{Mean}, Questions \ref{q:foreign_aid_belief}, \ref{q:foreign_aid_preferred})  \hfill (Back~to~Section~\ref{subsubsec:support_foreign_aid})}\label{fig:foreign_aid_amount}
    \makebox[\textwidth][c]{\includegraphics[width=.9\textwidth]{../figures/country_comparison/foreign_aid_amount_mean.pdf} } 
\end{figure}

% \begin{figure} 
%     \caption{Actual, perceived and preferred amount of foreign aid, with random info (or not) on actual amount. (\textit{Median}, Questions \ref{q:foreign_aid_belief}, \ref{q:foreign_aid_preferred})}\label{fig:foreign_aid_amount}
%     \makebox[\textwidth][c]{\includegraphics[width=.9\textwidth]{../figures/country_comparison/foreign_aid_amount_median.pdf} } % TODO? add? not necessary as the info on median can be deduced from below figures
% \end{figure}

\begin{figure} 
    % \caption{Support for increased foreign aid (vs. reduced or stable): from previous question, and directly asked (with info).}\vspace{-.2cm}
    % \includegraphics[height=.32\textheight]{../figures/country_comparison/foreign_aid_more_positive.pdf} 
    \caption[Preferred foreign aid (summary)]{Preferred foreign aid (after info or after perception). (Questions \ref{q:foreign_aid_belief} and \ref{q:foreign_aid_preferred})}\label{fig:foreign_aid_no_less_all}
    \makebox[\textwidth][c]{\includegraphics[width=\textwidth]{../figures/country_comparison/foreign_aid_no_less_all_positive.pdf} }
\end{figure} 

% \begin{figure}
%     \centering 
%     \caption{Your previous answer shows that you would like to increase [UK] foreign aid.\\How would you like to finance such increase in foreign aid? (Multiple answers possible)}
%     \includegraphics[width=\columnwidth]{../figures/all/foreign_aid_raise.pdf} 
% \end{figure}		
% \begin{figure}
%     \centering 
%     \caption{Your previous answer shows that you would like to reduce [UK] foreign aid.\\How would you like to use the freed budget? (Multiple answers possible)}
%     \includegraphics[width=\columnwidth]{../figures/all/foreign_aid_reduce.pdf} 
% \end{figure}

\begin{figure}[h!]
    \caption[Perceived foreign aid]{Perceived foreign aid. ``From your best guess, what percentage of [own country] government spending is allocated to foreign aid (that is, to reduce poverty in low-income countries)?'' (Question \ref{q:foreign_aid_belief})  \hfill (Back~to~Section~\ref{subsubsec:support_foreign_aid}) \\ Actual values: France: 0.8\%; Germany: 1.3\%; Spain: 0.5\%; UK: 1.7\%; U.S.: 0.4\%.}\label{fig:foreign_aid_belief}
    \makebox[\textwidth][c]{\includegraphics[width=\textwidth]{../figures/country_comparison/foreign_aid_belief_agg.pdf}} 
\end{figure}

\begin{figure}[h!]
    \caption[Preferred foreign aid (without info on actual amount)]{Preferred foreign aid (without info on actual amount). \\ ``If you could choose the government spending, what percentage would you allocate
    to foreign aid?'' (Question \ref{q:foreign_aid_preferred})  \hfill (Back~to~Section~\ref{subsubsec:support_foreign_aid})}\label{fig:foreign_aid_preferred_no_info}
    \makebox[\textwidth][c]{\includegraphics[width=\textwidth]{../figures/country_comparison/foreign_aid_preferred_no_info_agg.pdf}} 
\end{figure}

\begin{figure}[h!]
    \caption[Preferred foreign aid (after info on actual amount)]{Preferred foreign aid (after info on actual amount). \\ ``Actually,
    [US1: 0.4\%; FR: 0.8\%; DE: 1.3\%; ES: 0.5\%; UK: 1.7\%] of [own country] government spending is allocated to foreign aid. \\
    If you could choose the government spending, what percentage would you allocate
    to foreign aid?'' (Question \ref{q:foreign_aid_preferred})  \hfill (Back~to~Section~\ref{subsubsec:support_foreign_aid})}\label{fig:foreign_aid_preferred_info}
    \makebox[\textwidth][c]{\includegraphics[width=\textwidth]{../figures/country_comparison/foreign_aid_preferred_info_agg.pdf}} 
\end{figure}

\begin{figure}[h!]
    \caption[Preferences for funding increased foreign aid]{Preferences for funding increased foreign aid. [Asked iff preferred foreign aid is strictly greater than [Info: actual; No info: perceived] foreign aid] \\ ``How would you like to finance such increase in foreign aid? (Multiple answers possible)'' (in percent) (Question \ref{q:foreign_aid_raise_how})  \hfill (Back~to~Section~\ref{subsubsec:support_foreign_aid})}\label{fig:foreign_aid_raise_how}
    \makebox[\textwidth][c]{\includegraphics[width=.75\textwidth]{../figures/country_comparison/foreign_aid_raise_positive.pdf}} 
\end{figure}

\begin{figure}[h!]
    \caption[Preferences of spending following reduced foreign aid]{Preferences of spending following reduced foreign aid. [Asked iff preferred foreign aid is strictly lower than [Info: actual; No info: perceived] foreign aid] \\ ``How would you like to use the freed budget? (Multiple answers possible)'' (in percent) (Question \ref{q:foreign_aid_reduce_how})  \hfill (Back~to~Section~\ref{subsubsec:support_foreign_aid})}\label{fig:foreign_aid_reduce_how}
    \makebox[\textwidth][c]{\includegraphics[width=.75\textwidth]{../figures/country_comparison/foreign_aid_reduce_positive.pdf}} 
\end{figure}

% \begin{figure}[h!]
%     \caption[Attitudes on the evolution of foreign aid]{Attitudes regarding the evolution of [own country] foreign aid. (Question \ref{q:foreign_aid_raise_support})}\label{fig:foreign_aid_raise_support}
%     \makebox[\textwidth][c]{\includegraphics[width=\textwidth]{../figures/country_comparison/foreign_aid_raise_support.pdf}} 
% \end{figure}

% \begin{figure}[h!]
%     \caption[Conditions at which foreign aid should be increased]{Conditions at which foreign aid should be increased (in percent). [Asked to those who wish an increase of foreign aid at some conditions.] (Question \ref{q:foreign_aid_condition})}\label{fig:foreign_aid_condition}
%     \makebox[\textwidth][c]{\includegraphics[width=\textwidth]{../figures/country_comparison/foreign_aid_condition_positive.pdf}} 
% \end{figure}

% \begin{figure}[h!]
%     \caption[Reasons why foreign aid should not be increased]{Reasons why foreign aid should not be increased (in percent). [Asked to those who wish a decrease or stability of foreign aid.] (Question \ref{q:foreign_aid_no})}\label{fig:foreign_aid_no}
%     \makebox[\textwidth][c]{\includegraphics[width=\textwidth]{../figures/country_comparison/foreign_aid_no_positive.pdf}} 
% \end{figure}

% \begin{figure}[h!]
%     \caption[Willingness to sign a real-stake petition]{Willingness to sign real-stake petition for the Global Climate Scheme or National Redistribution. (Question \ref{q:petition})}\label{fig:petition}
%     \makebox[\textwidth][c]{\includegraphics[width=.8\textwidth]{../figures/country_comparison/petition_only_positive.pdf}} 
% \end{figure}

\begin{figure}[h!]
    \caption[Willingness to sign a real-stake petition]{Willingness to sign real-stake petition for the Global Climate Scheme or National Redistribution, compared to stated support in corresponding subsamples (e.g. support for the GCS in the branch where the petition was about the GCS). (Question \ref{q:petition})}\label{fig:petition}
    \makebox[\textwidth][c]{\includegraphics[width=.8\textwidth]{../figures/country_comparison/petition_comparable_positive.pdf}} 
\end{figure}

\begin{figure}[h!] % TODO? More details?
    \caption[Absolute support for various global policies]{Absolute support for various global policies (Percent of (\textit{somewhat} or \textit{strong}) support). (Questions \ref{q:climate_policies} and \ref{q:other_policies}. See Figure \ref{fig:support} for the relative support.)}\label{fig:support_likert_positive}
    \makebox[\textwidth][c]{\includegraphics[width=\textwidth]{../figures/country_comparison/support_likert_positive.pdf}} 
\end{figure}

% \begin{figure}[h!]
%     \caption{label}\label{fig:climate_policies}
%     \makebox[\textwidth][c]{\includegraphics[width=\textwidth]{../figures/country_comparison/climate_policies.pdf}} 
% \end{figure}

% \begin{figure}[h!]
%     \caption{label}\label{fig:global_policies}
%     \makebox[\textwidth][c]{\includegraphics[width=\textwidth]{../figures/country_comparison/global_policies.pdf}} 
% \end{figure}

\begin{figure}[h!]
    \caption[Preferred approach for international climate negotiations]{Preferred approach of diplomats at international climate negotiations. \\ In international climate negotiations, would you prefer [U.S.] diplomats to defend [own country] interests or global justice? (Question \ref{q:negotiation})}\label{fig:negotiation}
    \makebox[\textwidth][c]{\includegraphics[width=\textwidth]{../figures/country_comparison/negotiation.pdf}} 
\end{figure}

\begin{figure}[h!]
    \caption[Importance of selected issues]{Percent of selected issues viewed as important.\\ ``To what extent do you think the following issues are a problem?'' (Question \ref{q:problem})}\label{fig:problem}
    \makebox[\textwidth][c]{\includegraphics[width=.75\textwidth]{../figures/country_comparison/problem_positive.pdf}} 
\end{figure}

\begin{figure}[h!]
    \caption[Group defended when voting]{Group defended when voting. \\ ``What group do you defend when you vote?'' (Question \ref{q:group_defended})}\label{fig:group_defended}
    \makebox[\textwidth][c]{\includegraphics[width=\textwidth]{../figures/country_comparison/group_defended_agg2.pdf}} 
\end{figure}

% \begin{figure}[h!]
%     \caption{label}\label{fig:group_defended}
%     \makebox[\textwidth][c]{\includegraphics[width=\textwidth]{../figures/country_comparison/group_defended.pdf}} 
% \end{figure}

\begin{figure}[h!] 
    \caption[Mean prioritization of policies]{Mean prioritization of policies. \\Mean number of points allocated policies to express intensity of support (among six policies chosen at random). Blue color means that the policy has been awarded more points than the average policy. (Question \ref{q:points})}\label{fig:points}
    \makebox[\textwidth][c]{\includegraphics[width=\textwidth]{../figures/country_comparison/points_mean.pdf}} 
\end{figure}

\begin{figure}[h!] 
    \caption[Positive prioritization of policies]{Positive prioritization of policies. \\ Percent of people allocating a positive number of points to policies, expressing their support (among six policies chosen at random). (Question \ref{q:points})}\label{fig:points_positive}
    \makebox[\textwidth][c]{\includegraphics[width=\textwidth]{../figures/country_comparison/points_positive.pdf}} 
\end{figure}

\begin{figure}[h!]
    \caption[Charity donation]{Charity donation. \\ ``How much did you give to charities in 2022?'' (Question \ref{q:donation_charities})}\label{fig:donation_charities}
    \makebox[\textwidth][c]{\includegraphics[width=.8\textwidth]{../figures/country_comparison/donation_charities.pdf}} 
\end{figure}

\begin{figure}[h!] 
    \caption[Interest in politics]{Interest in politics. \\ ``To what extent are you interested in politics?'' (Question \ref{q:interested_politics})}\label{fig:interested_politics}
    \makebox[\textwidth][c]{\includegraphics[width=.8\textwidth]{../figures/country_comparison/interested_politics.pdf}} 
\end{figure}

\begin{figure}[h!] 
    \caption[Desired involvement of government]{Desired involvement of government (from 1 to 5). (Question \ref{q:involvement_govt})}\label{fig:involvement_govt}
    \makebox[\textwidth][c]{\includegraphics[width=.9\textwidth]{../figures/country_comparison/involvement_govt.pdf}} 
\end{figure}

\begin{figure}[h!] 
    \caption[Political leaning]{Political leaning on economics (from 1: Left to 5: Right). (Question \ref{q:left_right})}\label{fig:left_right}
    \makebox[\textwidth][c]{\includegraphics[width=.8\textwidth]{../figures/country_comparison/left_right.pdf}} 
\end{figure}

\begin{figure}[h!] 
    \caption[Voted in last election]{Voted in last election. (Question \ref{q:vote_participation})}\label{fig:vote_participation}
    \makebox[\textwidth][c]{\includegraphics[width=.8\textwidth]{../figures/country_comparison/vote_participation.pdf}} 
\end{figure}

\begin{figure}[h!] 
    \caption[Vote in last election]{Vote in last election (aggregated). \textit{PNR} includes people who did not vote or prefer not to answer. (Question \ref{q:vote})}\label{fig:vote}
    \makebox[\textwidth][c]{\includegraphics[width=.75\textwidth]{../figures/country_comparison/vote.pdf}} 
\end{figure}

\begin{figure}[h!] 
    \caption[Perception that survey was biased]{Perception that survey was biased. \\ ``Do you feel that this survey was politically biased?'' (Question \ref{q:survey_biased})}\label{fig:survey_biased}
    \makebox[\textwidth][c]{\includegraphics[width=.7\textwidth]{../figures/country_comparison/survey_biased.pdf}} 
\end{figure}

% \begin{columns}
% \begin{column}{.5\textwidth}
% \begin{multicols}{2}
    \begin{figure}[h!]
        \caption[Classification of open-ended field on extreme poverty]{Opinion on the fight against extreme poverty. \\ ``According to you, what should high-income countries do to fight extreme poverty in low-income countries?'' (Question \ref{q:poverty_field})  \hfill (Back~to~Section~\ref{subsubsec:support_foreign_aid})}\label{fig:poverty_field}
    \begin{subfigure}{.34\textwidth}
        \caption{Elements found in the open-ended field on the question (manually recoded, in percent)}.
        \includegraphics[width=\textwidth]{../figures/country_comparison/poverty_field_positive.pdf}        
    \end{subfigure}
    \hspace{.02\textwidth}
    \begin{subfigure}{.64\textwidth}
        \caption{Keywords found in the open-ended field on the GCS (automatic search ignoring case, in percent).}
        \includegraphics[width=\textwidth]{../figures/country_comparison/poverty_field_contains_positive.pdf}    
    \end{subfigure}
    \end{figure}
% \end{column}
% \begin{column}{.5\textwidth}
    % \begin{figure}[h!]
    %     \caption[Topics of open-ended field on extreme poverty]{Opinion on the fight against extreme poverty. \\ ``According to you, what should high-income countries do to fight extreme poverty in low-income countries?'' \\ Keywords found in the open-ended field on the GCS (automatic search ignoring case, in percent). (Question \ref{q:poverty_field})}\label{fig:poverty_field_contains}
    %     \makebox[\textwidth][c]{\includegraphics[width=\columnwidth]{../figures/country_comparison/poverty_field_contains_positive.pdf}} 
    % \end{figure}
% \end{multicols}
% \end{column}
% \end{columns}


\begin{figure}[h!] 
    \caption[Main attitudes by vote]{Main attitudes by vote (``Right'' spans from Center-right to Far right). \\ (Relative support in percent in Questions \ref{q:gcs_support}, \ref{q:global_tax}, \ref{q:other_policies}, \ref{q:foreign_aid_raise_support}, \ref{q:negotiation}) \hfill (Back~to~Section~\ref{subsec:universalistic})}\label{fig:main_by_vote}
    \makebox[\textwidth][c]{\includegraphics[width=\textwidth]{../figures/country_comparison/main_all_by_vote_share.pdf}} 
\end{figure}

% \begin{figure}[h!] 
%     \caption[Interested to be interviewed]{Interested to be interviewed by a researcher for 30 min through videoconference. (Question \ref{q:interview})}\label{fig:interview}
%     \makebox[\textwidth][c]{\includegraphics[width=\textwidth]{../figures/country_comparison/interview.pdf}} 
% \end{figure}    

% \begin{figure}[h!]
%     \caption{label}\label{fig:share_policies_supported}
%     \makebox[\textwidth][c]{\includegraphics[width=\textwidth]{../figures/country_comparison/share_policies_supported.pdf}} 
% \end{figure} % TODO? uncomment?

% \begin{figure}[h!]
%     \caption{label}\label{fig:vars}
%     \makebox[\textwidth][c]{\includegraphics[width=\textwidth]{../figures/country_comparison/vars.pdf}} 
% \end{figure}

% In Denmark, France and the U.S., the questions with an asterisk were asked differently, asking ``To achieve a given reduction of greenhouse gas emissions globally, costly investments are needed. Ideally, how should countries bear the costs of fighting climate change?''. Instead of the equal right per capita, the item was ``Countries should pay in proportion to their current emissions'', historical responsibilities was worded as ``Countries should pay in proportion to their past emissions (from 1990 onwards)'', then there was an item ``The richest countries should pay it all'', and compensating vulnerable countries was worded as ``The richest countries should pay even more, to help vulnerable countries face adverse consequences: vulnerable countries would then receive money instead of paying''.

\clearpage 
\section{Questionnaire of the global survey (section on global policies)}\label{app:questionnaire_oecd}
%\subsection*{International burden-sharing}
\renewcommand{\theenumi}{\Alph{enumi}}
\begin{enumerate} \item \label{q:scale} At which level(s) do you think public policies to tackle climate change need to be put in place? (Multiple answers are possible) [\textit{Figures \ref{fig:oecd} and \ref{fig:oecd_absolute}}]
\\ \textit{Global; [Federal / European / ...]; [State / National]; Local}
\item Do you agree or disagree with the following statement: ``[country] should take measures to fight climate change.''% TODO! figure
	\\ \textit{Strongly disagree; Somewhat disagree; Neither agree nor disagree; Somewhat agree; Strongly agree}
\item How should [country] climate policies depend on what other countries do?% TODO! figure
 \begin{itemize}
\item If other countries do more, [country] should do...
\item If other countries do less, [country] should do...
\end{itemize}
\textit{Much less; Less; About the same; More; Much more}
\item ~[In all countries but the U.S., Denmark and France]  All countries have signed the Paris agreement that aims to contain global warming ``well below +2 \textdegree{}C\''. To limit global warming to this level, there is a maximum amount of greenhouse gases we can emit globally, called the carbon budget. Each country could aim to emit less than a share of the carbon budget. To respect the global carbon budget, countries that emit more than their national share would pay a fee to countries that emit less than their share. \\ 
Do you support such a policy? [\textit{Figures \ref{fig:oecd} and \ref{fig:oecd_absolute}}]
\\ \textit{Strongly oppose; Somewhat oppose; Neither support nor oppose; Somewhat support; Strongly support}
\item ~[In all countries but the U.S., Denmark and France] Suppose the above policy is in place. How should the carbon budget be divided among countries? [\textit{Figures \ref{fig:oecd} and \ref{fig:oecd_absolute}}]
\\ \textit{The emission share of a country should be proportional to its population, so that each human has an equal right to emit.; The emission share of a country should be proportional to its current emissions, so that those who already emit more have more rights to emit.; Countries that have emitted more over the past decades (from 1990 onwards) should receive a lower emission share, because they have already used some of their fair share.; Countries that will be hurt more by climate change should receive a higher emission share, to compensate them for the damages.}
\item \label{q:burden_sharing_asterisk} ~[In the U.S., Denmark, and France only] To achieve a given reduction of greenhouse gas emissions globally, costly investments are needed. % TODO! figure
Ideally, how should countries bear the costs of fighting climate change?
 \begin{itemize}
\item Countries should pay in proportion to their income
\item Countries should pay in proportion to their current emissions [Used as a substitute to the equal right per capita in Figure \ref{fig:oecd}]
\item Countries should pay in proportion to their past emissions (from 1990 onwards) [Used as a substitute to historical responsibilities in Figure \ref{fig:oecd}]
\item The richest countries should pay it all, so that the poorest countries do not have to pay anything
\item The richest countries should pay even more, to help vulnerable countries face adverse consequences: vulnerable countries would then receive money instead of paying [Used as a substitute to compensating vulnerable countries in Figures \ref{fig:oecd} and \ref{fig:oecd_absolute}]
\end{itemize} 
\textit{Strongly disagree; Somewhat disagree; Neither agree nor disagree; Somewhat agree; Strongly agree}
\item Do you support or oppose establishing a global democratic assembly whose role would be to draft international treaties against climate change? Each adult across the world would have one vote to elect members of the assembly. [\textit{Figures \ref{fig:oecd} and \ref{fig:oecd_absolute}}]
\\ \textit{Strongly oppose; Somewhat oppose; Neither support nor oppose; Somewhat support; Strongly support}
\item Imagine the following policy: a global tax on greenhouse gas emissions funding a global basic income. 
Such a policy would progressively raise the price of fossil fuels (for example, the price of gasoline would increase by [40 cents per gallon] in the first years). Higher prices would encourage people and companies to use less fossil fuels, reducing greenhouse gas emissions. Revenues from the tax would be used to finance a basic income of [\$30] per month to each human adult, thereby lifting the 700 million people who earn less than \$2/day out of extreme poverty. 
The average British person would lose a bit from this policy as they would face [\$130] per month in price increases, which is higher than the [\$30] they would receive.

Do you support or oppose such a policy?  [\textit{Figures \ref{fig:oecd} and \ref{fig:oecd_absolute}}]
\\ \textit{Strongly oppose; Somewhat oppose; Neither support nor oppose; Somewhat support; Strongly support}
\item \label{q:millionaire_tax} Do you support or oppose a tax on all millionaires around the world to finance low-income countries that comply with international standards regarding climate action? 
This would finance infrastructure and public services such as access to drinking water, healthcare, and education. [\textit{Figures \ref{fig:oecd} and \ref{fig:oecd_absolute}}]
\\ \textit{Strongly oppose; Somewhat oppose; Neither support nor oppose; Somewhat support; Strongly support}
\end{enumerate}

% \clearpage
% \section{Questionnaire of US1 %the first U.S. complementary 
% survey}\label{app:questionnaire_US1}

% \begin{figure}[h!]
%     \caption{US1 survey structure}\label{fig:flow_US1}
%     \makebox[\textwidth][c]{\includegraphics[width=\textwidth]{../questionnaire/survey_flow_US1.pdf}} 
% \end{figure}

\renewcommand{\theenumi}{\arabic{enumi}}
\clearpage
\section{Questionnaire of the complementary surveys}\label{app:questionnaire}
\input{app_questionnaire}


\clearpage
\section{Net gains from the Global Climate Scheme}\label{app:gain_gcs}

To specify the GCS, we use the IEA's 2DS scenario \citep{iea_energy_2017}, which is consistent with limiting the global average temperature increase to 2\textdegree{}C with a probability of at least 50\%. The paper by \citet{hood_input_2017} contributing to the Report of the High-Level Commission on Carbon Prices \citep{stern_report_2017} presents a price corridor compatible with this emissions scenario, from which we take the midpoint. The product of these two series provides an estimate of the revenues expected from a global carbon price. We then use the UN median scenario of future population aged over 15 years (\textit{adults}, for short). We derive the basic income that could be paid to all adults by recycling the revenues from the global carbon price: evolving between \$20 and \$30 per month, with a peak in 2030. Accounting for the lower price levels in low-income countries, an additional income of \$30 per month would allow \href{https://data.worldbank.org/indicator/SI.POV.DDAY}{670 million people} to escape extreme poverty, defined with the threshold of \$2.15 per day in purchasing power parity.\footnote{By taking the \href{https://data.worldbank.org/indicator/PA.NUS.PPPC.RF}{ratio} of the World Bank series relating the GDP per capita of Sub-Saharan Africa in \href{https://data.worldbank.org/indicator/NY.GDP.PCAP.PP.KD?locations=ZG&year_high_desc=true}{PPP} and \href{https://data.worldbank.org/indicator/NY.GDP.PCAP.KD?locations=ZG&year_high_desc=true}{nominal}, we obtain the purchasing power of \$1 in this region: \$2.4 in 2019. %See also the price level ratio of PPP conversion factor to market exchange rate.
} 

To estimate the increase in fossil fuel expenditures (or ``cost'') in each country by 2030, we make a key assumption concerning the evolution of the carbon footprints per adult: that they will decrease by the same proportion %$\rho$ 
in each country. We use data from the Global Carbon Project \citep{peters_synthesis_2012}. 
% Noting $e_c$ (resp. $e_c^b$) the carbon footprint per adult of a country $c$ in 2030 (resp. in baseline year $b$), we have $e_c = \rho e_c^b$. Noting $a_c$ (resp. $a_c^b$) the adult population of a country $c$ in 2030 (resp. in baseline year $b$) and $E = \sum_c e_c a_c$ global emissions in 2030, we find $\rho = \frac{E}{\sum_c e_c^b a_c}$. Finally, the average cost per adult in year $y$ is $p \cdot e_c \frac{a_c}{a^y_c}$. %Multiplying country $c$'s carbon footprint per capita with the carbon price $p$ yields its average cost per adult: $p \cdot e_c$. %$\frac{s_c^y}{p^y_c} R$. 
In 2030, the average carbon footprint of a country $c$, $e_c$, evolves from baseline year $b$ proportionally to the evolution of its adult population $\Delta p_c = p^{2030}_c/p^b_c$. Thus, the global share of country $c$'s carbon footprint, $s_c$, is proportional to $\sigma_c = e_c \Delta p_c$, and as countries' shares sum to 1, $s_c = \frac{\sigma_c}{\sum_k \sigma_k}$. Multiplying country $c$'s emission share with global revenues in 2030, $R$, and dividing by $c$'s adult population in year $y$, yields its average cost per adult: $R \cdot s_c / p^y_c$. %$\frac{s_c^y}{p^y_c} R$. 
Using findings from \citet{ivanova_unequal_2020} for Europe and \citet{fremstad_impact_2019} for the U.S., we approximate the median cost as 90\% of the average cost. Finally, the net gain is given by the basic income (\$30 per month) minus the cost. We provided consistent estimates of net gains in all surveys (using $y = b = 2015$), though in the global survey we gave the average net gains vs. the median ones in the complementary surveys. The latter are shown in Figure \ref{fig:median_gain_2015}. 
For the record, Table \ref{tab:gain_gcs.tex} also provides an estimate of \textit{average} net gains (computed with $b = 2019$ and $y = 2030$).\footnote{2015 was the last year of data available when the global questionnaire was conceived (\href{https://stats.oecd.org/Index.aspx?DataSetCode=IO_GHG_2019}{OECD data} was then used -- it does not cover all countries but give identical rounded estimates than those recomputed from the Global Carbon Project data for our complementary surveys). 2030 was chosen as the reference year as it is the date at which global carbon price revenues are expected to peak (and the GCS redistributive effects would be largest), and the GCS could not realistically enter into force before that date. In the surveys, we chose $y = b = 2015$ rather than $b = 2019$ and $y = 2030$ to get more conservative estimates of the monthly cost in the U.S. (\$20 higher than the other option) and in Europe (\euro{5} or £10 higher).}% TODO? remove footnote?
%  ((e/E)*(f/a)*A/F)*R/a

Estimates of the net gains from the Global Climate Scheme are necessarily imprecise, given the uncertainties surrounding the carbon price required to achieve emissions reductions as well as each country's trajectory in terms of emissions and population. These values are highly dependent on future (non-price) climate policies, technical progress, and economic growth of each country, which are only partially known. Integrated Assessment Models have been used to derive a Global Energy Assessment \citep{johansson_global_2012}, a 100\% renewable scenario \citep{greenpeace_energy_2015} as well as Shared Socioeconomic Pathways (SSPs), which include consistent trajectories of population, emissions, and carbon price \citep{riahi_shared_2017,bauer_shared_2017,van_vuuren_energy_2017,fricko_marker_2017}. Instead of using some of these modelling trajectories, we relied on a simple and transparent formula, for a number of reasons. First and foremost, those trajectories describe territorial emissions while we need consumption-based emissions to compute the incidence of the GCS. Second, the carbon price is relatively low in trajectories of SSPs that contain global warming below 2\textdegree{}C (less than \$35/tCO$_\text{2}$ in 2030), so we conservatively chose a method yielding a higher carbon price (\$90 in 2030). Third, modelling results are available only for a few macro regions, while we wanted country by country estimates. Finally, we have checked that the emissions per capita given by our method are broadly in line with alternative methods, even if it tends to overestimate net gains in countries which will decarbonize less rapidly than average.\footnote{Computations with alternative methods can be found on \href{https://github.com/bixiou/global_tax_attitudes/blob/main/code_global/map_GCS_incidence.R}{our public repository}.} For example, although countries' decarbonization plans should realign with the GCS in place, India might still decarbonize less quickly than the European Union, so India's gain and the EU's loss might be overestimated in our computations. For a more sophisticated version of the Global Climate Scheme which includes participation mechanisms preventing middle-income countries (like China) to lose from it and estimations of the Net Present Value by country, see \citet{fabre_global_2023}.  \hfill (Back~to~Section~\ref{box:GCS})

\begin{figure}[h!]
    \caption{Net gains from the Global Climate Scheme.}\label{fig:median_gain_2015}
    \makebox[\textwidth][c]{\includegraphics[width=\textwidth]{../figures/maps/median_gain_2015.pdf}} 
\end{figure}

% \begin{table}[h]\label{tab:gain_gcs}
%     \caption{Net gains from the Global Climate Scheme.} 
%     \makebox[\textwidth][c]{
        % \resizebox*{!}{.7\textheight}{
\clearpage
\begin{multicols}{2}
    \setbox\ltmcbox\vbox{
    \makeatletter\col@number\@ne
        
\begin{longtable}[t]{lrr}
\caption{\label{tab:gain_gcs.tex}Estimated net gain from the GCS in 2030 and carbon footprint by country.}\\
\toprule
  & \makecell{Mean\\net gain\\from\\the GCS\\(\$/month)} & \makecell{CO$_\text{2}$\\footprint\\per adult\\in 2019\\(tCO$_\text{2}$/y)}\\
\midrule
Saudi Arabia & -93 & 24.0\\
United States & -77 & 21.0\\
Australia & -60 & 17.6\\
Canada & -56 & 16.7\\
South Korea & -50 & 15.6\\
Germany & -30 & 11.7\\
Russia & -29 & 11.5\\
Japan & -28 & 11.3\\
Malaysia & -21 & 10.0\\
Iran & -19 & 9.5\\
Poland & -19 & 9.5\\
United Kingdom & -18 & 9.4\\
China & -14 & 8.6\\
Italy & -13 & 8.4\\
South Africa & -11 & 8.0\\
France & -10 & 7.8\\
Iraq* & -8 & 7.4\\
Spain & -6 & 7.0\\
Turkey & -2 & 6.2\\
Algeria* & -1 & 6.0\\
Mexico & 2 & 5.6\\
Ukraine & 2 & 5.6\\
Uzbekistan* & 4 & 5.1\\
Argentina & 5 & 4.9\\
Thailand & 6 & 4.6\\
Egypt & 12 & 3.6\\
Indonesia & 13 & 3.3\\
Colombia & 15 & 3.0\\
Brazil & 15 & 2.9\\
Vietnam & 15 & 2.9\\
Peru & 16 & 2.8\\
Morocco & 16 & 2.7\\
North Korea* & 17 & 2.5\\
India & 18 & 2.4\\
Philippines & 18 & 2.3\\
Pakistan & 22 & 1.6\\
Bangladesh & 24 & 1.1\\
Nigeria & 25 & 1.0\\
Kenya & 25 & 0.9\\
Myanmar* & 26 & 0.9\\
Sudan* & 26 & 0.9\\
Tanzania & 27 & 0.5\\
Afghanistan* & 27 & 0.5\\
Uganda & 28 & 0.4\\
Ethiopia & 28 & 0.3\\
Venezuela & 29 & 0.3\\
DRC* & 30 & 0.1\\
\bottomrule
\end{longtable}
    \unskip
    \unpenalty
    \unpenalty}
    \unvbox\ltmcbox
\end{multicols}
        % }
%     }
    {\footnotesize \textit{Note}: %Emission data is from \cite{peters_synthesis_2012}. 
    Asterisks denote countries where footprint is missing and territorial emissions is used instead. %Estimation of net gains is described in the text. 
    Values differ from Figure \ref{fig:median_gain_2015} as this table present estimates of \textit{mean} net gain per adult in \textit{2030}, not at the present. Only the countries with more than 20 million adults (covering 87\% of the global total) are shown. 
    }
% \end{table}

% \clearpage
% \section{Sources}\label{app:sources}

\clearpage
\section{Determinants of support}\label{app:determinants}

\begin{table}[h]\label{tab:gcs_determinant}
    \caption[Determinants of support for the GCS]{Determinants of support for the Global Climate Scheme. (Back to \ref{subsubsec:support_gcs})} 
    \makebox[\textwidth][c]{
\resizebox*{!}{.73\textheight}{ % 73 is the max when there is a title
        
\begin{tabular}{@{\extracolsep{5pt}}lccccccc} 
\\[-1.8ex]\hline 
\hline \\[-1.8ex] 
 & \multicolumn{7}{c}{\makecell{Supports the Global Climate Scheme}} \\ 
\cline{2-8} 
\\[-1.8ex] & All & United States & Europe & France & Germany & Spain & United Kingdom \\ 
\hline \\[-1.8ex] 
 Country: Germany & $-$0.157$^{***}$ &  & $-$0.144$^{***}$ &  &  &  &  \\ 
  & (0.022) &  & (0.022) &  &  &  &  \\ 
  Country: Spain & $-$0.044$^{*}$ &  & $-$0.026 &  &  &  &  \\ 
  & (0.024) &  & (0.024) &  &  &  &  \\ 
  Country: United Kingdom & $-$0.079$^{***}$ &  & $-$0.104$^{***}$ &  &  &  &  \\ 
  & (0.024) &  & (0.023) &  &  &  &  \\ 
  Country: United States & $-$0.375$^{***}$ &  &  &  &  &  &  \\ 
  & (0.019) &  &  &  &  &  &  \\ 
  Income quartile: 2 & 0.037$^{**}$ & 0.031 & 0.038 & 0.047 & 0.058 & 0.013 & 0.023 \\ 
  & (0.017) & (0.022) & (0.023) & (0.043) & (0.049) & (0.053) & (0.043) \\ 
  Income quartile: 3 & 0.042$^{**}$ & 0.033 & 0.049$^{**}$ & 0.080$^{**}$ & 0.059 & 0.074 & $-$0.052 \\ 
  & (0.017) & (0.024) & (0.024) & (0.040) & (0.052) & (0.056) & (0.052) \\ 
  Income quartile: 4 & 0.056$^{***}$ & 0.063$^{**}$ & 0.010 & 0.018 & $-$0.015 & $-$0.001 & $-$0.005 \\ 
  & (0.018) & (0.026) & (0.026) & (0.047) & (0.055) & (0.056) & (0.057) \\ 
  Diploma: Post secondary & 0.023$^{*}$ & 0.033$^{*}$ & 0.010 & 0.007 & 0.045 & 0.007 & $-$0.010 \\ 
  & (0.012) & (0.017) & (0.018) & (0.029) & (0.039) & (0.039) & (0.039) \\ 
  Age: 25-34 & $-$0.076$^{***}$ & $-$0.083$^{***}$ & $-$0.044 & $-$0.031 & $-$0.077 & $-$0.050 & $-$0.103 \\ 
  & (0.025) & (0.031) & (0.035) & (0.057) & (0.083) & (0.066) & (0.091) \\ 
  Age: 35-49 & $-$0.101$^{***}$ & $-$0.108$^{***}$ & $-$0.069$^{**}$ & $-$0.094$^{*}$ & $-$0.009 & $-$0.168$^{**}$ & $-$0.050 \\ 
  & (0.024) & (0.030) & (0.034) & (0.055) & (0.077) & (0.070) & (0.090) \\ 
  Age: 50-64 & $-$0.137$^{***}$ & $-$0.164$^{***}$ & $-$0.038 & $-$0.039 & $-$0.020 & $-$0.146$^{**}$ & $-$0.017 \\ 
  & (0.024) & (0.030) & (0.035) & (0.056) & (0.082) & (0.067) & (0.087) \\ 
  Age: 65+ & $-$0.116$^{***}$ & $-$0.140$^{***}$ & $-$0.056 & 0.003 & $-$0.045 & $-$0.258$^{***}$ & 0.011 \\ 
  & (0.028) & (0.034) & (0.044) & (0.076) & (0.094) & (0.091) & (0.105) \\ 
  Gender: Man & 0.019$^{*}$ & 0.023 & $-$0.010 & $-$0.014 & $-$0.018 & 0.042 & $-$0.005 \\ 
  & (0.011) & (0.015) & (0.016) & (0.029) & (0.033) & (0.038) & (0.034) \\ 
  Lives with partner & 0.029$^{**}$ & 0.022 & 0.058$^{***}$ & 0.070$^{**}$ & 0.082$^{**}$ & 0.017 & 0.040 \\ 
  & (0.013) & (0.017) & (0.018) & (0.033) & (0.038) & (0.038) & (0.039) \\ 
  Employment status: Retired & $-$0.020 & $-$0.047 & 0.056 & 0.087 & 0.096 & 0.040 & 0.001 \\ 
  & (0.024) & (0.030) & (0.038) & (0.081) & (0.075) & (0.082) & (0.073) \\ 
  Employment status: Student & 0.045 & 0.063 & 0.101$^{**}$ & 0.165$^{*}$ & 0.192$^{**}$ & 0.116 & $-$0.021 \\ 
  & (0.033) & (0.048) & (0.044) & (0.085) & (0.087) & (0.074) & (0.107) \\ 
  Employment status: Working & $-$0.016 & $-$0.021 & 0.011 & 0.082 & 0.006 & $-$0.050 & 0.036 \\ 
  & (0.019) & (0.024) & (0.028) & (0.064) & (0.056) & (0.056) & (0.051) \\ 
  Vote: Center-right or Right & $-$0.331$^{***}$ & $-$0.435$^{***}$ & $-$0.106$^{***}$ & $-$0.131$^{***}$ & $-$0.004 & $-$0.114$^{***}$ & $-$0.081$^{**}$ \\ 
  & (0.013) & (0.017) & (0.019) & (0.035) & (0.044) & (0.038) & (0.041) \\ 
  Vote: PNR/Non-voter & $-$0.184$^{***}$ & $-$0.198$^{***}$ & $-$0.136$^{***}$ & $-$0.196$^{***}$ & $-$0.034 & $-$0.116$^{**}$ & $-$0.108$^{***}$ \\ 
  & (0.016) & (0.022) & (0.021) & (0.039) & (0.043) & (0.046) & (0.040) \\ 
  Vote: Far right & $-$0.396$^{***}$ &  & $-$0.308$^{***}$ & $-$0.493$^{***}$ & $-$0.168$^{***}$ & $-$0.130 & $-$0.314$^{***}$ \\ 
  & (0.032) &  & (0.033) & (0.064) & (0.051) & (0.102) & (0.080) \\ 
  Urban & 0.049$^{***}$ & 0.074$^{***}$ & 0.006 & $-$0.002 & 0.019 & $-$0.014 & 0.017 \\ 
  & (0.012) & (0.018) & (0.016) & (0.029) & (0.032) & (0.036) & (0.033) \\ 
  Race: White &  & $-$0.030 &  &  &  &  &  \\ 
  &  & (0.019) &  &  &  &  &  \\ 
  Region: Northeast &  & 0.009 &  &  &  &  &  \\ 
  &  & (0.023) &  &  &  &  &  \\ 
  Region: South &  & 0.011 &  &  &  &  &  \\ 
  &  & (0.020) &  &  &  &  &  \\ 
  Region: West &  & 0.011 &  &  &  &  &  \\ 
  &  & (0.022) &  &  &  &  &  \\ 
  Swing State &  & $-$0.019 &  &  &  &  &  \\ 
  &  & (0.017) &  &  &  &  &  \\ 
 \hline \\[-1.8ex] 
Constant & 1.048 & 0.729 & 0.89 & 0.7 & 0.732 & 0.935 & 0.886 \\ 
Observations & 7,986 & 4,992 & 2,994 & 977 & 727 & 748 & 542 \\ 
R$^{2}$ & 0.160 & 0.180 & 0.064 & 0.116 & 0.067 & 0.043 & 0.063 \\ 
\hline 
\hline \\[-1.8ex] 
\textit{Note:}  & \multicolumn{7}{r}{$^{*}$p$<$0.1; $^{**}$p$<$0.05; $^{***}$p$<$0.01} \\ 
\end{tabular} 
        }
    }
    {\footnotesize %\textit{Note}: 
    }
\end{table}


\clearpage
\section{Representativeness of the surveys}\label{app:representativeness}


\begin{table}[h!]
    \caption[Sample representativeness of US1, US2, Eu]{Sample representativeness of the complementary surveys. (Back to \ref{par:surveys}) } \label{tab:representativeness_waves}
    \makebox[\textwidth][c]{
        \resizebox*{!}{.80\textheight}{% 73 without notes cf. https://tex.stackexchange.com/questions/13809/resizing-a-table-by-textheight 
        
\begin{tabular}[t]{llllllllll}
\toprule
\multicolumn{1}{c}{} & \multicolumn{3}{c}{US1} & \multicolumn{3}{c}{US2} & \multicolumn{3}{c}{EU} \\
\cmidrule(l{3pt}r{3pt}){2-4} \cmidrule(l{3pt}r{3pt}){5-7} \cmidrule(l{3pt}r{3pt}){8-10}
  & Pop. & Sample & \makecell{Weighted\\sample} & Pop. & Sample & \makecell{Weighted\\sample} & Pop. & Sample & \makecell{Weighted\\sample}\\
\midrule
Sample size &  & 3,000 & 3,000 &  & 678 & 678 &  & 3,000 & 3,000\\
\addlinespace
Gender: Woman & 0.51 & 0.52 & 0.51 & 0.51 & 0.67 & 0.57 & 0.51 & 0.49 & 0.51\\
Gender: Man & 0.49 & 0.47 & 0.49 & 0.49 & 0.32 & 0.43 & 0.49 & 0.51 & 0.49\\
\addlinespace
Income\_quartile: 1 & 0.25 & 0.27 & 0.25 & 0.25 & 0.55 & 0.34 & 0.25 & 0.28 & 0.25\\
Income\_quartile: 2 & 0.25 & 0.24 & 0.25 & 0.25 & 0.29 & 0.32 & 0.25 & 0.23 & 0.25\\
Income\_quartile: 3 & 0.25 & 0.25 & 0.25 & 0.25 & 0.12 & 0.23 & 0.25 & 0.25 & 0.25\\
Income\_quartile: 4 & 0.25 & 0.23 & 0.25 & 0.25 & 0.04 & 0.12 & 0.25 & 0.24 & 0.25\\
\addlinespace
Age: 18-24 & 0.12 & 0.12 & 0.12 & 0.12 & 0.14 & 0.12 & 0.10 & 0.11 & 0.10\\
Age: 25-34 & 0.18 & 0.15 & 0.18 & 0.18 & 0.16 & 0.17 & 0.15 & 0.17 & 0.15\\
Age: 35-49 & 0.24 & 0.25 & 0.24 & 0.24 & 0.25 & 0.25 & 0.24 & 0.25 & 0.24\\
Age: 50-64 & 0.25 & 0.27 & 0.25 & 0.25 & 0.22 & 0.24 & 0.26 & 0.24 & 0.26\\
Age: 65+ & 0.21 & 0.21 & 0.21 & 0.21 & 0.22 & 0.22 & 0.25 & 0.23 & 0.25\\
\addlinespace
Diploma\_25\_64: Below upper secondary & 0.06 & 0.02 & 0.05 & 0.06 & 0.08 & 0.07 & 0.13 & 0.14 & 0.13\\
Diploma\_25\_64: Upper secondary & 0.28 & 0.25 & 0.28 & 0.28 & 0.33 & 0.30 & 0.23 & 0.19 & 0.23\\
Diploma\_25\_64: Post secondary & 0.34 & 0.40 & 0.34 & 0.34 & 0.23 & 0.28 & 0.29 & 0.33 & 0.29\\
\addlinespace
Race: White only & 0.60 & 0.67 & 0.61 & 0.60 & 0.20 & 0.46 &  &  & \\
Race: Hispanic & 0.18 & 0.15 & 0.19 & 0.18 & 0.41 & 0.27 &  &  & \\
Race: Black & 0.13 & 0.16 & 0.14 & 0.13 & 0.36 & 0.20 &  &  & \\
\addlinespace
Region: Northeast & 0.17 & 0.20 & 0.17 & 0.17 & 0.15 & 0.16 &  &  & \\
Region: Midwest & 0.21 & 0.22 & 0.21 & 0.21 & 0.15 & 0.20 &  &  & \\
Region: South & 0.38 & 0.39 & 0.38 & 0.38 & 0.50 & 0.45 &  &  & \\
Region: West & 0.24 & 0.20 & 0.24 & 0.24 & 0.20 & 0.20 &  &  & \\
\addlinespace
Urban: TRUE & 0.73 & 0.78 & 0.74 & 0.73 & 0.73 & 0.69 &  &  & \\
\addlinespace
Employment\_18\_64: Inactive & 0.20 & 0.16 & 0.16 & 0.20 & 0.18 & 0.15 & 0.17 & 0.15 & 0.15\\
Employment\_18\_64: Unemployed & 0.02 & 0.07 & 0.08 & 0.02 & 0.15 & 0.11 & 0.03 & 0.06 & 0.05\\
\addlinespace
Vote: Left & 0.32 & 0.47 & 0.45 & 0.32 & 0.48 & 0.42 & 0.30 & 0.32 & 0.32\\
Vote: Center-right or Right & 0.30 & 0.31 & 0.31 & 0.30 & 0.15 & 0.24 & 0.28 & 0.32 & 0.32\\
Vote: Far right &  &  &  &  &  &  & 0.10 & 0.10 & 0.10\\
\addlinespace
Country: FR &  &  &  &  &  &  & 0.24 & 0.24 & 0.24\\
Country: DE &  &  &  &  &  &  & 0.33 & 0.33 & 0.33\\
Country: ES &  &  &  &  &  &  & 0.18 & 0.18 & 0.18\\
Country: UK &  &  &  &  &  &  & 0.25 & 0.25 & 0.25\\
\addlinespace
Urbanity: Cities &  &  &  &  &  &  & 0.43 & 0.49 & 0.43\\
Urbanity: Towns and suburbs &  &  &  &  &  &  & 0.33 & 0.32 & 0.33\\
Urbanity: Rural &  &  &  &  &  &  & 0.25 & 0.20 & 0.25\\
\bottomrule
\end{tabular}
        }
    }
    {\footnotesize \textit{Note}: This table displays summary statistics of the samples alongside actual population frequencies. %For \textit{Vote}, we regroup candidates or parties into three broad categories and we take abstention into account (but omit this category). 
    %For \textit{Inactivity rate (15-64)}, the sample statistics include the share of respondents aged between 15 and 64 years old who indicated being either ``\textit{Inactive (not searching for a job)},'' a ``\textit{Student},'' or ``\textit{Retired}.'' For \textit{Unemployment rate (15-64)}, the sample statistics include the share of respondents aged between 15 and 64 years old who indicated being ``\textit{Unemployed (searching for a job)}'', (`\textit{Unemployed (searching for a job)},'' ``\textit{Full-time employed},'' ``\textit{Part-time employed},'' or ``\textit{Self-employed}''). For	\textit{Employment rate (15-64)}, the sample statistics include the share of respondents aged between 15 and 64 years old who indicated being either ``\textit{Full-time employed},'' ``\textit{Part-time employed},'' or ``\textit{Self-employed}.'' 
    Detailed sources for each variable and country population frequencies, as well as the definitions of regions, diploma, urbanity, employment, and vote are available in \href{https://github.com/bixiou/global_tax_attitudes/raw/main/questionnaire/specificities.xlsx}{this spreadsheet}. % TODO! Appendix \ref{app:sources}.
    } % TODO add hline before Urbanity, move Country/Urbanity above and add in Notes that quotas are those above the line
\end{table}

\begin{table}[h]
    \caption[Sample representativeness of each European country]{Sample representativeness for each European country. (Back to \ref{par:surveys})} \label{tab:representativeness_EU}
    \makebox[\textwidth][c]{
        \resizebox*{!}{.50\textheight}{% 73 without notes cf. https://tex.stackexchange.com/questions/13809/resizing-a-table-by-textheight 
        
\begin{tabular}[t]{lllllllllllll}
\toprule
\multicolumn{1}{c}{} & \multicolumn{3}{c}{FR} & \multicolumn{3}{c}{DE} & \multicolumn{3}{c}{ES} & \multicolumn{3}{c}{UK} \\
\cmidrule(l{3pt}r{3pt}){2-4} \cmidrule(l{3pt}r{3pt}){5-7} \cmidrule(l{3pt}r{3pt}){8-10} \cmidrule(l{3pt}r{3pt}){11-13}
  & Pop. & Sample & \makecell{Weighted\\sample} & Pop. & Sample & \makecell{Weighted\\sample} & Pop. & Sample & \makecell{Weighted\\sample} & Pop. & Sample & \makecell{Weighted\\sample}\\
\midrule
Sample size &  & 620 & 620 &  & 757 & 757 &  & 543 & 543 &  & 644 & 644\\
\addlinespace
Gender: Woman & 0.52 & 0.49 & 0.54 & 0.51 & 0.53 & 0.58 & 0.51 & 0.55 & 0.60 & 0.50 & 0.26 & 0.32\\
Gender: Man & 0.48 & 0.51 & 0.46 & 0.49 & 0.47 & 0.42 & 0.49 & 0.45 & 0.40 & 0.50 & 0.74 & 0.68\\
\addlinespace
Income\_quartile: 1 & 0.25 & 0.30 & 0.27 & 0.25 & 0.28 & 0.23 & 0.25 & 0.27 & 0.23 & 0.25 & 0.32 & 0.28\\
Income\_quartile: 2 & 0.25 & 0.17 & 0.17 & 0.25 & 0.25 & 0.24 & 0.25 & 0.32 & 0.33 & 0.25 & 0.29 & 0.28\\
Income\_quartile: 3 & 0.25 & 0.22 & 0.22 & 0.25 & 0.29 & 0.30 & 0.25 & 0.25 & 0.25 & 0.25 & 0.20 & 0.21\\
Income\_quartile: 4 & 0.25 & 0.32 & 0.34 & 0.25 & 0.18 & 0.23 & 0.25 & 0.15 & 0.19 & 0.25 & 0.19 & 0.23\\
\addlinespace
Age: 18-24 & 0.12 & 0.08 & 0.06 & 0.09 & 0.18 & 0.15 & 0.08 & 0.17 & 0.15 & 0.10 & 0.02 & 0.02\\
Age: 25-34 & 0.15 & 0.17 & 0.16 & 0.15 & 0.21 & 0.20 & 0.12 & 0.15 & 0.14 & 0.17 & 0.10 & 0.09\\
Age: 35-49 & 0.24 & 0.33 & 0.37 & 0.22 & 0.20 & 0.22 & 0.28 & 0.23 & 0.26 & 0.24 & 0.12 & 0.15\\
Age: 50-64 & 0.24 & 0.20 & 0.19 & 0.28 & 0.23 & 0.26 & 0.27 & 0.25 & 0.27 & 0.25 & 0.28 & 0.33\\
Age: 65+ & 0.25 & 0.23 & 0.22 & 0.26 & 0.18 & 0.18 & 0.25 & 0.19 & 0.19 & 0.24 & 0.48 & 0.42\\
\addlinespace
Urbanity: Cities & 0.47 & 0.51 & 0.43 & 0.37 & 0.47 & 0.40 & 0.52 & 0.67 & 0.62 & 0.40 & 0.37 & 0.31\\
Urbanity: Towns and suburbs & 0.19 & 0.18 & 0.18 & 0.40 & 0.34 & 0.34 & 0.22 & 0.27 & 0.29 & 0.42 & 0.46 & 0.47\\
Urbanity: Rural & 0.34 & 0.30 & 0.39 & 0.23 & 0.18 & 0.25 & 0.26 & 0.06 & 0.08 & 0.18 & 0.17 & 0.22\\
\addlinespace
Diploma\_25\_64: Below upper secondary & 0.11 & 0.22 & 0.18 & 0.10 & 0.17 & 0.16 & 0.24 & 0.10 & 0.09 & 0.12 & 0.10 & 0.08\\
Diploma\_25\_64: Upper secondary & 0.26 & 0.15 & 0.24 & 0.27 & 0.11 & 0.18 & 0.16 & 0.15 & 0.23 & 0.21 & 0.18 & 0.29\\
Diploma\_25\_64: Post secondary & 0.26 & 0.33 & 0.30 & 0.29 & 0.36 & 0.33 & 0.28 & 0.38 & 0.33 & 0.33 & 0.23 & 0.20\\
\addlinespace
Employment\_18\_64: Inactive & 0.20 & 0.18 & 0.16 & 0.15 & 0.16 & 0.14 & 0.20 & 0.16 & 0.15 & 0.16 & 0.14 & 0.15\\
Employment\_18\_64: Unemployed & 0.04 & 0.05 & 0.05 & 0.02 & 0.04 & 0.04 & 0.07 & 0.10 & 0.10 & 0.02 & 0.03 & 0.03\\
\addlinespace
Vote: Left & 0.23 & 0.18 & 0.17 & 0.37 & 0.42 & 0.42 & 0.33 & 0.37 & 0.38 & 0.25 & 0.27 & 0.27\\
Vote: Center-right or Right & 0.26 & 0.31 & 0.32 & 0.28 & 0.26 & 0.27 & 0.18 & 0.22 & 0.22 & 0.36 & 0.50 & 0.50\\
Vote: Far right & 0.23 & 0.23 & 0.24 & 0.08 & 0.07 & 0.08 & 0.09 & 0.08 & 0.07 & 0.01 & 0.03 & 0.04\\
\bottomrule
\end{tabular}
        }
    }
    % TODO add explanatory note
    {\footnotesize \textit{Note}: This table displays summary statistics of the samples alongside actual population frequencies. In this Table, weights are defined at the country level.  %For \textit{Vote}, we regroup candidates or parties into three broad categories and we take abstention into account (but omit this category). 
    %For \textit{Inactivity rate (15-64)}, the sample statistics include the share of respondents aged between 15 and 64 years old who indicated being either ``\textit{Inactive (not searching for a job)},'' a ``\textit{Student},'' or ``\textit{Retired}.'' For \textit{Unemployment rate (15-64)}, the sample statistics include the share of respondents aged between 15 and 64 years old who indicated being ``\textit{Unemployed (searching for a job)}'', (`\textit{Unemployed (searching for a job)},'' ``\textit{Full-time employed},'' ``\textit{Part-time employed},'' or ``\textit{Self-employed}''). For	\textit{Employment rate (15-64)}, the sample statistics include the share of respondents aged between 15 and 64 years old who indicated being either ``\textit{Full-time employed},'' ``\textit{Part-time employed},'' or ``\textit{Self-employed}.'' 
    Detailed sources for each variable and country population frequencies, as well as the definitions of regions, diploma, urbanity, employment, and vote are available in \href{https://github.com/bixiou/global_tax_attitudes/raw/main/questionnaire/specificities.xlsx}{this spreadsheet}. % TODO Appendix \ref{app:sources}.
    }
\end{table}

Similar tables for the global surveys can be found in \citet{dechezlepretre_fighting_2022}.

\clearpage
\section{Attrition analysis}\label{app:attrition}

\begin{table}[h]\label{tab:attrition_US1}
    \caption[Attrition analysis: US1]{Attrition analysis for the US1 survey.} 
    \makebox[\textwidth][c]{
\resizebox*{!}{.73\textheight}{ % 73 is the max when there is a title
        
\begin{tabular}{@{\extracolsep{5pt}}lccccc} 
\\[-1.8ex]\hline 
\hline \\[-1.8ex] 
\\[-1.8ex] & \makecell{Dropped out} & \makecell{Dropped out\\after\\socio-eco} & \makecell{Failed\\attention test} & \makecell{Duration\\(in min)} & \makecell{Duration\\below\\4 min} \\ 
\\[-1.8ex] & (1) & (2) & (3) & (4) & (5)\\ 
\hline \\[-1.8ex] 
Mean & 0.08 & 0.059 & 0.082 & 21.198 & 0.016  \\ \hline \\[-1.8ex]
 Income quartile: 3 & 0.001 & 0.001 & $-$0.022$^{*}$ & $-$0.770 & $-$0.009 \\ 
  & (0.010) & (0.010) & (0.012) & (3.203) & (0.006) \\ 
  Income quartile: 4 & 0.004 & 0.004 & $-$0.029$^{**}$ & 0.775 & $-$0.004 \\ 
  & (0.012) & (0.012) & (0.012) & (2.737) & (0.007) \\ 
  Diploma: Post secondary & $-$0.012 & $-$0.012 & 0.011 & $-$4.141 & $-$0.004 \\ 
  & (0.012) & (0.012) & (0.014) & (2.803) & (0.007) \\ 
  Age: 25-34 & 0.006 & 0.006 & 0.001 & 1.004 & 0.004 \\ 
  & (0.009) & (0.009) & (0.009) & (2.509) & (0.005) \\ 
  Age: 35-49 & $-$0.058$^{***}$ & $-$0.058$^{***}$ & 0.001 & $-$0.859 & $-$0.032$^{**}$ \\ 
  & (0.015) & (0.015) & (0.019) & (2.503) & (0.013) \\ 
  Age: 50-64 & $-$0.053$^{***}$ & $-$0.053$^{***}$ & 0.001 & 4.431 & $-$0.033$^{***}$ \\ 
  & (0.015) & (0.015) & (0.017) & (2.945) & (0.013) \\ 
  Age: 65+ & $-$0.031$^{**}$ & $-$0.031$^{**}$ & $-$0.055$^{***}$ & 5.358$^{**}$ & $-$0.041$^{***}$ \\ 
  & (0.015) & (0.015) & (0.016) & (2.556) & (0.012) \\ 
  Race: Black & 0.034$^{*}$ & 0.034$^{*}$ & $-$0.061$^{***}$ & 8.417$^{**}$ & $-$0.050$^{***}$ \\ 
  & (0.018) & (0.018) & (0.016) & (4.117) & (0.012) \\ 
  Race: Hispanic & 0.026$^{**}$ & 0.026$^{**}$ & 0.017 & 7.964$^{***}$ & 0.003 \\ 
  & (0.010) & (0.010) & (0.014) & (2.759) & (0.008) \\ 
  Gender: Man & 0.007 & 0.007 & 0.120$^{**}$ & $-$2.808 & 0.031 \\ 
  & (0.024) & (0.024) & (0.047) & (1.804) & (0.029) \\ 
  Region: Northeast & $-$0.049$^{***}$ & $-$0.049$^{***}$ & 0.020$^{**}$ & $-$0.344 & 0.00003 \\ 
  & (0.007) & (0.007) & (0.009) & (2.339) & (0.005) \\ 
  Region: South & 0.0002 & 0.0002 & 0.010 & $-$4.919 & $-$0.004 \\ 
  & (0.011) & (0.011) & (0.013) & (4.796) & (0.007) \\ 
  Region: West & $-$0.004 & $-$0.004 & 0.009 & $-$0.945 & $-$0.004 \\ 
  & (0.009) & (0.009) & (0.011) & (4.520) & (0.006) \\ 
  Urban & 0.005 & 0.005 & $-$0.020 & $-$4.232 & $-$0.004 \\ 
  & (0.011) & (0.011) & (0.013) & (4.485) & (0.007) \\ 
  urban & 0.001 & 0.001 & 0.008 & 4.599$^{**}$ & $-$0.005 \\ 
  & (0.009) & (0.009) & (0.010) & (2.221) & (0.006) \\ 
 \hline \\[-1.8ex] 

Observations & 5,719 & 5,719 & 3,252 & 3,044 & 3,044 \\ 
R$^{2}$ & 0.023 & 0.023 & 0.030 & 0.006 & 0.016 \\ 
\hline 
\hline \\[-1.8ex] 
\end{tabular} 
        }
    }
    {\footnotesize %\textit{Note}: 
    }
\end{table}

\begin{table}[h]\label{tab:attrition_US2}
    \caption[Attrition analysis: US2]{Attrition analysis for the US2 survey.} 
    \makebox[\textwidth][c]{
\resizebox*{!}{.73\textheight}{ % 73 is the max when there is a title
        
\begin{tabular}{@{\extracolsep{5pt}}lccccc} 
\\[-1.8ex]\hline 
\hline \\[-1.8ex] 
\\[-1.8ex] & \makecell{Dropped out} & \makecell{Dropped out\\after\\socio-eco} & \makecell{Failed\\attention test} & \makecell{Duration\\(in min)} & \makecell{Duration\\below\\4 min} \\ 
\\[-1.8ex] & (1) & (2) & (3) & (4) & (5)\\ 
\hline \\[-1.8ex] 
Mean & 0.105 & 0.08 & 0.112 & 21.78 & 0.041  \\ \hline \\[-1.8ex]
 Income quartile: 2 & 0.007 & 0.007 & $-$0.053$^{***}$ & 1.441 & $-$0.043$^{***}$ \\ 
  & (0.022) & (0.022) & (0.020) & (3.244) & (0.015) \\ 
  Income quartile: 3 & 0.020 & 0.020 & $-$0.011 & 45.106 & $-$0.033 \\ 
  & (0.030) & (0.030) & (0.034) & (46.289) & (0.025) \\ 
  Income quartile: 4 & $-$0.002 & $-$0.002 & $-$0.003 & 1.041 & $-$0.079$^{***}$ \\ 
  & (0.043) & (0.043) & (0.061) & (10.058) & (0.019) \\ 
  Diploma: Post secondary & $-$0.043$^{**}$ & $-$0.043$^{**}$ & $-$0.043$^{**}$ & 9.394 & 0.026 \\ 
  & (0.021) & (0.021) & (0.020) & (9.764) & (0.016) \\ 
  Age: 25-34 & 0.053$^{*}$ & 0.053$^{*}$ & $-$0.045 & $-$7.393 & 0.017 \\ 
  & (0.030) & (0.030) & (0.042) & (6.961) & (0.033) \\ 
  Age: 35-49 & 0.052$^{**}$ & 0.052$^{**}$ & $-$0.042 & 17.468 & 0.006 \\ 
  & (0.026) & (0.026) & (0.039) & (16.385) & (0.029) \\ 
  Age: 50-64 & 0.066$^{**}$ & 0.066$^{**}$ & $-$0.071$^{*}$ & $-$7.421 & $-$0.042$^{*}$ \\ 
  & (0.029) & (0.029) & (0.040) & (9.109) & (0.025) \\ 
  Age: 65+ & 0.057$^{*}$ & 0.057$^{*}$ & $-$0.107$^{***}$ & $-$1.734 & $-$0.052$^{**}$ \\ 
  & (0.030) & (0.030) & (0.037) & (9.343) & (0.025) \\ 
  Race: Black & 0.100$^{***}$ & 0.100$^{***}$ & $-$0.011 & 20.168 & $-$0.016 \\ 
  & (0.021) & (0.021) & (0.033) & (14.147) & (0.023) \\ 
  Race: Hispanic & 0.062$^{***}$ & 0.062$^{***}$ & $-$0.054 & $-$4.035 & $-$0.028 \\ 
  & (0.019) & (0.019) & (0.033) & (7.283) & (0.023) \\ 
  Gender: Man & $-$0.050$^{***}$ & $-$0.050$^{***}$ & 0.015 & 13.563 & 0.017 \\ 
  & (0.018) & (0.018) & (0.023) & (16.255) & (0.017) \\ 
  Region: Northeast & $-$0.018 & $-$0.018 & 0.030 & $-$4.964 & 0.014 \\ 
  & (0.030) & (0.030) & (0.043) & (4.837) & (0.029) \\ 
  Region: South & 0.013 & 0.013 & $-$0.029 & 10.628 & 0.007 \\ 
  & (0.024) & (0.024) & (0.034) & (13.411) & (0.022) \\ 
  Region: West & 0.006 & 0.006 & $-$0.023 & 0.452 & 0.010 \\ 
  & (0.029) & (0.029) & (0.038) & (5.076) & (0.027) \\ 
  Urban & 0.050$^{**}$ & 0.050$^{**}$ & 0.007 & 8.278 & 0.001 \\ 
  & (0.019) & (0.019) & (0.026) & (6.513) & (0.018) \\ 
 \hline \\[-1.8ex] 

Observations & 946 & 946 & 777 & 706 & 706 \\ 
R$^{2}$ & 0.042 & 0.042 & 0.046 & 0.023 & 0.043 \\ 
\hline 
\hline \\[-1.8ex] 
\end{tabular} 
        }
    }
    {\footnotesize %\textit{Note}: 
    }
\end{table}

\begin{table}[h]\label{tab:attrition_EU}
    \caption[Attrition analysis: Eu]{Attrition analysis for the Eu survey.} 
    \makebox[\textwidth][c]{
\resizebox*{!}{.73\textheight}{ % 73 is the max when there is a title
        
\begin{tabular}{@{\extracolsep{5pt}}lccccc} 
\\[-1.8ex]\hline 
\hline \\[-1.8ex] 
\\[-1.8ex] & \makecell{Dropped out} & \makecell{Dropped out\\after\\socio-eco} & \makecell{Failed\\attention test} & \makecell{Duration\\(in min)} & \makecell{Duration\\below\\6 min} \\ 
\\[-1.8ex] & (1) & (2) & (3) & (4) & (5)\\ 
\hline \\[-1.8ex] 
Mean & 0.067 & 0.044 & 0.151 & 54.602 & 0.039  \\ \hline \\[-1.8ex]
 Income quartile: 3 & 0.001 & $-$0.001 & $-$0.031$^{**}$ & 27.825 & $-$0.015 \\ 
  & (0.013) & (0.012) & (0.013) & (20.371) & (0.010) \\ 
  Income quartile: 4 & 0.002 & 0.001 & $-$0.061$^{***}$ & 0.612 & $-$0.022$^{**}$ \\ 
  & (0.014) & (0.013) & (0.011) & (11.887) & (0.010) \\ 
  Diploma: Post secondary & $-$0.022 & $-$0.024$^{*}$ & $-$0.042$^{***}$ & 13.029 & $-$0.019$^{*}$ \\ 
  & (0.014) & (0.014) & (0.013) & (19.608) & (0.010) \\ 
  Age: 25-34 & $-$0.006 & $-$0.005 & $-$0.033$^{***}$ & 5.978 & $-$0.008 \\ 
  & (0.011) & (0.010) & (0.009) & (12.265) & (0.007) \\ 
  Age: 35-49 & 0.028$^{**}$ & 0.025$^{**}$ & 0.033$^{*}$ & 33.335 & $-$0.004 \\ 
  & (0.013) & (0.013) & (0.018) & (20.624) & (0.018) \\ 
  Age: 50-64 & 0.048$^{***}$ & 0.047$^{***}$ & $-$0.006 & 32.456$^{**}$ & $-$0.013 \\ 
  & (0.013) & (0.012) & (0.016) & (14.803) & (0.016) \\ 
  Age: 65+ & 0.074$^{***}$ & 0.073$^{***}$ & $-$0.010 & 41.300$^{**}$ & $-$0.063$^{***}$ \\ 
  & (0.014) & (0.014) & (0.017) & (20.533) & (0.015) \\ 
  Gender: Man & 0.142$^{***}$ & 0.140$^{***}$ & $-$0.011 & 26.513$^{**}$ & $-$0.063$^{***}$ \\ 
  & (0.016) & (0.016) & (0.017) & (12.755) & (0.015) \\ 
  Urban & $-$0.031$^{***}$ & $-$0.031$^{***}$ & 0.013 & $-$24.850$^{*}$ & 0.010 \\ 
  & (0.009) & (0.009) & (0.009) & (14.378) & (0.007) \\ 
  urban & $-$0.010 & $-$0.009 & 0.016$^{*}$ & 13.704 & $-$0.005 \\ 
  & (0.009) & (0.009) & (0.008) & (15.465) & (0.007) \\ 
 \hline \\[-1.8ex] 

Observations & 3,963 & 3,963 & 3,326 & 3,115 & 3,115 \\ 
R$^{2}$ & 0.026 & 0.026 & 0.021 & 0.003 & 0.024 \\ 
\hline 
\hline \\[-1.8ex] 
\end{tabular} 
        }
    }
    {\footnotesize %\textit{Note}: 
    }
\end{table} 

% \clearpage
% \listoftables
% \listoffigures
% WPcomment
%% Here is the endmatter stuff: Supplementary Info, etc.
%% Use \item's to separate, default label is "Acknowledgements"
% \begin{addendum} % 177 words
\begin{itemize}
 \item[Acknowledgements] We are grateful for financial support from the University of Amsterdam and TU Berlin. Mattauch also thanks the Robert Bosch Foundation. We are grateful for financial support from the OECD, the French Ministry of Foreign Affairs, the French Conseil d'Analyse Economique and the Spanish Ministry for the Ecological Transition and Demographic Challenge. We also acknowledge support from the Grantham Foundation for the Protection of the Environment and the Economic and Social Research Council through the Centre for Climate Change Economics and Policy. 
We thank Antoine Dechezleprêtre, Tobias Kruse, Bluebery Planterose, Ana Sanchez Chico, and Stefanie Stantcheva for their invaluable inputs for the project. We thank Antonio Bento, Dietmar Fehr, and Auriane Meilland for feedback. We further thank Jakob Niemann, Laura Schepp, Martín Fernández-Sánchez, Samuel Gervais, Samuel Haddad, and Guadalupe Manzo for assistance. 
 \item[Registration] The project is approved by Economics \& Business Ethics Committee (EBEC) at the University of Amsterdam (EB-1113) and %IRB at Harvard University (IRB21-0137), and 
 was preregistered in the Open Science Foundation registry (osf.io/fy6gd).
 \item[Author Contributions] Fabre collected and analysed the data, and drafted the questionnaire and the paper. Douenne and Mattauch substantially revised the questionnaire and paper, and contributed to the conception and redaction.
 \item[Competing Interests] Fabre declares that he also serves as president of Global Redistribution Advocates.
\item[JEL codes] P48, Q58, H23, Q54.
\item[Keywords] Climate change, global policies, cap-and-trade, perceptions, survey, inequality, wealth tax.
 \item[Correspondence] Correspondence and requests for materials
should be addressed to Adrien Fabre~(email: adrien.fabre@cnrs.fr).
% \end{addendum}
\end{itemize}

%%
%% TABLES
%%
%% If there are any tables, put them here.
%%

% \begin{table}
% \centering
% \caption{This is a table with scientific results.}
% \medskip
% \begin{tabular}{ccccc}
% \hline
% 1 & 2 & 3 & 4 & 5\\
% \hline
% aaa & bbb & ccc & ddd & eee\\
% aaaa & bbbb & cccc & dddd & eeee\\
% aaaaa & bbbbb & ccccc & ddddd & eeeee\\
% aaaaaa & bbbbbb & cccccc & dddddd & eeeeee\\
% 1.000 & 2.000 & 3.000 & 4.000 & 5.000\\
% \hline
% \end{tabular}
% \end{table}

\end{document}


% General comments:
% I prefer my version of the intro
% I prefer to keep all graphs
% The new wording is often less direct/simple and more sophisticated/formal. e.g. "a large majority of Americans expressed that more help is needed" vs. "a significant majority of Americans emphasized the need for increased assistance". Not sure it's always better to be too formal. It is often less clear, and I think it add lengths. e.g. "In our conjoint analyses, we ask respondents to make five choices between pairs of political platforms." vs. "In order to assess the public support for the GCS in conjunction with other policies, we conducted a series of conjoint analyses. We asked respondents to make five choices between pairs of political platforms."
% Subsubsections, not paragraphs, for complementary survey results.
% Lacks results on how to finance more foreign aid. 
% present vs. past tense
% Influence of pros & cons
% Concluding paragraph in universalistic values?